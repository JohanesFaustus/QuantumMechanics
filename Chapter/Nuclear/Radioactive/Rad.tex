\documentclass[../../../main.tex]{subfiles}
\begin{document}
\subsection*{Nuclear Decay}

\subsubsection*{Alpha decay.} Occur when parent nucleon decay distribute among daughter nuclei 
\begin{equation*}
    \ce{^A_ZP}\rightarrow\ce{^{A-4}_{Z-2}D}+\ce{^4_2He}
\end{equation*}
The $Q$ value, which is defined as the total released in a given nuclear decay, of alpha decay is 
\begin{align*}
    Q&=\left[m\left(\ce{^A_ZP}\right)-m(\ce{^{A-4}_{Z-2}D})-m\left(\ce{^4_2He}\right)\right]c^2\\
    Q&=K_D+K_\alpha=\frac{A}{A-4}K_\alpha
\end{align*}
$Q$ can be determined by applying the energy conservation into given nuclear reaction. 

\subsubsection*{Beta decay.} Two example are electron capture and neutrino capture 
\begin{equation*}
    e^-+p\leftrightarrow \nu_e+n
\end{equation*} 
Other interaction can be found by moving particle to different side and changing them to their anti particle, such as 
\begin{equation*}
    \bar{\nu_e}+p\leftrightarrow e^++n
\end{equation*}
where the anti particle of electron $e^-$ and electron neutrino $\nu_e$ are positron $e^+$ and electron antineutrino $\bar{\nu_e}$. Another example is beta negative decay
\begin{equation*}
    \ce(^A_ZP)\rightarrow\ce{^A_{Z+1}D}+e^-+\bar{\nu_e}
\end{equation*} 
which is the neutron decay inside isotope, and beta positive decay
\begin{equation*}
    \ce(^A_ZP)\rightarrow\ce{^A_{Z-1}D}+e^-+{\nu_e}
\end{equation*}
which is the proton decay inside isotope. Electron capture inside isotope takes the form of 
\begin{equation*}
    \ce(^A_ZP)+e^-\rightarrow \ce{^A_{Z-1}}+\nu_e
\end{equation*}
The $Q$ value of beta negative is 
\begin{equation*}
    Q=\left[m\left(\ce(^A_ZP)\right)-m(\ce{^A_{Z-1}D})+m_e+m_{\nu_e}\right]
\end{equation*}
and beta negative decay 
\begin{equation*}
    Q=\left[m\left(\ce(^A_ZP)\right)-m(\ce{^A_{Z-1}D})+m_e+m_{\bar{\nu_e}}\right]
\end{equation*}

\subsection*{Radiation}
\subsubsection*{Radioactive decay.} The number of decay events over a time inteval is proportional to the number of particle
\begin{equation*}
    -\frac{dN}{dt}=\lambda N
\end{equation*}
This first order ODE is solved by integration
\begin{align*}
    -\int_{N_0}^{N}\frac{1}{N}\;dN&=\int_{0}^{t}\lambda t\;dt\\
    \ln\frac{N}{N_0}&=-\lambda t\\
    N(t)=N_0e^{-\lambda t}
\end{align*}

Now consider the case of chain radiation, that is the particle undergoes decay $N_A\rightarrow N_B\rightarrow N_C$. For the first decay of particles $N_A$ into $N_B$
\begin{equation*}
    \frac{dN_A}{dt}=-\lambda_A N_A\implies N_A=N_{A0}e^{-\lambda_A t}
\end{equation*}
Now the total rate of creation for the second particle $N_B$ is the sum of the decay of said particle and the decay of the first particle into the second particle.
\begin{align*}
    \frac{dN_B}{dt}=-\lambda_BN_B+\lambda_AN_A\\
    \frac{dN_B}{dt}+\lambda_BN_B=\lambda_AN_{A0}e^{-\lambda_A t}
\end{align*}
This ODE is solved by integrating factor method
\begin{equation*}
    I=\int \lambda_B\;dt=\lambda_B t
\end{equation*}
Then 
\begin{align*}
    N_B(t)&=e^{-\lambda_B t}\int\lambda_AN_{A0}e^{-\lambda_A t}\;dt +Ce^{-\lambda_B t}\\
    N_B(t)&=\frac{\lambda_AN_{A0}}{\lambda_B-\lambda_1}e^{-\lambda_A t}+Ce^{-\lambda_B t}
\end{align*}
If we assume at $t=0$ we have zero second particle, then 
\begin{equation*}
    N_B(0)=\frac{\lambda_AN_{A0}}{\lambda_B-\lambda_1}+C=0\implies C=-\frac{\lambda_AN_{A0}}{\lambda_B-\lambda_1}
\end{equation*}
Thus, the complete solution is 
\begin{equation*}
    N_B(t)=\frac{\lambda_AN_{A0}}{\lambda_B-\lambda_1}\left(e^{\lambda_A t}-e^{\lambda_Bt}\right)
\end{equation*}

In practice, often we are not given the number of particle, but rather the mol $n$ or mass $m$ (in gram) of said particle. Those quantities are related by 
\begin{equation*}
    n=\frac{m}{M_r}\quad \text{or}\quad n=\frac{N}{N_A}
\end{equation*}
where $M_r\approx A$ is the molecular mass (gram/mol) and $N_A=6.02\cdot 10^{23}\;(\text{mol}^{-1})$ is the Avogadro constant.

\subsubsection*{Halflife.} Defined as the time required for particles to reduce to half of its initial value. The value of halflife can be determined by considering $N(t)$ at $t_{1/2}$, which by definition
\begin{align*}
    \frac{1}{2}N_0&=N_0e^{-\lambda t_{1/2}}\\
    t_{1/2}&=\frac{\ln 2}{\lambda}
\end{align*}
This equation can also be used to determine the decay constant.

\subsubsection*{Activity.} Defined as the number of radioactive transformations per second
\begin{equation*}
    A\equiv-\frac{dN}{dt}=\lambda N
\end{equation*}
Using the solution for $N$, we can write
\begin{equation*}
    A=\lambda N_0 e^{-\lambda t}=A_0e^{-\lambda t}
\end{equation*}

\subsubsection*{Specific Activity.} Quantity related to activity; specific activity is the activity per unit mass
\begin{equation*}
    a\equiv\frac{A}{m}
\end{equation*}
Using the relation of mass with mol and halflife relation
\begin{equation*}
    a=\frac{\lambda N}{\frac{N}{N_A}M_r}=\frac{N_A \lambda }{M_r}=\frac{\ln 2 N_A}{t_{1/2}M_r}
\end{equation*}
On evaluating the numerator constant
\begin{equation*}
    a=\frac{1.32\cdot 10^{16}}{t_{1/2}M}
\end{equation*}
\subsection*{Nuclear Stability}
\begin{figure*}[ht]
    \centering
    \fullfig{../Rss/Nuclear/NuclearStability/NZfig.png}
    \caption*{Figure: Valley of stability in $ZN$ graph.}
\end{figure*}

\subsubsection*{Valley of stability.}  Consist of long-lived isotope that do not simultaneously decay.

\subsubsection*{Below the valley of stability.} Consist of isotopes with more $N$ than those of the valley of stability, thus they decay either by beta negative, more likely, or neutron decay, less likely.

\subsubsection*{Above the valley of stability.} Consist of isotope with more $Z$ than those of the valley of stability, thus it decays either by beta positive or electron capture.

\subsubsection*{Beyond the valley of stability.}  Consist of heavy isotope with $Z>83$, $N>126$, and $A>209$. They decay with alpha radiation.

\end{document}