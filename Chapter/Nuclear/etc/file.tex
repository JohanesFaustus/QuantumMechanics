\documentclass[../../../main.tex]{subfiles}
\begin{document}
\subsection{Atomic Species}
Characterized by the number of neutron $N$, number of proton $Z$, and mass number $A=N+Z$
\begin{equation*}
    (A,Z)= \;^A_ZX= \;^A_ZX_N
\end{equation*}

\subsection{Nucleon}
Defined as bound state of atomic nuclei. The two type are positively charged proton and neutral neutron. Nucleon constitutes three bound fermions called quark: up with charge ($2/3$) and down with charge $(-1/3)$
\begin{align*}
    \text{proton}&=\text{uud}\\
    \text{neutron}&=\text{udd}
\end{align*}

Both of them are fermion with mass 
\begin{align*}
    m_e&=939.56 \text{ MeV}/c^2\\ 
    m_p &= 938.27 \text{ MeV}/c^2\\
     m_n-m_e&=1.29 \text{ MeV}/c^2
\end{align*}

The magnetic moment projected by both are 
\begin{equation*}
    \mu_p=2.792 847 386  \;\mu_N\quad \mu_n=-1.913 042 75\;\mu_N
\end{equation*}
where $\mu_N$ denote nuclear magneton
\begin{equation*}
    \mu_N=\frac{e\hbar}{2m_p}=3.152 451 66\; 10^{-14}\;\text{MeV/T}
\end{equation*}

Here are the difference in unit used to describe nucleus compared to atom 
\begin{center}
    \begin{tabular}{c c c} 
        \toprule
        Properties&Atom&Nucleus\\ 
        \midrule
        Radius&Angstrom ($10^{-10}$ m)&Femto ($10^{-15}$ m)\\
        Energy&eV&MeV\\ \bottomrule
    \end{tabular}
\end{center}

\subsubsection{Radii.} In terms of their mass number $A$, their radius may be approximated as 
\begin{equation*}
    R=r_0A^{1/3}\quad\text{with}\quad r_0=1.2\;\text{fm}
\end{equation*}
This approximation comes from assuming the radius is proportional to the volume which is also assumed to be spherical. Then $\mathcal{V}=4\pi R^3/3\approx A$. 

\subsubsection{Binding energy.} Defined as the difference of the sum of nuclei mass and the nuclear mass 
\begin{equation*}
    B(A,Z)=Nm_nc^2+Zm_pc^2-m(A,Z)c^2
\end{equation*} 

\subsubsection{Mass.} Three unit most common are atomic mass unit (u), the kilogram (kg), and the electron-volt (eV). The atomic mass unit is defined as the mass of $^{12}$C atom divided by 12
\begin{equation*}
    1\;\text{u}=\frac{m (^{12}C)}{12}
\end{equation*}
electron volt is defined as the kinetic energy of an electron after being accelerated from rest through a potential difference of 1 V.

\subsection{Nuclear Relative}
\subsubsection{Isotope.} Same number of charge $Z$, but different number of neutron $N$. Isotope has identical chemical properties, since they have the same electron, but different nuclear properties. Example are
\begin{equation*}
    ^{238}_{92}\text{U}\quad\text{and}\quad    ^{235}_{92}\text{U}
\end{equation*}

\subsubsection{Isobar.} Same mass $A$. Frequently have the same nuclear properties due to the same number of nucleon. Example are
\begin{equation*}
    ^{3}\text{He}\quad\text{and}\quad    ^{3}\text{H}
\end{equation*}

\subsubsection{Isotone.} Same number of neutron $N$, but different number of proton $Z$. Example are
\begin{equation*}
    ^{14}\text{C}_6\quad\text{and}\quad    ^{16}\text{O}_8
\end{equation*}



\end{document}