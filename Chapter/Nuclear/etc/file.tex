\documentclass[../../../main.tex]{subfiles}
\begin{document}
\subsection*{Atomic Species}
Characterized by the number of neutron $N$, number of proton $Z$, and mass number $A=N+Z$
\begin{equation*}
    (A,Z)= \;^A_ZX= \;^A_ZX_N
\end{equation*}

\subsection*{Nucleon}
Defined as bound state of atomic nuclei. The two type are positively charged proton and neutral neutron. Nucleon constitutes three bound fermions called quark: up with charge ($2/3$) and down with charge $(-1/3)$
\begin{align*}
    \text{proton}&=\text{uud}\\
    \text{neutron}&=\text{udd}
\end{align*}

Both of them are fermion with mass 
\begin{align*}
    m_e&=939.56 \text{ MeV}/c^2\\ 
    m_p &= 938.27 \text{ MeV}/c^2\\
     m_n-m_e&=1.29 \text{ MeV}/c^2
\end{align*}

The magnetic moment projected by both are 
\begin{equation*}
    \mu_p=2.792 847 386  \;\mu_N\quad \mu_n=-1.913 042 75\;\mu_N
\end{equation*}
where $\mu_N$ denote nuclear magneton
\begin{equation*}
    \mu_N=\frac{e\hbar}{2m_p}=3.152 451 66\; 10^{-14}\;\text{MeV/T}
\end{equation*}

Here are the difference in unit used to describe nucleus compared to atom 
\begin{center}
    \begin{tabular}{c |c| c}
        Properties&Atom&Nucleus\\
        \hline\hline
        Radius&Angstrom ($10^{-10}$ m)&Femto ($10^{-15}$ m)\\
        Energy&eV&MeV
    \end{tabular}
\end{center}

\subsubsection*{Radii.} In terms of their mass number $A$, their radius may be approximated as 
\begin{equation*}
    R=r_0A^{1/3}\quad\text{with}\quad r_0=1.2\;\text{fm}
\end{equation*}
This approximation comes from assuming the radius is proportional to the volume which is also assumed to be spherical. Then $\mathcal{V}=4\pi R^3/3\approx A$. 

\subsubsection*{Binding energy.} Defined as the difference of the sum of nuclei mass and the nuclear mass 
\begin{equation*}
    B(A,Z)=Nm_nc^2+Zm_pc^2-m(A,Z)c^2
\end{equation*} 

\subsubsection*{Mass.} Three unit most common are atomic mass unit (u), the kilogram (kg), and the electron-volt (eV). The atomic mass unit is defined as the mass of $^{12}$C atom divided by 12
\begin{equation*}
    1\;\text{u}=\frac{m (^{12}C)}{12}
\end{equation*}
electron volt is defined as the kinetic energy of an electron after being accelerated from rest through a potential difference of 1 V.

\subsection*{Nuclear Relative}
\subsubsection*{Isotope.} Same number of charge $Z$, but different number of neutron $N$. Isotope has identical chemical properties, since they have the same electron, but different nuclear properties. Example are
\begin{equation*}
    ^{238}_{92}\text{U}\quad\text{and}\quad    ^{235}_{92}\text{U}
\end{equation*}

\subsubsection*{Isobar.} Same mass $A$. Frequently have the same nuclear properties due to the same number of nucleon. Example are
\begin{equation*}
    ^{3}\text{He}\quad\text{and}\quad    ^{3}\text{H}
\end{equation*}

\subsubsection*{Isotone.} Same number of neutron $N$, but different number of proton $Z$. Example are
\begin{equation*}
    ^{14}\text{C}_6\quad\text{and}\quad    ^{16}\text{O}_8
\end{equation*}


\subsection*{Nuclear Decay}

\subsubsection*{Alpha decay.} Occur when parent nucleon decay distribute among daughter nuclei 
\begin{equation*}
    \ce{^A_ZP}\rightarrow\ce{^{A-4}_{Z-2}D}+\ce{^4_2He}
\end{equation*}
The $Q$ value, which is defined as the total released in a given nuclear decay, of alpha decay is 
\begin{align*}
    Q&=\left[m\left(\ce{^A_ZP}\right)-m(\ce{^{A-4}_{Z-2}D})-m\left(\ce{^4_2He}\right)\right]c^2\\
    Q&=K_D+K_\alpha=\frac{A}{A-4}K_\alpha
\end{align*}
$Q$ can be determined by applying the energy conservation into given nuclear reaction. 

\subsubsection*{Beta decay.} Two example are electron capture and neutrino capture 
\begin{equation*}
    e^-+p\leftrightarrow \nu_e+n
\end{equation*} 
Other interaction can be found by moving particle to different side and changing them to their anti particle, such as 
\begin{equation*}
    \bar{\nu_e}+p\leftrightarrow e^++n
\end{equation*}
where the anti particle of electron $e^-$ and electron neutrino $\nu_e$ are positron $e^+$ and electron antineutrino $\bar{\nu_e}$. Another example is beta negative decay
\begin{equation*}
    \ce(^A_ZP)\rightarrow\ce{^A_{Z+1}D}+e^-+\bar{\nu_e}
\end{equation*} 
which is the neutron decay inside isotope, and beta positive decay
\begin{equation*}
    \ce(^A_ZP)\rightarrow\ce{^A_{Z-1}D}+e^-+{\nu_e}
\end{equation*}
which is the proton decay inside isotope. Electron capture inside isotope takes the form of 
\begin{equation*}
    \ce(^A_ZP)+e^-\rightarrow \ce{^A_{Z-1}}+\nu_e
\end{equation*}
The $Q$ value of beta negative is 
\begin{equation*}
    Q=\left[m\left(\ce(^A_ZP)\right)-m(\ce{^A_{Z-1}D})+m_e+m_{\nu_e}\right]
\end{equation*}
and beta negative decay 
\begin{equation*}
    Q=\left[m\left(\ce(^A_ZP)\right)-m(\ce{^A_{Z-1}D})+m_e+m_{\bar{\nu_e}}\right]
\end{equation*}

\subsection*{Radiation}
\subsubsection*{Radioactive decay.} The number of decay events over a time inteval is proportional to the number of particle
\begin{equation*}
    -\frac{dN}{dt}=\lambda N
\end{equation*}
This first order ODE is solved by integration
\begin{align}
    -\int_{N_0}^{N}\frac{1}{N}\;dN&=\int_{0}^{t}\lambda t\;dt\\
    \ln\frac{N}{N_0}&=-\lambda t\\
    N(t)=N_0e^{-\lambda t}
\end{align}

Now consider the case of chain radiation, that is the particle undergoes decay $N_A\rightarrow N_B\rightarrow N_C$. For the first decay of particles $N_A$ into $N_B$
\begin{equation*}
    \frac{dN_A}{dt}=-\lambda_A N_A\implies N_A=N_{A0}e^{-\lambda_A t}
\end{equation*}
Now the total rate of creation for the second particle $N_B$ is the sum of the decay of said particle and the decay of the first particle into the second particle.
\begin{align*}
    \frac{dN_B}{dt}=-\lambda_BN_B+\lambda_AN_A\\
    \frac{dN_B}{dt}+\lambda_BN_B=\lambda_AN_{A0}e^{-\lambda_A t}
\end{align*}
This ODE is solved by integrating factor method
\begin{equation*}
    I=\int \lambda_B\;dt=\lambda_B t
\end{equation*}
Then 
\begin{align*}
    N_B(t)&=e^{-\lambda_B t}\int\lambda_AN_{A0}e^{-\lambda_A t}\;dt +Ce^{-\lambda_B t}\\
    N_B(t)&=\frac{\lambda_AN_{A0}}{\lambda_B-\lambda_1}e^{-\lambda_A t}+Ce^{-\lambda_B t}
\end{align*}
If we assume at $t=0$ we have zero second particle, then 
\begin{equation*}
    N_B(0)=\frac{\lambda_AN_{A0}}{\lambda_B-\lambda_1}+C=0\implies C=-\frac{\lambda_AN_{A0}}{\lambda_B-\lambda_1}
\end{equation*}
Thus, the complete solution is 
\begin{equation*}
    N_B(t)=\frac{\lambda_AN_{A0}}{\lambda_B-\lambda_1}\left(e^{\lambda_A t}-e^{\lambda_Bt}\right)
\end{equation*}

In practice, often we are not given the number of particle, but rather the mol $n$ or mass $m$ (in gram) of said particle. Those quantities are related by 
\begin{equation*}
    n=\frac{m}{M_r}\quad \text{or}\quad n=\frac{N}{N_A}
\end{equation*}
where $M_r\approx A$ is the molecular mass (gram/mol) and $N_A=6.02\cdot 10^{23}\;(\text{mol}^{-1})$ is the Avogadro constant.

\subsubsection*{Halflife.} Defined as the time required for particles to reduce to half of its initial value. The value of halflife can be determined by considering $N(t)$ at $t_{1/2}$, which by definition
\begin{align*}
    \frac{1}{2}N_0&=N_0e^{-\lambda t_{1/2}}\\
    t_{1/2}&=\frac{\ln 2}{\lambda}
\end{align*}
This equation can also be used to determine the decay constant.

\subsubsection*{Activity.} Defined as the number of radioactive transformations per second
\begin{equation*}
    A\equiv-\frac{dN}{dt}=\lambda N
\end{equation*}
Using the solution for $N$, we can write
\begin{equation*}
    A=\lambda N_0 e^{-\lambda t}=A_0e^{-\lambda t}
\end{equation*}

\subsubsection*{Specific Activity.} Quantity related to activity; specific activity is the activity per unit mass
\begin{equation*}
    a\equiv\frac{A}{m}
\end{equation*}
Using the relation of mass with mol and halflife relation
\begin{equation*}
    a=\frac{\lambda N}{\frac{N}{N_A}M_r}=\frac{N_A \lambda }{M_r}=\frac{\ln 2 N_A}{t_{1/2}M_r}
\end{equation*}
On evaluating the numerator constant
\begin{equation*}
    a=\frac{1.32\cdot 10^{16}}{t_{1/2}M}
\end{equation*}
\subsection*{Nuclear Stability}
\begin{figure*}[h]
    \centering
    \fullfig{../Rss/Nuclear/NuclearStability/NZfig.png}
    \caption*{Figure: Valley of stability in $ZN$ graph.}
\end{figure*}

\subsubsection*{Valley of stability.}  Consist of long-lived isotope that do not simultaneously decay.

\subsubsection*{Below the valley of stability.} Consist of isotopes with more $N$ than those of the valley of stability, thus they decay either by beta negative, more likely, or neutron decay, less likely.

\subsubsection*{Above the valley of stability.} Consist of isotope with more $Z$ than those of the valley of stability, thus it decays either by beta positive or electron capture.

\subsubsection*{Beyond the valley of stability.}  Consist of heavy isotope with $Z>83$, $N>126$, and $A>209$. They decay with alpha radiation.

\subsection*{Conservation Laws}
\subsubsection*{Energy.} The total energy $E=K+mc^2$ must be conserved in all types of nuclear reaction.

\subsubsection*{Momentum.} Same as energy conservation law.

\subsubsection*{Mass number.} Same as energy conservation law.

\subsubsection*{Charge.} Same as energy conservation law.

\subsubsection*{Lepton number.} We assign lepton a lepton number $L=+1$, anti lepton $L=-1$, and non lepton $L=0$. Each family of lepton--such as electron $e$, muon $\mu$, tau $\tau$ and their neutrino sibling--has separate conserved neutrino number--$L_e$, $L_\mu$, $L_\tau$ repectively.

This quantum number almost conserved in all reaction, exception exist in neutrino oscillations; however only family lepton number is violated, while the total lepton number is conserved.

\subsubsection*{Baryon number.} Like before, we assign baryon a baryon number $L=+1$, anti baryon $L=-1$, and non baryon $L=0$. As an aside, baryon is the fined as the bound state of three quark and that strong force works on all of them.

\subsubsection*{Strangeness number.} Defined as the negative of the number of strange quarks in it, in particular strange quark $s$ has $S=-1$ and antistrange quark $\bar{s}$ has $S=+1$.

\subsubsection*{Isospin (Isotropic Spin).} We define isospin $I$ such that $2I+1$ is equal to the multiplet type of the baryon. Recall that the strong force does not differentiate between proton and nucleon, hence both particle can be defined as the different state of the same particle and thus can be categorized as doublet. To differentiate them, then, we define proton with $I_z=1/2$ and neutron with $I_z=-1/2$

Isospin is conserved in strong force interaction, but may not in other interaction.

\begin{comment}
\subsection*{Particle Physics}
\begin{figure*}
    \centering
    \fullfig{../Rss/QM/NuclearPhysics/Intro/StandardModel}
    \caption*{Figure: Standard model.}
\end{figure*}
\end{comment}

\end{document}