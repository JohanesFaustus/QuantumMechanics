\documentclass[../../../main.tex]{subfiles}
\begin{document}
\subsection*{Atomic Species}
Characterized by the number of neutron $N$, number of proton $Z$, and mass number $A=N+Z$
\begin{equation*}
    (A,Z)= \;^A_ZX= \;^A_ZX_N
\end{equation*}

\subsection*{Nucleon}
Defined as bound state of atomic nuclei. The two type are positively charged proton and neutral neutron. Nucleon constitutes three bound fermions called quark: up with charge ($2/3$) and down with charge $(-1/3)$
\begin{align*}
    \text{proton}&=\text{uud}\\
    \text{neutron}&=\text{udd}
\end{align*}

Both of them are fermion with mass 
\begin{align*}
    m_e&=939.56 \text{ MeV}/c^2\\ 
    m_p &= 938.27 \text{ MeV}/c^2\\
     m_n-m_e&=1.29 \text{ MeV}/c^2
\end{align*}

The magnetic moment projected by both are 
\begin{equation*}
    \mu_p=2.792 847 386  \;\mu_N\quad \mu_n=-1.913 042 75\;\mu_N
\end{equation*}
where $\mu_N$ denote nuclear magneton
\begin{equation*}
    \mu_N=\frac{e\hbar}{2m_p}=3.152 451 66\; 10^{-14}\;\text{MeV/T}
\end{equation*}

Here are the difference in unit used to describe nucleus compared to atom 
\begin{center}
    \begin{tabular}{c |c| c}
        Properties&Atom&Nucleus\\
        \hline\hline
        Radius&Angstrom ($10^{-10}$ m)&Femto ($10^{-15}$ m)\\
        Energy&eV&MeV
    \end{tabular}
\end{center}

\subsubsection*{Radii.} In terms of their mass number $A$, their radius may be approximated as 
\begin{equation*}
    R=r_0A^{1/3}\quad\text{with}\quad r_0=1.2\;\text{fm}
\end{equation*}
This approximation comes from assuming the radius is proportional to the volume which is also assumed to be spherical. Then $\mathcal{V}=4\pi R^3/3\approx A$. 

\subsubsection*{Binding energy.} Defined as the difference of the sum of nuclei mass and the nuclear mass 
\begin{equation*}
    B(A,Z)=Nm_nc^2+Zm_pc^2-m(A,Z)c^2
\end{equation*} 

\subsubsection*{Mass.} Three unit most common are atomic mass unit (u), the kilogram (kg), and the electron-volt (eV). The atomic mass unit is defined as the mass of $^{12}$C atom divided by 12
\begin{equation*}
    1\;\text{u}=\frac{m (^{12}C)}{12}
\end{equation*}
electron volt is defined as the kinetic energy of an electron after being accelerated from rest through a potential difference of 1 V.

\subsection*{Nuclear Relative}
\subsubsection*{Isotope.} Same number of charge $Z$, but different number of neutron $N$. Isotope has identical chemical properties, since they have the same electron, but different nuclear properties. Example are
\begin{equation*}
    ^{238}_{92}\text{U}\quad\text{and}\quad    ^{235}_{92}\text{U}
\end{equation*}

\subsubsection*{Isobar.} Same mass $A$. Frequently have the same nuclear properties due to the same number of nucleon. Example are
\begin{equation*}
    ^{3}\text{He}\quad\text{and}\quad    ^{3}\text{H}
\end{equation*}

\subsubsection*{Isotone.} Same number of neutron $N$, but different number of proton $Z$. Example are
\begin{equation*}
    ^{14}\text{C}_6\quad\text{and}\quad    ^{16}\text{O}_8
\end{equation*}


\subsection*{Nuclear Decay}

\subsubsection*{Alpha decay.} Occur when parent nucleon decay distribute among daughter nuclei 
\begin{equation*}
    \ce{^A_ZP}\rightarrow\ce{^{A-4}_{Z-2}D}+\ce{^4_2He}
\end{equation*}
The $Q$ value, which is defined as the total released in a given nuclear decay, of alpha decay is 
\begin{align*}
    Q&=\left[m\left(\ce{^A_ZP}\right)-m(\ce{^{A-4}_{Z-2}D})-m\left(\ce{^4_2He}\right)\right]c^2\\
    Q&=K_D+K_\alpha=\frac{A}{A-4}K_\alpha
\end{align*}
$Q$ can be determined by applying the energy conservation into given nuclear reaction. 

\subsubsection*{Beta decay.} Two example are electron capture and neutrino capture 
\begin{equation*}
    e^-+p\leftrightarrow \nu_e+n
\end{equation*} 
Other interaction can be found by moving particle to different side and changing them to their anti particle, such as 
\begin{equation*}
    \bar{\nu_e}+p\leftrightarrow e^++n
\end{equation*}
where the anti particle of electron $e^-$ and electron neutrino $\nu_e$ are positron $e^+$ and electron antineutrino $\bar{\nu_e}$. Another example is beta negative decay
\begin{equation*}
    \ce(^A_ZP)\rightarrow\ce{^A_{Z+1}D}+e^-+\bar{\nu_e}
\end{equation*} 
which is the neutron decay inside isotope, and beta positive decay
\begin{equation*}
    \ce(^A_ZP)\rightarrow\ce{^A_{Z-1}D}+e^-+{\nu_e}
\end{equation*}
which is the proton decay inside isotope. Electron capture inside inside isotope takes the form of 
\begin{equation*}
    \ce(^A_ZP)+e^-\rightarrow \ce{^A_{Z-1}}+\nu_e
\end{equation*}
The $Q$ value of beta negative is 
\begin{equation*}
    Q=\left[m\left(\ce(^A_ZP)\right)-m(\ce{^A_{Z-1}D})+m_e+m_{\nu_e}\right]
\end{equation*}
and beta negative decay 
\begin{equation*}
    Q=\left[m\left(\ce(^A_ZP)\right)-m(\ce{^A_{Z-1}D})+m_e+m_{\bar{\nu_e}}\right]
\end{equation*}


\subsection*{Nuclear Stability}
\subsubsection*{Valley of stability.}  Consist of long-lived isotope that do not mutinously decay.

\subsubsection*{Below the valley of stability.} Consist of isotope with more $N$ than those of the valley of stability, thus it decay with beta ne
\begin{figure*}
    \centering
    \fullfig{../Rss/Nuclear/NuclearStability/NZfig.png}
\end{figure*}
\end{document}