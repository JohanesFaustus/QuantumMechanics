\documentclass[../../../main.tex]{subfiles}
\begin{document}
\subsection*{Elementary Particle}
\begin{figure*}[b]
    \centering
    \fullfig{../Rss/QM/NuclearPhysics/Intro/StandardModel}
    \caption*{Figure: Standard model.}
\end{figure*}
Particle can be categorized by their spin, mass, and the type of interaction.

\subsubsection*{Boson.} A pseudo-particle with interger spin that mediate the interaction. Two types of boson are gauge, or force carrier, and scalar, whose function is to give particle mass. Also, $W^{\pm}$ and $Z^0$ boson both belong to the same isospin triplet with $I=1$
\begin{longtable}{cc}
    \caption*{Table: Boson properties}\\
    \toprule
    \multicolumn{2}{c}{Gauge Boson}\\
    \midrule
    Family&Function/Force\\
    \midrule
    Photon $\gamma$&Electromagnetic\\
    Gluon $g$&Strong: binds the quark\\
    W Boson $W^{\pm}$&Weak:radioactive decay\\
    Z Boson $Z^0$&Weak: same as above\\
    \bottomrule
    \toprule
    \multicolumn{2}{c}{Scalar Boson}\\
    \midrule
    Family&Function/Force\\
    \midrule
    Higgs Boson $H^0$&Gives particle mass\\
    \bottomrule
\end{longtable}

\subsubsection*{Fermion.} Consist of quark and lepton. All has half interger spin. Bound state of quark is called hadron and quark is able to experience all four fundamental forces.

Lepton experiences the weak force, however for charged lepton is also able to experience electromagnetic forces. This is why neutral lepton, that is neutrino, is hard to detect.

\subsubsection*{Hadron.} Two types of hadron are baryon, which consist of three quarks, and meson, which consist of two quarks and one antiquark. Since baryon is made up of three spin-1/2, its spin is always half interger, opposite with meson; thus, both are also able to experience all four fundamental forces. Anti baryon consist of antiquark with the same configuration, for example is anti proton $\bar{p}=\bar{u}\bar{u}\bar{d}$.

One difference is that, unlike baryon, meson do not follow Pauli exclusion principle, since its total spin is interger. Most mesons are not their own antiparticle: charged mesons  like $K^+$ with $K^-$; and neutral meson like $K^0$ with $\bar{K^0}$.

\subsubsection*{Quantum Number.} Summarized as follows. First we have the fundamental particles. 

\begin{longtable}{ccccccc}
    \caption*{Table: fundamental particle properties}\\
    \toprule
    \multicolumn{7}{c}{Boson}\\
    \midrule
    Name & $q$  & $L$        & $B$  & $S$ & $I$ & $I_z$ \\
    \midrule
    $\gamma$    & 0    & 0          & 0    & 0   & 0   & 0     \\
    $g$    & 0    & 0          & 0    & 0   & 0   & 0     \\
    $W^+$   & +1   & 0          & 0    & 0   & 1   & +1    \\
    $Z^0$    & 0    & 0          & 0    & 0   & 1   & 0     \\
    $W^-$   & -1   & 0          & 0    & 0   & 1   & -1    \\
    $H^0$    & 0    & 0          & 0    & 0   & 0   & 0     \\
    \bottomrule
    \toprule
    \multicolumn{7}{c}{Quark}                           \\
    \midrule
    Name & $q$  & $L$        & $B$  & $S$ & $I$ & $I_z$ \\
    \midrule
    u    & +2/3 & 0       & +1/3 & 0   & 1/2 & +1/2  \\
    d    & -1/3 & 0        & +1/3 & 0   & 1/2 & -1/2  \\
    s    & -1/3 & 0        & +1/3 & -1  & 0   & 0     \\
    c    & +2/3 & 0        & +1/3 & 0   & 0   & 0     \\
    b    & -1/3 & 0        & +1/3 & 0   & 0   & 0     \\
    t    & +2/3 & 0        & +1/3 & 0   & 0   & 0     \\
    \bottomrule
    \toprule
    \multicolumn{7}{c}{Lepton}                          \\
    \midrule
    Name & $q$  & $L$        & $B$  & $S$ & $I$ & $I_z$ \\
    \midrule
    $e$    & -1   & $L_e=1$    & 0    & 0   & 1/2 & +1/2  \\
    $\mu$   & -1   & $L_\mu=1$  & 0    & 0   & 1/2 & -1/2  \\
    $\tau$  & -1   & $L_\tau=1$ & 0    & 0   & 0   & 0     \\
    $\nu_e$ & 0    & $L_e=1$    & 0    & 0   & 0   & 0     \\
    $\nu_\mu$ & 0    & $L_\mu=1$  & 0    & 0   & 0   & 0     \\
    $\nu_\tau$ & 0    & $L_\tau=1$ & 0    & 0   & 0   & 0    \\
    \bottomrule
\end{longtable}
Then the Hadron.
\begin{longtable}{cccccccc}
    \caption*{Table: Hadron properties}\\
    \toprule
    \multicolumn{8}{c}{Baryon}                           \\ 
    \midrule
    Name & Content & $q$ & $L$ & $B$ & $S$ & $I$ & $I_z$ \\
    \midrule
    $p$    & uud     & +1  & 0   & +1  & 0   & 1/2 & +1/2  \\
    $n$    & udd     & 0   & 0   & +1  & 0   & 1/2 & -1/2  \\
    $\Lambda^0$   & uds     & 0   & 0   & +1  & -1  & I   & 0     \\
    $\Sigma^+$   & uus     & +1  & 0   & +1  & -1  & 1   & +1    \\
    $\Sigma^0$   & uds     & 0   & 0   & +1  & -1  & 1   & 0     \\
    $\Sigma^-$   & dds     & -1  & 0   & +1  & -1  & 1   & -1    \\
    $\Xi^0$   & uss     & 0   & 0   & +1  & -2  & 1/2 & +1/2  \\
    $\Xi^-$   & dss     & -1  & 0   & +1  & -2  & 1/2 & -1/2  \\
    $\Omega^-$   & sss     & -1  & 0   & +1  & -3  & 0   & 0     \\ 
    \bottomrule
    \toprule
    \multicolumn{8}{c}{Meson}                                                    \\ \midrule
    Name   & Content                       & $q$ & $L$ & $B$ & $S$ & $I$ & $I_z$ \\
    $\pi^+$     & $u \bar{d}$                    & +1  & 0   & 0   &    & 1   & +1    \\
    $\pi^0$     & $u\bar{u}, d\bar{d}$            & 0   & 0   & 0   & 0   & 1   & 0     \\
    $\pi^-$     & $d\bar{u}$                      & -1  & 0   & 0   & 0   & 1   & -1    \\
    $K^+$     & $d\bar{s}$                      & +1  & 0   & 0   & +1   & 1/2 & +1/2  \\
    $K^0$     & $d\bar{s}$                      & 0   & 0   & 0   & +1   & 1/2 & -1/2  \\
    $\bar{K^0}$ & $s\bar{d}$                      & 0   & 0   & 0   & -1   & 1/2 & +1/2  \\
    $K^-$     & $s\bar{u}$                      & -1  & 0   & 0   & -1   & 1/2 & -1/2  \\
    $\eta$    & $u\bar{u}, d\bar{d},s\bar{s}$   & 0   & 0   & 0   & 0   & 0   & 0     \\
    $\eta'$   & $u\bar{u},  d\bar{d}, s\bar{s}$ & 0   & 0   & 0   & 0   & 0   & 0     \\ \bottomrule
\end{longtable}

\subsection*{Conservation Laws}
\subsubsection*{Energy.} The total energy $E=K+mc^2$ must be conserved in all types of nuclear reaction.

\subsubsection*{Momentum.} Same as energy conservation law.

\subsubsection*{Mass number.} Same as energy conservation law.

\subsubsection*{Charge.} Same as energy conservation law.

\subsubsection*{Lepton number.} We assign lepton a lepton number $L=+1$, anti lepton $L=-1$, and non lepton $L=0$. Each family of lepton--such as electron $e$, muon $\mu$, tau $\tau$ and their neutrino sibling--has separate conserved neutrino number--$L_e$, $L_\mu$, $L_\tau$ repectively.

This quantum number almost conserved in all reaction, exception exist in neutrino oscillations; however only family lepton number is violated, while the total lepton number is conserved.

\subsubsection*{Baryon number.} Like before, we assign baryon a baryon number $L=+1$, anti baryon $L=-1$, and non baryon $L=0$. As an aside, baryon is the defined as the bound state of three quarke and that strong force works on all of them.

\subsubsection*{Strangeness number.} Defined as the negative of the number of strange quarks in it, in particular strange quark $s$ has $S=-1$ and antistrange quark $\bar{s}$ has $S=+1$.

\subsubsection*{Isospin (Isotropic Spin).} We define isospin $I$ such that $2I+1$ is equal to the multiplet type of the baryon. Recall that the strong force does not differentiate between proton and nucleon, hence both particle can be defined as the different state of the same particle and thus can be categorized as doublet. To differentiate them, then, we define proton with $I_z=1/2$ and neutron with $I_z=-1/2$. In general,
\begin{equation*}
    I_z=I,I-1,\dots,-I
\end{equation*}

Isospin is conserved in strong force interaction, but may not in other interaction.
\end{document}