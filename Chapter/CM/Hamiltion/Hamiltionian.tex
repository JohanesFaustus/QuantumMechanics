\documentclass[../../../main.tex]{subfiles}
\begin{document}
The Hamiltonian is defined as 
\begin{equation*}
    \mathcal{H}=\sum_{i=1}^{n}p_i\dot{q}_1-\mathcal{L}
\end{equation*}

\subsubsection*{Notation}
To avoid clutter, we define the following notation.
The $n$-dimensional system with $n$-generalized coordinate is represented by 
\begin{equation*}
    \mathbf{r}\equiv(q_1,\dots,wq_n)
\end{equation*}
while $n$-generalized velocity
\begin{equation*}
    \dot{\mathbf{q}}\equiv(\dot{q}_1,\dots,\dot{q}_n)
\end{equation*}
and generalized momentum
\begin{equation*}
    \mathbf{p}\equiv(p_1,\dots,p_n)
\end{equation*}
This $\mathbf{p}$ and $\mathbf{p}$ are $n$-dimensional vectors in the space of generalized position and generalized momentum.

\subsubsection*{Definition}
The definition Hamiltonian comes from Lagrangian.
To see this, consider the change of Lagrangian $\mathcal{L}(\mathbf{q},\dot{\mathbf{q}},t)$ as the time increase
\begin{align*}
    \frac{d\mathcal{L}}{dt}&=\frac{\partial\mathcal{L}}{\partial q_1}\frac{\partial q_1}{\partial t}\dots+\frac{\partial\mathcal{L}}{\partial q_n}\frac{\partial q_n}{\partial t}+\frac{\partial\mathcal{L}}{\partial \dot{q}_1}\frac{\partial \dot{q}_2}{\partial t}+\dots+\frac{\partial \mathcal{L}}{\partial \dot{q_n}}\frac{\partial\dot{q}_n}{\partial t}+\frac{\partial\mathcal{L}}{\partial t}\frac{\partial t}{\partial t}\\
    &=\sum_{i=1}^{n}\left(\frac{\partial \mathcal{L}}{\partial q_i}\dot{q}_i+\frac{\partial\mathcal{L}}{\partial\dot{q}_i}\ddot{q}_i\right)+\frac{\partial \mathcal{L}}{\partial t}\\
    &=\sum_{i=1}^{n}\left(\dot{p}_i\dot{q}_i+p_i\ddot{q}_i\right)+\frac{\partial \mathcal{L}}{\partial t}\\
    \frac{d\mathcal{L}}{dt}&=\frac{d}{dt}\sum_{i=1}^{n}\left(p_i\dot{q}_i\right)+\frac{\partial\mathcal{L}}{\partial t}
\end{align*}
If we assume the Lagrangian does not depend explicitly on time, say in the case of $T=T(\mathbf{q},\dot{\mathbf{q}})$ and $U=U(\mathbf{q})$, the $\partial\mathcal{L}/\partial t=0$. 
Thus, the quantity
\begin{equation*}
    \frac{d}{dt}\left(\sum_{i=1}^{n}p_i\dot{q}_i-\mathcal{L}\right)=0
\end{equation*}
which defined as Hamiltonian, is conserved. In other words,
\begin{equation*}
    \frac{\partial\mathcal{L}}{\partial t}=0\implies\frac{dH}{dt}=0
\end{equation*}
\end{document}