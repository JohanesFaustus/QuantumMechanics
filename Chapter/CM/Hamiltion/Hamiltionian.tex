\documentclass[../../../main.tex]{subfiles}
\begin{document}
The Hamiltonian is defined as 
\begin{equation*}
    \mathcal{H}=\sum_{i=1}^{n}p_i\dot{q}_1-\mathcal{L}
\end{equation*}

The conservation of Hamiltonian is stated as follows
\begin{equation*}
    \frac{\partial\mathcal{L}}{\partial t}=0\implies\frac{dH}{dt}=0
\end{equation*}

In the case of natural coordinate, that is the relation between the generalized coordinates and Cartesian is time-independent, the Hamiltonian takes the simple form $\mathcal{H}=T+U$
\begin{equation*}
    q_i=q_i(\mathbf{q})\implies\mathcal{H}=T+U
\end{equation*}

\subsection*{Notation}
To avoid clutter, we define the following notation.
The $n$-dimension system with $n$-generalized coordinate is represented by 
\begin{equation*}
    \mathbf{q}\equiv(q_1,\dots,q_n)
\end{equation*}
while $n$-generalized velocity
\begin{equation*}
    \dot{\mathbf{q}}\equiv(\dot{q}_1,\dots,\dot{q}_n)
\end{equation*}
and generalized momentum
\begin{equation*}
    \mathbf{p}\equiv(p_1,\dots,p_n)
\end{equation*}
In this case, $\mathbf{p}$ and $\mathbf{p}$ are $n$-dimensional vectors in the space of generalized position and generalized momentum.

\subsection*{Definition}
The definition Hamiltonian comes from Lagrangian.
To see this, consider the change of Lagrangian $\mathcal{L}(\mathbf{q},\dot{\mathbf{q}},t)$ as the time increase
\begin{align*}
    \frac{d\mathcal{L}}{dt}&=\frac{\partial\mathcal{L}}{\partial q_1}\frac{\partial q_1}{\partial t}\dots+\frac{\partial\mathcal{L}}{\partial q_n}\frac{\partial q_n}{\partial t}+\frac{\partial\mathcal{L}}{\partial \dot{q}_1}\frac{\partial \dot{q}_2}{\partial t}+\dots+\frac{\partial \mathcal{L}}{\partial \dot{q_n}}\frac{\partial\dot{q}_n}{\partial t}+\frac{\partial\mathcal{L}}{\partial t}\frac{\partial t}{\partial t}\\
    &=\sum_{i=1}^{n}\left(\frac{\partial \mathcal{L}}{\partial q_i}\dot{q}_i+\frac{\partial\mathcal{L}}{\partial\dot{q}_i}\ddot{q}_i\right)+\frac{\partial \mathcal{L}}{\partial t}\\
    &=\sum_{i=1}^{n}\left(\dot{p}_i\dot{q}_i+p_i\ddot{q}_i\right)+\frac{\partial \mathcal{L}}{\partial t}\\
    \frac{d\mathcal{L}}{dt}&=\frac{d}{dt}\sum_{i=1}^{n}\left(p_i\dot{q}_i\right)+\frac{\partial\mathcal{L}}{\partial t}
\end{align*}
If we assume the Lagrangian does not depend explicitly on time, say in the case of $T=T(\mathbf{q},\dot{\mathbf{q}})$ and $U=U(\mathbf{q})$, the $\partial\mathcal{L}/\partial t$ term is zero. 
Thus, the quantity
\begin{equation*}
    \frac{d}{dt}\left(\sum_{i=1}^{n}p_i\dot{q}_i-\mathcal{L}\right)=0
\end{equation*}
which is defined as Hamiltonian, is conserved. In other words,
\begin{equation*}
    \frac{\partial\mathcal{L}}{\partial t}=0\implies\frac{dH}{dt}=0
\end{equation*}

\subsection*{Special Case of Hamiltonian}
The special case we are referring is $\mathcal{H}=T+U$, which occur when the coordinates are natural
\begin{equation*}
    \mathbf{r}_\alpha=\mathbf{r}_\alpha(\mathbf{q})
\end{equation*}
where the $\alpha$ subscript used to denote the coordinate as $\alpha$-th particle's coordinate. 
We first begin by expressing the generalized velocity in terms of generalized coordinates, which can be obtained by performing partial derivative with respect to the generalized coordinate
\begin{equation*}
    \frac{\partial\mathbf{r}_\alpha}{\partial t}=\frac{\partial\mathbf{r}_\alpha}{\partial q_1}\frac{\partial q_1}{\partial t}\dots+\frac{\partial \mathbf{r}_\alpha}{\partial q_n}\frac{\partial q_n}{\partial t}=\sum_{i=1}^{n}\frac{\partial \mathbf{r}_\alpha}{\partial q}\dot{q}_i
\end{equation*}
Then its square is just the dot product with itself
\begin{equation*}
    \left(\frac{\partial\mathbf{r}_\alpha}{\partial t}\right)^2=\sum_{j=1}^{n}\frac{\partial\mathbf{r}_\alpha}{\partial q_i}\dot{q}_i\cdot\sum_{k=1}^{n}\frac{\partial\mathbf{r}_\alpha}{\partial q_j}\dot{q}_j
\end{equation*}
The kinetic energy is then product of triple sum
\begin{equation*}
    T=\frac{1}{2}\sum_{\alpha}m_\alpha\mathbf{r}_\alpha=\frac{1}{2}\sum_{\alpha}m_\alpha\sum_j\frac{\partial \mathbf{r}_\alpha}{\partial q_j}\dot{q}_j\cdot\sum_k\frac{\partial\mathbf{r_\alpha}}{\partial q_k}\dot{q}_k
\end{equation*}
If we define 
\begin{equation*}
    A_{jk}=\sum_\alpha m_\alpha\frac{\partial\mathbf{r}_\alpha}{\partial q_j}\cdot\frac{\partial \mathbf{r}_\alpha}{\partial q_k}
\end{equation*}
The expression for kinetic energy simplifies into 
\begin{equation*}
    T=\frac{1}{2}\sum_{j,k}A_{jk}\dot{q}_j\dot{q}_k
\end{equation*}

If we assume kinetic energy $T=T(\mathbf{q},\dot{\mathbf{q}})$ and potential energy $U=U(\mathbf{q})$, which is what natural coordinates imply anyway, then 
\begin{equation*}
    \frac{\partial \mathcal{L}}{\partial \dot{q}_i}=\frac{\partial T}{\partial \dot{q}_i}=p_i
\end{equation*}
To evaluate partial derivative, we need the following relation
\begin{equation*}
    \frac{d}{dv_i}\sum_{j,k}A_{jk}v_jv_k=2\sum_{j,k}A_{ij}v_j\quad\text{if }A_{jk}=A_{kj}
\end{equation*}
Applying the identity 
\begin{equation*}
    p_i=\frac{\partial}{\partial \dot{q}_i}\left(\frac{1}{2}\sum_{jk}A_{jk}\dot{q}_j\dot{q}_k\right)=\sum_jA_{ij}\dot{q}_j\end{equation*}
Therefore, the Hamiltonian now reads
\begin{align*}
    \mathcal{H}&=\sum_ip_i\dot{q}_i-\mathcal{L}=\sum_i\sum_jA_{ij}\dot{q}_i\dot{q}_j-(T-U)\\
    \mathcal{H}&=2T-T+U=T+U
\end{align*}
\end{document}