\documentclass[../../../main.tex]{subfiles}
\begin{document}
The Hamiltonian is defined as
\begin{equation*}
	\mathcal{H}=\sum_{i=1}^{n}p_i\dot{q}_i-\mathcal{L}
\end{equation*}
The Hamilton equation derived from this is
\begin{equation*}
	\dot{q}_i=\frac{\partial\mathcal{H}}{\partial p_i},\qquad\dot{p}_i=-\frac{\partial\mathcal{H}}{\partial q_i}
\end{equation*}

The conservation of Hamiltonian is stated as follows
\begin{equation*}
	\frac{\partial\mathcal{L}}{\partial t}=0\implies\frac{dH}{dt}=0
\end{equation*}

In the case of natural coordinate, that is the relation between the generalized coordinates and Cartesian is time-independent, the Hamiltonian takes the simple form $\mathcal{H}=T+U$
\begin{equation*}
	q_i=q_i(\mathbf{q})\implies\mathcal{H}=T+U
\end{equation*}

Consider the Hamiltonian 
\begin{equation*}
	\mathcal{H }=\mathcal{H }\left( \mathbf{q}, \mathbf{p},t \right) 
\end{equation*}
If the Hamiltonian does not depend on the coordinate $q_i$, the coordinate is said to be cyclic.
Correspondingly, the canonical momentum $p_i= \partial \mathcal{L }/\partial q_i$ is conserved.
The condition for $q_i$ to be cyclic can be written as 
\begin{equation*}
	\frac{\partial \mathcal{H }}{\partial q_i }=0
\end{equation*}

The steps to solve problem using Hamilton formalism are as follows.
\begin{enumerate}
    \item Choose the generalized coordinate $\mathbf{q}$.
    \item Write the $T$ and $U$ in terms of $(\mathbf{q},\dot{\mathbf{q}})$.
    \item Find the generalized momenta $p_i=\partial\mathbf{L}/\partial \dot{q}_i$.
    \item Solve for $\dot{\mathbf{q}}$ in terms $(\mathbf{q},\mathbf{p})$.
    \item Write $\mathcal{H}$ in terms $(\mathbf{q},\mathbf{p})$.
    \item Write the Hamilton equation for motion.
\end{enumerate}

\subsection*{Notation}
To avoid clutter, we define the following notation.
The $n$-dimension system with $n$-generalized coordinate is represented by
\begin{equation*}
	\mathbf{q}\equiv(q_1,\dots,q_n)
\end{equation*}
while $n$-generalized velocity
\begin{equation*}
	\dot{\mathbf{q}}\equiv(\dot{q}_1,\dots,\dot{q}_n)
\end{equation*}
and generalized momentum
\begin{equation*}
	\mathbf{p}\equiv(p_1,\dots,p_n)
\end{equation*}
In this case, $\mathbf{p}$ and $\mathbf{p}$ are $n$-dimensional vectors in the space of generalized position and generalized momentum.

\subsection*{Definition}
The definition Hamiltonian comes from Lagrangian.
To see this, consider the change of Lagrangian $\mathcal{L}(\mathbf{q},\dot{\mathbf{q}},t)$ as the time increase
\begin{align*}
	\frac{d\mathcal{L}}{dt} & =\frac{\partial\mathcal{L}}{\partial q_1}\frac{\partial q_1}{\partial t}\dots+\frac{\partial\mathcal{L}}{\partial q_n}\frac{\partial q_n}{\partial t}+\frac{\partial\mathcal{L}}{\partial \dot{q}_1}\frac{\partial \dot{q}_2}{\partial t}+\dots+\frac{\partial \mathcal{L}}{\partial \dot{q_n}}\frac{\partial\dot{q}_n}{\partial t}+\frac{\partial\mathcal{L}}{\partial t}\frac{\partial t}{\partial t} \\
	                        & =\sum_{i=1}^{n}\left(\frac{\partial \mathcal{L}}{\partial q_i}\dot{q}_i+\frac{\partial\mathcal{L}}{\partial\dot{q}_i}\ddot{q}_i\right)+\frac{\partial \mathcal{L}}{\partial t}                                                                                                                                                                                                                          \\
	                        & =\sum_{i=1}^{n}\left(\dot{p}_i\dot{q}_i+p_i\ddot{q}_i\right)+\frac{\partial \mathcal{L}}{\partial t}                                                                                                                                                                                                                                                                                                    \\
	\frac{d\mathcal{L}}{dt} & =\frac{d}{dt}\sum_{i=1}^{n}\left(p_i\dot{q}_i\right)+\frac{\partial\mathcal{L}}{\partial t}
\end{align*}
If we assume the Lagrangian does not depend explicitly on time, say in the case of $T=T(\mathbf{q},\dot{\mathbf{q}})$ and $U=U(\mathbf{q})$, the $\partial\mathcal{L}/\partial t$ term is zero.
Thus, the quantity
\begin{equation*}
	\frac{d}{dt}\left(\sum_{i=1}^{n}p_i\dot{q}_i-\mathcal{L}\right)=0
\end{equation*}
which is defined as Hamiltonian, is conserved. In other words,
\begin{equation*}
	\frac{\partial\mathcal{L}}{\partial t}=0\implies\frac{dH}{dt}=0
\end{equation*}

\subsection*{Special Case of Hamiltonian}
The special case we are referring is $\mathcal{H}=T+U$, which occur when the coordinates are natural
\begin{equation*}
	\mathbf{r}_\alpha=\mathbf{r}_\alpha(\mathbf{q})
\end{equation*}
where the $\alpha$ subscript used to denote the coordinate as $\alpha$-th particle's coordinate.
We first begin by expressing the generalized velocity in terms of generalized coordinates, which can be obtained by performing partial derivative with respect to the generalized coordinate
\begin{equation*}
	\frac{\partial\mathbf{r}_\alpha}{\partial t}=\frac{\partial\mathbf{r}_\alpha}{\partial q_1}\frac{\partial q_1}{\partial t}\dots+\frac{\partial \mathbf{r}_\alpha}{\partial q_n}\frac{\partial q_n}{\partial t}=\sum_{i=1}^{n}\frac{\partial \mathbf{r}_\alpha}{\partial q}\dot{q}_i
\end{equation*}
Then its square is just the dot product with itself
\begin{equation*}
	\left(\frac{\partial\mathbf{r}_\alpha}{\partial t}\right)^2=\sum_{j=1}^{n}\frac{\partial\mathbf{r}_\alpha}{\partial q_i}\dot{q}_i\cdot\sum_{k=1}^{n}\frac{\partial\mathbf{r}_\alpha}{\partial q_j}\dot{q}_j
\end{equation*}
The kinetic energy is then product of triple sum
\begin{equation*}
	T=\frac{1}{2}\sum_{\alpha}m_\alpha\mathbf{r}_\alpha=\frac{1}{2}\sum_{\alpha}m_\alpha\sum_j\frac{\partial \mathbf{r}_\alpha}{\partial q_j}\dot{q}_j\cdot\sum_k\frac{\partial\mathbf{r_\alpha}}{\partial q_k}\dot{q}_k
\end{equation*}
If we define
\begin{equation*}
	A_{jk}=\sum_\alpha m_\alpha\frac{\partial\mathbf{r}_\alpha}{\partial q_j}\cdot\frac{\partial \mathbf{r}_\alpha}{\partial q_k}
\end{equation*}
The expression for kinetic energy simplifies into
\begin{equation*}
	T=\frac{1}{2}\sum_{j,k}A_{jk}\dot{q}_j\dot{q}_k
\end{equation*}

If we assume kinetic energy $T=T(\mathbf{q},\dot{\mathbf{q}})$ and potential energy $U=U(\mathbf{q})$, which is what natural coordinates imply anyway, then
\begin{equation*}
	\frac{\partial \mathcal{L}}{\partial \dot{q}_i}=\frac{\partial T}{\partial \dot{q}_i}=p_i
\end{equation*}
To evaluate partial derivative, we need the following relation
\begin{equation*}
	\frac{d}{dv_i}\sum_{j,k}A_{jk}v_jv_k=2\sum_{j,k}A_{ij}v_j\quad\text{if }A_{jk}=A_{kj}
\end{equation*}
Applying the identity
\begin{equation*}
	p_i=\frac{\partial}{\partial \dot{q}_i}\left(\frac{1}{2}\sum_{jk}A_{jk}\dot{q}_j\dot{q}_k\right)=\sum_jA_{ij}\dot{q}_j\end{equation*}
Therefore, the Hamiltonian now reads
\begin{align*}
	\mathcal{H} & =\sum_ip_i\dot{q}_i-\mathcal{L}=\sum_i\sum_jA_{ij}\dot{q}_i\dot{q}_j-(T-U) \\
	\mathcal{H} & =2T-T+U=T+U
\end{align*}

\subsection*{Hamilton Equation Derivation}
The assumption that must be satisfied first are the constraint must be holonomic--the number of degree of freedom matches the number of generalized coordinates--and that the non constraint force must be derivable from potential energy $\mathbf{F}=-\nabla U$.
In principle, to obtain Hamilton equation of motion, we differentiate the Hamiltonian $\mathcal{H}(\mathbf{q},\mathbf{p})$ with both variable $q_i$ and $p_i$.
To evaluate the derivative, we need to express the Hamiltonian
\begin{equation*}
	\mathcal{H}=\sum_{i=1}^{n}p_i\dot{q}_1-\mathcal{L}
\end{equation*}
in terms of $\mathbf{q}$ and $\mathbf{p}$

First note that the generalized momentum is a function of generalized velocity and perhaps time, which can be seen from
\begin{equation*}
	p_i=\frac{\partial}{\partial \dot{q}_i}\mathcal{L}(\mathbf{q},\dot{q}_i,t)\implies p_i=p_i\left(\dot{q},t\right)
\end{equation*}
Therefore, in principle, we can solve $p_i$ to express $\dot{q}_i$ in terms of variables $\mathbf{p}$ and $\mathbf{q}$, also time perhaps
\begin{equation*}
	\dot{q}_i=\dot{q}_i(\mathbf{q},\mathbf{p}_i,t)
\end{equation*}

Then the next step is to actually perform the differentiation.
First we differentiate with respect to $q_i$
\begin{equation*}
	\frac{\partial \mathcal{H}}{\partial q_i}=\frac{\partial}{\partial q_i}\sum_{j=1}^{n}p_j\dot{q}_j-\frac{\partial \mathcal{L}}{\partial q_i}
\end{equation*}
The sum differentiation can be evaluated as follows
\begin{equation*}
	\frac{\partial}{\partial q_i}\sum_{j}p_j\dot{q}_j=\sum_j\left(\dot{q}_j\frac{\partial p_j}{\partial q_i}+p_j\frac{\partial\dot{q}_j}{\partial q_i}\right)=\sum_jp_j\frac{\partial \dot{q}_j}{\partial q_i}
\end{equation*}
While the Lagrangian term
\begin{align*}
	\frac{\partial \mathcal{L}}{\partial q_i} & =\frac{\partial \mathcal{L}}{\partial q_1}\frac{\partial q_1}{\partial q_i}+\dots\frac{\partial\mathcal{L}}{\partial q_n}\frac{\partial q_n}{\partial q_i}+\frac{\partial\mathcal{L}}{\partial \dot{q}_1}\frac{\partial\dot{q}_1}{\partial \dot{q}_i}+\dots+\frac{\partial \mathcal{L}}{\partial \dot{q}_n}\frac{\partial \dot{q}_n}{\partial q_i}+\frac{\partial\mathcal{L}}{\partial t}\frac{\partial t}{\partial q_i} \\
	                                          & =\sum_j\left(\frac{\partial\mathcal{L}}{\partial q_j}\frac{\partial q_j}{\partial q_i}+\frac{\partial\mathcal{L}}{\partial \dot{q}_j}\frac{\partial \dot{q}_j}{\partial q_i}\right)                                                                                                                                                                                                                                      \\
	\frac{\partial \mathcal{L}}{\partial q_i} & =\frac{\partial\mathcal{L}}{\partial q_i}+\sum_jp_i\frac{\partial \dot{q}_j}{\partial q_i}
\end{align*}
Substituting into the Hamiltonian
\begin{equation*}
	\frac{\partial \mathcal{H}}{\partial q_i}=\sum_jp_i\frac{\partial \dot{q}_j}{\partial q_i}-\frac{\partial \mathcal{L}}{\partial q_i}-\frac{\partial \dot{q}_j}{\partial q_i}=-\frac{\partial \mathcal{L}}{\partial q_i}=-\frac{d}{dt}\frac{\partial\mathcal{L}}{\partial \dot{q_i}}=-\dot{p}_i
\end{equation*}

Next is the derivative with respect to $p_i$
\begin{equation*}
	\frac{\partial \mathcal{H}}{\partial p_i}=\frac{\partial}{\partial p_i}\sum_{j=1}^{n}p_i\dot{q}_1-\frac{\partial \mathcal{L}}{\partial p_i}
\end{equation*}
The sum differentiation can be evaluated as follows
\begin{equation*}
	\frac{\partial}{\partial p_i}\sum_{j}p_j\dot{q}_j=\sum_j\left(\dot{q}_j\frac{\partial p_j}{\partial p_i}+p_j\frac{\partial\dot{q}_j}{\partial p_i}\right)=\dot{q}_i+\sum_jp_j\frac{\partial \dot{q}_j}{\partial p_j}
\end{equation*}
While the Lagrangian term
\begin{align*}
	\frac{\partial \mathcal{L}}{\partial p_i} & =\frac{\partial \mathcal{L}}{\partial q_1}\frac{\partial q_1}{\partial p_i}+\dots\frac{\partial\mathcal{L}}{\partial q_n}\frac{\partial q_n}{\partial p_i}+\frac{\partial\mathcal{L}}{\partial \dot{q}_1}\frac{\partial\dot{q}_1}{\partial p_i}+\dots+\frac{\partial \mathcal{L}}{\partial \dot{q}_n}\frac{\partial \dot{q}_n}{\partial p_i}+\frac{\partial\mathcal{L}}{\partial t}\frac{\partial t}{\partial p_i} \\
	                                          & =\sum_j\left(\frac{\partial\mathcal{L}}{\partial q_j}\frac{\partial q_j}{\partial p_i}+\frac{\partial\mathcal{L}}{\partial \dot{q}_j}\frac{\partial \dot{q}_j}{\partial p_i}\right)                                                                                                                                                                                                                                      \\
	\frac{\partial \mathcal{L}}{\partial p_i} & =\sum_jp_j\frac{\partial \dot{q}_j}{\partial p_i}
\end{align*}
Substituting into the Hamiltonian
\begin{equation*}
	\frac{\partial \mathcal{H}}{\partial q_i}=\dot{q}_i+\sum_jp_j\frac{\partial \dot{q}_j}{\partial p_j}-\sum_jp_j\frac{\partial \dot{q}_j}{\partial p_i}=\dot{q}_i
\end{equation*}

\subsection*{Hamiltonian Dependence on Time}
Consider the total derivative of the Hamiltonian with respect to time 
\begin{equation*}
    \frac{d\mathcal{H}}{\partial t}=\sum_{i}\left(\frac{\partial\mathcal{H}}{\partial q_i}\dot{q}+\frac{\partial\mathcal{H}}{\partial p_i}\dot{p}_i\right)+\frac{\partial\mathcal{H}}{\partial t}
\end{equation*}
Then using the Hamilton equation of motion
\begin{equation*}
    \frac{d\mathcal{H}}{dt}=\frac{\partial\mathcal{H}}{\partial t}
\end{equation*}
The total derivative $d\mathcal{H}/d t$ denote the actual changes as $t$ moves forward, which will changes both variables $(\mathbf{q},\mathbf{p})$ also.
In other hand, the partial derivative $\partial\mathcal{H}/\partial t$ denote the change of the Hamilton as $t$ moves forward with $(\mathbf{q},\mathbf{p})$ being held constant. 
This will be zero if $\mathcal{H}$ does not explicitly depend on $t$.
Thus, if $\mathcal{H}$ does not explicitly depend on $t$, then $\mathcal{H}$ is conserved.

\subsection*{Newton's Law in 2D}
Since we are using 2D Cartesian, the Hamiltonian is simply the system energy
\begin{equation*}
	\mathcal{H}=\frac{1}{2}m(\dot{x}^2+\dot{y}^2)+U(x,y)=\frac{1}{2m}\left(p_x^2+p_y^2\right)+U(x,y)
\end{equation*}
The $x$-th component reads 
\begin{equation*}
	\dot{x}=\frac{\partial \mathcal{H}}{\partial p_x}=\frac{p_x}{m}\qquad\dot{p}_x=-\frac{\partial \mathcal{H}}{\partial x}=-\frac{\partial U}{\partial x}
\end{equation*}
while the $y$-th component 
\begin{equation*}
	\dot{y}=\frac{\partial \mathcal{H}}{\partial p_y}=\frac{p_y}{m}\qquad\dot{p}_y=-\frac{\partial \mathcal{H}}{\partial y}=-\frac{\partial U}{\partial y}
\end{equation*}
Combining both, we obtain the Newton's definition of momentum and force 
\begin{equation*}
	\mathbf{p}=m\mathbf{r}\qquad\mathbf{F}=\frac{d\mathbf{p}}{dt}
\end{equation*}

\subsubsection*{Newton's Law in Polar Coordinate}
As previously, the Hamiltonian is the system energy
\begin{equation*}
	\mathcal{H}(r,\phi,\dot{r},\dot{\phi})=\frac{1}{2}m\left(\dot{r}^2+r^2\dot{\phi}^2\right)-U(r)
\end{equation*}
Next, we evaluate the momenta
\begin{equation*}
	p_r=\frac{\partial \mathcal{L}}{\partial\dot{r}}=m\dot{r},\qquad p_\phi=\frac{\partial\mathcal{L}}{\partial \dot{\phi}}=mr^2\dot{\phi}
\end{equation*}
and solve for generalized velocity
\begin{equation*}
	\dot{r}=\frac{p_r}{m}\qquad\dot{\phi}=\frac{p_\phi}{mr^2}
\end{equation*}
Then rewrite the Hamiltonian in terms of coordinate and momenta 
\begin{equation*}
	\mathcal{H}=\frac{1}{2}m\left(\frac{p_r^2}{m^2}+\frac{p_\phi^2}{m^2r^2}\right)+U(r)=\frac{1}{2}\left(p_r^2+\frac{p_\phi^2}{r^2}\right)+U(r)
\end{equation*}
The radial components are 
\begin{equation*}
	\dot{r}=\frac{\partial\mathcal{H}}{\partial p_r}=\frac{p_r}{m}\qquad p_r=-\frac{\partial\mathcal{H}}{\partial r}=\frac{p_\phi^2}{mr^3}-\frac{\partial U}{\partial r}
\end{equation*}
Combining both, we obtain
\begin{equation*}
	m\ddot{\mathbf{r}}=\frac{p_\phi^2}{mr^3}-\frac{\partial U}{\partial r}
\end{equation*}
which is the definition of total force $m\mathbf{\ddot{r}}$ as the sum of radial force $-\partial U/\partial r$ and centrifugal force $p_\phi^2/mr^3$.
Then the angular component
\begin{equation*}
	\dot{\phi}=\frac{\partial\mathcal{H}}{\partial p_\phi}=\frac{p_\phi}{mr^2}\qquad \dot{p}_\phi=-\frac{\partial\mathcal{H}}{\partial \phi}=0
\end{equation*}
which reproduce the definition of angular momentum and conservation of angular momentum.

\subsection*{Hamiltonian of a Charge in an Electromagnetic Field}
The Hamiltonian is 
\begin{equation*}
	\mathcal{H }=\frac{1 }{2m }\left( \mathbf{p }-q \mathbf{A } \right) ^2+qV
\end{equation*}
From this, we obtain the two equation of motion 
\begin{equation*}
	\dot{\mathbf{r}}=\frac{1 }{m }\left( \mathbf{p }-q \mathbf{A } \right) \qquad\text{and }\qquad \dot{\mathbf{p }}=q \left( \dot{\mathbf{r}}\cdot \nabla  \right)\mathbf{A }-q \nabla  V 
\end{equation*}
The momentum evolution is consistent with the Lorentz force law 
\begin{equation*}
	\mathbf{F }=q \left( \mathbf{E }+\mathbf{v }\times \mathbf{B } \right) 
\end{equation*}

\subsubsection*{Derivation.}
By definition, Hamiltonian in three dimension is 
\begin{equation*}
	\mathcal{H }=\sum_{i=1 }^{3 }q_ip_i-\mathcal{L }=\mathbf{p} \cdot \dot{\mathbf{r}}- \mathcal{L}
\end{equation*}
With the known Lagrangian of a particle in electromagnetic field
\begin{equation*}
	\mathcal{L}=\frac{1}{2}m\dot{\mathbf{r}}^2-q(V-\dot{\mathbf{r}}\cdot \mathbf{A})
\end{equation*}
and known canonical momentum
\begin{equation*}
	\mathbf{p }=m \dot{\mathbf{r }}+q \mathbf{A}
\end{equation*}
the Hamiltonian reads 
\begin{align*}
	\mathcal{H }&=m \dot{\mathbf{r }}\cdot \dot{\mathbf{r }}+q \mathbf{A }\cdot \dot{\mathbf{r }}-\frac{1 }{2 }m \dot{\mathbf{r }}\cdot \dot{\mathbf{r }} +q \left( V-\dot{\mathbf{r }}\cdot \mathbf{A} \right) \\
	\mathcal{H }&= \frac{1 }{2 }m \dot{\mathbf{r }}\cdot \dot{\mathbf{r }}+qV 
\end{align*}
Expressing the change of coordinate as momentum
\begin{equation*}
	\mathcal{H }=\frac{1 }{2m }\left( \mathbf{p }-q \mathbf{A} \right) ^2+qV
\end{equation*}

For the first equation of motion, consider
\begin{equation*}
	\dot{\mathbf{r }}=\frac{\partial \mathcal{H }}{\partial \mathbf{p }}=\frac{1 }{2m }2 \left( \mathbf{p } -q \mathbf{A }	\right)=\frac{1 }{m }\left( \mathbf{p }-q \mathbf{A } \right)  
\end{equation*}
On multiplying both side with $m$, we obtain the kinetic momentum $\pi =m \dot{\mathbf{r }}$, as opposed to canonical momentum $\pi=m \dot{\mathbf{r }}+q \mathbf{A}$.

Next we move to its conjugate.
Consider 
\begin{align*}
	\dot{\mathbf{p }}&=-\frac{\partial \mathcal{ H }}{\partial \mathbf{r }}=\frac{1 }{2m }2 \left( \mathbf{p }-q \mathbf{A } \right) q \cdot  \frac{\partial \mathbf{A 		}}{\partial \mathbf{r }	}-1 \frac{\partial V }{\partial \mathbf{r }}\\
	\dot{\mathbf{p }}&=\frac{q }{m}\left( \mathbf{p } -q \mathbf{A }	\right)\cdot \nabla \mathbf{A } -q \nabla V
\end{align*}
Substituting the value of kinetic momentum obtained previously
\begin{equation*}
	\dot{\mathbf{p }}=\frac{q }{m }m \dot{\mathbf{r }}\cdot \nabla \mathbf{A }-q \nabla V=q \dot{\mathbf{r }}\cdot \mathbf{A }	-q \nabla \mathbf{A }
\end{equation*}
Writing the first equation of motion as 
\begin{align*}
	m \dot{\mathbf{r }}&=\mathbf{p}-q \mathbf{A }\\
	m \ddot{\mathbf{r }} &= \dot{\mathbf{p }}-q \frac{d\mathbf{A }}{d t}
\end{align*}
and substituting it back 
\begin{align*}
	m \ddot{\mathbf{r }}+q \frac{\partial \mathbf{A }}{\partial t }&=  q \dot{\mathbf{r }}\cdot \nabla \mathbf{A }-q \nabla \mathbf{A }\\
	m \ddot{\mathbf{r }}&=q \dot{\mathbf{r }}\cdot \nabla \mathbf{A } -q \nabla V -q \frac{d \mathbf{A }}{dt}
\end{align*}
The total derivative of $\mathbf{A}(\mathbf{r },t)$ with respect to time has two part: spatial variation and explicit dependence on time
\begin{equation*}
	\frac{d\mbox{A }}{dt}=\sum_i \frac{\partial \mathbf{A }}{\partial q_i}\frac{\partial q_i }{\partial t }+\frac{\partial \mathbf{A }}{\partial t }=\sum_i \dot{q}_i \frac{\partial }{\partial q_i }\mathbf{A}+ \frac{\partial \mathbf{A }}{\partial t }= \left( \dot{\mathbf{r }}\cdot \nabla\right)\mathbf{A}+ \frac{\partial \mathbf{A }}{\partial t}
\end{equation*}
Hence we write
\begin{equation*}
	\mathbf{F }=-q \left( \nabla  V+\frac{\partial \mathbf{A }}{\partial t} \right) +q \left[ \nabla \left( \dot{\mathbf{r }\cdot \mathbf{A }} \right)-\left( \dot{\mathbf{r}}\cdot \nabla   \right)\mathbf{A}   \right] 
\end{equation*}
Recalling the vector triple product
\begin{equation*}
	\dot{\mathbf{r }}\times \left( \nabla \times \mathbf{A }  \right)=\nabla \left( \dot{\mathbf{r }}\cdot \mathbf{A} \right)  -\left( \dot{\mathbf{r }}\cdot \nabla   \right) \mathbf{A}
\end{equation*}
to write it as 
\begin{equation*}
	\mathbf{F }=-q \left( \nabla V+\frac{\partial \mathbf{A }}{\partial t } \right) +q \dot{\mathbf{r }}\times \left( \nabla \times \mathbf{A} \right) 
\end{equation*}
In terms of potential, both electromagnetic field maybe expressed as
\begin{equation*}
	\mathbf{E}=-\nabla V-\frac{\partial \mathbf{A }}{\partial t}\qquad \text{and}\qquad \mathbf{B} =\nabla \times \mathbf{A}
\end{equation*} 
Hence, we obtain our desired result
\begin{equation*}
	\mathbf{F}=q \left( \mathbf{E}+\dot{\mathbf{r}}\times \mathbf{B} \right) 
\end{equation*}
\end{document}