\documentclass[../../../main.tex]{subfiles}
\begin{document}
\subsection*{Application: Mass on a Cone}
Consider a mass $m$ which is constrained to move on the frictionless surface of a vertical cone $\rho=cz$.

Here we use cylindrical coordinate. 
The kinetic energy then is 
\begin{equation*}
   T=\frac{1}{2}m\left(\dot{\rho}^2+\rho^2\dot{\phi}^2+\dot{z}^2\right)=\frac{1}{2}m\left[(cz\dot{\phi})^2+(c^2+1)\dot{z}^2\right]
\end{equation*}
while the potential energy is simply due to gravity
\begin{equation*}
    U=mgz
\end{equation*}
Thus the Hamiltonian read
\begin{equation*}
   \mathcal{H}=\frac{1}{2}m\left[(cz\dot{\phi})^2+(c^2+1)\dot{z}^2\right]+mgz
\end{equation*}
Next we determine the generalized momenta 
\begin{equation*}
   p_z=\frac{\partial\mathcal{L}}{\partial \dot{z}}=(c^2+1)m\dot{z}\qquad p_\phi=\frac{\partial\mathcal{L}}{\partial \dot{\phi}}=(c^2z^2)m\dot{\phi}
\end{equation*}
Then we express the Hamiltonian as following
\begin{align*}
   \mathcal{H}&=\frac{1}{2}\left[\frac{(c^2+1)}{(c^2+1)^2m^2}p_z^2++\frac{c^2z^2}{m^2(c^2z^2)^2}p_\phi^2\right]+mgz\\
   \mathcal{H}&=\frac{1}{2m}\left[\frac{p_z^2}{c^2+1}+\frac{p_\phi^2}{c^2z^2}\right]+mgz
\end{align*}
Finally, we have the equations for $z$ component
\begin{equation*}
   \dot{z}=\frac{\partial\mathcal{H}}{\partial p_z}=\frac{p_z}{c^2+1}\qquad \dot{p}_z=-\frac{\partial\mathcal{H}}{\partial z}=\frac{p_\phi^2}{mc^2z^3}-mg
\end{equation*}
and $\phi$ component
\begin{equation*}
   \dot{\phi}=\frac{\partial\mathcal{H}}{\partial p_\phi}=\frac{p_\phi}{mc^2z^2}\qquad \dot{p_\phi}=-\frac{\partial\mathcal{H}}{\partial \phi}=0
\end{equation*}


\begin{figure*}
	\centering
	\dfig{../../../Rss/CM/Hamilton/Cone.png}
	\caption*{Figure: Particle constrained to move within a cone}
\end{figure*}


\end{document}