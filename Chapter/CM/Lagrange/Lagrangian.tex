\documentclass[../../../main.tex]{subfiles}
\begin{document}
The Lagrangian is defined as
\begin{equation*}
	\mathcal{L}=T-U
\end{equation*}
which mean that the Lagrangian is a function of position and velocity.
The path of particle is determined by Hamilton's principle
\begin{equation*}
	S=\int_{t_1}^{t_2}\mathcal{L}\;dt
\end{equation*}
that is, the particle's path is such that the action integral $S$ is stationary.

We can express the coordinate in other generalized coordinate
\begin{equation*}
	\mathbf{r}=\mathbf{r}(q_1,q_2,q_2)
\end{equation*}
And the same for velocity
\begin{equation*}
	\mathbf{v}=\mathbf{v}(\dot{q}_1,\dot{q}_2,\dot{q}_3)
\end{equation*}
Now the action integral reads
\begin{equation*}
	S=\int_{t_1}^{t_2}\mathcal{L}(q_1,q_2,q_3,\dot{q}_1,\dot{q}_2,\dot{q}_3,t)\;dt
\end{equation*}
Therefore we have three Euler-Lagrange equation that must be satisfied by the particle
\begin{align*}
	\frac{\partial \mathcal{L}}{\partial q_1}=\frac{d}{dt}\frac{\partial \mathcal{L}}{\partial \dot{q}_1}, \quad
	\frac{\partial \mathcal{L}}{\partial q_2}=\frac{d}{dt}\frac{\partial \mathcal{L}}{\partial \dot{q}_2}, \quad
	\frac{\partial \mathcal{L}}{\partial q_3}=\frac{d}{dt}\frac{\partial \mathcal{L}}{\partial \dot{q}_3}
\end{align*}
This is the case of one unconstrained particle. For $N$ unconstrained particle, then, we shall have $3N$ Lagrange equation
\begin{equation*}
	\frac{\partial\mathcal{L}}{\partial q_i}=\frac{d}{dt}\frac{\partial \mathcal{L}}{\partial \dot{q}_i}\qquad i=1,\dots,3N
\end{equation*}

The Lagrangian equation
\begin{equation*}
	\frac{\partial \mathcal{L}}{\partial q_i}=\frac{d}{dt}\frac{\partial \mathcal{L}}{\partial \dot{q}_i}
\end{equation*}
takes the form
\begin{equation*}
	\text{Generalized force}=\text{Rate of change of Generalized momentum}
\end{equation*}
where
\begin{equation*}
	F_i=\frac{\partial \mathcal{L}}{\partial q_i}=i\text{-th component of the Generalized force}
\end{equation*}
and
\begin{equation*}
	p_i=\frac{\partial \mathcal{L}}{\partial \dot{q}_i}=i\text{-th component of the Generalized momentum}
\end{equation*}

Set of generalized coordinates that minimize the number of Euler-Lagrangian equation and be able to uniquely describe said system is said to be natural.
This implies that natural coordinate also have minimum degree of freedom, that is the number of coordinate that can be varied independently.
In natural coordinate, the generalized coordinate has no time dependence with its Cartesian relative.

Consider the Lagrangian 
\begin{equation*}
	\mathcal{L }=\mathcal{L }\left( \mathbf{q}, \dot{\mathbf{q }},t \right) 
\end{equation*}
If the Lagrangian does not depend on the coordinate $q_i$, the coordinate is said to be cyclic.
Correspondingly, the canonical momentum $p_i= \partial \mathcal{L }/\partial q_i$ is conserved.
The condition for $q_i$ to be cyclic can be written as 
\begin{equation*}
	\frac{\partial \mathcal{L }}{\partial q_i }=0
\end{equation*}

System with $n$ degree of freedom that can be described by $n$ generalized coordinate is called holonomic.
In general
\begin{equation*}
	\text{DoF}=\text{No. of Coordinate}-\text{No. Constraint}
\end{equation*}
For example, double pendulum have 4 coordinates and two constraint, thus having two degree of freedom.

The steps to solve problem using Lagrangian formalism are as follows.
\begin{enumerate}
	\item Write down the Lagrangian $\mathcal{L}=T-U$.
	\item Choose generalized $n$ coordinate $q_n$ and $\dot{q}_n$.
	\item Rewrite $\mathcal{L}$ in terms of $q_n$ and $\dot{q}_n$.
	\item Write $n$ Lagrange equation.
\end{enumerate}

\subsection{Proof of Lagrange Equation with Constraint}
Suppose a particle has two degree of freedom with two kinds of forces act on it:
constraint force $\mathbf{F}_\text{cstr}$, say interatomic forces that bind rigid body atom together;
and the is non constraint conservative forces $\mathbf{F}$, which at minimum must be able to be derived from potential energy $F=-\nabla U(\mathbf{r},t)$, say gravitational force.
The total energy is then
\begin{equation*}
	\mathbf{F}_\text{tot}=\mathbf{F}_\text{cstr}+\mathbf{F}
\end{equation*}
and the Lagrangian
\begin{equation*}
	\mathcal{L}=T-U
\end{equation*}
where $U$ is the potential energy which can be derived into non constraint conservative force.

The path of the particle can be denoted as
\begin{equation*}
	\mathbf{R}(t)=\mathbf{r}(t)+\boldsymbol{\epsilon}(t)
\end{equation*}
with $\mathbf{r}(t)$ as the correct path and $\boldsymbol{\epsilon}(t)$ as infinitesimal vector pointing away from the correct path.
Then we have two Lagrangians
\begin{equation*}
	\mathcal{L}=\mathcal{L}(\mathbf{R},\dot{\mathbf{R}},t),\qquad
	\mathcal{L}_0=\mathcal{L}_0(\mathbf{r},\dot{\mathbf{r},t})
\end{equation*}
and two action integral
\begin{equation*}
	S=\int_{t_1}^{t_2}\mathcal{L}\;dt,\qquad S_0=\int_{t_1}^{t_2}\mathcal{L}_0\;dt
\end{equation*}
It can be proven that the difference in action integral $\delta S=S-S_0$ is zero, to the first order.

We write the difference in Lagrangian as
\begin{align*}
	\delta \mathcal{L} & =\frac{1}{2}m\dot{\mathbf{R}}^2-U(\mathbf{R},t)-\frac{1}{2}m\dot{\mathbf{r}}^2+U(\mathbf{r},t)                                                                     \\
	                   & =\frac{1}{2}m\left[(\dot{\mathbf{r}}^2+\dot{\boldsymbol{\epsilon}}^2)-\dot{\mathbf{r}}^2\right]-\left[U(\mathbf{r}+\boldsymbol{\epsilon},t)-U(\mathbf{r},t)\right] \\
	                   & =\frac{1}{2}m\left[\dot{\mathbf{r}}^2+\dot{\boldsymbol{\epsilon}}^2+2\dot{\mathbf{r}}\cdot\dot{\boldsymbol{\epsilon}}^2-\dot{\mathbf{r}}^2\right]-dU               \\
	\delta\mathcal{L}  & =\frac{1}{2}m\dot{\boldsymbol{\epsilon}}^2+m\dot{\mathbf{r}}\cdot\dot{\boldsymbol{\epsilon}}-{\boldsymbol{\epsilon}}\cdot \nabla U
\end{align*}
The difference in action integral, in the first order, is then
\begin{equation*}
	S=\int_{t_1}^{t_2}\mathcal{L}_0\;dt=\int_{t_1}^{t_2}\left(m\dot{\mathbf{r}}\cdot\dot{\boldsymbol{\epsilon}}-\boldsymbol{\epsilon}\cdot\nabla U\right)\;dt
\end{equation*}
Using integration by parts
\begin{equation*}
	\int_{a}^{b} f\bigg(\frac{dg}{dx}\bigg)\;dx=-\int_{a}^{b} g\bigg(\frac{df}{dx}\bigg)\;dx+ fg \bigg\lvert_{a}^{b}
\end{equation*}
on the first term
\begin{equation*}
	\delta S=-\int_{t_1}^{t_2}\boldsymbol{\epsilon}\cdot\left[m\ddot{\mathbf{r}}+\nabla U\right]\;dt+m\dot{\mathbf{r}}\boldsymbol{\epsilon}\bigg|_{t_1}^{t^2}
\end{equation*}
The difference of $\boldsymbol{\epsilon}$ is zero between two end point, so
\begin{equation*}
	\delta S=-\int_{t_1}^{t_2}\boldsymbol{\epsilon}\cdot\left[m\ddot{\mathbf{r}}+\nabla U\right]\;dt
\end{equation*}
The path $\mathbf{r}(t)$ satisfies Newton second law, thus $m\ddot{\mathbf{r}}=\mathbf{F}_\text{tot}$.
Meanwhile, the gradient of potential energy is the negative of non constraint force
\begin{equation*}
	\delta S=-\int_{t_1}^{t_2}\boldsymbol{\epsilon}\cdot\left[\mathbf{F}_\text{tot}-\mathbf{F}\right]\;dt=i\int_{t_1 }^{t_2}\boldsymbol{\epsilon}\cdot \mathbf{F}_\text{cstr}\;dt
\end{equation*}
Note that the constraint force is normal to the particle path and the $\boldsymbol{\epsilon}$, which lies to the same surface of particle path.
Therefore, their dot product is zero
\begin{equation*}
	\delta S=0
\end{equation*}
and the action integral is stationary.

This justifies the Lagrange equation for system with two degree of freedom where its constraint lie in the same surface as the particle path.
In other words, it only applies to particle, or particles in that case, constrained to move in two dimension.
Accordingly, the action integral in this case is written as
\begin{equation*}
	S=\int_{t_1}^{t_2}\mathcal{L}(q_1,q_2,\dot{q}_1,\dot{q}_2,t)\;dt
\end{equation*}
which will result in two Euler-Lagrange equation
\begin{equation*}
	\frac{\partial \mathcal{L}}{\partial q_1}=\frac{d}{dt}\frac{\partial \mathcal{L}}{\partial \dot{q}_1},\qquad\frac{\partial \mathcal{L}}{\partial q_2}=\frac{d}{dt}\frac{\partial \mathcal{L}}{\partial \dot{q}_2}
\end{equation*}
\subsection{Newton Law in 2D Cartesian}
The Lagrangian in this case is
\begin{equation*}
	\mathcal{L}(x,y,\dot{x},\dot{y})=\frac{1}{2}m(\dot{x}^2+\dot{y}^2)-U(x,y)
\end{equation*}
Here we have two Euler-Lagrange equation
\begin{equation*}
	\frac{\partial \mathcal{L}}{\partial x}=\frac{d}{dt}\frac{\partial \mathcal{L}}{\partial \dot{x}},\qquad
	\frac{\partial \mathcal{L}}{\partial y}=\frac{d}{dt}\frac{\partial \mathcal{L}}{\partial \dot{y}}
\end{equation*}
The derivative with respect to position is force
\begin{align*}
	\frac{\partial \mathcal{L}}{\partial x}=-\frac{\partial U}{\partial x}=F_x,\qquad
	\frac{\partial \mathcal{L}}{\partial y}=-\frac{\partial U}{\partial y}=F_y
\end{align*}
while the derivative with respect to velocity is momentum
\begin{align*}
	\frac{\partial\mathcal{L}}{\partial \dot{x}}=\frac{\partial T}{\partial \dot{x}}=m\dot{x}, \qquad
	\frac{\partial\mathcal{L}}{\partial \dot{y}}=\frac{\partial T}{\partial \dot{y}}=m\dot{y}
\end{align*}
Substituting this result in the two Euler-Lagrange equation, we have
\begin{equation*}
	F_x=m\ddot{x},\qquad F_y=m\ddot{y}
\end{equation*}
which are the two component of Newton second law $\mathbf{F}=m\mathbf{\ddot{r}}$.

\subsection{Newton Law in Polar Coordinate}
The Lagrangian is
\begin{equation*}
	\mathcal{L}(r,\phi,\dot{\mathbf{r}},\dot{\phi})=\frac{1}{2}m(\dot{\mathbf{r}}^2+r^2\dot{\phi}^2)-U(r,\phi)
\end{equation*}
which result into two Euler-Lagrange equation
\begin{equation*}
	\frac{\partial\mathcal{L}}{\partial r}=\frac{d}{dt}\frac{\partial\mathcal{L}}{\partial \dot{\mathbf{r}}}\qquad\frac{\partial\mathcal{L}}{\partial \phi}=\frac{d}{dt}\frac{\partial\mathcal{L}}{\partial \dot{\phi}}
\end{equation*}
Before moving into evaluating the derivative with respect to $r$ and $\phi$, recall the gradient of potential energy in polar coordinate
\begin{equation*}
	\nabla U=\frac{\partial U}{\partial r}\mathbf{\hat{{r}}}+ \frac{1}{r}\frac{\partial U}{\partial \phi}\boldsymbol{\hat{\phi}}-F_r\mathbf{\hat{r}}-F_{\phi}\boldsymbol{\hat{\phi}}
\end{equation*}

Evaluating the radial derivative
\begin{align*}
	mr\dot{\phi}-\frac{\partial U}{\partial r} & =m\ddot{r}               \\
	F_r                                        & =m(\ddot{r}+r\dot{\phi})
\end{align*}
which is simply the radial component of the Newton's second law in polar coordinate.
Evaluating the angular derivative
\begin{align*}
	-\frac{\partial U}{\partial \phi}            & =\frac{d}{dt}mr^2\dot{\phi}=m(2r\dot{\mathbf{r}}\dot{\phi}+r^2\ddot{\phi}) \\
	-\frac{1}{r}\frac{\partial U}{\partial \phi} & =m(2\dot{\mathbf{r}}\dot{\phi}+r^2\ddot{\phi})                             \\
	F_\phi                                       & =m(2\dot{\mathbf{r}}\dot{\phi}+r\ddot{\phi})
\end{align*}
which is, as previously mentioned, the angular component of Newton's second law. It should be noted that the quantity $mr^2\dot{\phi}$ can be recognized as angular momentum $L$, and the rate of change of it is torque $\Gamma$
\begin{equation*}
	\Gamma=F_\phi r=\frac{dL}{dt}=\frac{d}{dt}mr^2\dot{\phi}
\end{equation*}

\subsection{Lagrangian with Explicit Constraint Forces Using Lagrange Multipliers}
The modified Euler-Lagrangian that include constraining force can be obtained by Lagrange Multipliers.
For two dimension Lagrangian $\mathcal{L}(x,\dot{x},y,\dot{y},t)$ with one constraint $f(x,y)=C$, we must solve two modified Lagrange equation plus one for the constraint
\begin{align*}
	\frac{\partial \mathcal{L}}{\partial x}+\lambda\frac{\partial f}{\partial x} & =\frac{d}{dt}\frac{\partial \mathcal{L}}{\partial \dot{x}} \\
	\frac{\partial \mathcal{L}}{\partial y}+\lambda\frac{\partial f}{\partial y} & =\frac{d}{dt}\frac{\partial \mathcal{L}}{\partial \dot{y}} \\
	f(x,y)=\text{Constant}
\end{align*}

Lagrange multiplier is not simply mathematical technique, in fact given partial derivatives of the constraint function $f (x, y)$, the Lagrange multiplier $f(x,y)$ gives the corresponding components of the constraint force
\begin{equation*}
	\lambda\frac{\partial f}{\partial q_i}=F_i^{\text{cstr}}
\end{equation*}

\subsubsection{Derivation.}
We consider the case of two dimension Lagrange.
Then deviate the correct path $x(t)$ and $y(t)$ into
\begin{align*}
	x(t) & \rightarrow x(t)+\delta x \\
	y(t) & \rightarrow y(t)+\delta y
\end{align*}
With the Lagrangian of $\mathcal{L}(x,\dot{x},y,\dot{y})$, the action integral takes the form
\begin{equation*}
	S=\int_{t_1}^{t_2}\mathcal{L}\;dt
\end{equation*}
Assuming the constraint is consisted with the displacement, then, as proved before, the integral is unchanged and $\delta S=0$. In other words
\begin{equation*}
	\int\left(\frac{\partial \mathcal{L}}{\partial x}\delta x+\frac{\partial\mathcal{L}}{\partial \dot{x}}\delta \dot{x}+\frac{\partial\mathcal{L}}{\partial y}\delta y+\frac{\partial\mathcal{L}}{\partial\dot{y}}\delta \dot{y}\right)\;dt=0
\end{equation*}
Using integration by part on second and fourth term to move the derivative sign
\begin{equation*}
	\int\left(\frac{\partial\mathcal{L}}{\partial x}-\frac{d}{dt}\frac{\partial\mathcal{L}}{\partial \dot{x}}\right)\delta x\;dt+\int \left(\frac{\partial\mathcal{L}}{\partial y}-\frac{d}{dt}\frac{\partial\mathcal{L}}{\partial \dot{y}}\right)\delta y\;dt=0
\end{equation*}
For displacement $\delta x$ and $\delta y$, the terms inside parenthesis must be zero for the integral to be zero.
Now, this is just the proof of Euler-Lagrange equation.
However, we want to explicitly including the constraint, which can be achieved by multiplying the deviation of the constraint equation $\delta f$ with Lagrange multipliers
\begin{align*}
	\delta f & =\frac{\partial f}{\partial x}\delta x +\frac{\partial f}{\partial y}\delta y               \\
	0        & =\lambda\frac{\partial f}{\partial x}\delta x +\lambda\frac{\partial f}{\partial y}\delta y \\
\end{align*}
This step is justified because the value is zero anyway.
Then we can add it into the integral of $\delta S$ without changing the integral itself
\begin{equation*}
	\int\left(\frac{\partial\mathcal{L}}{\partial x}+\lambda\frac{\partial f}{\partial x}-\frac{d}{dt}\frac{\partial\mathcal{L}}{\partial \dot{x}}\right)\delta x\;dt+\int \left(\frac{\partial\mathcal{L}}{\partial y}+\lambda\frac{\partial f}{\partial \y}-\frac{d}{dt}\frac{\partial\mathcal{L}}{\partial \dot{y}}\right)\delta y\;dt=	0
\end{equation*}
Since the multipliers is arbitrary, we define Lagrange multipliers such that the terms inside parenthesis is zero, thus resulting in two modified Euler-Lagrange equation written previously
\begin{align*}
	\frac{\partial \mathcal{L}}{\partial x}+\lambda\frac{\partial f}{\partial x} & =\frac{d}{dt}\frac{\partial \mathcal{L}}{\partial \dot{x}} \\
	\frac{\partial \mathcal{L}}{\partial y}+\lambda\frac{\partial f}{\partial y} & =\frac{d}{dt}\frac{\partial \mathcal{L}}{\partial \dot{y}} 
\end{align*}
Now we add the constraint equation since we need three equation to find three equation in three unknown.

\subsubsection{Lagrange multipliers physical meaning.}
Given that the kinetic energy of the system does not depend on the position and the potential energy does not depend on velocity, the modified Euler-Lagrange equation reads
\begin{align*}
	-\frac{\partial U}{\partial q_i}+\lambda\frac{\partial d}{\partial q_i}&=m\ddot{q}_i\\
	\lambda\frac{\partial f}{\partial q_i}&=m\ddot{q}_i+\frac{\partial U}{\partial q_i}
\end{align*}
Recall that the negative gradient of potential energy is the non constraint force, while the product of mass and generalized acceleration is total force. 
The total force is the sum of non constraint force and constraint force, so 
\begin{equation*}
	\lambda\frac{\partial f}{\partial q_i}=F_i^{\text{cstr}}	
\end{equation*}


\subsubsection{Comparison with the usual function optimization using Lagrange multipliers.}
Let us compare the optimization method of Lagrange multipliers when the function we are concerned is $f(x,y)$ instead of the previous case of action integral $S$.
To optimize $f(x,y)$, set its derivative to zero
\begin{equation*}
	\frac{\partial f}{\partial x}=0,\qquad\frac{\partial f}{\partial y}=0
\end{equation*}
while to optimize $S$, we use the Euler-Lagrange equation
\begin{equation*}
	\frac{\partial \mathcal{L}}{\partial x}-\frac{d}{dt}\frac{\partial\mathcal{L}}{\partial\dot{x}}=0,\qquad\frac{\partial\mathcal{L}}{\partial y}-\frac{d}{dt}\frac{\partial \mathcal{L}}{\partial \dot{y}}=0
\end{equation*}
If we include constraint $\phi(x,y)$, we construct function $F=f+\lambda\phi$ such that the multipliers is defined
\begin{equation*}
	\frac{\partial f}{\partial x}+\lambda\frac{\partial \phi}{\partial x}=0,\qquad\frac{\partial f}{\partial y}+\lambda\frac{\partial \phi}{\partial y}=0
\end{equation*}
while in the action integral case
\begin{equation*}
	\frac{\partial \mathcal{L}}{\partial x}+\lambda\frac{\partial\phi}{\partial x}-\frac{d}{dt}\frac{\partial\mathcal{L}}{\partial\dot{x}}=0,
	\qquad\frac{\partial\mathcal{L}}{\partial y}+\lambda\frac{\partial\phi}{\partial y}-\frac{d}{dt}\frac{\partial \mathcal{L}}{\partial \dot{y}}=0
\end{equation*}
In any case, we need to solve those two resulting function alongside the constraint equation.

\subsection{Lagrangian for a Charge in an Electromagnetic Field}
The Lagrangian is 
\begin{equation*}
	\mathcal{L}=\frac{1}{2}m\dot{\mathbf{r}}^2-q(V-\dot{\mathbf{r}}\cdot \mathbf{A})
\end{equation*}
This Lagrangian is defined is such way to produce the Lorentz force law
\begin{equation*}
	\mathbf{F}=q \left( \mathbf{E}+\dot{\mathbf{r}}\times \mathbf{B} \right) 
\end{equation*}

\subsubsection{Derivation.}
The derivation of Lagrangian for particle in electromagnetic field assumes you know field and gauge theory.
Since you didn't, we shall only prove that the Lagrangian does give the correct equation of motion.

First we write the Euler-Lagrangian equation in three dimension as 
\begin{equation*}
	\frac{d }{dt }\frac{\partial \mathcal{L }}{\partial \dot{\mathbf{r}}}=\frac{\partial \mathcal{L }}{\partial \mathbf{r}}
\end{equation*}
where both scalar and vector potential are a function of space and time.
Feeding the given Lagrangian
\begin{equation*}
	\frac{d }{dt }\left( m \dot{\mathbf{r}}+q \mathbf{A} \right) =-q \nabla \left( V-\dot{\mathbf{r} }\cdot\mathbf{A} \right) 
\end{equation*}
then rewriting it as such
\begin{align*}
	\frac{d }{dt }\left( m \dot{\mathbf{r}} \right) =-q \nabla V-q \frac{d \mathbf{A }}{dt }+ q \nabla \left( \dot{\mathbf{r }}\cdot \mathbf{A} \right) 
\end{align*}
The total derivative of $\mathbf{A}(\mathbf{r },t)$ with respect to time has two part: spatial variation and explicit dependence on time
\begin{equation*}
	\frac{d\mbox{A }}{dt}=\sum_i \frac{\partial \mathbf{A }}{\partial q_i}\frac{\partial q_i }{\partial t }+\frac{\partial \mathbf{A }}{\partial t }=\sum_i \dot{q}_i \frac{\partial }{\partial q_i }\mathbf{A}+ \frac{\partial \mathbf{A }}{\partial t }= \left( \dot{\mathbf{r }}\cdot \nabla\right)\mathbf{A}+ \frac{\partial \mathbf{A }}{\partial t}
\end{equation*}
Now we have 
\begin{equation*}
	\mathbf{F }=-q \left( \nabla  V+\frac{\partial \mathbf{A }}{\partial t} \right) +q \left[ \nabla \left( \dot{\mathbf{r }\cdot \mathbf{A }} \right)-\left( \dot{\mathbf{r}}\cdot \nabla   \right)\mathbf{A}   \right] 
\end{equation*}
Recalling the vector triple product
\begin{equation*}
	\dot{\mathbf{r }}\times \left( \nabla \times \mathbf{A }  \right)=\nabla \left( \dot{\mathbf{r }}\cdot \mathbf{A} \right)  -\left( \dot{\mathbf{r }}\cdot \nabla   \right) \mathbf{A}
\end{equation*}
to write it as 
\begin{equation*}
	\mathbf{F }=-q \left( \nabla V+\frac{\partial \mathbf{A }}{\partial t } \right) +q \dot{\mathbf{r }}\times \left( \nabla \times \mathbf{A} \right) 
\end{equation*}
In terms of potential, both electromagnetic field maybe expressed as
\begin{equation*}
	\mathbf{E}=-\nabla V-\frac{\partial \mathbf{A }}{\partial t}\qquad \text{and}\qquad \mathbf{B} =\nabla \times \mathbf{A}
\end{equation*} 
Hence, we obtain our desired result
\begin{equation*}
	\mathbf{F}=q \left( \mathbf{E}+\dot{\mathbf{r}}\times \mathbf{B} \right) 
\end{equation*}
\end{document}