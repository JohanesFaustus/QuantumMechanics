\documentclass[../../../main.tex]{subfiles}
\begin{document}
The Lagrangian is defined as
\begin{equation*}
  \mathcal{L}=T-U
\end{equation*}
which mean that the Lagrangian is a function of position and velocity.
The path of particle is determined by Hamilton's principle
\begin{equation*}
  S=\int_{t_1}^{t_2}\mathcal{L}\;dt
\end{equation*}
that is, the particle's path is such that the action integral $S$ is stationary.

We can express the coordinate in other generalized coordinate
\begin{equation*}
  \mathbf{r}=\mathbf{r}(q_1,q_2,q_2)
\end{equation*}
And the same for velocity
\begin{equation*}
  \mathbf{v}=\mathbf{v}(\dot{q}_1,\dot{q}_2,\dot{q}_3)
\end{equation*}
Now the action integral reads
\begin{equation*}
  S=\int_{t_1}^{t_2}\mathcal{L}(q_1,q_2,q_3,\dot{q}_1,\dot{q}_2,\dot{q}_3,t)\;dt
\end{equation*}
Therefore we have three Euler-Lagrange equation that must be satisfied by the particle
\begin{align*}
    \frac{\partial \mathcal{L}}{\partial q_1}=\frac{d}{dt}\frac{\partial \mathcal{L}}{\partial \dot{q}_1}, \quad
    \frac{\partial \mathcal{L}}{\partial q_2}=\frac{d}{dt}\frac{\partial \mathcal{L}}{\partial \dot{q}_2}, \quad
    \frac{\partial \mathcal{L}}{\partial q_3}=\frac{d}{dt}\frac{\partial \mathcal{L}}{\partial \dot{q}_3}
\end{align*}
This is the case of one unconstrained particle. For $N$ unconstrained particle, then, we shall have $3N$ Lagrange equation
\begin{equation*}
    \frac{\partial\mathcal{L}}{\partial q_i}=\frac{d}{dt}\frac{\partial \mathcal{L}}{\partial \dot{q}_i}\qquad i=1,\dots,3N
\end{equation*}

The Lagrangian equation 
\begin{equation*}
  \frac{\partial \mathcal{L}}{\partial q_i}=\frac{d}{dt}\frac{\partial \mathcal{L}}{\partial \dot{q}_i}
\end{equation*}
takes the form
\begin{equation*}
    \text{Generalized force}=\text{Rate of change of Generalized momentum}
\end{equation*}
where 
\begin{equation*}
    \frac{\partial \mathcal{L}}{\partial q_i}=i\text{-th component of the Generalized force}
\end{equation*}
and
\begin{equation*}
    \frac{\partial \mathcal{L}}{\partial \dot{q}_i}=i\text{-th component of the Generalized momentum}
\end{equation*}

\subsection*{Newton Law in 2D Cartesian}
The Lagrangian in this case is 
\begin{equation*}
  \mathcal{L}(x,y,\dot{x},\dot{y})=\frac{1}{2}m(\dot{x}^2+\dot{y}^2)-U(x,y)
\end{equation*}
Here we have two Euler-Lagrange equation
\begin{equation*}
  \frac{\partial \mathcal{L}}{\partial x}=\frac{d}{dt}\frac{\partial \mathcal{L}}{\partial \dot{x}},\qquad
  \frac{\partial \mathcal{L}}{\partial y}=\frac{d}{dt}\frac{\partial \mathcal{L}}{\partial \dot{y}}
\end{equation*}
The derivative with respect to position is force
\begin{align*}
  \frac{\partial \mathcal{L}}{\partial x}=-\frac{\partial U}{\partial x}=F_x,\qquad
  \frac{\partial \mathcal{L}}{\partial y}=-\frac{\partial U}{\partial y}=F_y
\end{align*}
while the derivative with respect to velocity is momentum
\begin{align*}
  \frac{\partial\mathcal{L}}{\partial \dot{x}}=\frac{\partial T}{\partial \dot{x}}=m\dot{x}, \qquad
  \frac{\partial\mathcal{L}}{\partial \dot{y}}=\frac{\partial T}{\partial \dot{y}}=m\dot{y}
\end{align*}
Substituting this result in the two Euler-Lagrange equation, we have 
\begin{equation*}
  F_x=m\ddot{x},\qquad F_y=m\ddot{y}
\end{equation*}
which are the two component of Newton second law $\mathbf{F}=m\mathbf{\ddot{r}}$.

\subsection*{Newton Law in Polar Coordinate}
The Lagrangian is 
\begin{equation*}
  \mathcal{L}(r,\phi,\dot{r},\dot{\phi})=\frac{1}{2}m(\dot{r}^2+r^2\dot{\phi}^2)-U(r,\phi)
\end{equation*}
which result into two Euler-Lagrange equation
\begin{equation*}
  \frac{\partial\mathcal{L}}{\partial r}=\frac{d}{dt}\frac{\partial\mathcal{L}}{\partial \dot{r}}\qquad\frac{\partial\mathcal{L}}{\partial \phi}=\frac{d}{dt}\frac{\partial\mathcal{L}}{\partial \dot{\phi}}
\end{equation*}
Before moving into evaluating the derivative with respect to $r$ and $\phi$, recall the gradient of potential energy in polar coordinate
\begin{equation*}
  \nabla U=\frac{\partial U}{\partial r}\mathbf{\hat{{r}}}+ \frac{1}{r}\frac{\partial U}{\partial \phi}\boldsymbol{\hat{\phi}}-F_r\mathbf{\hat{r}}-F_{\phi}\boldsymbol{\hat{\phi}}
\end{equation*} 

Evaluating the radial derivative
\begin{align*}
  mr\dot{\phi}-\frac{\partial U}{\partial r}&=m\ddot{r}\\
  F_r&=m(\ddot{r}+r\dot{\phi})
\end{align*}
which is simply the radial component of the Newton's second law in polar coordinate. 
Evaluating the angular derivative
\begin{align*}
  -\frac{\partial U}{\partial \phi}&=\frac{d}{dt}mr^2\dot{\phi}=m(2r\dot{r}\dot{\phi}+r^2\ddot{\phi})\\
  -\frac{1}{r}\frac{\partial U}{\partial \phi}&=m(2\dot{r}\dot{\phi}+r^2\ddot{\phi})\\
  F_\phi&=m(2\dot{r}\dot{\phi}+r\ddot{\phi})
\end{align*}
which is, as previously mentioned, the angular component of Newton's second law. It should be noted that the quantity $mr^2\dot{\phi}$ can be recognized as angular momentum $L$, and the rate of change of it is torque $\Gamma$
\begin{equation*}
  \Gamma=F_\phi r=\frac{dL}{dt}=\frac{d}{dt}mr^2\dot{\phi}
\end{equation*} 
\end{document}