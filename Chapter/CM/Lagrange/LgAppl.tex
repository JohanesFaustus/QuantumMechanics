\documentclass[../../../main.tex]{subfiles}
\begin{document}
\subsection*{Application: Atwood's Machine}
Atwood's machine consist of  masses $m_1$ and $m_2$ are suspended by an in extensible string (length $l$) which passes over a massless pulley with frictionless bearings and radius $R$.
The length of the string acts as constraint
\begin{equation*}
    x+y+2\pi R=l
\end{equation*} 
This implies $y=-x+C$ and $\dot{y}=-\dot{x}$. 
Thus, the kinetic energy of both mass
\begin{equation*}
    T=\frac{1}{2}m_1\dot{x}^2+\frac{1}{2}m_2\dot{y}^2=\frac{1}{2}(m_1+m_2)\dot{x}^2
\end{equation*}
We define the downward displacement of $m_1$ and upward displacement of $m_2$ as positive displacement in our generalized coordinate $x$. 
This is the same case of upward acceleration of $m_2$ being the same as downward acceleration of $m_1$.
In any case, the potential energy is 
\begin{equation*}
    U=-m_1gx-m_2gy=-(m_1-m_2)gx+C_2
\end{equation*}
We can now write the Lagrangian as 
\begin{equation*}
    \mathcal{L}=\frac{1}{2}(m_1+m_2)\dot{x}^2+(m_1-m_2)gx
\end{equation*}
With only one generalized coordinate, we only have one Euler-Lagrange equation
\begin{equation*}
    \frac{\partial \mathcal{L}}{\partial x}=\frac{d}{dt}\frac{\partial \mathcal{L}}{\partial \dot{x}}
\end{equation*}
which on substituting the Lagrangian yield
\begin{align*}
    (m_1-m_2)g=(m_1+m_2)\ddot{x}\\
    \ddot{x}=\frac{m_1-m_2}{m_1+m_2}g
\end{align*}

Now let's compare it with Newtonian approach, we should obtain the same result.
Considering the acceleration direction for both masses, the net forces on both $m_1$ and $m_2$ respectively are 
\begin{align*}
    m_1g-F_t&=m_1\ddot{x}\\
    F_t-m_2g&=m_2\ddot{x}
\end{align*}
Adding both equation
\begin{align*}
    (m_1-m_2)g&=(m_1+m_2)\ddot{x}\\
    \ddot{x}&=\frac{m_1-m_2}{m_1+m_2}g
\end{align*}

\begin{figure*}
    \centering
    \dfig{../../../Rss/CM/Lagrange/Atwood.png}
    \caption*{Figure: Atwood's machine configuration}
\end{figure*}

\subsection*{Application: Particle Constrained on a Cylinder}
Consider a particle of mass $m$ constrained to move on a frictionless cylinder of radius $R$. 
Besides the force of constraint, the only force on the mass is a force $\mathbf{F}= -k\mathbf{r}$ directed toward the origin. 
With $\mathbf{r}$ as the position vector of the particle, this force is the three dimension version of Hooke's law.

We shall use cylindrical coordinate to solve this problem.
It is known that the radius component is fixed $\rho=R$, so we use $(\phi, z)$ as our generalized coordinate.
The kinetic energy is 
\begin{equation*}
    T=\frac{1}{2}mv^2=\frac{1}{2}m(R^2\dot{\phi}^2+\dot{x}^2)
\end{equation*}
Recall $\mathbf{F}=-\nabla U$, hence the potential energy
\begin{align*}
    -\frac{dU}{dr}\;\mathbf{\hat{r}}&=-kr\;\mathbf{\hat{r}}\\
    U&=\frac{1}{2}kr^2
\end{align*}
The distance of particle from origin is given by $r^2=R^2+z^2$, so 
\begin{equation*}
    U=\frac{1}{2}k(z^2+R^2)
\end{equation*}
Therefore, the Lagrangian is 
\begin{equation*}
    \mathcal{L}=\frac{1}{2}m(R^2\dot{\phi}^2+\dot{z}^2)-\frac{1}{2}k(R^2+z^2)
\end{equation*}
We use two generalized coordinates, thus we have two Euler-Lagrangian equation
\begin{equation*}
    \frac{\partial \mathcal{L}}{\partial z}=\frac{d}{dt}\frac{\partial \mathcal{text}}{\partial \dot{z}},\qquad\frac{\partial\mathcal{L}}{\partial \phi}=\frac{d}{dt}\frac{\partial\mathcal{L}}{\partial \dot{\phi}}
\end{equation*}
The $z$ equation is 
\begin{equation*}
    -kz=m\ddot{z}\implies z=A\cos(\omega t-\delta)
\end{equation*}
This mean that the mass perform simple harmonic motion in the $z$ direction. 
Now, the $\phi$ equation
\begin{align*}
    0=\frac{d}{dt}mR^2\dot{\phi}
\end{align*}
This mean that the angular momentum $L=mR^2\dot{\phi}$ is conserved and the particle rotate in constant velocity $\dot{\phi}$.

\begin{figure*}
    \centering
    \dfig{../../../Rss/CM/Lagrange/Cyllinder.png}
    \caption*{Figure: Particle constrained to move on a cylinder}
\end{figure*}

\end{document}
