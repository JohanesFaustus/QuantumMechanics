\documentclass[../../../main.tex]{subfiles}
\begin{document}
\subsection{Application: Atwood's Machine}
Atwood's machine consist of  masses $m_1$ and $m_2$ are suspended by an in extensible string (length $l$) which passes over a massless pulley with frictionless bearings and radius $R$.
The length of the string acts as constraint
\begin{equation*}
	x+y+2\pi R=l
\end{equation*}
This implies $y=-x+C$ and $\dot{y}=-\dot{x}$.
Thus, the kinetic energy of both mass
\begin{equation*}
	T=\frac{1}{2}m_1\dot{x}^2+\frac{1}{2}m_2\dot{y}^2=\frac{1}{2}(m_1+m_2)\dot{x}^2
\end{equation*}
We define the downward displacement of $m_1$ and upward displacement of $m_2$ as positive displacement in our generalized coordinate $x$.
This is the same case of upward acceleration of $m_2$ being the same as downward acceleration of $m_1$.
In any case, the potential energy is
\begin{equation*}
	U=-m_1gx-m_2gy=-(m_1-m_2)gx+C_2
\end{equation*}
We can now write the Lagrangian as
\begin{equation*}
	\mathcal{L}=\frac{1}{2}(m_1+m_2)\dot{x}^2+(m_1-m_2)gx
\end{equation*}
With only one generalized coordinate, we only have one Euler-Lagrange equation
\begin{equation*}
	\frac{\partial \mathcal{L}}{\partial x}=\frac{d}{dt}\frac{\partial \mathcal{L}}{\partial \dot{x}}
\end{equation*}
which on substituting the Lagrangian yield
\begin{align*}
	(m_1-m_2)g=(m_1+m_2)\ddot{x} \\
	\ddot{x}=\frac{m_1-m_2}{m_1+m_2}g
\end{align*}

Now let's compare it with Newtonian approach, we should obtain the same result.
Considering the acceleration direction for both masses, the net forces on both $m_1$ and $m_2$ respectively are
\begin{align*}
	m_1g-F_t & =m_1\ddot{x} \\
	F_t-m_2g & =m_2\ddot{x}
\end{align*}
Adding both equation
\begin{align*}
	(m_1-m_2)g & =(m_1+m_2)\ddot{x}        \\
	\ddot{x}   & =\frac{m_1-m_2}{m_1+m_2}g
\end{align*}

\begin{figure*}
	\centering
	\dfig{../../../Rss/CM/Lagrange/Atwood.png}
	\caption*{Figure: Atwood's machine configuration}
\end{figure*}

Now suppose we explicitly include the constraint using Lagrange multipliers.
Here, the constraint is the wire and the constraining force is the tension, which takes the form 
\begin{equation*}
    f(x,y)=x+y=\text{Constant}
\end{equation*}
Since we do not need to reduce the number of coordinate, we express the Lagrangian as 
\begin{equation*}
    \mathcal{L}=\frac{1}{2}m_1\dot{x}^2+\frac{1}{2}m_2\dot{y}^2+m_1gx+m_2gy
\end{equation*}

Then the equations that wee need to solve are as follows
\begin{align*}
    \frac{\partial \mathcal{L}}{\partial x}&=\frac{d}{dt}\frac{\partial\mathcal{L}}{\partial\dot{x}}\\
    \frac{\partial \mathcal{L}}{\partial y}&=\frac{d}{dt}\frac{\partial\mathcal{L}}{\partial\dot{y}}\\
    x+y=\text{Constant}
\end{align*}
The first two equation yield
\begin{equation*}
    m_1g+\lambda=m_1\ddot{x},\qquad m_2g+\lambda=m_2\ddot{y}
\end{equation*}
Performing second derivative to the third equation with respect to time results
\begin{equation*}
    \ddot{x}=-\ddot{y}
\end{equation*}
Using this and subtracting the first equation by the second 
\begin{align*}
    (m_1+m_2)\ddot{x}&=(m_1-m_2)g\\
    \ddot{x}&=\frac{m_1-m_2}{m_1+m_2}g
\end{align*}
which is the same result. 
It can be seen that, by comparing to the Newtonian result, that the Lagrange multipliers gives
\begin{equation*}
    \lambda=-F_t
\end{equation*} 

\subsection{Application: Particle Constrained on a Cylinder}
Consider a particle of mass $m$ constrained to move on a frictionless cylinder of radius $R$.
Besides the force of constraint, the only force on the mass is a force $\mathbf{F}= -k\mathbf{r}$ directed toward the origin.
With $\mathbf{r}$ as the position vector of the particle, this force is the three dimension version of Hooke's law.

We shall use cylindrical coordinate to solve this problem.
It is known that the radius component is fixed $\rho=R$, so we use $(\phi, z)$ as our generalized coordinate.
The kinetic energy is
\begin{equation*}
	T=\frac{1}{2}mv^2=\frac{1}{2}m(R^2\dot{\phi}^2+\dot{x}^2)
\end{equation*}
Recall $\mathbf{F}=-\nabla U$, hence the potential energy
\begin{align*}
	-\frac{dU}{dr}\;\mathbf{\hat{r}} & =-kr\;\mathbf{\hat{r}} \\
	U                                & =\frac{1}{2}kr^2
\end{align*}
The distance of particle from origin is given by $r^2=R^2+z^2$, so
\begin{equation*}
	U=\frac{1}{2}k(z^2+R^2)
\end{equation*}
Therefore, the Lagrangian is
\begin{equation*}
	\mathcal{L}=\frac{1}{2}m(R^2\dot{\phi}^2+\dot{z}^2)-\frac{1}{2}k(R^2+z^2)
\end{equation*}
We use two generalized coordinates, thus we have two Euler-Lagrangian equation
\begin{equation*}
	\frac{\partial \mathcal{L}}{\partial z}=\frac{d}{dt}\frac{\partial \mathcal{text}}{\partial \dot{z}},\qquad\frac{\partial\mathcal{L}}{\partial \phi}=\frac{d}{dt}\frac{\partial\mathcal{L}}{\partial \dot{\phi}}
\end{equation*}
The $z$ equation is
\begin{equation*}
	-kz=m\ddot{z}\implies z=A\cos(\omega t-\delta)
\end{equation*}
This mean that the mass perform simple harmonic motion in the $z$ direction.
Now, the $\phi$ equation
\begin{align*}
	0=\frac{d}{dt}mR^2\dot{\phi}
\end{align*}
This mean that the angular momentum $L=mR^2\dot{\phi}$ is conserved and the particle rotate in constant velocity $\dot{\phi}$.

\begin{figure*}
	\centering
	\dfig{../../../Rss/CM/Lagrange/Cyllinder.png}
	\caption*{Figure: Particle constrained to move on a cylinder}
\end{figure*}

\subsection{Application: Sliding Block on a Frictionless Wedge}
The block (mass $m$) is free to slide on the wedge, and the wedge (mass $M$) can slide on the horizontal table, both with negligible friction.
The block is released from the top of the wedge, with both initially at rest.

\begin{figure*}[b]
	\centering
	\normfigL{../../../Rss/CM/Lagrange/WedgeIncline.png}
	\caption*{Figure: Block slides on a wedge which is free to move without friction}
\end{figure*}

The system has two coordinates and no constraint whatsoever, so it has two degree of freedom.
We choose $q_1$ and $q_2$ as our generalized coordinate which denote the distance form the block from the top of the wedge and the distance of the wedge from convenient fixed point on the table.
We also define positive $x$ displacement to the right and downward as positive $y$ displacement.

The kinetic energy is
\begin{equation*}
	T=\frac{1}{2}mv_m^2+\frac{1}{2}Mv_M^2
\end{equation*}
The wedge velocity is simply
\begin{equation*}
	v_M=\dot{q}_2
\end{equation*}
Meanwhile, the block velocity have two component, which are
\begin{equation*}
	\mathbf{v}_m=\left(\dot{q}_2+\dot{q}_1\cos\alpha\right)\;\mathbf{\hat{x}}+\dot{q}_2\sin\alpha\;\mathbf{\hat{y}}
\end{equation*}
In terms of generalized coordinate, the kinetic energy reads
\begin{align*}
	T & =\frac{1}{2}m\left[\left(\dot{q}_2+\dot{q}_1\cos\alpha\right)^2+\dot{q}_2\sin\alpha^2\right]+\frac{1}{2}m\dot{q}_2^2 \\
	  & =\frac{1}{2}m\left[\dot{q}_1^2+\dot{q}_2^2+\dot{q}_1\dot{q}_2\cos\alpha+\right]+\frac{1}{2}M\dot{q}_2^2              \\
	T & =\frac{1}{2}(m+M)\dot{q}_2^2+\frac{1}{2}m\left(\dot{q}_1^2+2\dot{q}_1\dot{q}_2\cos\alpha\right)
\end{align*}
In other hand, we defined downward as positive displacement, so the potential energy reads
\begin{equation*}
	U=-mgy=-mgq_1\sin\alpha
\end{equation*}
The Lagrangian can be evaluated as
\begin{equation*}
	\mathcal{L}=\frac{1}{2}(m+M)\dot{q}_2^2+\frac{1}{2}m\left(\dot{q}_1^2+2\dot{q}_1\dot{q}_2\cos\alpha\right)+mgq_1\sin\alpha
\end{equation*}

With the generalized coordinates we used, we have two Euler-Lagrange equation
\begin{equation*}
	\frac{\partial \mathcal{L}}{\partial q_1}=\frac{d}{dt}\frac{\partial \mathcal{L}}{\partial \dot{q}_2}\qquad\frac{\partial \mathcal{L}}{\partial q_2}=\frac{d}{dt}\frac{\partial \mathcal{L}}{\partial \dot{q}_2}
\end{equation*}
The $q_1$ equation yield
\begin{align*}
	mg\sin\alpha & =\frac{d}{dt}\left[m(\dot{q}_1+\dot{q}_2\cos\alpha)\right] \\
	g\sin\alpha  & =\ddot{q}_1+\ddot{q}_2\cos\alpha
\end{align*}
while the $q_2$ equation yield
\begin{align*}
	0          & =\frac{d}{dt}\left[(M+m)\dot{q}_2+m(\dot{q}_2+\dot{q}_1\cos\alpha)\right] \\
	           & =(M+m)\ddot{q}_2+m(\ddot{q}_2+\ddot{q}_1\cos\alpha)                       \\
	\ddot{q}_2 & =-\frac{m}{M+m}\ddot{q}_1\cos\alpha
\end{align*}
which is just conservation of momentum in the $x$ direction
\begin{equation*}
	m\dot{q}_2+m\left(\dot{q_2}+\dot{q}_1\cos\alpha\right)=\text{Constant}
\end{equation*}
Now, combining the $q_1$ and $q_2$ result in
\begin{align*}
	\ddot{q}_1 & =g\sin\alpha+\frac{m}{M+m}\ddot{q}_1\cos^2\alpha \\
	\ddot{q}_1 & =\frac{g\sin\alpha}{1-\frac{m}{M+m}\cos^2\alpha}
\end{align*}
Suppose we want to determine the time it took for the block to reach the bottom of wedge, we can use the kinematic relation $x(t)=x_i+v_it+at^2/2$ or $t=\sqrt{2l/a}$, with $l$ as the length of the slope, to obtain
\begin{equation*}
	t=\left(2l\frac{1-\frac{m}{M+m}\cos^2\alpha}{g\sin\alpha}\right)^{1/2}
\end{equation*}

As a sanity check, consider the case for $\alpha=90^\circ$.
The acceleration $\ddot{q}_1=g$, which is correct.
Another is the case for $M\rightarrow\infty$.
The acceleration $\ddot{q}_1=g\sin\alpha$, which is the acceleration for a block on a fixed incline.


\subsection{Application: Simple Pendulum}
A bob of mass $m$ is fixed to a massless rod length $l=\sqrt{x^2+y^2}$, which is pivoted at $O$ and free to swing without friction in the $xy$ plane.
One way to integrate the constraint into the Lagrangian is by expressing both coordinate in terms single generalized coordinate $\phi$ or by writing one of them in terms other variable, say  $y=\sqrt{l^2-x^2}$.

In terms of $\phi$,the kinetic energy is 
\begin{equation*}
    T=\frac{1}{2}ml^2\dot{\phi}^2
\end{equation*}
and the potential energy 
\begin{equation*}
    U=mg(l-l\cos\phi)=mgl(1-\cos\phi)
\end{equation*}
So, the Lagrangian reads
\begin{equation*}
    \mathcal{L}=\frac{1}{2}ml^2\dot{\phi}^2-mgl(1-\cos\phi)
\end{equation*}
In this case, we only have one Euler-Lagrange equation
\begin{equation*}
    \frac{\partial\mathcal{L}}{\partial \phi}=\frac{d}{dt}\frac{\partial\mathcal{L}}{\partial \dot{\phi}}
\end{equation*}
Substituting the Lagrangian
\begin{align*}
    -mgl\sin\phi&=ml^2\ddot{\phi}\\
    \sin\phi&=-\frac{l}{g}\ddot{\phi}
\end{align*}
This is the differential equation describing simple pendulum motion, which, on assuming small angle $\phi$, has the solution $\phi=A\cos(\omega t+\delta)$. 
Also, recall $\ddot{\phi}$ is the angular acceleration. 
This mean that the Euler-Lagrange equation reproduce the formula for torque $\Gamma=I\alpha=ml^2\ddot{\phi}$ or $\Gamma=Fr=-mgl\sin\phi$. 

\begin{figure*}
	\centering
	\dfig{../../../Rss/CM/Lagrange/Pendulum.png}
	\caption*{Figure: A simple pendulum}
\end{figure*}

\subsection{Application: Bead Spinning on a Wire Hoop}
A bead of mass $m$ is threaded on a frictionless circular wire hoop of radius $R$.
The hoop lies in a vertical plane, which is forced to rotate about the hoop's vertical diameter with constant angular velocity $\dot{\phi} =\omega$.
The bead's position on the hoop is specified by the angle $\theta$ measured up from the rotation axis.
We shall use $\theta$ as our only generalized coordinate.

The kinetic energy is 
\begin{equation*}
    T=\frac{1}{2}m\left(v_\theta^2+v_\phi^2\right)
\end{equation*}
with $v_{\theta}$ denote the bead tangential velocity with respect to non-rotating hoop, while $v_{\phi}$ denote the rotation velocity of the hoop.
The tangential velocity is simply $v_\theta=R\dot{\theta}$, 
\begin{equation*}
    v_\theta=R\dot{\theta}
\end{equation*}with $R$ as the distance of the bead with the axis of rotation--the hoop radius in other words.
With the same principle, the hoop velocity is $v_\phi=\rho \dot{\phi}$.
From the figure and known quantity, $\rho=R\cos\theta$ and $\dot{\phi}=\omega$, thus 
\begin{equation*}
    v_\phi=R\sin\theta \omega
\end{equation*}
In terms of generalized coordinate, the kinetic energy reads
\begin{equation*}
    T=\frac{1}{2}mR^2\left(\dot{\theta}^2+\omega^2\sin^2\theta\right)
\end{equation*}
The potential energy is
\begin{equation*}
    U=mg(R-R\cos\theta)=mgR(1-\cos\theta)
\end{equation*}

Therefore, the Lagrangian is
\begin{equation*}
    \mathcal{L}=\frac{1}{2}mR^2\left(\dot{\theta}^2+\omega^2\sin^2\theta\right)-mgR(1-\cos\theta)
\end{equation*}
which yield one Euler-Lagrange equation
\begin{equation*}
    \frac{\partial\mathcal{L}}{\partial \theta}=\frac{d}{dt}\frac{\partial \mathcal{L}}{\partial \dot{\theta}}
\end{equation*}
Substituting the Lagrangian
\begin{align*}
    mR^2\omega^2\sin\theta\cos\theta-mgR\sin\theta&=mR^2\ddot{\theta}
\end{align*}
dividing by $mR^2$
\begin{align*}
    \ddot{\theta}&=\omega^2\sin\theta\cos\theta-\frac{g}{R}\sin\theta\\
    \ddot{\theta}&=\left(\omega^2\cos\theta-\frac{g}{R}\right)\sin\theta
\end{align*}
\begin{figure*}
	\centering
	\dfig{../../../Rss/CM/Lagrange/BeadWire.png}
	\caption*{Figure: Bead constrained into moving within wire hoop}
\end{figure*}

This equation can be used to determine the equilibrium point--that is, the point where the position of the system does not change--and the stable point--that is, the position at which the system returns after slightly disturbed--of the system.
The requirement of equilibrium point is $\dot{\theta}=0$, but we can obtain the same result by setting both $\dot{\theta}$ and $\ddot{\theta}$ to zero 
\begin{equation*}
    \left(\omega^2\cos\theta-\frac{g}{R}\right)\sin\theta=0
\end{equation*}
If we set $\sin\theta$ to zero, we obtain the following equilibrium points
\begin{equation*}
    \theta=0,\pi
\end{equation*}
If we set the term inside parenthesis to zero 
\begin{equation*}
    \cos\theta=\frac{g}{\omega^2R}
\end{equation*}
or, due to cosine being even function
\begin{equation*}
    \cos\left(-\theta\right)=\frac{g}{\omega^2R}
\end{equation*}
hence we have the following equilibrium
\begin{equation*}
    \theta=\pm\arccos\frac{g}{\omega^2R}
\end{equation*}
Arccos function only has real value at $\theta\in [-1,1]$, so when 
\begin{equation*}
    \bigg|\frac{g}{\omega^2R}\bigg|>1\quad\text{or}\quad\omega^2<\frac{g}{R}
\end{equation*} 
This equation undefined and the stable point disappear. 
These equilibrium points only appear when $\omega^2>g/R$ and located on either side of bottom $\theta=0$.

Out of these four equilibrium points, not all of them are stable.
\begin{enumerate}
    \item Top $\theta=\pi$ point is unstable point due to not having restorative force, both gravitational and centrifugal push the bead away.
    \item Bottom $\theta=0$ point depends on $\omega^2$.
    It is stable if $\omega^2<g/R$, but become unstable if $\omega^2>g/R$.
    This can be proven by approximating small $\theta$ displacement
    \begin{equation*}
        \ddot{\theta}=\left(\omega^2-g/R\right)\theta
    \end{equation*} 
    If $\omega^2<g/R$, then
    \begin{equation*}
        \ddot{\theta}=-\Omega^2\theta
    \end{equation*}
    with
    \begin{equation*}
        \Omega=\sqrt{\frac{g}{r}-\omega^2}
    \end{equation*}
    which mean the bead perform simple harmonic motion about the stable point.
    If $\omega^2>g/R$, then 
    \begin{equation*}
        \ddot{\theta}=\Omega^2\theta
    \end{equation*}
    which has the solution $\theta=Ae^{\Omega t}+Be^{-\Omega t}$, so it moves in exponential way and the point is unstable.
    \item [3\&4] These two point that comes after speeding up the rotation such that $\omega^2>g/R$ are stable.
    To proof this, we expand the equation
    \begin{equation*}
        \ddot{\theta}=\left(\omega^2\cos\theta-\frac{g}{R}\right)\sin\theta
    \end{equation*}
    around stable point $\theta_0$
    \begin{equation*}
        \theta\equiv\theta+\epsilon
    \end{equation*}
    using Taylor expansion
    \begin{equation*}
        \cos(\theta_0+\epsilon)\approx\cos(\theta_0)-\epsilon\sin\theta_0,\quad\sin(\theta_0+\epsilon)\approx\sin\theta_0+\epsilon\cos\theta_0
    \end{equation*}
    This result in 
    \begin{align*}
        \ddot{\theta}&=\left[\omega^2\cos\theta_0-\omega^2\epsilon\sin\theta_0 -\frac{g}{R}\right]\left[\sin\theta_0+\epsilon\cos\theta_0\right]\\
        \ddot{\theta}&=-\omega^2\epsilon\sin^2\theta_0-\omega^2\epsilon^2\sin\theta_0\cos\theta_0
    \end{align*}
    Since $\epsilon$ is small, we can ignore the second order
    \begin{equation*}
        \ddot{\theta}=-\epsilon\omega^2\sin^2\theta_0
    \end{equation*} 
    Since $\ddot{\theta}$ is the same as $\ddot{\epsilon}$
    \begin{equation*}
        \ddot{\epsilon}=-\omega^2\sin^2\theta=-\Omega'^2\epsilon
    \end{equation*}
    with 
    \begin{equation*}
        \Omega=\omega\sin\theta=\sqrt{\omega-\omega\cos^2\theta_0}=\sqrt{\omega^2-\left(\frac{g}{\omega R}\right)^2}
    \end{equation*}
    This mean $\epsilon$ oscillates about zero, and the bead itself oscillates about the equilibrium position $\theta_0$ with frequency $\Omega'$.
\end{enumerate}

\end{document}
