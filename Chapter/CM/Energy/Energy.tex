\documentclass[../../../main.tex]{subfiles}
\begin{document}
\subsection*{Potential Energy}
\subsubsection*{Gravitational.} 
The potential energy is defined 
\begin{equation*}
    U=mgh
\end{equation*}
where the reference point is choose to be the ground.
This makes it so that 
\begin{equation*}
    F=-\nabla U=-\frac{d}{dr}mgr\;\mathbf{\hat{r}}=-mg\;\mathbf{\hat{r}}
\end{equation*}
the gravitational force is negative, or point downward.
Now if we define the downward as positive displacement, the potential energy reads
\begin{equation*}
    U=-mgh
\end{equation*}
and the gravitational force
\begin{equation*}
    F=-\nabla U=-\frac{d}{dr}\left(-mgr\;\mathbf{\hat{r}}\right)=mg\;\mathbf{\hat{r}} 
\end{equation*}
is positive, or point downward all the same.
\end{document}