\documentclass[../../../main.tex]{subfiles}
\begin{document}
\subsection{Newton's Second Law in Polar Coordinate}
Acceleration in polar coordinate expressed as
\begin{equation*}
    \mathbf{\ddot{r}}=\left(\ddot{r}-r\dot{\phi}^2\right) \; \mathbf{\hat{r}} + \left(r\ddot{\phi}+2\dot{r}\dot{\phi}\right) \;\boldsymbol{ \hat{\phi}}
\end{equation*}
and velocity as
\begin{equation*}
    \mathbf{v}=\dot{r}\;\mathbf{\hat{r}} +r\dot{\phi}\;\boldsymbol{ \hat{\phi}}
\end{equation*}
Hence Newton's law transform into
\begin{equation*}
    \mathbf{F}=m\mathbf{a}=\begin{cases}
        F_r    & =m\left(\ddot{r}-r\dot{\phi}^2\right)          \\
        F_\phi & =m\left(r\ddot{\phi}+2\dot{r}\dot{\phi}\right)
    \end{cases}
\end{equation*}

\subsection{Derivation}
\begin{figure*}[h]
    \centering
    \normfigL{../Rss/CM/CentralForce/drValue.png}
    \normfigL{../Rss/CM/CentralForce/dphiValue.png}
    \caption*{The value of $d\mathbf{\hat{r}}$ and $d\boldsymbol{\hat{\phi}}$.}
\end{figure*}

\subsubsection{Newtonian derivation.}
From the figure, we have
\begin{equation*}
    d\mathbf{\hat{r}}=d\phi\;\boldsymbol{\hat{\phi}},\quad d\boldsymbol{\hat{\phi}}=-d\phi\;\mathbf{\hat{r}}
\end{equation*}
or equivalently
\begin{equation*}
    \frac{d\mathbf{\hat{r}}}{dt}=\dot{\phi}\;\boldsymbol{\hat{\phi}},\quad \frac{d\boldsymbol{\hat{\phi}}}{dt}=-\dot{\phi}\;\mathbf{\hat{r}}
\end{equation*}
Using these we can now proceed to derive the Newton's law in polar coordinate. In cartesian coordinate, position vector can be writen as
\begin{equation*}
    \mathbf{r}=x\;\mathbf{\hat{x}} + y\;\mathbf{\hat{y}}
\end{equation*}
converting it into polar
\begin{equation*}
    \mathbf{r}=r\;\mathbf{\hat{r}}
\end{equation*}
Next, we determine the velocity as
\begin{equation*}
    \mathbf{\dot{r}}=\frac{d}{dt}r\;\mathbf{\hat{r}}=\dot{r}\;\mathbf{\hat{r}} +r\frac{d\mathbf{\hat{r}}}{dt}= \dot{r}\;\mathbf{\hat{r}} +r\dot{\phi}\;\boldsymbol{\hat{\phi}}
\end{equation*}
and acceleration as
\begin{align*}
    \ddot{r} & =\frac{d}{dt}\left(\dot{r}\;\mathbf{\hat{r}} +r\dot{\phi}\;\boldsymbol{\hat{\phi}}\right)                                                              \\
             & = \ddot{r}\;\mathbf{\hat{r}}+ \dot{r}\dot{\phi}\;\boldsymbol{\hat{\phi}} +r\frac{d}{dt}\left(\dot{\phi}\;\boldsymbol{\hat{\phi}}\right)                \\
             & =\ddot{r}\mathbf{\hat{r}}+ 2\dot{r}\dot{\phi}\;\boldsymbol{\hat{\phi}}+r\left( \ddot{\phi}\;\boldsymbol{\hat{\phi}}-\dot{\phi}\mathbf{\hat{ r}}\right) \\
    \ddot{r} & = \left(\ddot{r}-r\dot{\phi}^2\right) \; \mathbf{\hat{r}} + \left(r\ddot{\phi}+2\dot{r}\dot{\phi}\right) \;\boldsymbol{ \hat{\phi}}
\end{align*}
Finally
\begin{equation*}
    \mathbf{F}=F_r\;\mathbf{\hat{r}} + F_\phi\;\boldsymbol{ \hat{\phi}}
    \begin{cases}
        F_r    & =m\left(\ddot{r}-r\dot{\phi}^2\right)          \\
        F_\phi & =m\left(r\ddot{\phi}+2\dot{r}\dot{\phi}\right)
    \end{cases}
\end{equation*}

\subsubsection{Lagrangian derivation.}
We write the Lagrangian as
\begin{equation*}
    L = T - V = \frac{1}{2} m (\dot{r}^2 + r^2 \dot{\theta}^2) - V(r,\theta)
\end{equation*}
The radial equation reads
\begin{equation*}
    \frac{\partial L}{\partial r} =\frac{d}{dt} \frac{\partial L}{\partial \dot{r}}
\end{equation*}
which gives
\begin{align*}
    m \ddot{r}                      & =  m r \dot{\theta}^2 - \frac{\partial V}{\partial r} \\
    - \frac{\partial V}{\partial r} & =  m\left( \ddot{r}-r \dot{\theta}^2  \right)
\end{align*}
The angular equation reads
\begin{equation*}
    \frac{\partial L}{\partial \theta} =\frac{d}{dt} \frac{\partial L}{\partial \dot{\theta}}
\end{equation*}
which gives
\begin{align*}
    m \frac{d}{dt}(r^2 \dot{\theta})   & =- \frac{\partial V}{\partial \theta} \\
    m (r^2 \ddot{\theta} + 2 r \dot{r} \dot{\theta}) &= - \frac{\partial V}{\partial \theta}\\
    m (r \ddot{\theta} + 2\dot{r} \dot{\theta}) &= - \frac{1 }{r}\frac{\partial V}{\partial \theta}   
\end{align*}
Each of them are the radial and angular component of force.
\end{document}