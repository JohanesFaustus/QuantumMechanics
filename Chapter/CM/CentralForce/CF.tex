\documentclass[../../../main.tex]{subfiles}
\begin{document}
\subsection{Newton's Second Law in Polar Coordinate}
Acceleration in polar coordinate expressed as
\begin{equation*}
    \mathbf{\ddot{r}}=\left(\ddot{r}-r\dot{\phi}^2\right) \; \mathbf{\hat{r}} + \left(r\ddot{\phi}+2\dot{r}\dot{\phi}\right) \;\boldsymbol{ \hat{\phi}}
\end{equation*}
and velocity as
\begin{equation*}
    \mathbf{v}=\dot{r}\;\mathbf{\hat{r}} +r\dot{\phi}\;\boldsymbol{ \hat{\phi}}
\end{equation*}
Hence Newton's law transform into
\begin{equation*}
    \mathbf{F}=m\mathbf{a}=\begin{cases}
        F_r    & =m\left(\ddot{r}-r\dot{\phi}^2\right)          \\
        F_\phi & =m\left(r\ddot{\phi}+2\dot{r}\dot{\phi}\right)
    \end{cases}
\end{equation*}
if $F_\theta=0$, we have the definition of central force.
That is
\begin{equation*}
    \mathbf{F}=F_r\;\mathbf{\hat{r}}
\end{equation*}

\subsection{Derivation}
\begin{figure*}[h]
    \centering
    \normfigL{../Rss/CM/CentralForce/drValue.png}
    \normfigL{../Rss/CM/CentralForce/dphiValue.png}
    \caption*{The value of $d\mathbf{\hat{r}}$ and $d\boldsymbol{\hat{\phi}}$.}
\end{figure*}

\subsubsection{Newtonian derivation.}
From the figure, we have
\begin{equation*}
    d\mathbf{\hat{r}}=d\phi\;\boldsymbol{\hat{\phi}},\quad d\boldsymbol{\hat{\phi}}=-d\phi\;\mathbf{\hat{r}}
\end{equation*}
or equivalently
\begin{equation*}
    \frac{d\mathbf{\hat{r}}}{dt}=\dot{\phi}\;\boldsymbol{\hat{\phi}},\quad \frac{d\boldsymbol{\hat{\phi}}}{dt}=-\dot{\phi}\;\mathbf{\hat{r}}
\end{equation*}
Using these we can now proceed to derive the Newton's law in polar coordinate. In cartesian coordinate, position vector can be writen as
\begin{equation*}
    \mathbf{r}=x\;\mathbf{\hat{x}} + y\;\mathbf{\hat{y}}
\end{equation*}
converting it into polar
\begin{equation*}
    \mathbf{r}=r\;\mathbf{\hat{r}}
\end{equation*}
Next, we determine the velocity as
\begin{equation*}
    \mathbf{\dot{r}}=\frac{d}{dt}r\;\mathbf{\hat{r}}=\dot{r}\;\mathbf{\hat{r}} +r\frac{d\mathbf{\hat{r}}}{dt}= \dot{r}\;\mathbf{\hat{r}} +r\dot{\phi}\;\boldsymbol{\hat{\phi}}
\end{equation*}
and acceleration as
\begin{align*}
    \ddot{r} & =\frac{d}{dt}\left(\dot{r}\;\mathbf{\hat{r}} +r\dot{\phi}\;\boldsymbol{\hat{\phi}}\right)                                                              \\
             & = \ddot{r}\;\mathbf{\hat{r}}+ \dot{r}\dot{\phi}\;\boldsymbol{\hat{\phi}} +r\frac{d}{dt}\left(\dot{\phi}\;\boldsymbol{\hat{\phi}}\right)                \\
             & =\ddot{r}\mathbf{\hat{r}}+ 2\dot{r}\dot{\phi}\;\boldsymbol{\hat{\phi}}+r\left( \ddot{\phi}\;\boldsymbol{\hat{\phi}}-\dot{\phi}\mathbf{\hat{ r}}\right) \\
    \ddot{r} & = \left(\ddot{r}-r\dot{\phi}^2\right) \; \mathbf{\hat{r}} + \left(r\ddot{\phi}+2\dot{r}\dot{\phi}\right) \;\boldsymbol{ \hat{\phi}}
\end{align*}
Finally
\begin{equation*}
    \mathbf{F}=F_r\;\mathbf{\hat{r}} + F_\phi\;\boldsymbol{ \hat{\phi}}
    \begin{cases}
        F_r    & =m\left(\ddot{r}-r\dot{\phi}^2\right)          \\
        F_\phi & =m\left(r\ddot{\phi}+2\dot{r}\dot{\phi}\right)
    \end{cases}
\end{equation*}

\subsubsection{Lagrangian derivation.}
We write the Lagrangian as
\begin{equation*}
    \mathcal{L} = T - U = \frac{1}{2} m (\dot{r}^2 + r^2 \dot{\theta}^2) - U(r,\theta)
\end{equation*}
The radial equation reads
\begin{equation*}
    \frac{\partial \mathcal{L}}{\partial r} =\frac{d}{dt} \frac{\partial \mathcal{L}}{\partial \dot{r}}
\end{equation*}
which gives
\begin{align*}
    m r \dot{\theta}^2 - \frac{\partial U}{\partial r} & = m \ddot{r}                                  \\
    - \frac{\partial U}{\partial r}                    & =  m\left( \ddot{r}-r \dot{\theta}^2  \right)
\end{align*}
The angular equation reads
\begin{equation*}
    \frac{\partial \mathcal{L}}{\partial \theta} =\frac{d}{dt} \frac{\partial \mathcal{L}}{\partial \dot{\theta}}
\end{equation*}
which gives
\begin{align*}
    m \frac{d}{dt}(r^2 \dot{\theta})                 & =- \frac{\partial U}{\partial \theta}              \\
    m (r^2 \ddot{\theta} + 2 r \dot{r} \dot{\theta}) & = - \frac{\partial U}{\partial \theta}             \\
    m (r \ddot{\theta} + 2\dot{r} \dot{\theta})      & = - \frac{1 }{r}\frac{\partial U}{\partial \theta}
\end{align*}
Each of them are the radial and angular component of force.

\subsection{Orbit in a Central Force}
The general orbit differential equation is
\begin{equation*}
    u''(\theta)= u(\theta)-\frac{\mu }{\ell^2 u(\theta)^2}F
\end{equation*}
with $u=1/r$.
The solution as a variable of time can be obtained from
\begin{equation*}
    \mu \ddot{r}=-\frac{d }{dr }U_{eff}(r)
\end{equation*}
Few important parameters introduced in the derivation are as follows
\begin{flalign*}
     & \begin{aligned}
            & F_{cf}=\frac{\ell^2 }{\mu r^3}=-\frac{d }{dr }U_{cf}          \\
            & \mathbf{R }=\frac{m_1 \mathbf{r}_1+m_2 \mathbf{r}_2}{m_1+m_2} \\
            & \mu= \frac{m_1m_2}{m_1+m_2}
           U_{eff}=U(r)+U_{cf}=U(r)+\frac{\ell^2 }{2\mu r^2}
       \end{aligned} &
\end{flalign*}

\subsubsection{Derivation.}
In the case of two astronomical bodies and Coulomb potential of Hydrogen atom, the potential may be expressed as
\begin{equation*}
    U_G(\mathbf{r })=-\frac{Gm_1m_2}{\mathbf{r}},\qquad
    U_C(\mathbf{r})=-\frac{ke^2 }{\mathbf{r}}
\end{equation*}
with $r=\mathbf{r}_1-\mathbf{r}_2$ as the relative potential.
The result is that the potential energy $U$ depends only on the magnitude $r$ of the relative position $\mathbf{r}$
\begin{equation*}
    U=U(r)
\end{equation*}
The Lagrangian then
\begin{equation*}
    \mathcal{L }=\frac{1 }{2 }m_1 \mathbf{r }_1^2+\frac{1 }{2 }m_2 \mathbf{r }_2^2+U(r)
\end{equation*}

The best choice of the generalized coordinate is the center of mass $\mathbf{R}$, which defined as
\begin{equation*}
    \mathbf{R }=\frac{m_1 \mathbf{r}_1+m_2 \mathbf{r}_2}{m_1+m_2}
\end{equation*}
With this, we can express each mass coordinate $\mathbf{r}_i$ in terms of relative coordinate and CM
\begin{equation*}
    \mathbf{r}_1=\mathbf{R}+\frac{m_2}{M}\mathbf{r}
    \qquad
    \mathbf{r}_2=\mathbf{R}-\frac{m_1}{M}\mathbf{r}
\end{equation*}
with $M=m_1+m_2$.
Thus the kinetic energy is
\begin{equation*}
    T=\frac{1 }{2 }\left( m_1 \dot{\mathbf{r}}_1^2+m_2 \dot{\mathbf{r}}_2^2 \right) =\frac{1 }{2}\left( M \mathbf{\dot{R}}^2+\frac{m_1m_2}{M}\dot{\mathbf{r}}^2 \right)
\end{equation*}
or by using reduced mass $\mu=m_1m_2/m_1+m_2$
\begin{equation*}
    T=\frac{1 }{2}\left( M \mathbf{\dot{R}}^2+\mu\dot{\mathbf{r}}^2 \right)
\end{equation*}
The corresponding result for the Lagrangian
\begin{equation*}
    \mathcal{L}=\frac{1 }{2}M \dot{\mathbf{R}}^2+\left( \frac{1 }{2}\mu \dot{\mathbf{r}}^2-U(r) \right)=\mathcal{L}_{CM}+\mathcal{L}_{rel}
\end{equation*}
We see that by using the CM and relative positions as our generalized coordinates, we have split the Lagrangian into two separate pieces, one of which involves only the CM coordinate and the other only the relative coordinate.

The CM coordinate is ignoble, so
\begin{equation*}
    \frac{\partial \mathcal{L }}{\partial \mathbf{R}}=\frac{d }{dt }\frac{\partial \mathcal{L }}{\partial \dot{\mathbf{R}}}
    \implies
    M \ddot{\mathbf{R}}=0
\end{equation*}
The consequence of conservation of momentum in other words.
The Lagrange equation for the relative coordinate is
\begin{equation*}
    \frac{\partial \mathcal{L }}{\partial \mathbf{r}}=\frac{d }{dt }\frac{\partial \mathcal{L }}{\partial \dot{\mathbf{r}}}
    \implies
    \mu \ddot{\mathbf{r}}=- \nabla U(\mathbf{r})
\end{equation*}

Now we move the CM reference frame.
In this frame, $\mathbf{\dot{R}} = 0$ and the CM part of the Lagrangian is zero $\mathcal{L}_{CM} = 0$.
Thus in the CM frame
\begin{equation*}
    \mathcal{L}=\frac{1 }{2 }\mu \mathbf{\dot{r}}^2-U(r)
\end{equation*}
In other words, in the CM frame, the two-body problem with central conservative forces is reduced to a two-dimensional problem.

To set up the equations of motion for the remaining two-dimensional problem, we  choose the polar coordinate
\begin{equation*}
    \mathcal{L}=\frac{1 }{2 }\mu \left( \dot{r}^2 +r^2\dot{\theta}\right) -U(r)
\end{equation*}
The radial equation reads
\begin{equation*}
    \frac{\partial \mathcal{L}}{\partial r} =\frac{d}{dt} \frac{\partial \mathcal{L}}{\partial \dot{r}}
\end{equation*}
which gives
\begin{align*}
    \mu r \dot{\theta}^2 - \frac{\partial U}{\partial r} & = \mu \ddot{r}
\end{align*}
The angular equation reads
\begin{equation*}
    \frac{\partial \mathcal{L}}{\partial \theta} =\frac{d}{dt} \frac{\partial \mathcal{L}}{\partial \dot{\theta}}
\end{equation*}
which gives the conservation of angular momentum
\begin{align*}
    \frac{d}{dt}(\mu r^2 \dot{\theta}) & =0    \\
    \mu r^2 \dot{\theta}^2             & =\ell
\end{align*}

The radial equation can be rewritten as
\begin{equation*}
    \mu \ddot{r}=-\frac{dU }{dr }+F_{cf}
\end{equation*}
which has the form of Newton's second law for a particle subject to the actual force $-dU /dr$ plus a "fictitious" outward centrifugal force
\begin{equation*}
    F_{cf}=\mu r \dot{\phi}^2=\frac{\ell^2 }{\mu r^3}
\end{equation*}
where we have used the $\phi$ equation to solve for $\dot{\phi}={\ell }/{\mu r^2}$.
We can now express the centrifugal force in terms of a centrifugal potential energy
\begin{equation*}
    F_{cf}=-\frac{d }{dr }\left( \frac{\ell^2 }{2\mu r^2} \right)
    \implies
    U_{cf}=\frac{\ell^2 }{2 \mu r^2}
\end{equation*}
Defining effective potential
\begin{equation*}
    U_{eff}=U(r)+U_{cf}=U(r)+\frac{\ell^2 }{2\mu r^2}
\end{equation*}
where $F(r)$ is the actual central force, $F = -dU/dr$.
We can write the radial equation as
\begin{equation*}
    \mu \ddot{r}=-\frac{d }{dr }U_{eff}(r)
\end{equation*}
This equation determine $r$ as a function of $t$.
It also implies that the radial motion of the particle is exactly the same as if the particle were moving in one dimension with an effective potential energy $U_{eff}$.

The function $r = r (\theta)$ will tell us the shape of the orbit more directly.
This can be achieved by writing the radial equation in terms of forces
\begin{equation*}
    \mu \ddot{r}=F(r)+\frac{\ell^2 }{\mu r^3}
\end{equation*}
Then make the substitution $u=1/r$, so that we can use the chain rule
\begin{equation*}
    \frac{d }{dt}=
    \frac{d }{d\theta}\frac{d\theta }{dt}=
    \frac{\ell}{\mu r^2}\frac{d }{dt}=
    \frac{\ell u^2}{\mu}\frac{d   }{d\theta}
\end{equation*}
We use this to rewrite the radial equation.
The first derivative
\begin{equation*}
    \frac{dr }{dt}=
    \frac{\ell u^2}{\mu}\frac{d   }{d\theta}\frac{1 }{u}=
    \frac{\ell u^2}{\mu}\frac{d   }{du}\left(\frac{1 }{u}\right)\frac{du }{d\theta}=
    -\frac{\ell }{\mu}\frac{du }{d\theta}
\end{equation*}
and the second
\begin{equation*}
    \frac{d \dot{r}}{dt}=
    \frac{\ell u^2}{\mu}\frac{d   }{d\theta}\left(     -\frac{\ell }{\mu}\frac{du }{d\theta} \right)=
    -\frac{\ell^2u^2 }{\mu^2}\frac{d^2u }{d\theta^2}
\end{equation*}
Substituting back
\begin{align*}
    -\frac{\ell^2u^2 }{\mu}\frac{d^2u }{d\theta^2} & = F+ \frac{\ell^2u^3 }{\mu}                   \\
    u''(\theta)                                    & =  u(\theta)-\frac{\mu }{\ell^2 u(\theta)^2}F
\end{align*}

\subsection{Kepler Orbit}
\subsubsection{Orbit trajectory.}1
The Kepler orbit refers to the problem of finding the possible orbits of a comet or any other object subject to an inverse-square force
\begin{equation*}
    F(r)=-\frac{\gamma }{r^2}=-\gamma u^2
\end{equation*}
Inserting into the orbit equation
\begin{equation*}
    u''(\theta)=-u(\theta)+\frac{\gamma \mu}{\ell^2}
\end{equation*}
Substitute $w=u-\gamma\mu/\ell^2$
\begin{equation*}
    w''(\theta)=-w(\theta)
\end{equation*}
which has the general solution
\begin{equation*}
    w(\theta)=A \cos (\theta-\delta)
\end{equation*}
Taking $\delta=0$, the general solution in of $u(\theta)$ is 
\begin{equation*}
    u(\theta)=\frac{\gamma\mu }{\ell^2 }+A\cos \theta=\frac{\gamma\mu }{\ell^2}\left( 1+\epsilon \cos \theta \right) 
\end{equation*}
or in terms of $r(\theta)$
\begin{equation*}
    r(\theta)=\frac{\ell^2 /\gamma\mu }{1+\epsilon \cos \theta}=\frac{c }{1+\epsilon \cos \theta}
\end{equation*}

\subsubsection{Bounded orbit.}
\begin{figure*}
    \centering
    \normfigL{../Rss/CM/CentralForce/KeplerOrbit.png}
    \caption*{Figure: The bounded orbits of a comet or planet are ellipses}
\end{figure*}
The value eccentricity $\epsilon = 1$ is the boundary between the bounded and unbounded orbits. 
In the case that the constant $\epsilon$ is less than 1, the denominator of $r(\theta)$  between the values $1 \pm \epsilon$. 
Therefore, $r(\theta)$ oscillates between
\begin{equation*}
    r_{min}=\frac{c}{1+\epsilon},\qquad
    r_{max}=\frac{c}{1-\epsilon}
\end{equation*}
where $r(\theta=0)=r_{min}$ as the perihelion and $r(\theta=\pi)=r_{max}$ as the aphelion.

The orbit trajectory $r(\theta)$ can be transformed into cartesian as 
\begin{equation*}
    \frac{(x+d)^2}{a^2  }+\frac{y^2 }{b^2 }=1
\end{equation*}
the semi–major axis $a$, semi–minor axis $b$, semi–latus rectum $c$, and the focal offset $d$ are related by
\begin{equation*}
    a=\frac{c }{1-\epsilon^2 }\qquad
    b=\frac{c }{\sqrt{1-\epsilon^2}}\qquad
    d=a\epsilon
\end{equation*}
Accordingly, the ratio of the major to minor axes is
\begin{equation*}
    \frac{b }{a }\sqrt{1-\epsilon^2}
\end{equation*}
\end{document}