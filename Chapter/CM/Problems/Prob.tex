\documentclass[../../../main.tex]{subfiles}
\begin{document}
\subsection*{Incline Problem}
Consider:
\begin{quotation}
    A block of mass $m$ is observed accelerating from rest down an incline that has coefficient of friction $\mu$ and is at angle $\theta$ from the horizontal. How far will it travel in time $t$?
\end{quotation}
\begin{figure*}[h]
    \centering
    \dfig{../Rss/CM/Problem/Incline}
    \dfig{../Rss/CM/Problem/Incline-1}
\end{figure*}

First we define the direction of displacement as positive $x$ axis and the normal force as positive $y$ axis. The resultant force in $y$ axis written as 
\begin{equation*}
    \sum F_y=N-mg\cos\theta=0 \implies N=mg\cos \theta
\end{equation*}
and $x$ axis
\begin{equation*}
    \sum F_x=mg\sin\theta -f= mg\sin\theta -\mu mg\cos \theta=m\ddot{x}
\end{equation*}
we then solve for $x$ by 
\begin{align*}
    \ddot{x}&=g\left(\sin\theta -\mu \cos \theta\right)\\
    \dot{x}&=g\left(\sin\theta -\mu \cos \theta\right)t\\
    x(t)&=\frac{1}{2}g\left(\sin\theta -\mu \cos \theta\right)t^2
\end{align*}
Since the block stared from rest, its constant of integration is zero.

\subsection*{Central Force Problem}
Consider also:
\begin{quotation}
    A "half-pipe" at a skateboard park consists of a concrete trough with a semicircular cross section of radius $R = 5 $m. I hold a frictionless skateboard on the side of the trough pointing down toward the bottom and release it. Find the equation of motion for this system.
\end{quotation}

\begin{figure*}[h]
    \centering
    \normfig{../Rss/CM/Problem/CF.png}
\end{figure*}

In this case $r$ is held constant, thus the expression for resultant force in polar coordinate reads
\begin{equation*}
    \mathbf{F}=-m\dot{\phi}^2R\;\mathbf{\hat{r}}+ mR\ddot{\phi} \;\boldsymbol{ \hat{\phi}}
\end{equation*}
We also know that the acting force in this system are the normal and the skateboard weigh. Applying this force into equation above
\begin{equation*}
    \left(mg\cos\phi-N\right) \;\mathbf{\hat{r}}-mg\sin\phi\; \boldsymbol{ \hat{\phi}}=-m\dot{\phi}^2R\;\mathbf{\hat{r}}+ mR\ddot{\phi} \;\boldsymbol{ \hat{\phi}}
\end{equation*}
We can't do anything with the radial component, we only use the angular component
\begin{align*}
    mR\ddot{\phi} = mg\sin\phi \\
    \ddot{\phi} = {\frac{g}{R}} \sin\phi
\end{align*}
This differential equation is solved by 
\begin{equation*}
    \phi(t)=A\sin \sqrt{\frac{g}{R}}t +B\cos \sqrt{\frac{g}{R}}t
\end{equation*}
Since this is released from rest, we have the initial condition of $\phi(0)=\phi_0$ and $\dot{\phi}(0)=0$. Apllying the first condition
\begin{equation*}
    \phi_0=B
\end{equation*}
and the second
\begin{align*}
    \dot{\phi}(t)&=A\sqrt{\frac{g}{R}} \cos \sqrt{\frac{g}{R}}t -\phi_0\sqrt{\frac{g}{R}}\sin \sqrt{\frac{g}{R}}t\\
    \dot{\phi}(0)&=0=A\sqrt{\frac{g}{R}}
\end{align*}
Hence the equation of motion reads
\begin{equation*}
    \phi(t)=\phi_0\cos \sqrt{\frac{g}{R}}t
\end{equation*}
\end{document}