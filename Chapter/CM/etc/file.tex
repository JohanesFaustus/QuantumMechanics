\documentclass[../../../main.tex]{subfiles}
\begin{document}
\subsection*{Newton's Law}
\subsubsection*{First law.} In the absence of an external force, when viewed from an inertial frame, an object at rest remains at rest and an object in uniform motion in a straight line maintains that motion.

\subsubsection*{Second law.} Simply put 
\begin{equation*}
    \mathbf{F}=\frac{d\mathbf{p}}{dt}
\end{equation*}

\subsubsection*{Third law.} States that if two objects interact, the force exerted by object 1 on object 2 is equal in magnitude and opposite in direction to the force exerted by object 2 on object 1.

\subsection*{Particle Under Constant Acceleration}
Here's some kinematics equation for position
\begin{align*}
    x(t)&=x_i+\frac{1}{2}(v_i+v_f)t\\
    x(t)&=x_i+v_it+\frac{1}{2}at^2
\end{align*}
and for velocity
\begin{align*}
    v(t)&=v_i+at\\
    v(t)^2&=v_i^2+2a(x_f-x_i)
\end{align*}

\subsection*{Particle in Uniform Circular Motion}
If a particle moves in a circular path of radius $r$ with a constant speed $v$, the magnitude of its centripetal acceleration is given by
\begin{equation*}
    a_r=\frac{v^2}{r}
\end{equation*}
while its period and angular velocity is 
\begin{equation*}
    T=\frac{2\pi r}{v},\quad \omega=\frac{2\pi}{T}
\end{equation*}
Applying Newton's second law 
\begin{equation*}
    \sum F=ma_r=m\frac{v^2}{r}
\end{equation*}

\subsection*{Rigid Object Under Constant Angular Acceleration} Analogous to those for translational motion of a particle under constant acceleration
\begin{align*}
    \omega(t)&=\omega_i+\alpha t\\
    \omega(t)^2&=\omega_i^2+2\alpha(\theta_t-\theta_i)\\
    \theta(t)&=\theta_i +\omega t+\frac{1}{2}\alpha t^2\\
    \theta(t)&=\theta_i+\frac{1}{2}(\omega_i+\omega_f)t
\end{align*}

\subsection*{Relation of Linear and Rotational Motion}
The following equations show the relation of linear and rotational motion
\begin{equation*}
    s=r\theta,\quad v=r\omega,\quad a_t=r\alpha
\end{equation*}

\subsection*{Torque}
The torque associated with a force $\mathbf{F}$ acting on an object
\begin{equation*}
    \boldsymbol{\tau}=\mathbf{r}\times\mathbf{F}=I \mathbf{\alpha} =\frac{d\mathbf{L}}{dt}
\end{equation*}

\subsection*{Moment of Inertia}
The moment of inertia of a rigid object is 
\begin{equation*}
    I=\sum mr^2=\int r^2 dm
\end{equation*}

\subsubsection*{Parallel Axis Theorem.} To calculate the moment inertia from any axis, we use parallel axis theorem 
\begin{equation*}
    I=I_\text{CM}+Md^2
\end{equation*}

\subsection*{Terminal velocity}
\subsubsection*{$\boldsymbol{r\propto v}$.} The velocity as a function of time is 
\begin{equation*}
    v=\frac{mg}{b}\left[1-\exp\left(-\frac{bt}{m}\right)\right]=v_T\left[1-\exp\left(-\frac{bt}{m}\right)\right]
\end{equation*}
where $b$ is a resistive constant whose value depends on the properties of the medium.
\subsubsection*{$\boldsymbol{r\propto v^2}$.} Given by 
\begin{equation*}
    v_T=\sqrt{\frac{2mg}{D\rho A}}
\end{equation*}
where $D$ is a dimensionless empirical quantity called the drag coefficient, $\rho$ is the density of air, and $A$ is the cross-sectional area of the moving object.

\subsubsection*{Escape velocity.} The speed required by an object to escape from any planet orbit is 
\begin{equation*}
    v_\text{esc}=\sqrt{\frac{2GM}{R}}
\end{equation*}

\subsection*{Work Energy Theorem}
It states that if work is done on a system by external forces and
the only change in the system is in its speed,
\begin{equation*}
    W=\Delta T
\end{equation*}

\subsection*{Kinetic Energy}
For an object in linear motion, the kinetic energy of said object is 
\begin{equation*}
    T=\frac{1}{2}mv^2
\end{equation*}
whereas for rotational motion
\begin{equation*}
    T=\frac{1}{2}I\omega^2
\end{equation*}
Hence the total kinetic energy of a rigid object rolling on a rough surface without slipping
\begin{equation*}
    T=\frac{1}{2}mv_\text{CM}^2+\frac{1}{2}I\omega_\text{CM}^2
\end{equation*}

\subsection*{Potential Energy Function}
For conservative energy $\mathbf{F}$, applies
\begin{equation*}
    V_f-V_i=-\int_{\mathbf{r_i}}^{\mathbf{r_f}}\mathbf{F}\cdot d\mathbf{r}
\end{equation*}

For particle-Earth system, the gravitational potential energy is
\begin{equation*}
    V=mgy
\end{equation*}
and elastic potential stored in spring
\begin{equation*}
    V=\frac{1}{2}kx^2
\end{equation*}

\subsection*{Effective potential}
Effective potential energy $U_\text{eff} (r)$ is the sum of the actual potential energy $U(r)$ and the centrifugal $U_\text{cf} (r)$:
\begin{equation*}
    U_\text{eff} (r)=U(r)+\frac{l^2}{2\mu r^2}
\end{equation*}
where $l$ is the angular momentum and $\mu$ is the reduced mass
\begin{equation*}
    \mu=\frac{m_1m_2}{m_1+m_2}
\end{equation*}

\subsection*{Momentum Impulse}
The linear momentum and impulse are defined as 
\begin{equation*}
    \mathbf{p}=m\mathbf{v},\quad \mathbf{I}=\int_{t_i}^{t_f}\sum \mathbf{F}\;dt
\end{equation*}

\subsubsection*{Angular Momentum} 
The angular momentum about an axis through the origin of a particle having linear momentum
\begin{equation*}
    \mathbf{L}=\mathbf{r}\times \mathbf{p}
\end{equation*}
The $z$ component of angular momentum of a rigid object rotating about a fixed $z$ axis is
\begin{equation*}
    L_z=I\omega
\end{equation*}


\subsection*{Center of Mass and Velocity}
The position vector of the center of mass of a system of particles is defined as
\begin{equation*}
    \mathbf{r}_\text{CM}=\frac{1}{M}\sum m\mathbf{r}=\frac{1}{M}\int \mathbf{r}\;dm
\end{equation*}
where $M$ is the total mass. The velocity of the center of mass for a system of particles is
\begin{equation*}
    \mathbf{v}_\text{CM}=\frac{1}{M}\sum m\mathbf{v}=
\end{equation*}

\subsection*{Collision}
\subsubsection*{Inelastic collision.} One for which the total kinetic energy of the system of colliding particles is not conserved.

\subsubsection*{Elastic collision.} One in which the kinetic energy of the
system is conserved.

\subsubsection*{Perfectly inelastic.} A collision which the colliding particles stick together after the collision.

\subsubsection*{Rocket Propulsion} The expression for rocket propulsion
is 
\begin{equation*}
    v_f-v_i=v_e\ln\frac{M_i}{M_f}
\end{equation*}

\subsection*{Power}
The rate at which work is done by an external force, called power, is
\begin{equation*}
    P=\frac{dE}{dt}=Fv=\tau\omega
\end{equation*}

\subsection*{Newton's Law on Gravity}
\begin{equation*}
    \mathbf{F}=G\frac{m_1m_2}{r^2}\mathbf{\hat{r}}
\end{equation*}
For an object at a distance $h$ above the Earth's, the gravitational acceleration is 
\begin{equation*}
    g=\frac{GM_E}{r^2}=\frac{GM_E}{(R_E+h)^2}
\end{equation*}
In general, the gravitational field experienced by mass $m$ is 
\begin{equation*}
    \mathbf{g}=\frac{\mathbf{F}}{m}
\end{equation*}


\subsection*{Kepler's Law}
\subsubsection*{First Law.} All planets move in elliptical orbits with the Sun at one focus.
\subsubsection*{Second Law} The radius vector drawn from the Sun to a planet sweeps out equal areas in equal time intervals.
\subsubsection*{Third Law} Simply put 
\begin{equation*}
    T^2=\frac{4\pi^2a^3}{GM_S}
\end{equation*}
where $a$ is semimajor axis and $M_S$ is the mass of the sun. 

\subsection*{Energy of Gravitational system}
\subsubsection*{Potential energy.} The gravitational potential energy associated with a system of two particles is
\begin{equation*}
    V=-\frac{Gm_1m_2}{r}
\end{equation*}
\subsubsection*{Total energy.} The total energy of the system is the sum
of the kinetic and potential energies
\begin{equation*}
    E=\frac{1}{2}mv^2-G\frac{Mm}{r}=-\frac{GMm}{2r}
\end{equation*}
\end{document}