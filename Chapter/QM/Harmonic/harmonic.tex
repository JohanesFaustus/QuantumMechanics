\documentclass[../../../main.tex]{subfiles}
\begin{document}
\subsection{1D Harmonic Potential}
Particle under harmonic potential is subjected to the potential in the following form
\begin{equation*}
    V(x) = \frac{1}{2} m \omega^{2} x^{2},
\end{equation*}
In the $X$ basis, the Schrödinger equation now reads
\begin{equation*}
    -\frac{\hbar^{2}}{2m} \frac{d^{2}\psi(x)}{dx^{2}} + \frac{1}{2}m\omega^{2}x^{2}\psi(x) = E\psi(x).
\end{equation*}
Solving this yields discrete energy eigenvalues
\[
    E_{n} = \hbar\omega \left(n + \frac{1}{2}\right), \quad n = 0, 1, 2, \ldots
\]
and normalized eigenfunctions
\[
    \psi_{n}(x) = \left(\frac{m\omega}{\pi\hbar}\right)^{1/4} \frac{1}{\sqrt{2^{n} n!}} H_{n}\left(\sqrt{\frac{m\omega}{\hbar}} x\right) e^{-\frac{m\omega x^{2}}{2\hbar}}
\]
where \(H_{n}\) are Hermite polynomials.
Here are the first few.
\begin{align*}
    H_{0}(x) & = 1,                     \\
    H_{1}(x) & = 2x,                    \\
    H_{2}(x) & = 4x^{2} - 2,            \\
    H_{3}(x) & = 8x^{3} - 12x,          \\
    H_{4}(x) & = 16x^{4} - 48x^{2} + 12
\end{align*}

\subsection{Derivation}
From the Schrödinger equation
\begin{equation*}
    \frac{d^{2}\psi(x)}{dx^{2}} + \frac{2m }{\hbar^2 }\left(E-\omega^{2}x^{2}\right)\psi(x)
\end{equation*}
we make the substitution of $x=by$, which result in differential
\begin{align*}
    \frac{d\psi }{dx}       & =\frac{d\psi }{dy}\frac{dy }{dx}=\frac{1 }{b }\frac{d\psi }{dy}                                                                       \\
    \frac{d^2 \psi }{dx^2 } & = \frac{d }{dx }\left( \frac{1 }{b }\frac{d\psi }{dy} \right) =\frac{1 }{b}\frac{d^2\psi }{dy\;dx}=\frac{1 }{b }\frac{d^2\psi }{dy^2}
\end{align*}
Now Schrödinger equation reads
\begin{equation*}
    \frac{1 }{b^2 }\frac{d^2\psi }{dy^2 }+\frac{2m }{\hbar^2 }\left( E-\frac{1 }{2 }m\omega^2b^2y^2   \right) \psi=0
\end{equation*}
or
\begin{equation*}
    \frac{d^2\psi }{dy^2 }+\left( \frac{2mEb^2 }{\hbar^2 }-\frac{m^2\omega^2b^4y^2 }{\hbar^2}   \right) \psi=0
\end{equation*}
Now define
\begin{equation*}
    \epsilon=\frac{mEb^2 }{\hbar^2}=\frac{E }{\hbar\omega},\qquad b^2=\frac{\hbar }{m\omega}
\end{equation*}
Then
\begin{equation*}
    \frac{d^2 \psi}{dy^2}+(2\epsilon-y^2)\psi=0
\end{equation*}
Compare this to the Hermite equation
\begin{equation*}
    y_n''+(2n+1-x^2)y_n=0
\end{equation*}
which has the solution called the Hermite function
\begin{equation*}
    y_n=e^{x^2/2}\left(\frac{d}{dx}\right)^ne^{-x^2}=e^{-x^2/2}H_n(x)
\end{equation*}
On comparing $2\epsilon$ with $2n+1$, we obtain the expression for energy.
Then the solution to the Schrödinger equation is
\begin{equation*}
    \psi(y)=e^{-y^2/2}H_n(y)\qquad\text{or}\qquad\psi(x)=\exp \left[ -\frac{m\omega }{2 \hbar}x^2 \right] H_n \left[ \left( \frac{m \omega }{\hbar }^{1/2}x \right)  \right]
\end{equation*}

All that left is normalization.
To do so consider the orthogonality of Hermite polynomials
\begin{equation*}
    \int_{-\infty}^{\infty}e^{-x^2}H_n(x)H_m(x)\;dx=\sqrt{\pi}2^n n!\delta_{nm}
\end{equation*}
Therefore
\begin{align*}
    \braket{\psi|\psi} & = \int A^2\exp \left[ -\frac{m\omega }{ \hbar}x^2 \right] H_n \left\{\left[ \left( \frac{m \omega }{\hbar }^{1/2}x \right)  \right] \right\}^2\;dx                                                                                 \\
                       & = A^2 \sqrt{\frac{\hbar }{m\omega}}  \int A^2\exp \left[ -\frac{m\omega }{ \hbar}x^2 \right] H_n \left\{\left[ \left( \frac{m \omega }{\hbar }^{1/2}x \right)  \right] \right\}^2\;d \left( \sqrt{\frac{m\omega }{\hbar}x} \right) \\
    1                  & = A^2  \sqrt{\frac{\hbar }{m\omega }}(\pi)^{1/2}2^nn!                                                                                                                                                                              \\
    A                  & = \left( \frac{m\omega }{\pi \hbar 2^{2n}(n!)^2 } \right) ^{1/4}
\end{align*}
Inserting this into the wavefunction yields the normalized wavefunction.

\subsection{Ladder Operator}
The ladder operator are operators that raise or lower the eigenvalue of another operator, here it is the Hamiltonian.
They are defined as
\begin{align*}
    a           & =  \left( \frac{m \omega}{2 \hbar} \right)^{1/2} X + i \left( \frac{1}{2 m \omega \hbar} \right)^{1/2} P \\
    a^{\dagger} & =  \left( \frac{m \omega}{2 \hbar} \right)^{1/2} X - i \left( \frac{1}{2 m \omega \hbar} \right)^{1/2} P
\end{align*}
which satisfy $[a,a^\dagger]=1$ and referred as lowering $a$  and raising $a^\dagger$  operators because of how they modify the quantum state’s excitation number.
They are also called destruction and creation operators since they destroy or create quanta of energy $\hbar \omega$.
We can also write the position and momentum in terms of these operators
\begin{gather*}
    X = \left( \frac{\hbar}{2m\omega} \right)^{1/2} (a + a^{\dagger}) \\
    P = i \left( \frac{m \omega \hbar}{2} \right)^{1/2} (a^{\dagger} - a)
\end{gather*}

From the physical Hamiltonian
\begin{equation*}
    H=\frac{P^2}{2m }+\frac{1 }{2 }m\omega^2 X^2=\left( a a ^\dagger+\frac{1 }{2 } \right)\hbar \omega
\end{equation*}
We define an operator
\begin{equation*}
    \hat{H }=\frac{H }{m\omega }=\left( a a ^\dagger+\frac{1 }{2 } \right)
\end{equation*}
whose eigenvalues $\epsilon$ measure energy in units of $\hbar \omega$.
We have then relation
\begin{equation*}
    [a,\hat{H } ]=a,\qquad [a ^\dagger,\hat{H }]=-a ^\dagger
\end{equation*}
From this can be obtained the action of ladder operator on the eigenstate $\ket{n}$
\begin{equation*}
    a |n\rangle = n^{1/2} |n-1\rangle\qquad a^{\dagger} |n\rangle = (n+1)^{1/2} |n+1\rangle
\end{equation*}

The following are matrix components of some operator mentioned
\begin{gather*}
    a ^\dagger\leftrightarrow\begin{bmatrix}
        0 & 0 & 0 & \cdots \\ 1^{1/2} & 0 & 0 & \\ 0 & 2^{1/2} & 0 & \\ 0 & 0 & 3^{1/2} & \\ \vdots & & &
    \end{bmatrix}                                                                                                                                         \\
    a \leftrightarrow\begin{bmatrix}
        0 & 1^{1/2} & 0 & 0 & \cdots \\ 0 & 0 & 2^{1/2} & 0 & \\ 0 & 0 & 0 & 3^{1/2} & \\ \vdots & & & &
    \end{bmatrix}                                                                                                                                          \\
    X \leftrightarrow \left( \frac{\hbar}{2m\omega} \right)^{1/2} \begin{bmatrix} 0 & 1^{1/2} & 0 & 0 & ... \\ 1^{1/2} & 0 & 2^{1/2} & 0 & \\ 0 & 2^{1/2} & 0 & 3^{1/2} & \\ 0 & 0 & 3^{1/2} & 0 & \\ \vdots & & & & \end{bmatrix}           \\
    P \leftrightarrow i \left( \frac{m \omega \hbar}{2} \right)^{1/2} \begin{bmatrix} 0 & -1^{1/2} & 0 & 0 & \cdots \\ 1^{1/2} & 0 & -2^{1/2} & 0 & \\ 0 & 2^{1/2} & 0 & -3^{1/2} & \\ 0 & 0 & 3^{1/2} & 0 & \\ \vdots & & & & \end{bmatrix} \\
    H \leftrightarrow \hbar \omega \begin{bmatrix} \frac{1}{2} & 0 & 0 & 0 & \cdots \\ 0 & \frac{3}{2} & 0 & 0 & \\ 0 & 0 & \frac{5}{2} & & \\ \vdots & & & & \end{bmatrix}
\end{gather*}

\subsection{3D Harmonic Potential}
The generalized form of 3D harmonic potential is
\begin{equation*}
    V(x, y, z) = \frac{1}{2}m(\omega_{x}^{2}x^{2} + \omega_{y}^{2}y^{2} + \omega_{z}^{2}z^{2}),
\end{equation*}
Each coordinate direction oscillates independently with its own frequency, and the system loses spherical symmetry.
In the case of isotropic potential
\begin{equation*}
    V(r) = \frac{1}{2}m\omega^{2}r^{2}
\end{equation*}
This symmetry produces degenerate energy levels because the potential is identical in every direction.
Unlike isotropic potential, anisotropic potential cannot be expressed in a simple form in spherical coordinates because it lacks spherical symmetry.

\subsubsection{Cartesian coordinates.}
In Cartesian coordinates the 3D eigenfunctions are products of 1D harmonic-oscillator eigenfunctions
\begin{equation*}
    \Psi_n(x, y, z) = \psi_{n_x}(x) \psi_{n_y}(y) \psi_{n_z}(z)
\end{equation*}
As such
\begin{equation*}
    \Psi_n(x, y, z) = \left( \frac{\alpha^2}{\pi } \right)^{3/4} \frac{H(\xi_x) H(\xi_y) H(\xi_z)}{\sqrt{2^{n}n_x!  n_y! n_z!}}  \exp \left[ -\frac{1}{2} (\xi_x^2 + \xi_y^2 + \xi_z^2) \right]
\end{equation*}
with $n=n_x+n_y+n_z$, $\alpha = \sqrt{m\omega/\hbar }$ and $ \xi_i = \alpha i$.
The corresponding energy, assuming isotropic, is
\begin{equation*}
    E_{n_{x},n_{y},n_{z}} = \hbar \omega \left(n_{x} + n_{y} + n_{z} + \frac{3}{2}  \right)
\end{equation*}
The degeneracy of the shell with total quantum number $n=n_x+n_y+n_z$ is
\begin{equation*}
    g_n=\frac{(n+1)(n+2 )}{2}
\end{equation*}

In the case of anisotropic harmonic potential, the wavefunction remains a product of 1D harmonic oscillator with differing value of $\omega$
\begin{equation*}
    \Psi_n(x, y, z) = \psi_{n_x}(x; \omega_x) \psi_{n_y}(y; \omega_y) \psi_{n_z}(z; \omega_z)
\end{equation*}
As such
\begin{equation*}
    \Psi_n(x, y, z) = \left( \frac{\alpha_x \alpha_y \alpha_z}{\pi^{3/2}} \right)^{1/2} \frac{H(\xi_x) H(\xi_y) H(\xi_z)}{\sqrt{2^{n}n_x!  n_y! n_z!}}  \exp \left[ -\frac{1}{2} (\xi_x^2 + \xi_y^2 + \xi_z^2) \right]
\end{equation*}
with $n=n_x+n_y+n_z$, $\alpha_i = \sqrt{m\omega_i/\hbar }$ and $ \xi_i = \alpha i$.
Whereas the energy is
\begin{equation*}
    E_n = \hbar \left[  \omega_x \left(n_x + \frac{1}{2}  \right) + \omega_y \left(n_y + \frac{1}{2}  \right) + \omega_z \left(n_z + \frac{1}{2}  \right)  \right]
\end{equation*}

\subsubsection{Spherical coordinates.}
In spherical coordinates they are products of a radial function built from generalized Laguerre polynomials and spherical harmonics
\begin{equation*}
    \Psi_{n_{r},l,m}(r, \theta, \phi) = R_{n_{r},l}(r) Y_{l}^{m}(\theta, \phi)
\end{equation*}
With the radial equation
\begin{equation*}
    R_{n_{r},l}(r) = N_{n_{r},l} r^{l} e^{-\frac{1}{2}\alpha r^{2}} L_{n_{r}}^{(l + \frac{1}{2})}(\alpha r^{2}),
\end{equation*}
where $\beta=m\omega/\hbar$$ L_n^{(a)}$ is the generalized Laguerre polynomial and the normalization constant
\begin{equation*}
    N_{n_r,l} = \sqrt{\frac{2 \alpha^{l+\frac{3}{2}} n_r!}{\Gamma(n_r + l + \frac{3}{2})}}
\end{equation*}
The spherical harmonic is
\begin{equation*}
    Y_{l}^{m}(\theta, \phi) = (-1)^{m} \sqrt{\frac{(2l + 1)}{4\pi} \frac{(l - m)!}{(l + m)!}} P_{l}^{m}(\cos\theta) e^{im\phi}
\end{equation*}
Both components obey the orthonormality
\begin{align*}
    \int_{0}^{2\pi} \int_{0}^{\pi} Y_{l}^{m}(\theta, \phi) Y_{l'}^{m'*}(\theta, \phi) \sin \theta \,d\theta \,d\phi & =  \delta_{ll'} \delta_{mm'} \\
    \int_{0}^{\infty} R_{n_r,l}(r) R_{n'_r,l}(r) r^2 dr                                                             & =  \delta_{n_r n'_r}
\end{align*}
such that
\begin{equation*}
    \int \Psi_{n_r,l,m}^{*}(r, \theta, \phi) \Psi_{n'_r,l',m'}(r, \theta, \phi) d^3r = \delta_{n_r n'_r} \delta_{ll'} \delta_{mm'}
\end{equation*}

\end{document}