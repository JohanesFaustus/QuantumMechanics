\documentclass[../../../main.tex]{subfiles}
\begin{document}
\section{Angular Momentum}
From the definition $\mathbf{L}= \mathbf{r} \times \mathbf{p}$, we can obtain the quantum angular momentum
\begin{equation*}
    L_x = YP_z - ZP_y \qquad L_y = ZP_x - XP_z \qquad L_z = XP_y - YP_x
\end{equation*}
The basic commutation relation are as follows
\begin{equation*}
    [L_{x}, L_{y}] = i \hbar L_{z}\qquad
    [L_{y}, L_{z}] = i \hbar L_{x}\qquad
    [L_{z}, L_{x}] = i \hbar L_{y}
\end{equation*}
This simplified by
\begin{equation*}
    [L_{i}, L_{j}] = i \hbar \epsilon_{ijk} L_{k}
\end{equation*}
Another one for the position case
\begin{equation*}
    [x_{i}, L_{j}] = i \hbar \epsilon_{ijk} x_{k}
\end{equation*}
and momentum
\begin{equation*}
    [P_{i}, L_{j}] = i \hbar \epsilon_{ijk} P_{k}
\end{equation*}

\subsubsection*{Derivation.}
Before we begin, consider the identities of commutator
\begin{align*}
    [\Omega,\Lambda\theta]     & =\Lambda[\Omega,\theta]+[\Omega,\Lambda]\theta \\
    [\Lambda\Omega,\Theta]     & =\Lambda[\Omega,\theta]+[\Lambda,\theta]\Omega \\
    [\Omega\Lambda,\Theta\Phi] & =
    \Omega[\Lambda,\Theta]\Phi
    + \Omega\Theta[\Lambda,\Phi]
    + [\Omega,\Theta]\Lambda\Phi
    + \Theta[\Omega,\Phi]\Lambda
\end{align*}
First let us derive the basic commutation
\begin{align*}
    [L_{x}, L_{y}] & =  [YP_{z} - ZP_{y}, ZP_{x} - XP_{z}]                                        \\
    [L_{x}, L_{y}] & =  [YP_{z}, ZP_{x}] - [YP_{z}, XP_{z}] - [ZP_{y}, ZP_{x}] + [ZP_{y}, XP_{z}] \\
\end{align*}
Each terms expand as
\begin{align*}
    [YP_z,ZP_x]
     & = Y[P_z,Z]P_x + YZ[P_z,P_x] + [Y,Z]P_zP_x + Z[Y,P_x]P_z \\
     & = Y[P_z,Z]P_x                                           \\
     & = -i\hbar\,Y P_x,                                       \\
    [YP_z,XP_z]
     & = Y[P_z,X]P_z + YX[P_z,P_z] + [Y,X]P_zP_z + X[Y,P_z]P_z \\
     & = 0,                                                    \\
    [ZP_y,ZP_x]
     & = Z[P_y,Z]P_x + ZZ[P_y,P_x] + [Z,Z]P_yP_x + Z[Z,P_x]P_y \\
     & = 0,                                                    \\
    [ZP_y,XP_z]
     & = Z[P_y,X]P_z + ZX[P_y,P_z] + [Z,X]P_yP_z + X[Z,P_z]P_y \\
     & = X[Z,P_z]P_y                                           \\
     & = i\hbar\,X P_y
\end{align*}
Summing the four results gives
\begin{align*}
    [L_{x}, L_{y}] & =  Y[P_{z}, Z]P_{x} + X[Z, P_{z}]P_{y} \\
                   & =  i\hbar (XP_{y} - YP_{x})            \\
    [L_{x}, L_{y}] & =  i\hbar L_{z}.
\end{align*}

For the posistion relation, we only derive for $X$ case, the other follows immediately
\begin{flalign*}
     & [X, L_x] = [X, Y P_z - Z P_y] = 0                                      \\
     & [X, L_y] = [X, Z P_x - X P_z] = [X, Z P_x] = Z[X, P_x] = i \hbar Z     \\
     & [X, L_z] = [X, X P_y - Y P_x] = -[X, Y P_x] = -Y[X, P_x] = -i \hbar Y.
\end{flalign*}
Same for the momentum
\begin{flalign*}
     & [P_{x}, L_{x}] = [P_{x}, Y P_{z} - Z P_{y}] = 0                                                    \\
     & [P_{x}, L_{y}] = [P_{x}, Z P_{x} - X P_{z}] = -[P_{x}, X P_{z}] = -[P_{x}, X]P_{z} = i \hbar P_{z} \\
     & [P_{x}, L_{z}] = [P_{x}, X P_{y} - Y P_{x}] = [P_{x}, X P_{y}] = [P_{x}, X]P_{y} = -i \hbar P_{y}  \\
\end{flalign*}

Another derivation just for fun
\begin{align*}
    [X, L^2] & =  [X, L_x^2] + [X, L_y^2] + [X, L_z^2]                      \\
             & =  0 + L_y[X, L_y] + [X, L_y]L_y + L_z[X, L_z] + [X, L_z]L_z \\
             & =  i\hbar (L_yZ + ZL_y - L_zY - YL_y),
\end{align*}
and
\begin{align*}
    [P_x, L^2] & =  [P_x, L_x^2] + [P_x, L_y^2] + [P_x, L_z^2]                        \\
               & =  0 + L_y[P_x, L_y] + [P_x, L_y]L_y + L_z[P_x, L_z] + [P_x, L_z]L_z \\
               & =  i\hbar(L_yP_z + P_zL_y - L_zP_y - P_yL_z).
\end{align*}

\subsubsection*{Total Angular Momentum}
The total angular momentum operator is defined as the sum of the angular momentum and spin $\mathbf{J}=\mathbf{L }+\mathbf{S}$.
Ladder operator for angular momentum is expressed as
\begin{equation*}
    J_{\pm} = J_x \pm i J_y
\end{equation*}
Their action on eigenstates $\ket{j,m}$ are
\begin{equation*}
    J_{\pm} \, |j, m\rangle = \hbar \sqrt{j(j+1) - m(m \pm 1)} \, |j, m \pm 1\rangle
\end{equation*}
Eigenstate $\ket{j,m}$ is simultaneous eigenstate of the total angular momentum operator $J^2$ with $j $ as the total angular momentum quantum number and $m$ as the magnetic quantum number.
For given $j$, $m$ can take the value $m=-j,j$ such that there are $2j+1$ degeneracy level.
The reason it being a degenerate is that the total angular momentum
\begin{equation*}
    J^2 |j, m\rangle = \hbar^2 j(j + 1) |j, m\rangle
\end{equation*}
are the same, yet the $z$ component
\begin{equation*}
    J_z |j, m\rangle = \hbar m |j, m\rangle,
\end{equation*}
are different.
The operator also satisfies
\begin{equation*}
    [J_z, J_{\pm}] = \pm \hbar J_{\pm}\qquad[J_+, J_-] = 2\hbar J_z
\end{equation*}

\subsubsection*{Derivation.}
Using the definition of Ladder operator
\begin{align*}
    [J_{z}, J_{+}] & =  [J_{z}, J_{x} + i J_{y}] = [J_{z}, J_{x}] + i [J_{z}, J_{y}] = i\hbar J_{y} + i (-i\hbar J_{x}) \\
                   & =  \hbar (J_{x} + i J_{y})                                                                         \\
    [J_{z}, J_{+}] & =  \hbar J_{+}                                                                                     \\
    [J_{z}, J_{-}] & =  [J_{z}, J_{x} - i J_{y}] = [J_{z}, J_{x}] - i [J_{z}, J_{y}] = i\hbar J_{y} - i (-i\hbar J_{x}) \\
                   & =  -\hbar (J_{x} - i J_{y})                                                                        \\
    [J_{z}, J_{-}] & =  -\hbar J_{-}
\end{align*}
And for other one
\begin{align*}
    [J_+, J_-] & =  (J_x + iJ_y)(J_x - iJ_y) - (J_x - iJ_y)(J_x + iJ_y)                     \\
               & =  J_x^2 - iJ_xJ_y + iJ_yJ_x + J_y^2 - (J_x^2 + iJ_xJ_y - iJ_yJ_x + J_y^2) \\
               & =  (-iJ_xJ_y + iJ_yJ_x) - (iJ_xJ_y - iJ_yJ_x)                              \\
               & =  -2iJ_xJ_y + 2iJ_yJ_x                                                    \\
               & =  2i(J_yJ_x - J_xJ_y)=  2i[J_y, J_x]                                      \\
    [J_+, J_-] & =  2i(-i\hbar J_z) = 2\hbar J_z
\end{align*}
\end{document}