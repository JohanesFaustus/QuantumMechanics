\documentclass[../../../main.tex]{subfiles}
\begin{document}
The electron of the hydrogen atom  moving in a Coulomb potential due to the proton
\begin{equation*}
    V(r) = - \frac{e^2}{4 \pi \epsilon_0 r}
\end{equation*}
obey th Schrödinger equation in spherical coordinates $(r,\theta,\phi)$
\begin{equation*}
    -\frac{\hbar^{2}}{2m}\nabla^{2}\psi(r, \theta, \phi) - \frac{e^{2}}{4\pi\varepsilon_{0}r}\psi(r, \theta, \phi) = E\psi(r, \theta, \phi)
\end{equation*}
since the potential is spherically symmetric.
Using the separation of variable, the solution reads
\begin{equation*}
    \psi_{nlm}(r, \theta, \phi) = R_{n\ell}(r) Y_{\ell}^{m}(\theta, \phi)
\end{equation*}
$R_{n\ell}(r)$ the radial function, which depends on the principal quantum number $n$ and angular momentum quantum number $\ell$
\begin{equation*}
    R_{n\ell}(r) = \sqrt{\left( \frac{2}{na_0} \right)^3 \frac{(n-\ell-1)!}{2n[(n+\ell)!]} e^{-r/(na_0)} \left( \frac{2r}{na_0} \right)^{\ell} L_{n-\ell-1}^{2\ell+1} \left( \frac{2r}{na_0} \right)}
\end{equation*}
are the spherical harmonics, which depend on $\ell$ and the magnetic quantum number $m$
\begin{equation*}
    Y_{\ell}^{m}(\theta, \phi) = (-1)^{m} \sqrt{\frac{(2\ell + 1)(\ell - m)!}{4\pi (\ell + m)!}} P_{\ell}^{m}(\cos \theta)e^{im\phi}
\end{equation*}

The complete description of electron is specified by four quantum number $(n,\ell,m_l,m_s)$, where $m_s=\pm 1/2$.
The wavefunction of hydrogen atom must be the product of spatial part $\psi_{n\ell m}$ and the spin part $\ket{1/2,m_s}$
\begin{equation*}
    \Psi_{n\ell m_\ell m_s}=R_{n\ell}(r) Y_{\ell}^{m}(\theta, \phi) \Bigg| \frac{1}{2}, \pm \frac{1}{2} \Bigg\rangle
\end{equation*}
Or by using spinor
\begin{equation*}
    \Psi_{n\ell m_\ell 1/2}=\psi_{n\ell m_l}
    \begin{bmatrix}
        1 \\0
    \end{bmatrix}
    =
    \begin{bmatrix}
        \psi_{n\ell m_l} \\0
    \end{bmatrix}
\end{equation*}
and
\begin{equation*}
    \Psi_{n\ell m_\ell -1/2}=\psi_{n\ell m_l}
    \begin{bmatrix}
        0 \\1
    \end{bmatrix}
    =
    \begin{bmatrix}
        0 \\\psi_{n\ell m_l}
    \end{bmatrix}
\end{equation*}
As such, the ground state then may be written as
\begin{align*}
    \Psi_{100,+\frac{1}{2}} & =  \begin{bmatrix} \psi_{100} \\ 0 \end{bmatrix} = \begin{bmatrix} \frac{1}{\sqrt{\pi a_0^3}}e^{-r/a_0} \\ 0 \end{bmatrix} \\
    \Psi_{100,-\frac{1}{2}} & =  \begin{bmatrix}0\\ \psi_{100} \end{bmatrix} = \begin{bmatrix} 0\\ \frac{1}{\sqrt{\pi a_0^3}}e^{-r/a_0}  \end{bmatrix}
\end{align*}

\subsection{Energy and Degeneracy}
The energy eigenvalues are quantized and depend only on the principal quantum number $n$
\begin{equation*}
    E_{n} = -\frac{13.6 }{n^{2}} \text{ eV}
\end{equation*}
The dependence of $n$ only is a property of central potential.
The additional degeneracy in $\ell$ is a property of Coulomb potential only, however.
The degeneracy of the state $n$ is given by
\begin{equation*}
    g_n = \sum_{\ell=0 }^{n-1 }(2l+1)=n^2
\end{equation*}
It is common to refer to the states with $\ell = 0, 1, 2, 3, 4, \dots$ as $s$, $p$, $d$,$f$, $g$, $h$, $\dots$ states.
In this spectroscopic notation, $1s$ denotes the state $(n= 1, \ell=0)$; $2s$ and $2p$ the $l=O$
and $l = 1$ states at $n = 2$; $3s$, $3p$, and $3d$ the $l = 0$, 1, and 2 states at $n = 3$, and so on.
No attempt is made to keep track of $m$.

\end{document}