\documentclass[../../../main.tex]{subfiles}
\begin{document}
\subsection{Infinite Potential Well}
Consider the one--dimensional infinite square well of width \(L\).
The potential shape is
\begin{equation*}
    V(x)=
    \begin{cases}
        0,      & |x|<L/2      \\
        \infty, & |x| \leq L/2
    \end{cases}
\end{equation*}
The wave function in this system takes the form 
\begin{equation*}
    \psi_n^{\text{odd}}(x) = \sqrt{\frac{2  }{L }} \sin\left(\frac{2 n \pi x}{L}\right)
    \qquad
    \psi_n^{\text{even}}(x) =\sqrt{\frac{2 }{L }} \cos\left(\frac{(2n-1)\pi x}{L}\right)
\end{equation*}
We also have the energy eigenstate
\begin{equation*}
    E_n^{\text{even}} = \frac{\hbar^2}{2 m} \left[\frac{(2n-1)\pi}{L}\right]^2, \qquad
    E_n^{\text{odd}} = \frac{\hbar^2}{2 m} \left(\frac{2 n \pi}{L}\right)^2
\end{equation*}

\subsubsection{Derivation.}
We begin by partitioning space into three regions I, II, and III.
According to the Schrödinger equation, the wave function outside the well is simply zero since the potential is infinite.
\begin{equation*}
    -\frac{\hbar^2 }{2m }\frac{d^2 }{dx^2 }\psi_I+\infty \psi_I=E\psi,\implies \psi_I(x)=\psi_{III}=0
\end{equation*}

Next we move to region $II$.
The Schrödinger equation now reads
\begin{align*}
    -\frac{\hbar^2}{2m} \frac{d^2 \psi(x)}{dx^2} & =  E \psi(x)   \\
    \frac{d^2 \psi(x)}{dx^2}                     & = -k^2 \psi(x)
\end{align*}
with $k^2 =2 m E/\hbar^2.$
Recall the boundaries condition of
\begin{equation*}
    \psi \left( -\frac{L }{2 } \right) =\psi \left( \frac{L }{2 } \right) =0
\end{equation*}
Because the potential is symmetric, the Hamiltonian commutes with the parity operator $[H,\Pi] = 0$.
For even parity, we have
\begin{equation*}
    \psi(-x) = \psi(x) \implies B = 0, \quad \psi(x) = A \cos(kx)
\end{equation*}
Applying the second boundaries condition
\begin{equation*}
    \cos\left(\frac{k L}{2}\right) = 0\implies k^{\text{even}} = \frac{(2n-1)\pi}{L},
\end{equation*}
This yield the even parity wave function
\begin{equation*}
    \psi_n^{\text{even}}(x) =A \cos\left(\frac{(2n-1)\pi x}{L}\right)
\end{equation*}
Next for odd parity
\begin{equation*}
    \psi(-x) = -\psi(x) \implies A = 0, \quad \psi(x) = B \sin(kx)
\end{equation*}
Applying the first boundaries condition
\begin{equation*}
    \sin\left(\frac{k L}{2}\right) = 0 \implies k^{\text{odd}} = \frac{2 n \pi}{L}
\end{equation*}
This yield the odd parity wave function
\begin{equation*}
    \psi_n^{\text{odd}}(x) = B \sin\left(\frac{2 n \pi x}{L}\right)
\end{equation*}

Next we begin the normalization step.
To normalize $\psi_n^{\text{even}}$, we perform
\begin{align*}
    \int_{-L/2}^{L/2} \sin^2\left(\frac{2 n \pi x}{L}\right) dx & =  \int_0^{L/2} \left[1 - \cos\left(\frac{4 n \pi x}{L}\right)\right] dx                 \\
                                                                   & = \frac{L }{2 }+ \frac{\sin\left(\frac{4 n \pi x}{L}\right)}{4 n \pi / L} \Bigg|_0^{L/2} \\
                                                                   & = \frac{L }{2 }+\frac{\sin(2 n \pi) - \sin 0}{4 n \pi / L}                               \\
    \int_{-L/2}^{L/2} \sin^2\left(\frac{2 n \pi x}{L}\right) dx & =\frac{L }{2}                                                                         \\
\end{align*}
which meant 
\begin{equation*}
    B^2 \frac{L }{2 }=1\implies B=\frac{2 }{L}
\end{equation*}
Then we normalize $\psi_n^{\text{odd}}$
\begin{align*}
    \int_{-L/2}^{L/2} \cos^2\left(\frac{(2n-1) \pi x}{L}\right) dx & =  \int_0^{L/2} \left[1 + \cos\left(\frac{2 (2n-1) \pi x}{L}\right)\right] dx                   \\
                                                                      & = \frac{L }{2 }+\frac{\sin\left(\frac{2 (2n-1) \pi x}{L}\right)}{2(2n-1)\pi / L} \Bigg|_0^{L/2} \\
                                                                      & = \frac{L }{2 }+ \frac{\sin((2n-1)\pi)-\sin0}{2(2n-1)\pi / L}                                   \\
    \int_{-L/2}^{L/2} \cos^2\left(\frac{(2n-1) \pi x}{L}\right) dx & = \frac{L }{2 }\\
\end{align*}
which meant 
\begin{equation*}
    A^2 \frac{L }{2 }=1\implies A=\frac{2 }{L}
\end{equation*}

Using the expression for $k^{\text{odd}}$ and $k^{\text{even}}$, we can obtain the expression for energy eigenstate for both parity.

\subsection{Infinite Potential Well: Antisymmetric Case}
The potential is in the shape
\begin{equation*}
    V(x)=
    \begin{cases}
        \infty, & x \leq0  \\
        0,      & 0<x<L    \\
        \infty, & L \leq L
    \end{cases}
\end{equation*}
With wave function in the form 
\begin{equation*}
    \psi(x)=\begin{cases}
        0                                          & x\leq 0   \\
        \sqrt{\dfrac{2}{L}}\sin( \dfrac{n\pi}{L}x) & 0 < x < L \\
        0                                          & L \leq x
    \end{cases}
\end{equation*}
and energy eigenstate
\begin{equation*}
    E_n=\frac{\pi^2\hbar^2}{2mL^2}n^2
\end{equation*}
This is different from the symmetric potential case.
Even if the “shape” or magnitude of the potential is similar, changing its symmetry changes the mathematical constraints on eigenfunctions, which directly affects the allowed energies.
In quantum mechanics, small changes in the Hamiltonian can produce different spectra, because energies are eigenvalues of the operator, not just classical measures of potential height.

\subsubsection{Derivation.}
As usual, the wave function at infinite potential is zero $\psi(x \leq 0)=\psi(L\leq x)=0$.
In $0 < x < L$, where $ V = 0$,
\begin{align*}
    \frac{d^2\psi}{dx^2} & = -\frac{2mE}{\hbar^2}\psi \\
    \frac{d^2\psi}{dx^2} & = -k^2\psi                 \\
\end{align*}
with $k^2 =2 m E/\hbar^2$.
The solution reads
\begin{equation*}
    \psi(x)=A\sin (kx)+ B \cos (kx)
\end{equation*}
Imposing boundary condition $\psi(x)=0$, we see that $B=0$
\begin{equation*}
    \psi(x)=A\sin (kx)
\end{equation*}
Imposing the second boundary condition $\psi(L)=0$, and considering that sinus function goes to zero at $n\pi$,
\begin{equation*}
    kL=n\pi
\end{equation*}
Normalization also requires $A = \sqrt{2/L}$.
\end{document}