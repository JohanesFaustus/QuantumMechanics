\documentclass[../../../main.tex]{subfiles}
\begin{document}
\subsection{Infinite Potential Well}
Consider the one--dimensional infinite square well of width \(L\).
The potential shape is
\begin{equation*}
    V(x)=
    \begin{cases}
        0,      & |x|<L/2      \\
        \infty, & |x| \leq L/2
    \end{cases}
\end{equation*}
The wave function in this system takes the form
\begin{equation*}
    \psi_n^{\text{odd}}(x) = \sqrt{\frac{2  }{L }} \sin\left(\frac{2 n \pi x}{L}\right)
    \qquad
    \psi_n^{\text{even}}(x) =\sqrt{\frac{2 }{L }} \cos\left(\frac{(2n-1)\pi x}{L}\right)
\end{equation*}
We also have the energy eigenstate
\begin{equation*}
    E_n^{\text{even}} = \frac{\hbar^2}{2 m} \left[\frac{(2n-1)\pi}{L}\right]^2, \qquad
    E_n^{\text{odd}} = \frac{\hbar^2}{2 m} \left(\frac{2 n \pi}{L}\right)^2
\end{equation*}

\subsubsection{Derivation.}
We begin by partitioning space into three regions I, II, and III.
According to the Schrödinger equation, the wave function outside the well is simply zero since the potential is infinite.
\begin{equation*}
    -\frac{\hbar^2 }{2m }\frac{d^2 }{dx^2 }\psi_I+\infty \psi_I=E\psi,\implies \psi_I(x)=\psi_{III}=0
\end{equation*}

Next we move to region $II$.
The Schrödinger equation now reads
\begin{align*}
    -\frac{\hbar^2}{2m} \frac{d^2 \psi(x)}{dx^2} & =  E \psi(x)   \\
    \frac{d^2 \psi(x)}{dx^2}                     & = -k^2 \psi(x)
\end{align*}
with $k^2 =2 m E/\hbar^2.$
Recall the boundaries condition of
\begin{equation*}
    \psi \left( -\frac{L }{2 } \right) =\psi \left( \frac{L }{2 } \right) =0
\end{equation*}
Because the potential is symmetric, the Hamiltonian commutes with the parity operator $[H,\Pi] = 0$.
For even parity, we have
\begin{equation*}
    \psi(-x) = \psi(x) \implies B = 0, \quad \psi(x) = A \cos(kx)
\end{equation*}
Applying the second boundaries condition
\begin{equation*}
    \cos\left(\frac{k L}{2}\right) = 0\implies k^{\text{even}} = \frac{(2n-1)\pi}{L},
\end{equation*}
This yield the even parity wave function
\begin{equation*}
    \psi_n^{\text{even}}(x) =A \cos\left(\frac{(2n-1)\pi x}{L}\right)
\end{equation*}
Next for odd parity
\begin{equation*}
    \psi(-x) = -\psi(x) \implies A = 0, \quad \psi(x) = B \sin(kx)
\end{equation*}
Applying the first boundaries condition
\begin{equation*}
    \sin\left(\frac{k L}{2}\right) = 0 \implies k^{\text{odd}} = \frac{2 n \pi}{L}
\end{equation*}
This yield the odd parity wave function
\begin{equation*}
    \psi_n^{\text{odd}}(x) = B \sin\left(\frac{2 n \pi x}{L}\right)
\end{equation*}

Next we begin the normalization step.
To normalize $\psi_n^{\text{even}}$, we perform
\begin{align*}
    \int_{-L/2}^{L/2} \sin^2\left(\frac{2 n \pi x}{L}\right) dx & =  \int_0^{L/2} \left[1 - \cos\left(\frac{4 n \pi x}{L}\right)\right] dx                 \\
                                                                & = \frac{L }{2 }+ \frac{\sin\left(\frac{4 n \pi x}{L}\right)}{4 n \pi / L} \Bigg|_0^{L/2} \\
                                                                & = \frac{L }{2 }+\frac{\sin(2 n \pi) - \sin 0}{4 n \pi / L}                               \\
    \int_{-L/2}^{L/2} \sin^2\left(\frac{2 n \pi x}{L}\right) dx & =\frac{L }{2}                                                                            \\
\end{align*}
which meant
\begin{equation*}
    B^2 \frac{L }{2 }=1\implies B=\frac{2 }{L}
\end{equation*}
Then we normalize $\psi_n^{\text{odd}}$
\begin{align*}
    \int_{-L/2}^{L/2} \cos^2\left(\frac{(2n-1) \pi x}{L}\right) dx & =  \int_0^{L/2} \left[1 + \cos\left(\frac{2 (2n-1) \pi x}{L}\right)\right] dx                   \\
                                                                   & = \frac{L }{2 }+\frac{\sin\left(\frac{2 (2n-1) \pi x}{L}\right)}{2(2n-1)\pi / L} \Bigg|_0^{L/2} \\
                                                                   & = \frac{L }{2 }+ \frac{\sin((2n-1)\pi)-\sin0}{2(2n-1)\pi / L}                                   \\
    \int_{-L/2}^{L/2} \cos^2\left(\frac{(2n-1) \pi x}{L}\right) dx & = \frac{L }{2 }                                                                                 \\
\end{align*}
which meant
\begin{equation*}
    A^2 \frac{L }{2 }=1\implies A=\frac{2 }{L}
\end{equation*}

Using the expression for $k^{\text{odd}}$ and $k^{\text{even}}$, we can obtain the expression for energy eigenstate for both parity.

\subsection{Infinite Potential Well: Antisymmetric Case}
The potential is in the shape
\begin{equation*}
    V(x)=
    \begin{cases}
        \infty, & x \leq0  \\
        0,      & 0<x<L    \\
        \infty, & L \leq L
    \end{cases}
\end{equation*}
With wave function in the form
\begin{equation*}
    \psi(x)=\begin{cases}
        0                                          & x\leq 0   \\
        \sqrt{\dfrac{2}{L}}\sin( \dfrac{n\pi}{L}x) & 0 < x < L \\
        0                                          & L \leq x
    \end{cases}
\end{equation*}
and energy eigenstate
\begin{equation*}
    E_n=\frac{\pi^2\hbar^2}{2mL^2}n^2
\end{equation*}
This is different from the symmetric potential case.
Even if the “shape” or magnitude of the potential is similar, changing its symmetry changes the mathematical constraints on eigenfunctions, which directly affects the allowed energies.
In quantum mechanics, small changes in the Hamiltonian can produce different spectra, because energies are eigenvalues of the operator, not just classical measures of potential height.

\subsubsection{Derivation.}
As usual, the wave function at infinite potential is zero $\psi(x \leq 0)=\psi(L\leq x)=0$.
In $0 < x < L$, where $ V = 0$,
\begin{align*}
    \frac{d^2\psi}{dx^2} & = -\frac{2mE}{\hbar^2}\psi \\
    \frac{d^2\psi}{dx^2} & = -k^2\psi                 \\
\end{align*}
with $k^2 =2 m E/\hbar^2$.
The solution reads
\begin{equation*}
    \psi(x)=A\sin (kx)+ B \cos (kx)
\end{equation*}
Imposing boundary condition $\psi(x)=0$, we see that $B=0$
\begin{equation*}
    \psi(x)=A\sin (kx)
\end{equation*}
Imposing the second boundary condition $\psi(L)=0$, and considering that sinus function goes to zero at $n\pi$,
\begin{equation*}
    kL=n\pi
\end{equation*}
Normalization also requires $A = \sqrt{2/L}$.

\subsection{3D Infinite Potential Well}
We assume antisymmetric 3D potential
\begin{equation*}
    V(x,y,z) =
    \begin{cases}
        0,      & 0 < x < L_x, \ 0 < y < L_y, \ 0 < z < L_z, \\
        \infty, & \text{otherwise}.
    \end{cases}
\end{equation*}

The full normalized 3D wavefunction is
\begin{equation*}
    \psi_{nml}(x,y,z) = \sqrt{\frac{8}{L_x L_y L_z}} \,
    \sin\left(\frac{n \pi x}{L_x}\right)
    \sin\left(\frac{m \pi y}{L_y}\right)
    \sin\left(\frac{l \pi z}{L_z}\right).
\end{equation*}
We derive this by the method of separation variable, that is assuming $\psi(x,y,z) = X(x) Y(y) Z(z)$, where
\begin{align*}
    X_n(x) & = \sqrt{\frac{2}{L_x}} \sin\left(\frac{n \pi x}{L_x}\right) \\
    Y_m(y) & = \sqrt{\frac{2}{L_y}} \sin\left(\frac{m \pi y}{L_y}\right) \\
    Z_l(z) & = \sqrt{\frac{2}{L_z}} \sin\left(\frac{l \pi z}{L_z}\right)
\end{align*}

The energy eigenstate is
\begin{equation*}
    E_{nml} =\frac{\hbar^2 \pi^2}{2 m} \left( \frac{n^2}{L_x^2} + \frac{m^2}{L_y^2} + \frac{l^2}{L_z^2} \right)
\end{equation*}
For a cubic box \(L_x = L_y = L_z = L\), the energies simplify to
\begin{equation*}
    E_{nml} = \frac{\hbar^2 \pi^2}{2 m L^2} (n^2 + m^2 + l^2).
\end{equation*}
The total energy $E$ is the sum of energies from each dimension $= E_x + E_y + E_z$.
The expression for each dimension is that of one dimensional case
\begin{equation*}
    E_i=\frac{\hbar^2 \pi^2 }{2mL_i^2}i^2
\end{equation*}

\subsubsection{Derivation.}
Outside the box, the wave function is zero; while inside, the wave function according to Schrödinger equation
\begin{equation*}
    -\frac{\hbar^2}{2m} \nabla^2 \psi(x,y,z) = E \psi(x,y,z)
\end{equation*}
First, by separation variable, we assume the solution in the form
\begin{equation*}
    \psi(x,y,z) = X(x) Y(y) Z(z).
\end{equation*}
With the second differentiation of
\begin{align*}
    \frac{\partial^2 \psi}{\partial x^2} & = Y(y) Z(z) X''(x), \\
    \frac{\partial^2 \psi}{\partial y^2} & = X(x) Z(z) Y''(y), \\
    \frac{\partial^2 \psi}{\partial z^2} & = X(x) Y(y) Z''(z).
\end{align*}
the Laplacian becomes
\begin{equation*}
    \nabla^2 \psi = Y Z X'' + X Z Y'' + X Y Z''.
\end{equation*}
Substituting into the Schrödinger equation and dividing both sides by \(\psi = X Y Z\) gives
\begin{equation*}
    -\frac{\hbar^2}{2m} \left( \frac{X''}{X} + \frac{Y''}{Y} + \frac{Z''}{Z} \right) = E.
\end{equation*}
Each term depends on only one variable, which allows the equation to be separated into the Schrödinger equation gives
\begin{equation*}
    -\frac{\hbar^2}{2m} \left( \frac{X''(x)}{X(x)} + \frac{Y''(y)}{Y(y)} + \frac{Z''(z)}{Z(z)} \right) = E,
\end{equation*}
This leads to separate equations for each dimension
\begin{align*}
    -\frac{\hbar^2}{2m} X''(x) & = E_x X(x), \\
    -\frac{\hbar^2}{2m} Y''(y) & = E_y Y(y), \\
    -\frac{\hbar^2}{2m} Z''(z) & = E_z Z(z),
\end{align*}
with $E_i$ as the constant of separation.
Similar to 1D case, the solution to each equation is
\begin{align*}
    X_n(x) & = \sqrt{\frac{2}{L_x}} \sin\left(\frac{n \pi x}{L_x}\right) \\
    Y_m(y) & = \sqrt{\frac{2}{L_y}} \sin\left(\frac{m \pi y}{L_y}\right) \\
    Z_l(z) & = \sqrt{\frac{2}{L_z}} \sin\left(\frac{l \pi z}{L_z}\right)
\end{align*}

Each dimension contributes independently to the expression of energy
\begin{align*}
    E_x & = \frac{\hbar^2 \pi^2 n^2}{2 m L_x^2}, \\
    E_y & = \frac{\hbar^2 \pi^2 m^2}{2 m L_y^2}, \\
    E_z & = \frac{\hbar^2 \pi^2 l^2}{2 m L_z^2}.
\end{align*}
The total energy is
\begin{equation*}
    E_{nml} = E_x + E_y + E_z = \frac{\hbar^2 \pi^2}{2 m} \left( \frac{n^2}{L_x^2} + \frac{m^2}{L_y^2} + \frac{l^2}{L_z^2} \right)
\end{equation*}
\end{document}