\documentclass[../../../main.tex]{subfiles}
\begin{document}
\subsection{Infinite Potential Well}
Consider the one--dimensional infinite square well of width \(L\).
The potential shape is
\begin{equation*}
    V(x)=
    \begin{cases}
        0,      & |x|<L/2      \\
        \infty, & |x| \leq L/2
    \end{cases}
\end{equation*}
The wave function in this system takes the form
\begin{equation*}
    \psi_n^{\text{odd}}(x) = \sqrt{\frac{2  }{L }} \sin\left(\frac{2 n \pi x}{L}\right)
    \qquad
    \psi_n^{\text{even}}(x) =\sqrt{\frac{2 }{L }} \cos\left(\frac{(2n-1)\pi x}{L}\right)
\end{equation*}
We also have the energy eigenvalue
\begin{equation*}
    E_n^{\text{even}} = \frac{\hbar^2}{2 m} \left[\frac{(2n-1)\pi}{L}\right]^2, \qquad
    E_n^{\text{odd}} = \frac{\hbar^2}{2 m} \left(\frac{2 n \pi}{L}\right)^2
\end{equation*}

\subsubsection{Derivation.}
We begin by partitioning space into three regions I, II, and III.
According to the Schrödinger equation, the wave function outside the well is simply zero since the potential is infinite.
\begin{equation*}
    -\frac{\hbar^2 }{2m }\frac{d^2 }{dx^2 }\psi_I+\infty \psi_I=E\psi,\implies \psi_I(x)=\psi_{III}=0
\end{equation*}

Next we move to region $II$.
The Schrödinger equation now reads
\begin{align*}
    -\frac{\hbar^2}{2m} \frac{d^2 \psi(x)}{dx^2} & =  E \psi(x)   \\
    \frac{d^2 \psi(x)}{dx^2}                     & = -k^2 \psi(x)
\end{align*}
with $k^2 =2 m E/\hbar^2.$
Recall the boundaries condition of
\begin{equation*}
    \psi \left( -\frac{L }{2 } \right) =\psi \left( \frac{L }{2 } \right) =0
\end{equation*}
Because the potential is symmetric, the Hamiltonian commutes with the parity operator $[H,\Pi] = 0$.
For even parity, we have
\begin{equation*}
    \psi(-x) = \psi(x) \implies B = 0, \quad \psi(x) = A \cos(kx)
\end{equation*}
Applying the second boundaries condition
\begin{equation*}
    \cos\left(\frac{k L}{2}\right) = 0\implies k^{\text{even}} = \frac{(2n-1)\pi}{L},
\end{equation*}
This yield the even parity wave function
\begin{equation*}
    \psi_n^{\text{even}}(x) =A \cos\left(\frac{(2n-1)\pi x}{L}\right)
\end{equation*}
Next for odd parity
\begin{equation*}
    \psi(-x) = -\psi(x) \implies A = 0, \quad \psi(x) = B \sin(kx)
\end{equation*}
Applying the first boundaries condition
\begin{equation*}
    \sin\left(\frac{k L}{2}\right) = 0 \implies k^{\text{odd}} = \frac{2 n \pi}{L}
\end{equation*}
This yield the odd parity wave function
\begin{equation*}
    \psi_n^{\text{odd}}(x) = B \sin\left(\frac{2 n \pi x}{L}\right)
\end{equation*}

Next we begin the normalization step.
To normalize $\psi_n^{\text{even}}$, we perform
\begin{align*}
    \int_{-L/2}^{L/2} \sin^2\left(\frac{2 n \pi x}{L}\right) dx & =  \int_0^{L/2} \left[1 - \cos\left(\frac{4 n \pi x}{L}\right)\right] dx                 \\
                                                                & = \frac{L }{2 }+ \frac{\sin\left(\frac{4 n \pi x}{L}\right)}{4 n \pi / L} \Bigg|_0^{L/2} \\
                                                                & = \frac{L }{2 }+\frac{\sin(2 n \pi) - \sin 0}{4 n \pi / L}                               \\
    \int_{-L/2}^{L/2} \sin^2\left(\frac{2 n \pi x}{L}\right) dx & =\frac{L }{2}                                                                            \\
\end{align*}
which meant
\begin{equation*}
    B^2 \frac{L }{2 }=1\implies B=\frac{2 }{L}
\end{equation*}
Then we normalize $\psi_n^{\text{odd}}$
\begin{align*}
    \int_{-L/2}^{L/2} \cos^2\left(\frac{(2n-1) \pi x}{L}\right) dx & =  \int_0^{L/2} \left[1 + \cos\left(\frac{2 (2n-1) \pi x}{L}\right)\right] dx                   \\
                                                                   & = \frac{L }{2 }+\frac{\sin\left(\frac{2 (2n-1) \pi x}{L}\right)}{2(2n-1)\pi / L} \Bigg|_0^{L/2} \\
                                                                   & = \frac{L }{2 }+ \frac{\sin((2n-1)\pi)-\sin0}{2(2n-1)\pi / L}                                   \\
    \int_{-L/2}^{L/2} \cos^2\left(\frac{(2n-1) \pi x}{L}\right) dx & = \frac{L }{2 }                                                                                 \\
\end{align*}
which meant
\begin{equation*}
    A^2 \frac{L }{2 }=1\implies A=\frac{2 }{L}
\end{equation*}

Using the expression for $k^{\text{odd}}$ and $k^{\text{even}}$, we can obtain the expression for energy eigenvalue for both parity.

\subsection{Infinite Potential Well: Antisymmetric Case}
The potential is in the shape
\begin{equation*}
    V(x)=
    \begin{cases}
        \infty, & x \leq0  \\
        0,      & 0<x<L    \\
        \infty, & L \leq L
    \end{cases}
\end{equation*}
With wave function in the form
\begin{equation*}
    \psi(x)=\begin{cases}
        0                                          & x\leq 0   \\
        \sqrt{\dfrac{2}{L}}\sin( \dfrac{n\pi}{L}x) & 0 < x < L \\
        0                                          & L \leq x
    \end{cases}
\end{equation*}
and energy eigenvalue
\begin{equation*}
    E_n=\frac{\pi^2\hbar^2}{2mL^2}n^2
\end{equation*}
This is different from the symmetric potential case.
Even if the “shape” or magnitude of the potential is similar, changing its symmetry changes the mathematical constraints on eigenfunctions, which directly affects the allowed energies.
In quantum mechanics, small changes in the Hamiltonian can produce different spectra, because energies are eigenvalues of the operator, not just classical measures of potential height.

\subsubsection{Derivation.}
As usual, the wave function at infinite potential is zero $\psi(x \leq 0)=\psi(L\leq x)=0$.
In $0 < x < L$, where $ V = 0$,
\begin{align*}
    \frac{d^2\psi}{dx^2} & = -\frac{2mE}{\hbar^2}\psi \\
    \frac{d^2\psi}{dx^2} & = -k^2\psi                 \\
\end{align*}
with $k^2 =2 m E/\hbar^2$.
The solution reads
\begin{equation*}
    \psi(x)=A\sin (kx)+ B \cos (kx)
\end{equation*}
Imposing boundary condition $\psi(x)=0$, we see that $B=0$
\begin{equation*}
    \psi(x)=A\sin (kx)
\end{equation*}
Imposing the second boundary condition $\psi(L)=0$, and considering that sinus function goes to zero at $n\pi$,
\begin{equation*}
    kL=n\pi
\end{equation*}
Normalization also requires $A = \sqrt{2/L}$.

\subsection{3D Infinite Potential Well}
We assume antisymmetric 3D potential
\begin{equation*}
    V(x,y,z) =
    \begin{cases}
        0,      & 0 < x < L_x, \ 0 < y < L_y, \ 0 < z < L_z, \\
        \infty, & \text{otherwise}.
    \end{cases}
\end{equation*}

The full normalized 3D wavefunction is
\begin{equation*}
    \psi_{nml}(x,y,z) = \sqrt{\frac{8}{L_x L_y L_z}} \,
    \sin\left(\frac{n \pi x}{L_x}\right)
    \sin\left(\frac{m \pi y}{L_y}\right)
    \sin\left(\frac{l \pi z}{L_z}\right).
\end{equation*}
We derive this by the method of separation variable, that is assuming $\psi(x,y,z) = X(x) Y(y) Z(z)$, where
\begin{align*}
    X_n(x) & = \sqrt{\frac{2}{L_x}} \sin\left(\frac{n \pi x}{L_x}\right) \\
    Y_m(y) & = \sqrt{\frac{2}{L_y}} \sin\left(\frac{m \pi y}{L_y}\right) \\
    Z_l(z) & = \sqrt{\frac{2}{L_z}} \sin\left(\frac{l \pi z}{L_z}\right)
\end{align*}

The energy eigenvalue is
\begin{equation*}
    E_{nml} =\frac{\hbar^2 \pi^2}{2 m} \left( \frac{n^2}{L_x^2} + \frac{m^2}{L_y^2} + \frac{l^2}{L_z^2} \right)
\end{equation*}
For a cubic box \(L_x = L_y = L_z = L\), the energies simplify to
\begin{equation*}
    E_{nml} = \frac{\hbar^2 \pi^2}{2 m L^2} (n^2 + m^2 + l^2).
\end{equation*}
The total energy $E$ is the sum of energies from each dimension $= E_x + E_y + E_z$.
The expression for each dimension is that of one dimensional case
\begin{equation*}
    E_i=\frac{\hbar^2 \pi^2 }{2mL_i^2}i^2
\end{equation*}

\subsubsection{Derivation.}
Outside the box, the wave function is zero; while inside, the wave function according to Schrödinger equation
\begin{equation*}
    -\frac{\hbar^2}{2m} \nabla^2 \psi(x,y,z) = E \psi(x,y,z)
\end{equation*}
First, by separation variable, we assume the solution in the form
\begin{equation*}
    \psi(x,y,z) = X(x) Y(y) Z(z).
\end{equation*}
With the second differentiation of
\begin{align*}
    \frac{\partial^2 \psi}{\partial x^2} & = Y(y) Z(z) X''(x), \\
    \frac{\partial^2 \psi}{\partial y^2} & = X(x) Z(z) Y''(y), \\
    \frac{\partial^2 \psi}{\partial z^2} & = X(x) Y(y) Z''(z).
\end{align*}
The Laplacian becomes
\begin{equation*}
    \nabla^2 \psi = Y Z X'' + X Z Y'' + X Y Z''.
\end{equation*}
Substituting into the Schrödinger equation and dividing both sides by \(\psi = X Y Z\) gives
\begin{equation*}
    -\frac{\hbar^2}{2m} \left( \frac{X''}{X} + \frac{Y''}{Y} + \frac{Z''}{Z} \right) = E.
\end{equation*}
Each term depends on only one variable, which allows the equation to be separated into the Schrödinger equation gives
\begin{equation*}
    -\frac{\hbar^2}{2m} \left( \frac{X''(x)}{X(x)} + \frac{Y''(y)}{Y(y)} + \frac{Z''(z)}{Z(z)} \right) = E,
\end{equation*}
This leads to separate equations for each dimension
\begin{align*}
    -\frac{\hbar^2}{2m} X''(x) & = E_x X(x), \\
    -\frac{\hbar^2}{2m} Y''(y) & = E_y Y(y), \\
    -\frac{\hbar^2}{2m} Z''(z) & = E_z Z(z),
\end{align*}
with $E_i$ as the constant of separation.
Similar to 1D case, the solution to each equation is
\begin{align*}
    X_n(x) & = \sqrt{\frac{2}{L_x}} \sin\left(\frac{n \pi x}{L_x}\right) \\
    Y_m(y) & = \sqrt{\frac{2}{L_y}} \sin\left(\frac{m \pi y}{L_y}\right) \\
    Z_l(z) & = \sqrt{\frac{2}{L_z}} \sin\left(\frac{l \pi z}{L_z}\right)
\end{align*}

Each dimension contributes independently to the expression of energy
\begin{align*}
    E_x & = \frac{\hbar^2 \pi^2 n^2}{2 m L_x^2}, \\
    E_y & = \frac{\hbar^2 \pi^2 m^2}{2 m L_y^2}, \\
    E_z & = \frac{\hbar^2 \pi^2 l^2}{2 m L_z^2}.
\end{align*}
The total energy is
\begin{equation*}
    E_{nml} = E_x + E_y + E_z = \frac{\hbar^2 \pi^2}{2 m} \left( \frac{n^2}{L_x^2} + \frac{m^2}{L_y^2} + \frac{l^2}{L_z^2} \right)
\end{equation*}

\subsection{Step Potential}
The system has the potential at the shape
\begin{equation*}
    V(x) =
    \begin{cases}
        0,   & x < 0,   \\
        V_0, & 0\leq x.
    \end{cases}
\end{equation*}

As usual, the wavefunction depends on the region and the value of $E$
\begin{equation*}
    \psi(x)=
    \begin{cases}
        A e^{i k_1 x} + B e^{-i k_1 x} & x <0                \\
        C e^{i k_2 x} + D e^{-i k_2 x} & 0 \leq x,\; V_0 < E \\
        C e^{-\kappa x}                & 0 \leq x,\;E < V_0
    \end{cases}
\end{equation*}
where
\begin{equation*}
    k_1^2 =\frac{2 m E}{\hbar^2}    ,
    \qquad
    k_2 =
    \begin{cases}
        \sqrt{\dfrac{2 m (E - V_0)}{\hbar^2}}                           & V_0<E   \\
        i \kappa, \quad \kappa = \sqrt{\dfrac{2 m (V_0 - E)}{\hbar^2}}, & E < V_0
    \end{cases}
\end{equation*}

In the case of $E<V_0$, our wave function reads
\begin{equation*}
    \psi(x)=Ae^{ik_1x}+Be^{-ik_1x}+Ce^{-\kappa x}
\end{equation*}
$Ae^{ik_1x}$ corresponds to a wave incident from the left, traveling toward the potential step, while $Be^{-ik_1x}$ corresponds to a reflected wave, moving backward, away from the step.
These are oscillatory plane waves, representing the particle moving freely in the $x<0$.
$Ce^{-\kappa x}$ represents an evanescent wave, i.e., the tunneling tail inside the classically forbidden region.

Unlike the case of $E<V_0$, particle can move freely in the case of $V_0<E$.
The wavefunction in the form
\begin{equation*}
    \psi(x)=Ae^{ik_1x}+Be^{-ik_1x}+Ce^{ik_2 x}+De^{-ik_2x}
\end{equation*}
represents this.
$Ae^{ik_1x}$ represents particle incident form left, $Be^{-ik_1x}$ represents it being reflected back, and $Ce^{ik_2 x}$ represents it being transmitted.
$De^{-ik_2x}$ represents wave incident from right, although we often set $D=0$ since there is no particle incident from right.

The absolute magnitude of incident wave $A$ is arbitrary because multiplying the wavefunction by a constant does not change the reflection $R$ or transmission probabilities $T$
\begin{equation*}
    R=\frac{|B|^2}{|A|^2}=\left|\frac{k_1-k_2 }{k_1+k_2}\right|^2
    \qquad
    T = \frac{k_2}{k_1} \frac{|C|^2}{|A|^2} = \frac{k_2}{k_1} \left| \frac{2 k_1}{k_1 + k_2} \right|^2
\end{equation*}
Since both are probabilities, $R+T=1$ should apply.
We define $r=B/A$ and $t=C/A$ as the reflection and transmission amplitude respectively.

\subsubsection{Derivation.}
We begin by dividing the regions into I and II.
At the region I, the Schrödinger equation reads
\begin{align*}
    -\frac{\hbar^2}{2m} \frac{d^2 \psi_{I}}{dx^2} & =  E \psi_I        \\
    \frac{d^2 \psi_{I}}{dx^2}                     & = - k_1^2 \psi_{I}
\end{align*}
with
\begin{equation*}
    k_1^2 =\frac{2 m E}{\hbar^2}
\end{equation*}
The general solution is
\begin{equation*}
    \psi_I(x) = A e^{i k_1 x} + B e^{-i k_1 x}
\end{equation*}

Now we move into region II.
The Schrödinger equation now reads
\begin{align*}
    -\frac{\hbar^2}{2m} \frac{d^2 \psi_{II}}{dx^2} + V_0 \psi_{II} & =  E \psi_{II}      \\
    \frac{d^2 \psi_{II}}{dx^2}                                     & = - k_2^2 \psi_{II}
\end{align*}
The value  of $k_2$ depends on the value of $E$
\begin{equation*}
    k_2 =
    \begin{cases}
        \sqrt{\dfrac{2 m (E - V_0)}{\hbar^2}}                           & V_0<E   \\
        i \kappa, \quad \kappa = \sqrt{\dfrac{2 m (V_0 - E)}{\hbar^2}}, & E < V_0
    \end{cases}
\end{equation*}
It also follows that the solution also depends on $E$.
The general form in the region II is
\begin{equation*}
    \psi_{II}(x)=C e^{i k_2 x} + D e^{-i k_2 x}
\end{equation*}
For the case of $V_0<E$, the exponential is complex.
As such, we have the solution
\begin{equation*}
    \psi_{II}(x) = C e^{i k_2 x} + D e^{-i k_2 x}
\end{equation*}
For the case of $E<V_0$, the exponential is real.
As such, we throw the $D$ constant to avoid divergence as $x \rightarrow \infty$
\begin{equation*}
    \psi_{II}(x) =C e^{-\kappa x}
\end{equation*}

\subsubsection{Reflection and transmission probabilities.}
The continuity for wavefunction and its first derivative state
\begin{equation*}
    \begin{cases}
        \psi_I(0)  & =\psi_{II}(0)  \\
        \psi'_I(0) & =\psi'_{II}(0) \\
    \end{cases}
\end{equation*}
In the case of the step potential, this reads
\begin{equation*}
    \begin{cases}
        A+B       & =C       \\
        ik_1(A-B) & =  ik_2C
    \end{cases}
\end{equation*}
if $D=0$.
Plugging wavefunction condition into derivative and dividing by $i$ to solve $B$ in  terms of $A$
\begin{align*}
    k_1(A-B)      & =  k_2(A+B)                 \\
    (k_1 + k_2) B & =  (k_1 - k_2) A            \\
    B             & =  \frac{k_1-k_2 }{k_1+k_2}
\end{align*}
Now solve $C$ in terms of $A$ by substituting $B$ into wavefunction condition
\begin{equation*}
    C = A + B = A + \frac{k_1 - k_2}{k_1 + k_2} A = \frac{2 k_1}{k_1 + k_2} A
\end{equation*}

\subsection{Finite Potential Barrier}
Generalization of step potential such that the potential form a bump relative to the surroundings
\begin{equation*}
    V(x) =
    \begin{cases}
        0,   & x < 0         \\
        V_0, & 0 \le x \le L \\
        0,   & L < x
    \end{cases}
\end{equation*}

The wavefunction in this system is
\begin{equation*}
    \psi(x) =
    \begin{cases}
        A e^{i k_1 x} + B e^{-i k_1 x},   & x < 0                        \\[1mm]
        C e^{i k_2 x} + D e^{-i k_2 x},   & 0 \le x \le L, \quad E > V_0 \\[1mm]
        C e^{\kappa x} + D e^{-\kappa x}, & 0 \le x \le L, \quad E < V_0 \\[1mm]
        F e^{i k_1 x},                    & L<x
    \end{cases}
\end{equation*}
$A$ represents incident amplitude, $B$ reflected amplitude, $C$ and $D$ amplitudes inside the barrier (right/left moving or decaying/growing), $F$ represents transmitted amplitude.

Physically, if there is only one particle coming from the left, there is no source on the right.
Therefore, we set $G=0$.

For a finite potential barrier or well, there is generally no simple closed-form expression for reflection and transmission amplitude.
They still can be written as
\begin{equation*}
    r=\frac{B }{A }\qquad t=\frac{F }{A}
\end{equation*}
For the probabilities
\begin{equation*}
    R=|r|^2=\frac{|A|^2}{|B|^2}\qquad T=\frac{k_3 }{k_1 }|t^2|=\frac{|F|^2}{|A|^2}
\end{equation*}
\subsubsection{Derivation.}
Mostly the same as the infinite case.
In the incident region
\begin{align*}
    -\frac{\hbar^2}{2m} \frac{d^2 \psi_{I}}{dx^2} & =  E \psi_I        \\
    \frac{d^2 \psi_{I}}{dx^2}                     & = - k_1^2 \psi_{I}
\end{align*}
with
\begin{equation*}
    k_1^2 =\frac{2 m E}{\hbar^2}
\end{equation*}
The general solution is
\begin{equation*}
    \psi_I(x) = A e^{i k_1 x} + B e^{-i k_1 x}
\end{equation*}

Now, the region II
\begin{align*}
    -\frac{\hbar^2}{2m} \frac{d^2 \psi_{II}}{dx^2} + V_0 \psi_{II} & =  E \psi_{II}      \\
    \frac{d^2 \psi_{II}}{dx^2}                                     & = - k_2^2 \psi_{II}
\end{align*}
The value  of $k_2$ depends on the value of $E$
\begin{equation*}
    k_2 =
    \begin{cases}
        \sqrt{\dfrac{2 m (E - V_0)}{\hbar^2}}                           & V_0<E   \\
        i \kappa, \quad \kappa = \sqrt{\dfrac{2 m (V_0 - E)}{\hbar^2}}, & E < V_0
    \end{cases}
\end{equation*}
The general form in the region II is
\begin{equation*}
    \psi_{II}(x)=C e^{i k_2 x} + D e^{-i k_2 x}
\end{equation*}
For the case of $V_0<E$,
\begin{equation*}
    \psi_{II}(x) = C e^{i k_2 x} + D e^{-i k_2 x}
\end{equation*}
For the case of $E<V_0$,
\begin{equation*}
    \psi_{II}(x) =C e^{-\kappa x}+ + D e^{-i\kappa x}
\end{equation*}
Here we do not throw $D$ away since $x$ does not approach infinity.

Region III is the same as region I
\begin{equation*}
    \frac{d^2 \psi_{III}}{dx^2} = - k_3^2 \psi_{III}
\end{equation*}
with
\begin{equation*}
    k_3^2 =k_1=\frac{2 m E}{\hbar^2}
\end{equation*}
The general solution is
\begin{equation*}
    \psi_I(x) = F e^{i k_1 x} + G e^{-i k_1 x}
\end{equation*}
If there are no incident waves from right, we throw $G$ away

\subsection{Finite Potential Well}
Generalization of potential well such that the potential form a dip relative to the surroundings
\begin{equation*}
    V(x)=
    \begin{cases}
        0,   & |x|\leq L/2 \\
        V_0, & L/2 <|x|
    \end{cases}
\end{equation*}

For $E<V_0$, the wavefunction is
\begin{equation*}
    \psi^{\text{even}}(x) =
    \begin{cases}
        F e^{\kappa (x + L/2)}, \\
        C \cos(k x),            \\
        F e^{-\kappa (x - L/2)},
    \end{cases}
    \quad
    \psi^{\text{odd}}(x) =
    \begin{cases}
        - F e^{\kappa (x + L/2)}, & x < -L/2    \\
        D \sin(k x),              & |x| \le L/2 \\
        F e^{-\kappa (x - L/2)},  & x > L/2
    \end{cases}
\end{equation*}
For $V_0<E$,
\begin{align*}
    \psi^{\text{even}}(x) & =
    \begin{cases}
        A e^{i k_1 x} , & x < -L/     \\
        C \cos(k x),    & |x| \le L/2 \\
        A e^{-i k_1 x}, & x > L/2
    \end{cases}
    \\
    \psi^{\text{odd}}(x)  & =
    \begin{cases}
        A e^{i k_1 x},   & x < -L/2    \\
        D \sin(k x),     & |x| \le L/2 \\
        -A e^{i k_1 x} , & x > L/2
    \end{cases}
\end{align*}

We have used the parameter
\begin{equation*}
    k = \sqrt{\frac{2 m E}{\hbar^2}}, \quad
    k_1 =
    \begin{cases}
        \sqrt{\dfrac{2 m (E - V_0)}{\hbar^2}}                           & V_0<E   \\
        i \kappa, \quad \kappa = \sqrt{\dfrac{2 m (V_0 - E)}{\hbar^2}}, & E < V_0
    \end{cases}
\end{equation*}

Using this parameter, we express the transcendental equation to find the allowed energy eigenvalue.
With $E<V_0$ 
\begin{align*}
    \sqrt{E} \tan \left( \frac{L }{2 }\sqrt{\frac{2mE}{\hbar^2}} \right) &= \sqrt{V_0-E}\\
    \sqrt{E} \cot \left( \frac{L }{2 }\sqrt{\frac{2mE}{\hbar^2}} \right) &= -\sqrt{V_0-E}
\end{align*}
for even and odd parity respectively.
With $V_0<E$
\begin{align*}
    \sqrt{E} \tan \left( \frac{L }{2 }\sqrt{\frac{2mE}{\hbar^2}} \right) &= i\sqrt{E-V_0}\\
    \sqrt{E} \cot \left( \frac{L }{2 }\sqrt{\frac{2mE}{\hbar^2}} \right) &= -i\sqrt{E-V_0}
\end{align*}
for even and odd parity respectively.


\subsubsection{Derivation.}
In the region I, the Schrödinger equation reads
\begin{align*}
    -\frac{\hbar^2}{2 m} \frac{d^2 \psi_{I}}{dx^2} + V_0 \psi_{I} & =  E \psi_{I}   \\
    \frac{d^2 \psi_{I}}{dx^2}                                     & = k_1^2\psi_{I}
\end{align*}
with
\begin{equation*}
    k_1 =
    \begin{cases}
        \sqrt{\dfrac{2 m (E - V_0)}{\hbar^2}}                           & V_0<E   \\
        i \kappa, \quad \kappa = \sqrt{\dfrac{2 m (V_0 - E)}{\hbar^2}}, & E < V_0
    \end{cases}
\end{equation*}
The general solution is
\begin{equation*}
    \psi_{I}(x) = A e^{ik_1x} + B e^{-ik_1 x}
\end{equation*}

For the case of $V_0<E$,
\begin{equation*}
    \psi_{I}(x) = A e^{i k_1 x} + B e^{-i k_1 x}
\end{equation*}
Now enforcing odd parity $\psi(x)=-\psi(-x)$ and even  parity $\psi(x)=\psi(-x)$, we have
\begin{equation*}
    \psi_{III}^{\text{odd}}=-A e^{i k_1 x}
    \qquad
    \psi_{III}^{\text{odd}}=A e^{-i k_1 x}
\end{equation*}
where we have also considered single-sided scattering such that $B=0$.

For the case of $E<V_0$, we throw $A$ away since it diverges as $x \rightarrow -\infty$
\begin{equation*}
    \psi_{I}(x) = B e^{\kappa x}
\end{equation*}
Define a new amplitude at the boundary \(x = -L/2\) as $F = B e^{\kappa L/2}$, so that
\begin{equation*}
    \psi_{I}(x) = F e^{\kappa (x + L/2)}
\end{equation*}
when applying boundaries $\psi_{I}(-L/2) = F$.
Enforcing odd parity $\psi(x)=-\psi(-x)$ and even  parity $\psi(x)=\psi(-x)$, we have
\begin{equation*}
    \psi_{III}^{\text{odd}}(x) = -F e^{-\kappa (x - L/2)}
    \qquad
    \psi_{III}^{\text{even}}(x) = F e^{-\kappa (x - L/2)}
\end{equation*}
Now we can use this wavefunction for both odd and even parity, nothing would matter physically.
However, the $\psi_{III}^{\text{odd}}$ is negative; and by convention we want to have positive right wavefunction.
As such, we reversed the order, and choose
\begin{equation*}
    \psi_{III}(x) = F e^{-\kappa (x - L/2)}
\end{equation*}
as reference.
Now enforcing odd and even parity yield the region I wavefunction
\begin{equation*}
    \psi_{I}^{\text{odd}}(x) = -F e^{\kappa (x + L/2)},
    \qquad
    \psi_{I}^{\text{even}}(x) = F e^{\kappa (x + L/2)}
\end{equation*}

In the region II, the Schrödinger equation reads
\begin{align*}
    -\frac{\hbar^2}{2 m} \frac{d^2 \psi_{II}}{dx^2} & =  E \psi_{II}      \\
    \frac{d^2 \psi_{II}}{dx^2}                      & = - k_2^2 \psi_{II}
\end{align*}
with $k_2^2=\sqrt{2mE/\hbar^2}$.
The general solution is
\begin{equation*}
    \psi_{II}(x)=C\sin (k_2x)+D \cos (k_2x)
\end{equation*}
Enforcing the parity
\begin{equation*}
    \psi_{II}^{\text{odd}}(x) = C \sin (k_2x)
    \qquad
    \psi_{II}^{\text{even}}(x) =  D \cos (k_2x)
\end{equation*}

\subsubsection{Energy eigenvalues.}
Consider the case of $E<V_0$ first.
For even parity, continuity at $x=L/2$ demands
\begin{equation*}
    \begin{cases}
        F=C \cos \left(  kL/2 \right) \\
        -\kappa F =-C k \sin  \left( kL/2\right)
    \end{cases}
\end{equation*}
Divide the derivative continuity with the wavefunction continuity
\begin{equation*}
    k \tan \left( \frac{k L}{2} \right) =\kappa
\end{equation*}
For even parity,
\begin{equation*}
    \begin{cases}
        D \sin (kL/2)=F \\
        kD \cos (kL/2)=\kappa F
    \end{cases}
\end{equation*}
Divide the derivative continuity with the wavefunction continuity
\begin{equation*}
    k \cot \left( \frac{k L}{2} \right) =\kappa
\end{equation*}

Now consider the case $V_0<E$.
For even parity, continuity at $x=L/2$ demands
\begin{equation*}
    \begin{cases}
        Ae^{ik_1L/2}=C \cos (kL/2) \\
        -ik_1 Ae^{ik_1L/2}=-C k \sin  \left( kL/2\right)
    \end{cases}
\end{equation*}
Divide the derivative continuity with the wavefunction continuity
\begin{equation*}
    k \tan \left( \frac{k L}{2} \right) =ik_1
\end{equation*}
For even parity,
\begin{equation*}
    \begin{cases}
        D \sin (kL/2)=-Ae^{ik_1L/2} \\
        kD \cos (kL/2)=ik_1Ae^{ik_1L/2} 
    \end{cases}
\end{equation*}
Divide the derivative continuity with the wavefunction continuity
\begin{equation*}
    k \cot \left( \frac{k L}{2} \right) =ik_1
\end{equation*}

\end{document}