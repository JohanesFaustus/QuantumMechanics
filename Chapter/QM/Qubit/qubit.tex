\documentclass[../../../main.tex]{subfiles}
\begin{document}

\subsection{Bit}
\begin{figure}[b]
  \centering
  \normfigL{../../../Rss/QM/Qubit/QubitState.png}
  \caption{Figure: States of deterministic classical, probabilistic classical, and quantum bits.}
\end{figure}

A bit is, in essence, an abstraction of a two-level system.
It characterizes the state space of a system restricted to two mutually distinguishable alternatives, independent of the system’s physical realization.
Within this framework, three conceptually distinct types of bits may be identified: deterministic classical bit, probabilistic classical bit, and qubit.

The state of a classical bit can be described by a single binary value $\psi$, which can be equal to either 0 or 1.
The state can be indicated by a point in one of two positions, indicated by the two points labelled 0 and 1 (see figure).

Next consider probabilistic classical bit whose value is not known exactly, but is known to be either 0 or 1 with corresponding probabilities $p_0$ and $p_1$, which satisfies $p_0+p_1=1$.
We can represent these two probabilities by the 2-dimensional unit vector
\begin{equation*}
  \psi = \begin{bmatrix}
    p_0 \\p_1
  \end{bmatrix}
\end{equation*}
which does not need to belong to a Hilbert space.
In the figure, the probabilities $p_0$ and $p_1$ are represented by the position of a point on the line segment between the points representing 0 and 1.

The state of a quantum bit is described by a complex unit vector $\ket{\psi}$ in a 2-dimensional Hilbert space
\begin{equation*}
  | \psi \rangle = \cos\left( \frac{\theta}{2} \right) | 0 \rangle + e^{i \phi} \sin\left( \frac{\theta}{2} \right) | 1 \rangle
\end{equation*}
Such a state vector is often depicted as a point on the surface of a 3-dimensional sphere, known as the Bloch sphere,
Points on the surface of the Bloch sphere can also be expressed in Cartesian coordinates as
\begin{equation*}
  (x, y, z) = \left( \sin \theta \cos \varphi, \sin \theta \sin \varphi, \cos \theta \right)
\end{equation*}

\subsection{Quantum Gate}
In quantum computing, we refer to a unitary operator $U$ acting on a single-qubit as a 1-qubit (unitary) gate.
We can represent operators on the $2$-dimensional Hilbert space of a single qubit as $2 \times  2$ matrices.
More generally, an $n$-qubit system is described by a $2n$-dimensional Hilbert space, and operators acting on it are represented by $2^n \times 2^n$ matrices.

\subsubsection{Pauli gates.}
The Pauli gates $X$, $Y $, and $Z$ correspond to rotations about the $x$-, $y$- and $z$-axes of the Bloch sphere, respectively
\begin{align*}
  \sigma_0 \equiv I \equiv \begin{bmatrix}
                             1 & 0 \\
                             0 & 1
                           \end{bmatrix}
   &  & \sigma_1 \equiv \sigma_x \equiv X \equiv \begin{bmatrix}
                                                   0 & 1 \\
                                                   1 & 0
                                                 \end{bmatrix} \\
  \sigma_2 \equiv \sigma_y \equiv Y \equiv \begin{bmatrix}
                                             0 & -i \\
                                             i & 0
                                           \end{bmatrix}
   &  & \sigma_3 \equiv \sigma_z \equiv Z \equiv \begin{bmatrix}
                                                   1 & 0  \\
                                                   0 & -1
                                                 \end{bmatrix}
\end{align*}

The $X$ operator is also known as $\mathrm{NOT}$ whose action is to map $\ket{0}$ to $\ket{1}$ and  $\ket{1}$ to $\ket{0}$.
\begin{equation*}
  \mathrm{NOT} :  \begin{pmatrix} 1 \\ 0 \end{pmatrix} \rightarrow \begin{pmatrix} 0 \\ 1 \end{pmatrix}, \quad \begin{pmatrix} 0 \\ 1 \end{pmatrix} \rightarrow \begin{pmatrix} 1 \\ 0 \end{pmatrix}.
\end{equation*}
The \(Y\) operator combines a bit flip with a phase.
Its action maps \(\ket{0}\) to \(i\ket{1}\) and \(\ket{1}\) to \(-i\ket{0}\).
\begin{equation*}
  Y : \begin{pmatrix} 1 \\ 0 \end{pmatrix} \rightarrow \begin{pmatrix} 0 \\ i \end{pmatrix},
  \quad
  \begin{pmatrix} 0 \\ 1 \end{pmatrix} \rightarrow \begin{pmatrix} -i \\ 0 \end{pmatrix}.
\end{equation*}
The \(Z\) operator is known as the phase-flip operator. Its action is to leave \(\ket{0}\) unchanged while introducing a relative phase of \(-1\) on \(\ket{1}\).
\begin{equation*}
  Z : \begin{pmatrix} 1 \\ 0 \end{pmatrix} \rightarrow \begin{pmatrix} 1 \\ 0 \end{pmatrix},
  \quad
  \begin{pmatrix} 0 \\ 1 \end{pmatrix} \rightarrow \begin{pmatrix} 0 \\ -1 \end{pmatrix}.
\end{equation*}
The identity operator \(I\) acts trivially on the computational basis and leaves all states unchanged.
\begin{equation*}
  I : \begin{pmatrix} 1 \\ 0 \end{pmatrix} \rightarrow \begin{pmatrix} 1 \\ 0 \end{pmatrix},
  \quad
  \begin{pmatrix} 0 \\ 1 \end{pmatrix} \rightarrow \begin{pmatrix} 0 \\ 1 \end{pmatrix}.
\end{equation*}


\end{document}