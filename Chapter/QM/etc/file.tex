\documentclass[../../../main.tex]{subfiles}
\begin{document}
\subsection*{Braket notation}
\subsubsection*{Ket.} $\ket{\psi}$ represents quantum state. Written in matrix form as 
\begin{equation*}
    \ket{\psi}=\begin{pmatrix}
        \psi_0\\
        \psi_1\\
        \vdots\\
        \psi_n
    \end{pmatrix}
\end{equation*}
\subsubsection*{Bra.} $\bra{\psi}$ is the Hermitian conjugate (complex conjugate transpose) of the ket $\ket{\psi}$
\begin{equation*}
    \bra{\psi}=\begin{pmatrix}
        \psi_0&
        \psi_1&
        \ldots&
        \psi_n
    \end{pmatrix}
\end{equation*}

\subsubsection*{Inner Product.} Written
\begin{equation*}
    \braket{\phi|\psi}=\begin{cases}
        0, \text{ if orthogonal}\\
        1, \text{ if orthonormal}
    \end{cases}
\end{equation*}

\subsection*{Operator}
\subsubsection*{Position Operator.} Represents the position of a particle.
\begin{equation*}
    \hat{x}=x
\end{equation*}

\subsubsection*{Momentum Operator.} 
\begin{equation*}
    \hat{p}= -i\hbar\nabla
\end{equation*}

\subsubsection*{Energy Operator.} 
\begin{equation*}
    \hat{E}=i \hbar \frac{\partial}{\partial t}
\end{equation*}

\subsubsection*{Hamiltonian Operator.}
\begin{equation*}
    \hat{H}=-\frac{\hbar^2}{2m}\nabla^2 +V(x)
\end{equation*}
The Hamiltonian can be written in terms of ladder operators as:
\begin{equation*}
H = \hbar\omega \left( a^\dagger a + \frac{1}{2} \right)
\end{equation*}
Its action on the energy eigenstates $\ket{n}$ is given by:
\begin{equation*}
    H \ket{n} = E_n\ket{n}
\end{equation*}
where the energy eigenvalues are
\begin{equation*}
    E_n=\hbar\omega\left(n+\frac{1}{2}\right)
\end{equation*}

\subsubsection*{Creation operator.} Increases the system's energy, thus often said to be raising operator. Defined as 
\begin{equation*}
    a^\dagger = \frac{1}{\sqrt{2\hbar m \omega}} \left( m\omega x - i p \right)
\end{equation*}
\begin{equation*}
    a^\dagger=\begin{pmatrix}
    0 & 0 & 0 & 0 & \dots & 0 & \dots \\
    \sqrt{1} & 0 & 0 & 0 & \dots & 0 & \dots \\
    0 & \sqrt{2} & 0 & 0 & \dots & 0 & \dots \\
    0 & 0 & \sqrt{3} & 0 & \dots & 0 & \dots \\
    \vdots & \vdots & \vdots & \ddots & \ddots & \dots & \dots \\
    0 & 0 & 0 & \dots & \sqrt{n} & 0 & \dots & \\
    \vdots & \vdots & \vdots & \vdots & \vdots & \ddots & \ddots 
\end{pmatrix}
\end{equation*}
Its action on the energy eigenstates $\ket{n}$ is given by:
\begin{equation*}
a^\dagger \ket{n} = \sqrt{n+1} \ket{n+1}
\end{equation*}

\subsubsection*{Annihilation operator.} Decrease the system's energy, thus often said to be lowering operator. Defined as
\begin{equation*}
    a = \frac{1}{\sqrt{2\hbar m \omega}} \left( m\omega x + i p \right)
\end{equation*}
in matrix representation
\begin{equation*}
    a=\begin{pmatrix}
        0 & \sqrt{1} & 0 & 0 & \dots & 0 & \dots \\
        0 & 0 & \sqrt{2} & 0 & \dots & 0 & \dots \\
        0 & 0 & 0 & \sqrt{3} & \dots & 0 & \dots \\
        0 & 0 & 0 & 0 & \ddots & \vdots & \dots \\
        \vdots & \vdots & \vdots & \vdots & \ddots & \sqrt{n} & \dots \\
        0 & 0 & 0 & 0 & \dots & 0 & \ddots \\
        \vdots & \vdots & \vdots & \vdots & \vdots & \vdots & \ddots 
    \end{pmatrix}
\end{equation*}
Its action on the energy eigenstates $\ket{n}$ is given by:
\begin{equation*}
    a \ket{n} = \sqrt{n} \ket{n-1}
\end{equation*}

\subsection*{Commutator}
Commutator measures how much two physical quantities fail to be simultaneously measurable or well-defined. It is defined as 
\begin{equation*}
    [A,B]=AB-BA
\end{equation*}
If $[A,B]=0$, then $A$ and $B$ commute and can be simultaneously measured with arbitrary precision. If not, their measurement outcomes interfere with each other.

\subsection*{Expectation value}
\subsubsection*{Braket notation.}
\begin{equation*}
    \braket{\hat{A}}=\braket{\psi|\hat{A}|\psi}
\end{equation*}

\subsubsection*{Matrix notation.}
\begin{equation*}
    \braket{\hat{A}}=\psi^{\dagger}\hat{A}\psi
\end{equation*}

\subsubsection*{Integral notation.} If $\psi(x)$ is the wavefunction in the position representation
\begin{equation*}
    \braket{\hat{A}}=\int_{-\infty}^{\infty}\psi^*(x)\hat{A}\psi(x)\;dx
\end{equation*}

\subsection*{Normalization}
\subsubsection*{Braket notation.}
\begin{equation*}
    \braket{\psi|\psi}=1
\end{equation*}

\subsubsection*{Integral notation.} If $\psi(x)$ is the wavefunction in the position representation
\begin{equation*}
    \int_{-\infty}^{\infty}\psi^*(x)\hat{A}\psi(x)\;dx=1
\end{equation*}
\end{document}