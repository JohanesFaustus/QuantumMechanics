\documentclass[../../../main.tex]{subfiles}
\begin{document}
\subsection{First Postulate}
As the following
\begin{quotation}
    The state of the particle is represented by a vector $\ket{\psi(t)}$ in a Hilbert space.
\end{quotation}
Compare it with the classical mechanics postulate via Hamiltonian formalism
\begin{quotation}
    The state of a particle at any given time is specified by the two variables $x(t)$ and $p(t)$, i.e., as a point in a two-dimensional phase space.
\end{quotation}

\subsubsection{Discussion: general discussion.}
The first three tell us how the system is depicted at a given time, and the fourth specifies how this picture changes with time.
The first postulate states that a particle is described by a ket $\ket{\psi} $ in a Hilbert space which, you will recall, contains proper vectors normalizable to unity as well as improper vectors, normalizable only to the Dirac delta functions.

Now, a ket in a Hilbert space has in general an infinite number of components in a given basis. 
A classical particle has, at any given time, a definite position. 
One simply has to give this value of $x$ in specifying the state. 
A quantum particle, on the other hand, can take on any value of $x$ upon measurement and one must give the relative probabilities for all possible outcomes.

\subsection{Second Postulate}
As the following
\begin{quotation}
    The independent variables $x$ and $p$ of classical mechanics are represented by Hermitian operators $X$ and $P$ with the following matrix elements in the eigenbasis of $X$
    \begin{align*}
        \braket{x|X|x'}&=x\delta(x-x')\\
        \braket{x|P|x'}&=-i\hbar \delta'(x-x')
    \end{align*}

    The operators corresponding to dependent variables $\omega=\omega(x,p)$ are given Hermitian operators
    \begin{equation*}
        \Omega(X,P)=\omega \left( x\rightarrow X, p\rightarrow P \right) 
    \end{equation*}
    where it is understood that $\Omega$ is a function of $X$ and $P$ just like $\omega$ is a function of $x$ and $p$.
\end{quotation}
Compare it with the classical mechanics postulate via Hamiltonian formalism
\begin{quotation}
    Every dynamical variable $\omega$ is a function of $x$ and $p$, that is $\omega=\omega(x,p)$.
\end{quotation}

\subsubsection{Discussion: ambiguity of operator.}
The definition $        \Omega(X,P)=\omega \left( x\rightarrow X, p\rightarrow P \right) $ is ambiguous.
Suppose $\omega=xp$.
We do not know $\Omega=PX$ or $\Omega=XP$ since $px=xp$ classically.
The rule is to use symmetric sum
\begin{equation*}
    \Omega=\frac{XP+PX }{2}
\end{equation*}
which also render $\Omega$ Hermitian.

\subsubsection{Discussion: continuous spectrum of operator.}
In the case of continuous eigenvalues of $\Omega$, ket $\ket{\psi}$ expands as
\begin{equation*}
    \ket{\psi}=\int \ket{\omega}\braket{\omega|\psi}\;d\omega=\int \ket{\omega}\psi(\omega)\;d\omega
\end{equation*}
Since $\omega$ varies continuously, so will $\braket{\omega|\psi}=\psi(\omega)$, which imply that $\psi(\omega)$ is a smooth function.

$P(\omega)$ in this case refer to the probability density, in particular $P(\omega)\;d\omega$ is the probability of obtaining $\omega$ between $\omega$ and $\omega+d\omega$.
This definition must meet the unity definition
\begin{equation*}
    \int P(\omega)\;d\omega=\int \braket{\psi|\omega}\braket{\omega|\psi}\;d\omega=\braket{\psi|I\psi}=1
\end{equation*}

\subsection{Third Postulate}
As the following
\begin{quotation}
    If the particle is in a state $\ket{\psi}$, measurement of the variable corresponding to $\Omega$ will yield one of the eigenvalues $\omega$ with probability $P(\omega)\propto |\braket{\omega|\psi}|^2$. The state of the system will change from $\ket{\psi}$ to $\ket{\omega}$ as a result of the measurement.
\end{quotation}
Compare it with the classical mechanics postulate via Hamiltonian formalism
\begin{quotation}
    If the particle is in a state given by $x$ and $p$, the measurement of the variable $\omega$ will yield a value $\omega(x,p)$. The state will remain unaffected.
\end{quotation}

\subsubsection{Discussion: probability.}
The theory makes only probabilistic predictions for the result of a measurement of $\Omega$, which only have the possible values of its eigenvalues, which also all real by the postulate.
The postulate also state the relative probabilities, to get the absolute probability, we divide by the sum of all relative probabilities
\begin{equation*}
    P(\omega_i)=\frac{|\braket{\omega_i|\psi}|^2 }{\sum_j \braket{\omega_j | \psi}^2}=\frac{|\braket{\omega_i|\psi}|^2}{\braket{\psi|\psi}}
\end{equation*}
As a side note, the probability interpretation breaks down if $\ket{\psi}$ is an improper vector.

Suppose there exist state described by
\begin{equation*}
    \ket{\psi}=\frac{\alpha \ket{\omega_1 }+\ket{\omega_2}}{\left( |\alpha|^2+|\beta|^2 \right)^{1/2} }
\end{equation*}
Measurement can either yield $\omega_1$ or $\omega_2$ with the probabilities $|\alpha^2|/(\alpha^2+\beta^2)$ and $|\beta^2|/(\alpha^2+\beta^2)$ respectively.

\subsubsection{Discussion: probability for degenerate operator.}
Say that we have degenerate operator $\Omega$ with degenerate eigenvalues $\omega_1=\omega_2=\omega$ with orthonormal basis $\ket{\omega,1}$ and $\ket{\omega,2}$. 
Then the probability is 
\begin{equation*}
    P(\omega)=|\braket{\omega,1|\psi}|^2+|\braket{\omega,2|\psi}|^2
\end{equation*}
In terms of projection operator in this eigenspace
\begin{equation*}
    \mathbb{P }_\omega=\ket{\omega,1}\bra{\omega,1}+\ket{\omega,1 }\bra{\omega,2}
\end{equation*}
we have 
\begin{equation*}
    P(\omega)=\braket{\psi|\mathbb{P}_\omega|\psi}=\braket{\mathbb{P}_\omega \psi|\mathbb{P}_\omega \psi}
\end{equation*}

\subsubsection{Discussion: change of basis.}
Suppose our interest switch to measuring $\Lambda$ from the measurement of $\Omega$--note that this does not mean a successive measurement.
Recall that to obtain the probability of $\omega$, we need to $\ket{\psi}$ in the $\Omega$ basis.
To obtain the probability of $\lambda$, then, we can do the same; but there is no need to expand $\ket{\psi}$ in $\Lambda$ given we already expanded $\ket{\psi}$ in $\Omega$ basis.
Working in $\Omega$ basis, we have 
\begin{equation*}
    \ket{\psi }=\sum_i \ket{\omega_i } \braket{\omega_i |\psi}\qquad\text{and}\quad P(\omega_i)=|\braket{\omega_i|\psi}|^2
\end{equation*}
If we wanted the probability of $\lambda$, we can simply project $\ket{\psi}$ in the eigenvectors $\ket{\lambda}$
\begin{equation*}
    \braket{\lambda_i|\psi}=\sum_j \braket{\lambda_i|\omega_j}\braket{\omega_j|\psi}
\end{equation*}


\subsection{Fourth Postulate: Schrödinger equation}
As the following
\begin{quotation}
    The state vector $\ket{\psi}$ obeys the Schrödinger equation
    \begin{equation*}
        i\hbar \frac{d }{dt }\ket{\psi(t )}= H \ket{\psi(t)}
    \end{equation*}
    where $H(X,P)=\mathcal{H }(x\rightarrow X, p\rightarrow P  )$ is the quantum Hamiltonian operator and $\mathcal{H }$ is the Hamiltonian for the corresponding classical problem
\end{quotation}
Compare it with the classical mechanics postulate via Hamiltonian formalism
\begin{quotation}
The state variables change with time according to Hamilton's equations
\begin{equation*}
    \dot{x}=\frac{\partial \mathcal{H }}{\partial p}\qquad \dot{p }=-\frac{\partial \mathcal{H }}{\partial x}
\end{equation*}
\end{quotation}

\subsubsection{Discussion: Constructing Hamiltonian.}
Hamiltonian is constructed by making the substitution $\mathcal{H}(x,p) \rightarrow H(X,P)$, where $\mathcal{H }$ is the classical Hamiltonian for the same case.

Consider a harmonic oscillator.
The classical Hamiltonian is 
\begin{equation*}
    \mathcal{H }=\frac{p^2 }{2m }+\frac{1 }{2 }m\omega^2x^2
\end{equation*}
Thus the Hamiltonian for quantum harmonic oscillator is 
\begin{equation*}
    H=\frac{P^2 }{2m }+\frac{1 }{2 }m\omega^2X^2
\end{equation*}
or in three dimension
\begin{equation*}
    H=\frac{P_x^2+P_y^2+P_z^2}{2m  }+\frac{1 }{2}m\omega \left[ X^2+Y^2+Z^2 \right] 
\end{equation*}
assuming the force constant is the same in all directions.
It is the isotropic harmonic oscillator, meaning the restoring force has the same frequency $\omega$ in every spatial direction.
This is derived from the potential
\begin{equation*}
    V(x,y,z) = \frac{1}{2} m \omega^2 (x^2 + y^2 + z^2)
\end{equation*}
which generate the isotropic force 
\begin{equation*}
    \mathbf{F }= -m \omega^2 \mathbf{r}
\end{equation*}

Now consider a particle in one dimension is subject to a constant force $f$.
The classical Hamiltonian is 
\begin{equation*}
    \mathcal{H }=\frac{p^2 }{2m }-fx
\end{equation*}
Thus the quantum Hamiltonian  
\begin{equation*}
    H=\frac{P^2 }{2m }-fX
\end{equation*}

Now consider particle of charge $q$ in an electromagnetic field.
The classical Hamiltonian is 
\begin{equation*}
    \mathcal{H }=\frac{|\mathbf{p }-(q/c) \mathbf{A}(\mathbf{r},t)|^2 }{2m }+q\phi(\mathbf{r },t)
\end{equation*}
The Hamiltonian is constructed by using the symmetric form 
\begin{equation*}
    H=\frac{P^2 }{2m }\left( \mathbf{P }\cdot \mathbf{P }-\frac{q }{c }\mathbf{P }\cdot \mathbf{A}-\frac{q }{c}\mathbf{A}\cdot \mathbf{P}+\frac{q^2 }{c^2}\mathbf{A }\cdot \mathbf{A}\cdot \mathbf{A} \right) +q\phi
\end{equation*}

\subsubsection{Discussion: The propagator equation.}
Let us first assume that $H$ has no explicit $t$ dependence. In this case
\begin{equation*}
    i \hbar \ket{\dot{\psi}(t)}=H \ket{\psi(t)}
\end{equation*}
On using the propagator, the time evolution of the solution is written as 
\begin{equation*}
    \ket{\psi(t)}=U(t)\ket{\psi(0)}
\end{equation*}

\subsubsection{Discussion: The propagator expression.}
The propagator for time independent Hamiltonian can be constructed as 
\begin{equation*}
    U(t)=\sum_E \ket{E } \bra{E }e^{-iEt/\hbar}=\int \ket{E} \bra{E}e^{-iEt/\hbar}\;dE
\end{equation*}
Another expression is 
\begin{equation*}
    U(t)=e^{-iHt/\hbar}
\end{equation*}

The expression is derived as follows.
First consider the time independent Schrödinger equation that describe the action of Hamilton operator on energy eigenstate
\begin{equation*}
    H \ket{E}=E \ket{E}
\end{equation*}
We can expand the state in this basis 
\begin{equation*}
    \ket{\psi(t)}=\sum \ket{E }\braket{E|\psi(t)}=\sum a_E(t)\ket{E}
\end{equation*}
Operate both $\ket{\psi(t)}$ with $i \hbar d/dt-H$ to obtain the Schrödinger equation and more
\begin{align*}
    \left( i \hbar \frac{\partial }{\partial t }-H  \right) \ket{\psi(t)}&= i \hbar \ket{\dot{\psi}(t)}-H \ket{\psi(y)}\\
    &= i \hbar \sum \dot{a}_E \ket{E }- H\sum a_E \ket{E }\\
    &= i \hbar \sum \dot{a}_E \ket{E }- \sum a_E E\ket{E }\\
    0&= \sum \left( i \hbar \dot{a }_E-a_EE  \right) \ket{E}
\end{align*}
which imply
\begin{equation*}
   i \hbar \dot{a }_E=a_EE 
\end{equation*}
A simple differential equation
\begin{align*}
    \frac{d a_E }{dt }&= i \frac{E }{\hbar }a_E\\
    \ln a_E &= -i \frac{E }{\hbar }+C\\
    a_E(t)&= C e^{-i E t/\hbar}
\end{align*}
with $a_E(0)=0$, this makes 
\begin{equation*}
    a_E(t)= a_E(0)e^{-i E t/\hbar}
\end{equation*}

With the expression for $a_E(t)$, we can now expand 
\begin{equation*}
    \ket{\psi(t)}=\sum a_E(0)e^{-i E t/\hbar}\ket{E}
\end{equation*}
Now we project it onto $\bra{E }$ at $t=t$ and $t=0$
\begin{align*}
    \braket{E|\psi(t)}&= \bra{E }\sum a_E(0)e^{-i E t/\hbar}\ket{E}=a_E(0)e^{-i Et/\hbar}\\
    \braket{E|\psi(0)}&= \bra{E }\sum a_E(0)\ket{E}=a_E(0)
\end{align*}
or 
\begin{equation*}
    \braket{E|\psi(t)}=a_E(0)e^{-i Et/\hbar}=\braket{E|\psi(t)}e^{-iEt/\hbar}
\end{equation*}
Thus 
\begin{equation*}
    \ket{\psi(t)}=\ket{\psi(0)}e^{-iEt/\hbar}
\end{equation*}
On comparing this with the propagator equation
\begin{equation*}
    \ket{\psi(t)}=U(t)\ket{\psi(0)}    
\end{equation*}
We can written the propagator as 
\begin{equation*}
    U(t)=\sum_E \ket{E }\bra{E }e^{-iE/\hbar}=e^{-iEt/\hbar}
\end{equation*}

\subsection{Measurement}
Given state $(x,p)$, we can say that dynamical value has the value $\omega(x,p)$ in classical mechanics.
In quantum mechanics, when a particle is in a state $\ket{\psi } $ the particle can have the value of $\omega$ for the quantum operator $\Omega$ with probability $P(\omega)\propto|\braket{\omega|\psi}  |^2$. 
This value is obtained by the following method.
\begin{enumerate}
    \item Construct the corresponding quantum operator $\Omega=\omega(x \rightarrow ,p \rightarrow P)$ where $X$ and $P$ are the operators defined in postulate II.
    \item Find the orthonormal eigenvectors $\ket{\omega} $ and eigenvalues $\omega_i$ of $\Omega$.
    \item Expand $\ket{\psi } $ in this basis
    \begin{equation*}
        \ket{\psi }=\sum_i  \ket{\omega_i } \braket{\omega_i |\psi}  
    \end{equation*}
    \item The probability $P(\omega_i)$ that the result $\omega_i$ will be obtained is $P(\omega)\propto|\braket{\omega|\psi}  |^2$, or in terms of projection vector
    \begin{equation*}
        P(\omega)\propto \braket{\psi|\omega } \braket{\omega|\psi}=\braket{\psi|\mathbb{P}_\omega|\psi  }=\braket{\psi|\mathbb{P } _\omega \mathbb{P } _\omega |\psi   }=\braket{\mathbb{P } _\omega \psi| \mathbb{P } _\omega \psi  }    
    \end{equation*}
\end{enumerate}
particle of charge q in an electromagnetic field
\subsubsection{Collapse.}
Measurement by the operator $\Omega$ changes the state vector $\ket{\psi}$, which on this basis expands as 
\begin{equation*}
    \ket{\psi}=\sum_\omega \ket{\omega}\braket{\omega|\psi}
\end{equation*}
into the eigenstate $\ket{\omega}$ corresponding to the eigenvalue of $\omega$  obtained in measurement.

The effect of the measurement, then, can be represented as 
\begin{equation*}
    \ket{\psi}\xrightarrow{\Omega\text{ measured, }\omega\text{ obtained}}\frac{\mathbb{P}_\omega \ket{\omega }}{\braket{\mathbb{P}_\omega\psi|\mathbb{P }_\omega \psi}^{1/2}}
\end{equation*}
with $\mathbb{P }_\omega$ as the projection operator to $\ket{\omega}$. 
If the eigenvalues $\omega$ is degenerate, the $\mathbb{P }_\omega$ is the projection operator for the eigenspace $\mathbb{V}_\omega$. 

\subsubsection{Testing quantum theory.}
Unlike classical mechanics, quantum mechanics does not make deterministic predictions but statistical predictions.
To test the predictions of the quantum mechanics, we must be able to create well-defined state $\ket{\psi}$ and check the probabilistic predictions at any time.
The procedure are as follows.
\begin{enumerate}
    \item Begin with particle in an arbitrary state $\ket{\psi}$ and measure a variable $\Omega$.
    \item If we get nondegenerate value $\omega$, we have the state $\ket{\omega}$.
    \item We immediately thereafter measure variable $\Lambda$ next so that the state could not change form $\ket{\omega}$.
    If say the theory state 
    \begin{equation*}
        \ket{\omega}=\alpha \ket{\lambda_1}+\beta \ket{\lambda_2}
    \end{equation*}
    then the theory predict that $\lambda_1$ and $\lambda_2$ will be obtained with probability $\alpha^2$ and $\beta^2$ respectively.
\end{enumerate}
If the measurement give any other eigenvalue $\lambda_i$ or other noneigenvalue value, that is the end of the theory.
Suppose the resulting measurement give the correct eigenvalue, we can repeat the measurement with quantum ensemble of $N$ particles all in this same state $\ket{\omega}$.
The measurement should match with the predictions $N/\alpha^2$ particle in $\ket{\lambda_1}$ and $N/\beta^2$ in $\ket{\lambda_2}$.

\subsection{Information Inside Abstract State}
\subsubsection{Probabilities.}
The probability for obtaining $\omega$ for variable $\Omega$ is 
\begin{equation*}
    P(\omega)=|\braket{\omega|\psi}|^2
\end{equation*}
or continuous eigenvalues, the probability $\omega$ between $a$ and $b$.
\begin{equation*}
    P(a,b)=\int_{a }^{b }|\braket{\omega|\psi}|^2\;d\omega
\end{equation*}

\subsubsection{Expectation value.}
The expectation value is the mean value defined in statistics
\begin{equation*}
    \braket{\Omega}=\sum_i P(\omega_i)\omega_i=\braket{\psi|\Omega|\psi}
\end{equation*}
If the particle is an eigenstate of $\Omega$, that is $\Omega \ket{\psi}=\omega \ket{\psi}$, the $\braket{\Omega}=\omega$.

To see how the third term comes to be, we can expand the statistical definition as follows.
\begin{align*}
    \braket{\Omega }&= \sum_i \braket{\psi|\omega_i }\braket{\omega_i|\psi}\omega_i =\sum_i \bra{\psi}\omega_i \ket{\omega_i }\braket{\omega_i|\psi}=\sum_i \braket{\psi|\Omega|\omega_i}\braket{\omega_i|\psi}\\
    \braket{\Omega}&= \braket{\psi|\Omega|\psi} 
\end{align*}
where we have used the completion relation $\sum_i \ket{\omega_i}\bra{\omega_i }$.

For continuous variable, we write 
\begin{equation*}
    \braket{\Omega}=\int P(\omega)\omega\;d \omega=\int \braket{\psi|\omega}\omega \braket{\omega|\psi}\; d\omega
\end{equation*}

\subsubsection{Uncertainty.}
The uncertainty is the standard deviation defined in statistics, which measure the average fluctuation around the mean 
\begin{equation*}
    \Delta\Omega=\left(\sum_i P(\omega_i)\left( \omega_i -\braket{\Omega} \right)^2 \right)^{1/2}=\left\langle \left( \Omega-\braket{\Omega} \right)^2 \right\rangle  ^{1/2}
\end{equation*}
Another expression for the uncertainty is 
\begin{equation*}
    \Delta\Omega=\left( \braket{\Omega^2}-\braket{\Omega}^2 \right) ^{1/2}
\end{equation*}
This expression is related to the definition of standard deviation by expanding the square term 
\begin{align*}
    \Delta\Omega=\left\langle \left( \Omega-\braket{\Omega} \right)^2 \right\rangle  ^{1/2} &= \left\langle \Omega^2+\braket{\Omega}^2-2\Omega \braket{\Omega } \right\rangle^{1/2}\\
    &= \left( \braket{\Omega^2}+\braket{\Omega}^2-2 \braket{\Omega}^2 \right) ^{1/2}\\
    \Delta\Omega &= \left( \braket{\Omega^2}-\braket{\Omega}^2 \right) ^{1/2}
\end{align*} 

As it the case with the expectation value, the uncertainty for continuous variable is 
\begin{equation*}
    \Delta\Omega=\left( \int P(\omega)\left( \omega-\braket{\Omega}^2 \right)^2  \;d \omega\right) ^{1/2}
\end{equation*} 

\subsection{Compatibility}
The compatibility measure the degree of incompatibility between two observables; it is defined by the commutator
\begin{equation*}
    [\Omega\Lambda]=\Omega\Lambda-\Lambda\Omega
\end{equation*}

\subsubsection{Procedure.}
Here are the scheme by which the compatibility of two operators is measured.
Let us first measure $\Omega$ on the ensemble described by $\ket{\psi}$ and take the particles that yield a result $\omega$. 
We immediately measure $\Lambda$ and pick those particles that give a result $\lambda$.
\begin{equation*}
    \ket{\psi}\xrightarrow{\Omega=\omega}\ket{\omega}\xrightarrow{\Lambda=\lambda}\ket{\lambda}
\end{equation*}
Here the two operator are said to be incompatible since the second measurement change the state produced by the first measurement.

Opposite case occur if the first eigenstate $\ket{\omega }$ is also an eigenstate of the second operator $\Lambda$.
We denote this simultaneous eigenstates as $\ket{\omega\lambda}$.
\begin{equation*}
    \ket{\psi}\xrightarrow{\Omega=\omega}\ket{\omega\lambda}\xrightarrow{\Lambda=\lambda}\ket{\omega\lambda}
\end{equation*}
This eigenstate obey the following equations
\begin{align*}
    \Omega \ket{\omega\lambda}&= \omega \ket{\omega\lambda}\\
    \Lambda\ket{\omega\lambda}&= \lambda \ket{\omega\lambda}
\end{align*}
Now operate the first $\Lambda$ and the second by $\Omega$
\begin{align*}
    \Lambda\Omega \ket{\omega\lambda}&= \omega\lambda \ket{\omega\lambda}\\
    \Omega\Lambda\ket{\omega\lambda}&= \omega\lambda \ket{\omega\lambda}
\end{align*}
then subtract the second with the first
\begin{align*}
    (\Omega\Lambda-\Lambda\Omega)\ket{\omega\lambda}&= (\omega\lambda-\omega\lambda)\ket{\omega\lambda}\\
    [\Omega,\Lambda]\ket{\omega\lambda}&= 0 \ket{\omega\lambda}
\end{align*}
Thus $[\Omega,\Lambda]$ must have eigenkets with zero eigenvalue if simultaneous eigenkets are to exist.
\end{document}