\documentclass[../../../main.tex]{subfiles}
\begin{document}
\subsection{First Postulate}
As the following
\begin{quotation}
    The state of the particle is represented by a vector $\ket{\psi(t)}$ in a Hilbert space.
\end{quotation}
Compare it with the classical mechanics postulate via Hamiltonian formalism
\begin{quotation}
    The state of a particle at any given time is specified by the two variables $x(t)$ and $p(t)$, i.e., as a point in a two-dimensional phase space.
\end{quotation}

\subsubsection{Discussion: general discussion.}
The first three tell us how the system is depicted at a given time, and the fourth specifies how this picture changes with time.
The first postulate states that a particle is described by a ket $\ket{\psi} $ in a Hilbert space which, you will recall, contains proper vectors normalizable to unity as well as improper vectors, normalizable only to the Dirac delta functions.

Now, a ket in a Hilbert space has in general an infinite number of components in a given basis. 
A classical particle has, at any given time, a definite position. 
One simply has to give this value of $x$ in specifying the state. 
A quantum particle, on the other hand, can take on any value of $x$ upon measurement and one must give the relative probabilities for all possible outcomes.

\subsection{Second Postulate}
As the following
\begin{quotation}
    The independent variables $x$ and $p$ of classical mechanics are represented by Hermitian operators $X$ and $P$ with the following matrix elements in the eigenbasis of $X$
    \begin{align*}
        \braket{x|X|x'}&=x\delta(x-x')\\
        \braket{x|P|x'}&=-i\hbar \delta'(x-x')
    \end{align*}

    The operators corresponding to dependent variables $\omega=\omega(x,p)$ are given Hermitian operators
    \begin{equation*}
        \Omega(X,P)=\omega \left( x\rightarrow X, p\rightarrow P \right) 
    \end{equation*}
    where it is understood that $\Omega$ is a function of $X$ and $P$ just like $\omega$ is a function of $x$ and $p$.
\end{quotation}
Compare it with the classical mechanics postulate via Hamiltonian formalism
\begin{quotation}
    Every dynamical variable $\omega$ is a function of $x$ and $p$, that is $\omega=\omega(x,p)$.
\end{quotation}

\subsection{Third Postulate}
As the following
\begin{quotation}
    If the particle is in a state $\ket{\psi}$, measurement of the variable corresponding to $\Omega$ will yield one of the eigenvalues $\omega$ with probability $P(\omega)\propto |\braket{\omega|\psi}|^2$. The state of the system will change from $\ket{\psi}$ to $\ket{\omega}$ as a result of the measurement.
\end{quotation}
Compare it with the classical mechanics postulate via Hamiltonian formalism
\begin{quotation}
    If the particle is in a state given by $x$ and $p$, the measurement of the variable $\omega$ will yield a value $\omega(x,p)$. The state will remain unaffected.
\end{quotation}

\subsection{Fourth Postulate}
As the following
\begin{quotation}
    The state vector $\ket{\psi}$ obeys the Schrödinger equation
    \begin{equation*}
        i\hbar \frac{d }{dt }\ket{\psi(t )}= H \ket{\psi(t)}
    \end{equation*}
    where $H(X,P)=\mathcal{H }(x\rightarrow X, p\rightarrow P  )$ is the quantum Hamiltonian operator and $\mathcal{H }$ is the Hamiltonian for the corresponding classical problem
\end{quotation}
Compare it with the classical mechanics postulate via Hamiltonian formalism
\begin{quotation}
The state variables change with time according to Hamilton's equations
\begin{equation*}
    \dot{x}=\frac{\partial \mathcal{H }}{\partial p}\qquad \dot{p }=\frac{\partial \mathcal{H }}{\partial x}
\end{equation*}
\end{quotation}

\subsection{Measurement}
Given state $(x,p)$, we can say that dynamical value has the value $\omega(x,p)$ in classical mechanics.
In quantum mechanics, when a particle is in a state $\ket{\psi } $ the particle can have the value of $\omega$ for the quantum operator $\Omega$ with probability $P(\omega)\propto|\braket{\omega|\psi}  |^2$. 
This value is obtained by the following method.
\begin{enumerate}
    \item Construct the corresponding quantum operator $\Omega=\omega(x \rightarrow ,p \rightarrow P)$ where $X$ and $P$ are the operators defined in postulate II.
    \item Find the orthonormal eigenvectors $\ket{\omega} $ and eigenvalues $\omega_i$ of $\Omega$.
    \item Expand $\ket{\psi } $ in this basis
    \begin{equation*}
        \ket{\psi }=\sum_i  \ket{\omega_i } \braket{\omega_i |\psi}  
    \end{equation*}
    \item The probability $P(\omega_i)$ that the result $\omega_i$ will be obtained is $P(\omega)\propto|\braket{\omega|\psi}  |^2$, or in terms of projection vector
    \begin{equation*}
        P(\omega)\propto \braket{\psi|\omega } \braket{\omega|\psi}=\braket{\psi|\mathbb{P}_\omega|\psi  }=\braket{\psi|\mathbb{P } _\omega \mathbb{P } _\omega |\psi   }=\braket{\mathbb{P } _\omega \psi| \mathbb{P } _\omega \psi  }    
    \end{equation*}
\end{enumerate}
\end{document}