\documentclass[../../../main.tex]{subfiles}
\begin{document}
\subsection{First Postulate}
As the following
\begin{quotation}
    The state of the particle is represented by a vector $\ket{\psi(t)}$ in a Hilbert space.
\end{quotation}
Compare it with the classical mechanics postulate via Hamiltonian formalism
\begin{quotation}
    The state of a particle at any given time is specified by the two variables $x(t)$ and $p(t)$, i.e., as a point in a two-dimensional phase space.
\end{quotation}

\subsubsection{Discussion: general discussion.}
The first three tell us how the system is depicted at a given time, and the fourth specifies how this picture changes with time.
The first postulate states that a particle is described by a ket $\ket{\psi} $ in a Hilbert space which, you will recall, contains proper vectors normalizable to unity as well as improper vectors, normalizable only to the Dirac delta functions.

Now, a ket in a Hilbert space has in general an infinite number of components in a given basis. 
A classical particle has, at any given time, a definite position. 
One simply has to give this value of $x$ in specifying the state. 
A quantum particle, on the other hand, can take on any value of $x$ upon measurement and one must give the relative probabilities for all possible outcomes.

\subsection{Second Postulate}
As the following
\begin{quotation}
    The independent variables $x$ and $p$ of classical mechanics are represented by Hermitian operators $X$ and $P$ with the following matrix elements in the eigenbasis of $X$
    \begin{align*}
        \braket{x|X|x'}&=x\delta(x-x')\\
        \braket{x|P|x'}&=-i\hbar \delta'(x-x')
    \end{align*}

    The operators corresponding to dependent variables $\omega=\omega(x,p)$ are given Hermitian operators
    \begin{equation*}
        \Omega(X,P)=\omega \left( x\rightarrow X, p\rightarrow P \right) 
    \end{equation*}
    where it is understood that $\Omega$ is a function of $X$ and $P$ just like $\omega$ is a function of $x$ and $p$.
\end{quotation}
Compare it with the classical mechanics postulate via Hamiltonian formalism
\begin{quotation}
    Every dynamical variable $\omega$ is a function of $x$ and $p$, that is $\omega=\omega(x,p)$.
\end{quotation}

\subsubsection{Discussion: ambiguity of operator.}
The definition $        \Omega(X,P)=\omega \left( x\rightarrow X, p\rightarrow P \right) $ is ambiguous.
Suppose $\omega=xp$.
We do not know $\Omega=PX$ or $\Omega=XP$ since $px=xp$ classically.
The rule is to use symmetric sum
\begin{equation*}
    \Omega=\frac{XP+PX }{2}
\end{equation*}
which also render $\Omega$ Hermitian.

\subsubsection{Discussion: continuous spectrum of operator.}
In the case of continuous eigenvalues of $\Omega$, ket $\ket{\psi}$ expands as
\begin{equation*}
    \ket{\psi}=\int \ket{\omega}\braket{\omega|\psi}\;d\omega=\int \ket{\omega}\psi(\omega)\;d\omega
\end{equation*}
Since $\omega$ varies continuously, so will $\braket{\omega|\psi}=\psi(\omega)$, which imply that $\psi(\omega)$ is a smooth function.

$P(\omega)$ in this case refer to the probability density, in particular $P(\omega)\;d\omega$ is the probability of obtaining $\omega$ between $\omega$ and $\omega+d\omega$.
This definition must meet the unity definition
\begin{equation*}
    \int P(\omega)\;d\omega=\int \braket{\psi|\omega}\braket{\omega|\psi}\;d\omega=\braket{\psi|I\psi}=1
\end{equation*}

\subsection{Third Postulate}
As the following
\begin{quotation}
    If the particle is in a state $\ket{\psi}$, measurement of the variable corresponding to $\Omega$ will yield one of the eigenvalues $\omega$ with probability $P(\omega)\propto |\braket{\omega|\psi}|^2$. The state of the system will change from $\ket{\psi}$ to $\ket{\omega}$ as a result of the measurement.
\end{quotation}
Compare it with the classical mechanics postulate via Hamiltonian formalism
\begin{quotation}
    If the particle is in a state given by $x$ and $p$, the measurement of the variable $\omega$ will yield a value $\omega(x,p)$. The state will remain unaffected.
\end{quotation}

\subsubsection{Discussion: probability.}
The theory makes only probabilistic predictions for the result of a measurement of $\Omega$, which only have the possible values of its eigenvalues, which also all real by the postulate.
The postulate also state the relative probabilities, to get the absolute probability, we divide by the sum of all relative probabilities
\begin{equation*}
    P(\omega_i)=\frac{|\braket{\omega_i|\psi}|^2 }{\sum_j \braket{\omega_j | \psi}^2}=\frac{|\braket{\omega_i|\psi}|^2}{\braket{\psi|\psi}}
\end{equation*}
As a side note, the probability interpretation breaks down if $\ket{\psi}$ is an improper vector.

Suppose there exist state described by
\begin{equation*}
    \ket{\psi}=\frac{\alpha \ket{\omega_1 }+\ket{\omega_2}}{\left( |\alpha|^2+|\beta|^2 \right)^{1/2} }
\end{equation*}
Measurement can either yield $\omega_1$ or $\omega_2$ with the probabilities $|\alpha^2|/(\alpha^2+\beta^2)$ and $|\beta^2|/(\alpha^2+\beta^2)$ respectively.

\subsubsection{Discussion: probability for degenerate operator.}
Say that we have degenerate operator $\Omega$ with degenerate eigenvalues $\omega_1=\omega_2=\omega$ with orthonormal basis $\ket{\omega,1}$ and $\ket{\omega,2}$. 
Then the probability is 
\begin{equation*}
    P(\omega)=|\braket{\omega,1|\psi}|^2+|\braket{\omega,2|\psi}|^2
\end{equation*}
In terms of projection operator in this eigenspace
\begin{equation*}
    \mathbb{P }_\omega=\ket{\omega,1}\bra{\omega,1}+\ket{\omega,1 }\bra{\omega,2}
\end{equation*}
we have 
\begin{equation*}
    P(\omega)=\braket{\psi|\mathbb{P}_\omega|\psi}=\braket{\mathbb{P}_\omega \psi|\mathbb{P}_\omega \psi}
\end{equation*}

\subsubsection{Discussion: change of basis.}
Suppose our interest switch to measuring $\Lambda$ from the measurement of $\Omega$--note that this does not mean a successive measurement.
Recall that to obtain the probability of $\omega$, we need to $\ket{\psi}$ in the $\Omega$ basis.
To obtain the probability of $\lambda$, then, we can do the same; but there is no need to expand $\ket{\psi}$ in $\Lambda$ given we already expanded $\ket{\psi}$ in $\Omega$ basis.
Working in $\Omega$ basis, we have 
\begin{equation*}
    \ket{\psi }=\sum_i \ket{\omega_i } \braket{\omega_i |\psi}\qquad\text{and}\quad P(\omega_i)=|\braket{\omega_i|\psi}|^2
\end{equation*}
If we wanted the probability of $\lambda$, we can simply project $\ket{\psi}$ in the eigenvectors $\ket{\lambda}$
\begin{equation*}
    \braket{\lambda_i|\psi}=\sum_j \braket{\lambda_i|\omega_j}\braket{\omega_j|\psi}
\end{equation*}


\subsection{Fourth Postulate}
As the following
\begin{quotation}
    The state vector $\ket{\psi}$ obeys the Schrödinger equation
    \begin{equation*}
        i\hbar \frac{d }{dt }\ket{\psi(t )}= H \ket{\psi(t)}
    \end{equation*}
    where $H(X,P)=\mathcal{H }(x\rightarrow X, p\rightarrow P  )$ is the quantum Hamiltonian operator and $\mathcal{H }$ is the Hamiltonian for the corresponding classical problem
\end{quotation}
Compare it with the classical mechanics postulate via Hamiltonian formalism
\begin{quotation}
The state variables change with time according to Hamilton's equations
\begin{equation*}
    \dot{x}=\frac{\partial \mathcal{H }}{\partial p}\qquad \dot{p }=-\frac{\partial \mathcal{H }}{\partial x}
\end{equation*}
\end{quotation}

\subsection{Measurement}
Given state $(x,p)$, we can say that dynamical value has the value $\omega(x,p)$ in classical mechanics.
In quantum mechanics, when a particle is in a state $\ket{\psi } $ the particle can have the value of $\omega$ for the quantum operator $\Omega$ with probability $P(\omega)\propto|\braket{\omega|\psi}  |^2$. 
This value is obtained by the following method.
\begin{enumerate}
    \item Construct the corresponding quantum operator $\Omega=\omega(x \rightarrow ,p \rightarrow P)$ where $X$ and $P$ are the operators defined in postulate II.
    \item Find the orthonormal eigenvectors $\ket{\omega} $ and eigenvalues $\omega_i$ of $\Omega$.
    \item Expand $\ket{\psi } $ in this basis
    \begin{equation*}
        \ket{\psi }=\sum_i  \ket{\omega_i } \braket{\omega_i |\psi}  
    \end{equation*}
    \item The probability $P(\omega_i)$ that the result $\omega_i$ will be obtained is $P(\omega)\propto|\braket{\omega|\psi}  |^2$, or in terms of projection vector
    \begin{equation*}
        P(\omega)\propto \braket{\psi|\omega } \braket{\omega|\psi}=\braket{\psi|\mathbb{P}_\omega|\psi  }=\braket{\psi|\mathbb{P } _\omega \mathbb{P } _\omega |\psi   }=\braket{\mathbb{P } _\omega \psi| \mathbb{P } _\omega \psi  }    
    \end{equation*}
\end{enumerate}

\subsubsection{Collapse.}
Measurement by the operator $\Omega$ changes the state vector $\ket{\psi}$, which on this basis expands as 
\begin{equation*}
    \ket{\psi}=\sum_\omega \ket{\omega}\braket{\omega|\psi}
\end{equation*}
into the eigenstate $\ket{\omega}$ corresponding to the eigenvalue of $\omega$  obtained in measurement.

The effect of the measurement, then, can be represented as 
\begin{equation*}
    \ket{\psi}\xrightarrow{\Omega\text{ measured, }\omega\text{ obtained}}\frac{\mathbb{P}_\omega \ket{\omega }}{\braket{\mathbb{P}_\omega\psi|\mathbb{P }_\omega \psi}}
\end{equation*}
with $\mathbb{P }_\omega$ as the projection operator to $\ket{\omega}$. 
If the eigenvalues $\omega$ is degenerate, the $\mathbb{P }_\omega$ is the projection operator for the eigenspace $\mathbb{V}_\omega$. 
\end{document}