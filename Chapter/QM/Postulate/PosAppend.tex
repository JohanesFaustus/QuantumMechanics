\documentclass[../../../main.tex]{subfiles}
\begin{document}
\subsection{Appendix: Measurement Example}
\subsubsection{State collapse example.}
Suppose there exist  state described as 
\begin{equation*}
    \ket{\psi}=\frac{1 }{2}\ket{\omega,1}+\frac{1 }{2}\ket{\psi}+\sum_{i=3} c_i \ket{\omega_i}
\end{equation*}
It is also known that the orthonormal basis $\ket{\omega,1  }$ and $\ket{\omega,2}$ is degenerate and so do their eigenvalues $\omega_1=\omega_2=\omega$.
If measurement reveal the result $\omega$, the collapsed state is written as 
\begin{equation*}
    \ket{\psi}\xrightarrow{\omega\text{ obtained}}\frac{\mathbb{P}_\omega \ket{\omega }}{\braket{\mathbb{P}_\omega\psi|\mathbb{P }_\omega \psi}}^{1/2}
\end{equation*}
where the projection operator $\mathbb{P }_\omega$ for the eigenspace is written as 
\begin{equation*}
    \mathbb{P }\omega=\ket{\omega,1}\bra{\omega,1}+\ket{\omega,2}\bra{\omega,2}
\end{equation*} 
As such, its action on the state vector is 
\begin{equation*}
    \mathbb{P }\omega \ket{\psi}=\ket{\omega,1}\braket{\omega,1|\psi}+\ket{\omega,2}\braket{\omega,2|\psi}
    =\frac{1 }{2 }\ket{\omega,1}+\frac{1 }{2 }\ket{\omega,2}
\end{equation*}
Therefore
\begin{equation*}
    \ket{\psi}=\frac{1}{\left( 1/4+1/4 \right) ^{1/2}}\left(\frac{1 }{2 }\ket{\omega,1}+\frac{1 }{2 }\ket{\omega,2}\right)
    =\frac{1 }{\sqrt{2} }\ket{\omega,1}+\frac{1 }{\sqrt{2} }\ket{\omega,2}
\end{equation*}

\subsection{Appendix: Exercise 4.2.1 from Shankar's Book}
Consider the following operators on a Hilbert space $\mathbb{V}^3(C)$
\begin{equation*}
    L_x=\frac{1 }{\sqrt{2 }}\begin{bmatrix}
        0&1&0\\
        1&0&1\\
        0&1&0
    \end{bmatrix}\;
    L_y=\frac{1 }{\sqrt{2 }}\begin{bmatrix}
        0&-i&0\\
        i&0&-i\\
        0&i&0
    \end{bmatrix}\;
    L_z=\frac{1 }{\sqrt{2 }}\begin{bmatrix}
        1&0&0\\
        0&0&0\\
        0&0&-1
    \end{bmatrix}
\end{equation*}
\subsubsection{First question.}
What are the possible values one can obtain if $L_z$ is measured?

Since the $L_z$ is already diagonalized, the eigenvalues are its diagonal entries: $1,0,-1$.

\subsubsection{Second question.}
Take the state in which $L_z=1$. 
In this state what are $\braket{L_x}$, $\braket{L_x^2}$ and $\braket{\Delta L_x^2}$?

The state at which $L_z=1$ is written in braket notation as 
\begin{align*}
    L_z \ket{\psi}&=  1 \ket{\psi}\\
    \frac{1 }{\sqrt{2 }}\begin{bmatrix}
        1&0&0\\
        0&0&0\\
        0&0&-1\\
    \end{bmatrix}
    \ket{\psi} &= 
    \begin{bmatrix}
        1&&\\
        &1&\\
        &&1\\
    \end{bmatrix}
    \ket{\psi}
\end{align*}
It can be easily seen that the state in question is 
\begin{align*}
    \bra{\psi}=\begin{bmatrix}
        1&0&0
    \end{bmatrix}
\end{align*}

Then let us determine the expectation value of $L_x$
\begin{align*}
    \braket{L_x}=\braket{\psi|L_x|\psi}&= 
    \begin{bmatrix}
        1&0&0
    \end{bmatrix}
    \frac{1 }{\sqrt{2 }}\begin{bmatrix}
        &1&\\
        1&&1\\
        &1&\\
    \end{bmatrix}
    \begin{bmatrix}
        1\\
        0\\
        0\\
    \end{bmatrix}\\
    &= 
    \frac{1 }{\sqrt{2}}
    \begin{bmatrix}
        1&0&0
    \end{bmatrix}
    \begin{bmatrix}
        0\\
        1\\
        0\\
    \end{bmatrix}
    =
    0
\end{align*}

Then $\braket{L_x^2}$
\begin{align*}
    \braket{L_z^2}&= 
    \begin{bmatrix}
        1&0&0
    \end{bmatrix}
    \frac{1 }{2}
    \begin{bmatrix}
        &1&\\
        1&&1\\
        &1&\\
    \end{bmatrix}
    \begin{bmatrix}
        &1&\\
        1&&1\\
        &1&\\
    \end{bmatrix}
    \begin{bmatrix}
        1\\
        0\\
        0\\
    \end{bmatrix}\\
    &= 
    \frac{1 }{2}
    \begin{bmatrix}
        1&0&0
    \end{bmatrix}
    \begin{bmatrix}
        1&&1\\
        &2&\\
        1&&1\\
    \end{bmatrix}
    \begin{bmatrix}
        1\\
        0\\
        0\\
    \end{bmatrix}\\
    &= 
   \frac{1 }{2}
    \begin{bmatrix}
        1&0&0
    \end{bmatrix}
    \begin{bmatrix}
        1\\
        0\\
        1\\
    \end{bmatrix}
    =
    \frac{1 }{2 }
\end{align*}

And the uncertainty $\Delta L_x$
\begin{equation*}
    \Delta L_x =  \left( \braket{L_x^2}-\braket{L_x}^2 \right) ^{1/2 }=\left( \frac{1 }{2 } -0\right) ^{1/2 }=\frac{1 }{\sqrt{2 }}
\end{equation*}

\subsubsection{Third question.}
Find the normalized eigenstates and the eigenvalues of $L_x$ in $L_z$ basis.

First solve the eigenvalue problem 
\begin{align*}
    \left( L_x-\omega I  \right) \ket{\omega}&= 0
    \begin{bmatrix}
        -\omega & 1& \\
        1&-\omega &1 \\
        & 1&-\omega \\
    \end{bmatrix}
    \ket{\omega}=0
\end{align*}
We have the equation
\begin{equation*}
    \det
    \begin{pmatrix}
        -\omega & 1/\sqrt{2 }& \\
        1/\sqrt{2}&-\omega &1/\sqrt{2} \\
        & 1/\sqrt{2}&-\omega \\
    \end{pmatrix}
    =0
\end{equation*}
and the characteristic equation
\begin{align*}
    -\omega \left( \omega^2- \frac{1 }{2} \right) +\frac{\omega }{2}&= 0\\
    \omega \left( -\omega^2+1  \right) &= 0
\end{align*}
The eigenvalues are thus $\omega=0,\pm 1$.

Moving on to find the eigenstates.
First the eigenstate corresponding to the eigenvalue $\omega=1$
\begin{align*}
    \left( L_x -\omega I  \right) \ket{L_x=1}&= 0\\
    \begin{bmatrix}
        -1 & 1/\sqrt{2 }& \\
        1/\sqrt{2}&-1 &1/\sqrt{2} \\
        & 1/\sqrt{2}&-1 \\
    \end{bmatrix}
    \ket{L_x=1}&= 0\\
    \begin{bmatrix}
        -x+\frac{y }{\sqrt{2 }}\\
        \frac{x }{\sqrt{2 }}-y +\frac{z }{\sqrt{2 }}\\
        \frac{y }{\sqrt{2 }}-z
    \end{bmatrix}
    &= 0
    \begin{cases}
        \sqrt{2 }x=y\\
        1/2-y+y/2=0\\
        y=\sqrt{2 }z
    \end{cases}
\end{align*}
If we choose $x=1$, the eigenstate $\ket{L_x'=1}$ and its normalized state $\ket{L_x=1}$ are 
\begin{equation*}
    \ket{L_x'=1}=
    \begin{bmatrix}
        1\\
        \sqrt{2}\\
        1\\
    \end{bmatrix}
    \qquad
    \ket{L_x=1}=
    \frac{1}{\sqrt{1+2+1}}
    \begin{bmatrix}
        1\\
        \sqrt{2}\\
        1\\
    \end{bmatrix}
    =
        \begin{bmatrix}
        1/2\\
        \sqrt{2}/2\\
        1/2\\
    \end{bmatrix}
\end{equation*}

Then the eigenstate corresponding to $\omega=0$
\begin{align*}
    \left( L_x -\omega I  \right) \ket{L_x=0}&= 0\\
    \begin{bmatrix}
        0& 1/\sqrt{2 }& \\
        1/\sqrt{2}&0 &1/\sqrt{2} \\
        & 1/\sqrt{2}&0 \\
    \end{bmatrix}
    \ket{L_x=0}&= 0\\
    \begin{bmatrix}
        y/\sqrt{2}\\
        x/\sqrt{2 }+z/\sqrt{2 }\\
        y/\sqrt{2}
    \end{bmatrix}
    &= 0
    \begin{cases}
        y/\sqrt{2}=0\\
        x+z=0
    \end{cases}
\end{align*}
If we choose $x=1$, the normalized eigenstate is
\begin{equation*}
    \ket{L_x=1}=
    \frac{1}{\sqrt{1+1}}
    \begin{bmatrix}
        1\\
        0\\
        -1\\
    \end{bmatrix}
    =
        \begin{bmatrix}
        1/2\\
        0\\
        -1/2\\
    \end{bmatrix}
\end{equation*}

Then the eigenstate corresponding to $\omega=-1$
\begin{align*}
    \left( L_x -\omega I  \right) \ket{L_x=-1}&= 0\\
    \begin{bmatrix}
        1& 1/\sqrt{2 }& \\
        1/\sqrt{2}&1 &1/\sqrt{2} \\
        & 1/\sqrt{2}&1 \\
    \end{bmatrix}
    \ket{L_x=-1}&= 0\\
    \begin{bmatrix}
        x+y/\sqrt{2}\\
        x/\sqrt{2 }+y+z/\sqrt{2}=0\\
        y/\sqrt{2 }+z\\
    \end{bmatrix}
    &= 0
    \begin{cases}
        - \sqrt{2}x=y\\
        -y/2+y-y/2=0\\
        y=-\sqrt{2}z\\
    \end{cases}
\end{align*}
If we choose $x=1$, the normalized eigenstate is
\begin{equation*}
    \ket{L_x=1}=
    \frac{1}{\sqrt{1+1+2}}
    \begin{bmatrix}
        1\\
        -\sqrt{2}\\
        1\\
    \end{bmatrix}
    =
        \begin{bmatrix}
        1/2\\
        -1/\sqrt{2}\\
        1/2\\
    \end{bmatrix}
\end{equation*}

\subsubsection{Fourth question.}
If the particle is in the state with $L_z = -1$, and $L_x$ is measured, what are the possible outcomes and their probabilities?

State at which $L_z=-1$ can be obtained from 
\begin{align*}
    L_z \ket{\psi}&=-  1 \ket{\psi}\\
    \frac{1 }{\sqrt{2 }}\begin{bmatrix}
        1&0&0\\
        0&0&0\\
        0&0&-1\\
    \end{bmatrix}
    \ket{\psi} &= 
    \begin{bmatrix}
        -1&&\\
        &-1&\\
        &&-1\\
    \end{bmatrix}
    \ket{\psi}
\end{align*}
The state in question is 
\begin{equation*}
    \ket{\psi}=
    \begin{bmatrix}
        0\\0\\-1\\
    \end{bmatrix}
\end{equation*}

Since the eigenvalues of $L_x$ are $\omega=0,\pm 1$, their probabilities are 
\begin{align*}
    P(L_x=1) &= \left( 
    \begin{bmatrix}
        1/2&\sqrt{2}/2&2&1/2
    \end{bmatrix} 
    \begin{bmatrix}
        0\\0\\-1
    \end{bmatrix}
    \right) ^2=\frac{1 }{4}\\
    P(L_x=0) &= \left( 
    \begin{bmatrix}
        1/\sqrt{2}&0&2&-1/\sqrt{2}
    \end{bmatrix} 
    \begin{bmatrix}
        0\\0\\-1
    \end{bmatrix}
    \right) ^2=\frac{1 }{/2}\\
    P(L_x=-1) &= \left( 
    \begin{bmatrix}
        -1/2&0&2&-1/2
    \end{bmatrix} 
    \begin{bmatrix}
        0\\0\\-1
    \end{bmatrix}
    \right) ^2=\frac{1 }{4}\\
\end{align*}

\subsubsection{Fifth question.}
Consider the state, in bra to save space 
\begin{equation*}
    \bra{\psi}=\begin{bmatrix}
        \frac{1 }{2}&\frac{1 }{2}&\frac{1 }{\sqrt{2}}
    \end{bmatrix}
\end{equation*}
in the $L_z$ basis.
If $L_z^2$ is measured in this state and a result $+1$ is obtained, what is the state after the measurement? 
How probable was this result? 
If $L_z$ is measured, what are the outcomes and respective probabilities?

\subsubsection{Sixth question.}
A particle is in a state for which the probabilities are $P(L_z=1)=1/4$, $P(L_z=O)=1/2$, and $P(L_z =-1)= 1 /4$. 
Convince yourself that the most general, normalized state with this property is
\begin{equation*}
    \ket{\psi}=\frac{e^{i\delta_1 }}{2 }\ket{L_z=1}+\frac{e^{i \delta_2 }}{\sqrt{2 }}\ket{L_z=0}+\frac{e^{i \delta_3 }}{2}\ket{L_z=-1}
\end{equation*}
It was stated earlier on that if $\ket{\psi}$ is a normalized state then the state $e^{i\delta}\ket{y}$ is a physically equivalent normalized state. 
Does this mean that the factors $e^{i\delta}$ multiplying the $L_z$ eigenstates are irrelevant?
\end{document}