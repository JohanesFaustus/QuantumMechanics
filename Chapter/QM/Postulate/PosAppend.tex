\documentclass[../../../main.tex]{subfiles}
\begin{document}
\subsection{Appendix: Measurement Example}
\subsubsection{State collapse example.}
Suppose there exist  state described as
\begin{equation*}
    \ket{\psi}=\frac{1 }{2}\ket{\omega,1}+\frac{1 }{2}\ket{\psi}+\sum_{i=3} c_i \ket{\omega_i}
\end{equation*}
It is also known that the orthonormal basis $\ket{\omega,1  }$ and $\ket{\omega,2}$ is degenerate and so do their eigenvalues $\omega_1=\omega_2=\omega$.
If measurement reveal the result $\omega$, the collapsed state is written as
\begin{equation*}
    \ket{\psi}\xrightarrow{\omega\text{ obtained}}\frac{\mathbb{P}_\omega \ket{\omega }}{\braket{\mathbb{P}_\omega\psi|\mathbb{P }_\omega \psi}}^{1/2}
\end{equation*}
where the projection operator $\mathbb{P }_\omega$ for the eigenspace is written as
\begin{equation*}
    \mathbb{P }\omega=\ket{\omega,1}\bra{\omega,1}+\ket{\omega,2}\bra{\omega,2}
\end{equation*}
As such, its action on the state vector is
\begin{equation*}
    \mathbb{P }\omega \ket{\psi}=\ket{\omega,1}\braket{\omega,1|\psi}+\ket{\omega,2}\braket{\omega,2|\psi}
    =\frac{1 }{2 }\ket{\omega,1}+\frac{1 }{2 }\ket{\omega,2}
\end{equation*}
Therefore
\begin{equation*}
    \ket{\psi}=\frac{1}{\left( 1/4+1/4 \right) ^{1/2}}\left(\frac{1 }{2 }\ket{\omega,1}+\frac{1 }{2 }\ket{\omega,2}\right)
    =\frac{1 }{\sqrt{2} }\ket{\omega,1}+\frac{1 }{\sqrt{2} }\ket{\omega,2}
\end{equation*}

\subsection{Appendix: Exercise 4.2.1 from Shankar's Book}
Consider the following operators on a Hilbert space $\mathbb{V}^3(C)$
\begin{equation*}
    L_x=\frac{1 }{\sqrt{2 }}\begin{bmatrix}
        0 & 1 & 0 \\
        1 & 0 & 1 \\
        0 & 1 & 0
    \end{bmatrix}\;
    L_y=\frac{1 }{\sqrt{2 }}\begin{bmatrix}
        0 & -i & 0  \\
        i & 0  & -i \\
        0 & i  & 0
    \end{bmatrix}\;
    L_z= \begin{bmatrix}
        1 & 0 & 0  \\
        0 & 0 & 0  \\
        0 & 0 & -1
    \end{bmatrix}
\end{equation*}
\subsubsection{First question.}
What are the possible values one can obtain if $L_z$ is measured?

Since the $L_z$ is already diagonalized, the eigenvalues are its diagonal entries: $1,0,-1$.

\subsubsection{Second question.}
Take the state in which $L_z=1$.
In this state what are $\braket{L_x}$, $\braket{L_x^2}$ and $\braket{\Delta L_x^2}$?

The state at which $L_z=1$ is written in braket notation as
\begin{align*}
    L_z \ket{\psi} & =  1 \ket{\psi} \\
    \begin{bmatrix}
        1 & 0 & 0  \\
        0 & 0 & 0  \\
        0 & 0 & -1 \\
    \end{bmatrix}
    \ket{\psi}     & =
    \begin{bmatrix}
        1 &   &   \\
          & 1 &   \\
          &   & 1 \\
    \end{bmatrix}
    \ket{\psi}
\end{align*}
It can be easily seen that the state in question is
\begin{align*}
    \bra{\psi}=\begin{bmatrix}
                   1 & 0 & 0
               \end{bmatrix}
\end{align*}

Then let us determine the expectation value of $L_x$
\begin{align*}
    \braket{L_x}=\braket{\psi|L_x|\psi} & =
    \begin{bmatrix}
        1 & 0 & 0
    \end{bmatrix}
    \frac{1 }{\sqrt{2 }}\begin{bmatrix}
                              & 1 &   \\
                            1 &   & 1 \\
                              & 1 &   \\
                        \end{bmatrix}
    \begin{bmatrix}
        1 \\
        0 \\
        0 \\
    \end{bmatrix}                          \\
                                        & =
    \frac{1 }{\sqrt{2}}
    \begin{bmatrix}
        1 & 0 & 0
    \end{bmatrix}
    \begin{bmatrix}
        0 \\
        1 \\
        0 \\
    \end{bmatrix}
    =
    0
\end{align*}

Then $\braket{L_x^2}$
\begin{align*}
    \braket{L_z^2} & =
    \begin{bmatrix}
        1 & 0 & 0
    \end{bmatrix}
    \frac{1 }{2}
    \begin{bmatrix}
          & 1 &   \\
        1 &   & 1 \\
          & 1 &   \\
    \end{bmatrix}
    \begin{bmatrix}
          & 1 &   \\
        1 &   & 1 \\
          & 1 &   \\
    \end{bmatrix}
    \begin{bmatrix}
        1 \\
        0 \\
        0 \\
    \end{bmatrix}     \\
                   & =
    \frac{1 }{2}
    \begin{bmatrix}
        1 & 0 & 0
    \end{bmatrix}
    \begin{bmatrix}
        1 &   & 1 \\
          & 2 &   \\
        1 &   & 1 \\
    \end{bmatrix}
    \begin{bmatrix}
        1 \\
        0 \\
        0 \\
    \end{bmatrix}     \\
                   & =
    \frac{1 }{2}
    \begin{bmatrix}
        1 & 0 & 0
    \end{bmatrix}
    \begin{bmatrix}
        1 \\
        0 \\
        1 \\
    \end{bmatrix}
    =
    \frac{1 }{2 }
\end{align*}

And the uncertainty $\Delta L_x$
\begin{equation*}
    \Delta L_x =  \left( \braket{L_x^2}-\braket{L_x}^2 \right) ^{1/2 }=\left( \frac{1 }{2 } -0\right) ^{1/2 }=\frac{1 }{\sqrt{2 }}
\end{equation*}

\subsubsection{Third question.}
Find the normalized eigenstates and the eigenvalues of $L_x$ in $L_z$ basis.

First solve the eigenvalue problem
\begin{align*}
    \left( L_x-\omega I  \right) \ket{\omega} & = 0
    \begin{bmatrix}
        -\omega & 1       &         \\
        1       & -\omega & 1       \\
                & 1       & -\omega \\
    \end{bmatrix}
    \ket{\omega}=0
\end{align*}
We have the equation
\begin{equation*}
    \det
    \begin{pmatrix}
        -\omega    & 1/\sqrt{2 } &            \\
        1/\sqrt{2} & -\omega     & 1/\sqrt{2} \\
                   & 1/\sqrt{2}  & -\omega    \\
    \end{pmatrix}
    =0
\end{equation*}
and the characteristic equation
\begin{align*}
    -\omega \left( \omega^2- \frac{1 }{2} \right) +\frac{\omega }{2} & = 0 \\
    \omega \left( -\omega^2+1  \right)                               & = 0
\end{align*}
The eigenvalues are thus $\omega=0,\pm 1$.

Moving on to find the eigenstates.
First the eigenstate corresponding to the eigenvalue $\omega=1$
\begin{align*}
    \left( L_x -\omega I  \right) \ket{L_x=1} & = 0 \\
    \begin{bmatrix}
        -1         & 1/\sqrt{2 } &            \\
        1/\sqrt{2} & -1          & 1/\sqrt{2} \\
                   & 1/\sqrt{2}  & -1         \\
    \end{bmatrix}
    \ket{L_x=1}                               & = 0 \\
    \begin{bmatrix}
        -x+\frac{y }{\sqrt{2 }}                      \\
        \frac{x }{\sqrt{2 }}-y +\frac{z }{\sqrt{2 }} \\
        \frac{y }{\sqrt{2 }}-z
    \end{bmatrix}
                                              & = 0
    \begin{cases}
        \sqrt{2 }x=y \\
        1/2-y+y/2=0  \\
        y=\sqrt{2 }z
    \end{cases}
\end{align*}
If we choose $x=1$, the eigenstate $\ket{L_x'=1}$ and its normalized state $\ket{L_x=1}$ are
\begin{equation*}
    \ket{L_x'=1}=
    \begin{bmatrix}
        1        \\
        \sqrt{2} \\
        1        \\
    \end{bmatrix}
    \qquad
    \ket{L_x=1}=
    \frac{1}{\sqrt{1+2+1}}
    \begin{bmatrix}
        1        \\
        \sqrt{2} \\
        1        \\
    \end{bmatrix}
    =
    \begin{bmatrix}
        1/2        \\
        \sqrt{2}/2 \\
        1/2        \\
    \end{bmatrix}
\end{equation*}

Then the eigenstate corresponding to $\omega=0$
\begin{align*}
    \left( L_x -\omega I  \right) \ket{L_x=0} & = 0 \\
    \begin{bmatrix}
        0          & 1/\sqrt{2 } &            \\
        1/\sqrt{2} & 0           & 1/\sqrt{2} \\
                   & 1/\sqrt{2}  & 0          \\
    \end{bmatrix}
    \ket{L_x=0}                               & = 0 \\
    \begin{bmatrix}
        y/\sqrt{2}              \\
        x/\sqrt{2 }+z/\sqrt{2 } \\
        y/\sqrt{2}
    \end{bmatrix}
                                              & = 0
    \begin{cases}
        y/\sqrt{2}=0 \\
        x+z=0
    \end{cases}
\end{align*}
If we choose $x=1$, the normalized eigenstate is
\begin{equation*}
    \ket{L_x=1}=
    \frac{1}{\sqrt{1+1}}
    \begin{bmatrix}
        1  \\
        0  \\
        -1 \\
    \end{bmatrix}
    =
    \begin{bmatrix}
        1/2  \\
        0    \\
        -1/2 \\
    \end{bmatrix}
\end{equation*}

Then the eigenstate corresponding to $\omega=-1$
\begin{align*}
    \left( L_x -\omega I  \right) \ket{L_x=-1} & = 0 \\
    \begin{bmatrix}
        1          & 1/\sqrt{2 } &            \\
        1/\sqrt{2} & 1           & 1/\sqrt{2} \\
                   & 1/\sqrt{2}  & 1          \\
    \end{bmatrix}
    \ket{L_x=-1}                               & = 0 \\
    \begin{bmatrix}
        x+y/\sqrt{2}               \\
        x/\sqrt{2 }+y+z/\sqrt{2}=0 \\
        y/\sqrt{2 }+z              \\
    \end{bmatrix}
                                               & = 0
    \begin{cases}
        - \sqrt{2}x=y \\
        -y/2+y-y/2=0  \\
        y=-\sqrt{2}z  \\
    \end{cases}
\end{align*}
If we choose $x=1$, the normalized eigenstate is
\begin{equation*}
    \ket{L_x=1}=
    \frac{1}{\sqrt{1+1+2}}
    \begin{bmatrix}
        1         \\
        -\sqrt{2} \\
        1         \\
    \end{bmatrix}
    =
    \begin{bmatrix}
        1/2         \\
        -1/\sqrt{2} \\
        1/2         \\
    \end{bmatrix}
\end{equation*}

\subsubsection{Fourth question.}
If the particle is in the state with $L_z = -1$, and $L_x$ is measured, what are the possible outcomes and their probabilities?

State at which $L_z=-1$ can be obtained from
\begin{align*}
    L_z \ket{\psi} & =-  1 \ket{\psi} \\
    \begin{bmatrix}
        1 & 0 & 0  \\
        0 & 0 & 0  \\
        0 & 0 & -1 \\
    \end{bmatrix}
    \ket{\psi}     & =
    \begin{bmatrix}
        -1 &    &    \\
           & -1 &    \\
           &    & -1 \\
    \end{bmatrix}
    \ket{\psi}
\end{align*}
The state in question is
\begin{equation*}
    \ket{\psi}=
    \begin{bmatrix}
        0 \\0\\-1\\
    \end{bmatrix}
\end{equation*}

Since the eigenvalues of $L_x$ are $\omega=0,\pm 1$, their probabilities are
\begin{align*}
    P(L_x=1)  & = \left(
    \begin{bmatrix}
            1/2 & \sqrt{2}/2 & 2 & 1/2
        \end{bmatrix}
    \begin{bmatrix}
            0 \\0\\-1
        \end{bmatrix}
    \right) ^2=\frac{1 }{4}  \\
    P(L_x=0)  & = \left(
    \begin{bmatrix}
            1/\sqrt{2} & 0 & 2 & -1/\sqrt{2}
        \end{bmatrix}
    \begin{bmatrix}
            0 \\0\\-1
        \end{bmatrix}
    \right) ^2=\frac{1 }{/2} \\
    P(L_x=-1) & = \left(
    \begin{bmatrix}
            -1/2 & 0 & 2 & -1/2
        \end{bmatrix}
    \begin{bmatrix}
            0 \\0\\-1
        \end{bmatrix}
    \right) ^2=\frac{1 }{4}  \\
\end{align*}

\subsubsection{Fifth question.}
Consider the state, in bra to save space
\begin{equation*}
    \bra{\psi}=\begin{bmatrix}
        \frac{1 }{2} & \frac{1 }{2} & \frac{1 }{\sqrt{2}}
    \end{bmatrix}
\end{equation*}
in the $L_z$ basis.
If $L_z^2$ is measured in this state and a result $+1$ is obtained, what is the state after the measurement?
How probable was this result?
If $L_z$ is measured, what are the outcomes and respective probabilities?

The square of $L_z$ operator is
\begin{equation*}
    L_z=\frac{1 }{2}
    \begin{bmatrix}
        1 & 0 & 0  \\
        0 & 0 & 0  \\
        0 & 0 & -1
    \end{bmatrix}
    \begin{bmatrix}
        1 & 0 & 0  \\
        0 & 0 & 0  \\
        0 & 0 & -1
    \end{bmatrix}
    =
    \begin{bmatrix}
        1 & 0 & 0 \\
        0 & 0 & 0 \\
        0 & 0 & 1
    \end{bmatrix}
\end{equation*}
with eigenvalue of 0 and degenerate value of 1.

The eigenstate corresponding the degenerate eigenvalue is
\begin{align*}
    (L_x^2-\omega I)\ket{L_x^2=1} & = 0 \\
    \begin{bmatrix}
        0 & 0  & 0 \\
        0 & -1 & 0 \\
        0 & 0  & 0
    \end{bmatrix}
    \ket{L_x^2=1}                 & = 0 \\
    \begin{bmatrix}
        0 & -y & 0
    \end{bmatrix}
    =0
\end{align*}
With arbitrary choices of $x$ and $z$, we choose such that the two eigenstate are orthonormal to each other
\begin{equation*}
    \ket{L_x^2=1,1}=
    \begin{bmatrix}
        1 \\0\\0
    \end{bmatrix}\qquad
    \ket{L_x^2=1,2}=
    \begin{bmatrix}
        0 \\0\\1
    \end{bmatrix}
\end{equation*}

Now the eigenstate corresponding to the eigenvalue of 0
\begin{align*}
    (L_x^2-\omega I) \ket{L_x^2} & = 0 \\
    \begin{bmatrix}
        1 &   &   \\
          & 0 &   \\
          &   & 1
    \end{bmatrix}
    \ket{L_x^2}                  & = 0 \\
    \begin{bmatrix}
        x \\0\\z
    \end{bmatrix}
                                 & = 0
\end{align*}
Since $y$ is arbitrary, we choose so that it normalize the eigenstate
\begin{equation*}
    \ket{L_x^2=0}=
    \begin{bmatrix}
        0 \\1\\0
    \end{bmatrix}
\end{equation*}

From each eigenstate, we can construct the state vector in terms of the eigenstates
\begin{align*}
    \ket{\psi } & =  \sum_{i} \ket{L_x^2=i}\braket{L_x^2=1|\psi}                                            \\
                & = \frac{1 }{2 }\ket{L_x^2=0}+\frac{1 }{2 }\ket{L_x^2,1}+\frac{1 }{\sqrt{2 }}\ket{L_x^2,2}
\end{align*}
It can be easily seen that the probability of measuring the eigenstates $P(L_x^2=1)=3/4$

We know that the state changes after the measurement of the degenerate eigenvalue.
To find changed state, we construct the projection operator on this eigenspace
\begin{equation*}
    \mathbb{P }_1=\sum_i \ket{L_x^2,i}\bra{L_x^2,i} =
    \begin{bmatrix}
        1 &   &   \\
          & 0 &   \\
          &   & 0 \\
    \end{bmatrix}
    +
    \begin{bmatrix}
        0 &   &   \\
          & 0 &   \\
        1 &   & 0 \\
    \end{bmatrix}
\end{equation*}
So its action on the state is
\begin{align*}
    P_1 \ket{\psi} & =   \begin{bmatrix}
                             1 &   &   \\
                               & 0 &   \\
                               &   & 0
                         \end{bmatrix}
    \begin{bmatrix}
        \frac{1 }{2} \\ \frac{1 }{2} \\ \frac{1 }{\sqrt{2}}
    \end{bmatrix}
    +
    \begin{bmatrix}
        0 &   &   \\
          & 0 &   \\
        1 &   & 0
    \end{bmatrix}
    \begin{bmatrix}
        \frac{1 }{2} \\ \frac{1 }{2} \\ \frac{1 }{\sqrt{2}}
    \end{bmatrix}                   \\
                   & =
    \begin{bmatrix}
        1/2 \\0\\0
    \end{bmatrix}
    +
    \begin{bmatrix}
        0 \\0\\1/\sqrt{2 }
    \end{bmatrix}                                                                   \\
                   & = \frac{1 }{2 }\ket{L_x^2=1,1}+\frac{1 }{\sqrt{2 }}\ket{L_x^2=1,2}
\end{align*}

\subsubsection{Sixth question.}
A particle is in a state for which the probabilities are $P(L_z=1)=1/4$, $P(L_z=O)=1/2$, and $P(L_z =-1)= 1 /4$.
Convince yourself that the most general, normalized state with this property is
\begin{equation*}
    \ket{\psi}=\frac{e^{i\delta_1 }}{2 }\ket{L_z=1}+\frac{e^{i \delta_2 }}{\sqrt{2 }}\ket{L_z=0}+\frac{e^{i \delta_3 }}{2}\ket{L_z=-1}
\end{equation*}
It was stated earlier on that if $\ket{\psi}$ is a normalized state then the state $e^{i\delta}\ket{y}$ is a physically equivalent normalized state.
Does this mean that the factors $e^{i\delta}$ multiplying the $L_z$ eigenstates are irrelevant?

No.
The vectors $\ket{\psi}$ and $e^{i\theta}\ket{\psi}$ are physically equivalent only in the sense that they generate the same probability distribution for any observable.
This does not mean that when the vector $\psi$ appears as a part of a linear combination it can be multiplied by an arbitrary phase factor

\subsection{Appendix: Shankar's Example to Determine the Expectation Value and Uncertainty of Position Operator}
Suppose that the wave function is $X$ basis $\braket{x|\psi}=\psi(x)$ is a Gaussian, that is
\begin{equation*}
    \psi(x)=A \exp \left[ \-\frac{(X-a)^2 }{2\Delta^2} \right]
\end{equation*}
The normalization of $\psi(x )$ is
\begin{equation*}
    \psi(x)=\frac{1 }{(\pi\Delta^2)^{1/4}}\exp \left[ -\frac{(x-a )^2}{2\Delta^2} \right]
\end{equation*}

Form this, the particle is most likely to be found around $x =a$, and chances of finding it away from this point drop rapidly beyond a distance $\Delta$.
Mathematically
\begin{equation*}
    \braket{X}=a,\quad\Delta X=\frac{\Delta}{\sqrt{2}}
\end{equation*}

\subsubsection{Normalization.}
From the  unity of eigenstate,
\begin{align*}
    \braket{\psi|\psi} & = \int_{-\infty}^{\infty} \braket{\psi|x}\braket{x|\psi}\;dx                    \\
                       & = \int_{-\infty}^{\infty} A^2\exp \left[ -\frac{(x-a)^2}{\Delta^2} \right] \;dx \\
                       & = A^2 \sqrt{\frac{\pi }{1/\Delta^2}}=A^2 \sqrt{\pi \Delta^2}
\end{align*}
As such, $A=(\pi\Delta^2)^{-1/4}$.

\subsubsection{Expectation value.}
With the now normalized state, we can determine the expectation  value of position operator $X$.
The definition of expectation value give
\begin{align*}
    \braket{\psi|X|\psi} & = \int_{-\infty}^{\infty} \braket{\psi|X|x}\braket{x|\psi}\;dx                                                   \\
                         & = \int_{-\infty}^{\infty} \left(\int_{-\infty}^{\infty}  \braket{x|X|x'}\braket{x'|\psi}\;dx' \right)\psi(x)\;dx \\
                         & = \int_{-\infty}^{\infty} \left(\int_{-\infty}^{\infty}  x\delta(x-x')\psi(x')\;dx' \right)\psi(x)\;dx           \\
                         & = \int_{-\infty}^{\infty} \left(\int_{-\infty}^{\infty}  \delta(x'-x)x\psi(x')\;dx' \right)\psi(x)\;dx           \\
                         & = \int_{-\infty}^{\infty} \psi ^*(x)x\psi(x)                                                                     \\
                         & = \frac{1 }{(\pi \Delta^2)^{1/2}}\int_{-\infty}^{\infty} x \exp\left[ -\frac{(x-a)^2}{\Delta^2} \right] dx       \\
                         & = \frac{1 }{(\pi\Delta^2)^{1/2}}\int_{-\infty}^{\infty} (x+a)\exp \left( -\frac{u^2 }{\Delta^2} \right) du       \\
    \braket{X}           & =\frac{1 }{(\pi\Delta^2)^{1/2}} \left( 0+a (\pi\Delta)^{1/2} \right) =a
\end{align*}

\subsubsection{Uncertainty.}
Using the expression for uncertainty
\begin{equation*}
    \Delta\Omega=\left( \braket{\Omega^2}-\braket{\Omega}^2 \right) ^{1/2}
\end{equation*}
we need to determine the expectation value of the square of position operator.
We simply need to evaluate
\begin{align*}
    \braket{X^2} & = \frac{1 }{(\pi\Delta^2)^{1/2}} \int_{-\infty}^{\infty} \psi ^*(x)x^2\psi(x)\;dx                                   \\
                 & = \frac{1 }{(\pi\Delta^2)^{1/2}}\int_{-\infty}^{\infty} (u^2+2ua+a^2)\exp \left( -\frac{u ^2}{\Delta^2} \right)\;dx \\
    \braket{X^2} & = \frac{\Delta^2 }{2}+0+a^2
\end{align*}
So
\begin{equation*}
    \Delta X=\left[ \frac{\Delta^2}{2}+a^{2}-a^{2} \right]^{1/2}=\frac{\Delta}{\sqrt{2}}
\end{equation*}

\subsection{Appendix: Shankar's Example to Determine the Expectation Value and Uncertainty of Momentum Operator}
The solution of the Dirac delta $\braket{x|p}=\psi_p(x)$ is
\begin{equation*}
    \psi_p(x)=\frac{1 }{\sqrt{2\pi \hbar}}e^{ipx/\hbar}
\end{equation*}

The expectation value and uncertainty the of momentum operator is
\begin{equation*}
    \braket{P}=0,\quad\Delta P=\frac{\hbar}{\Delta\sqrt{2}}
\end{equation*}


\subsection{Appendix: Simple System of Free Particle}
The Schrödinger equation in this case
\begin{equation*}
    i \hbar \ket{\dot{\psi}}=H \ket{\psi}=\frac{p^2 }{2m }\ket{\psi}
\end{equation*}
The normal mode are the solution in the form of $\ket{\psi }= \ket{E }e^{-iE t}/\hbar$.
Feeding the normal mode into the Schrödinger equation
\begin{equation*}
    H \ket{E}=\frac{P^2 }{2m }\ket{E}=E \ket{E}
\end{equation*}
Since $P$ is also an eigenstate of $P^2 $, we feed the trial solution $\ket{p}$
\begin{align*}
    \frac{P^2 }{2m }\ket{p }                    & =  E \ket{p } \\
    \left( \frac{p^2 }{2m }-E  \right) \ket{p } & = 0
\end{align*}
Since $\ket{p}$ is not a null vector, therefore
\begin{equation*}
    p=\pm \sqrt{2mE}
\end{equation*}
The energy eigenvalue $E$ is doubly degenerate because two distinct momentum eigenvalues $p=\pm \sqrt{2mE}$.

Suppose we expand the Hamiltonian in $X$ basis
\begin{equation*}
    -\frac{\hbar^2 }{2m }\frac{d^2 }{dx } \psi(x)=E\psi
\end{equation*}
or
\begin{equation*}
    \frac{d^2 }{dx^2}\psi+k^2\psi=0
\end{equation*}
with
\begin{equation*}
    k^2=\frac{2mE }{\hbar^2}
\end{equation*}
The solution is
\begin{equation*}
    \psi(x)=Ae^{ikx}+Be^{-ikx}
\end{equation*}
From $E=p^2/2m$, we also obtain the relation $p=\hbar k$.

\subsection{Appendix: Simple System of Step Potential}
\subsubsection{Infinite Potential Well.}
Consider the one--dimensional infinite square well of width \(L\).
The potential shape is
\begin{equation*}
    V(x)=
    \begin{cases}
        0,      & |x|<L/2      \\
        \infty, & |x| \leq L/2
    \end{cases}
\end{equation*}
The wave function in this system takes the form
\begin{equation*}
    \psi_n^{\text{odd}}(x) = \sqrt{\frac{2  }{L }} \sin\left(\frac{2 n \pi x}{L}\right)
    \qquad
    \psi_n^{\text{even}}(x) =\sqrt{\frac{2 }{L }} \cos\left(\frac{(2n-1)\pi x}{L}\right)
\end{equation*}
We also have the energy eigenvalue
\begin{equation*}
    E_n^{\text{even}} = \frac{\hbar^2}{2 m} \left[\frac{(2n-1)\pi}{L}\right]^2, \qquad
    E_n^{\text{odd}} = \frac{\hbar^2}{2 m} \left(\frac{2 n \pi}{L}\right)^2
\end{equation*}

This is the part where we derive it.
We begin by partitioning space into three regions I, II, and III.
According to the Schrödinger equation, the wave function outside the well is simply zero since the potential is infinite.
\begin{equation*}
    -\frac{\hbar^2 }{2m }\frac{d^2 }{dx^2 }\psi_I+\infty \psi_I=E\psi,\implies \psi_I(x)=\psi_{III}=0
\end{equation*}

Next we move to region $II$.
The Schrödinger equation now reads
\begin{align*}
    -\frac{\hbar^2}{2m} \frac{d^2 \psi(x)}{dx^2} & =  E \psi(x)   \\
    \frac{d^2 \psi(x)}{dx^2}                     & = -k^2 \psi(x)
\end{align*}
with $k^2 =2 m E/\hbar^2.$
Recall the boundaries condition of
\begin{equation*}
    \psi \left( -\frac{L }{2 } \right) =\psi \left( \frac{L }{2 } \right) =0
\end{equation*}
Because the potential is symmetric, the Hamiltonian commutes with the parity operator $[H,\Pi] = 0$.
For even parity, we have
\begin{equation*}
    \psi(-x) = \psi(x) \implies B = 0, \quad \psi(x) = A \cos(kx)
\end{equation*}
Applying the second boundaries condition
\begin{equation*}
    \cos\left(\frac{k L}{2}\right) = 0\implies k^{\text{even}} = \frac{(2n-1)\pi}{L},
\end{equation*}
This yield the even parity wave function
\begin{equation*}
    \psi_n^{\text{even}}(x) =A \cos\left(\frac{(2n-1)\pi x}{L}\right)
\end{equation*}
Next for odd parity
\begin{equation*}
    \psi(-x) = -\psi(x) \implies A = 0, \quad \psi(x) = B \sin(kx)
\end{equation*}
Applying the first boundaries condition
\begin{equation*}
    \sin\left(\frac{k L}{2}\right) = 0 \implies k^{\text{odd}} = \frac{2 n \pi}{L}
\end{equation*}
This yield the odd parity wave function
\begin{equation*}
    \psi_n^{\text{odd}}(x) = B \sin\left(\frac{2 n \pi x}{L}\right)
\end{equation*}

Next we begin the normalization step.
To normalize $\psi_n^{\text{even}}$, we perform
\begin{align*}
    \int_{-L/2}^{L/2} \sin^2\left(\frac{2 n \pi x}{L}\right) dx & =  \int_0^{L/2} \left[1 - \cos\left(\frac{4 n \pi x}{L}\right)\right] dx                 \\
                                                                & = \frac{L }{2 }+ \frac{\sin\left(\frac{4 n \pi x}{L}\right)}{4 n \pi / L} \Bigg|_0^{L/2} \\
                                                                & = \frac{L }{2 }+\frac{\sin(2 n \pi) - \sin 0}{4 n \pi / L}                               \\
    \int_{-L/2}^{L/2} \sin^2\left(\frac{2 n \pi x}{L}\right) dx & =\frac{L }{2}                                                                            \\
\end{align*}
which meant
\begin{equation*}
    B^2 \frac{L }{2 }=1\implies B=\frac{2 }{L}
\end{equation*}
Then we normalize $\psi_n^{\text{odd}}$
\begin{align*}
    \int_{-L/2}^{L/2} \cos^2\left(\frac{(2n-1) \pi x}{L}\right) dx & =  \int_0^{L/2} \left[1 + \cos\left(\frac{2 (2n-1) \pi x}{L}\right)\right] dx                   \\
                                                                   & = \frac{L }{2 }+\frac{\sin\left(\frac{2 (2n-1) \pi x}{L}\right)}{2(2n-1)\pi / L} \Bigg|_0^{L/2} \\
                                                                   & = \frac{L }{2 }+ \frac{\sin((2n-1)\pi)-\sin0}{2(2n-1)\pi / L}                                   \\
    \int_{-L/2}^{L/2} \cos^2\left(\frac{(2n-1) \pi x}{L}\right) dx & = \frac{L }{2 }                                                                                 \\
\end{align*}
which meant
\begin{equation*}
    A^2 \frac{L }{2 }=1\implies A=\frac{2 }{L}
\end{equation*}

Using the expression for $k^{\text{odd}}$ and $k^{\text{even}}$, we can obtain the expression for energy eigenvalue for both parity.

\subsubsection{Infinite Potential Well: Antisymmetric Case.}
The potential is in the shape
\begin{equation*}
    V(x)=
    \begin{cases}
        \infty, & x \leq0  \\
        0,      & 0<x<L    \\
        \infty, & L \leq L
    \end{cases}
\end{equation*}
With wave function in the form
\begin{equation*}
    \psi(x)=\begin{cases}
        0                                          & x\leq 0   \\
        \sqrt{\dfrac{2}{L}}\sin( \dfrac{n\pi}{L}x) & 0 < x < L \\
        0                                          & L \leq x
    \end{cases}
\end{equation*}
and energy eigenvalue
\begin{equation*}
    E_n=\frac{\pi^2\hbar^2}{2mL^2}n^2
\end{equation*}
This is different from the symmetric potential case.
Even if the “shape” or magnitude of the potential is similar, changing its symmetry changes the mathematical constraints on eigenfunctions, which directly affects the allowed energies.
In quantum mechanics, small changes in the Hamiltonian can produce different spectra, because energies are eigenvalues of the operator, not just classical measures of potential height.

The derivation begin with setting the wave function at infinite potential as zero $\psi(x \leq 0)=\psi(L\leq x)=0$.
In $0 < x < L$, where $ V = 0$,
\begin{align*}
    \frac{d^2\psi}{dx^2} & = -\frac{2mE}{\hbar^2}\psi \\
    \frac{d^2\psi}{dx^2} & = -k^2\psi                 \\
\end{align*}
with $k^2 =2 m E/\hbar^2$.
The solution reads
\begin{equation*}
    \psi(x)=A\sin (kx)+ B \cos (kx)
\end{equation*}
Imposing boundary condition $\psi(x)=0$, we see that $B=0$
\begin{equation*}
    \psi(x)=A\sin (kx)
\end{equation*}
Imposing the second boundary condition $\psi(L)=0$, and considering that sinus function goes to zero at $n\pi$,
\begin{equation*}
    kL=n\pi
\end{equation*}
Normalization also requires $A = \sqrt{2/L}$.

\subsubsection{3D Infinite Potential Well.}
We assume antisymmetric 3D potential
\begin{equation*}
    V(x,y,z) =
    \begin{cases}
        0,      & 0 < x < L_x, \ 0 < y < L_y, \ 0 < z < L_z, \\
        \infty, & \text{otherwise}.
    \end{cases}
\end{equation*}

The full normalized 3D wavefunction is
\begin{equation*}
    \psi_{nml}(x,y,z) = \sqrt{\frac{8}{L_x L_y L_z}} \,
    \sin\left(\frac{n \pi x}{L_x}\right)
    \sin\left(\frac{m \pi y}{L_y}\right)
    \sin\left(\frac{l \pi z}{L_z}\right).
\end{equation*}
We derive this by the method of separation variable, that is assuming $\psi(x,y,z) = X(x) Y(y) Z(z)$, where
\begin{align*}
    X_n(x) & = \sqrt{\frac{2}{L_x}} \sin\left(\frac{n \pi x}{L_x}\right) \\
    Y_m(y) & = \sqrt{\frac{2}{L_y}} \sin\left(\frac{m \pi y}{L_y}\right) \\
    Z_l(z) & = \sqrt{\frac{2}{L_z}} \sin\left(\frac{l \pi z}{L_z}\right)
\end{align*}

The energy eigenvalue is
\begin{equation*}
    E_{nml} =\frac{\hbar^2 \pi^2}{2 m} \left( \frac{n^2}{L_x^2} + \frac{m^2}{L_y^2} + \frac{l^2}{L_z^2} \right)
\end{equation*}
For a cubic box \(L_x = L_y = L_z = L\), the energies simplify to
\begin{equation*}
    E_{nml} = \frac{\hbar^2 \pi^2}{2 m L^2} (n^2 + m^2 + l^2).
\end{equation*}
The total energy $E$ is the sum of energies from each dimension $= E_x + E_y + E_z$.
The expression for each dimension is that of one dimensional case
\begin{equation*}
    E_i=\frac{\hbar^2 \pi^2 }{2mL_i^2}i^2
\end{equation*}

The derivation is as follows.
Outside the box, the wave function is zero; while inside, the wave function according to Schrödinger equation
\begin{equation*}
    -\frac{\hbar^2}{2m} \nabla^2 \psi(x,y,z) = E \psi(x,y,z)
\end{equation*}
First, by separation variable, we assume the solution in the form
\begin{equation*}
    \psi(x,y,z) = X(x) Y(y) Z(z).
\end{equation*}
With the second differentiation of
\begin{align*}
    \frac{\partial^2 \psi}{\partial x^2} & = Y(y) Z(z) X''(x), \\
    \frac{\partial^2 \psi}{\partial y^2} & = X(x) Z(z) Y''(y), \\
    \frac{\partial^2 \psi}{\partial z^2} & = X(x) Y(y) Z''(z).
\end{align*}
The Laplacian becomes
\begin{equation*}
    \nabla^2 \psi = Y Z X'' + X Z Y'' + X Y Z''.
\end{equation*}
Substituting into the Schrödinger equation and dividing both sides by \(\psi = X Y Z\) gives
\begin{equation*}
    -\frac{\hbar^2}{2m} \left( \frac{X''}{X} + \frac{Y''}{Y} + \frac{Z''}{Z} \right) = E.
\end{equation*}
Each term depends on only one variable, which allows the equation to be separated into the Schrödinger equation gives
\begin{equation*}
    -\frac{\hbar^2}{2m} \left( \frac{X''(x)}{X(x)} + \frac{Y''(y)}{Y(y)} + \frac{Z''(z)}{Z(z)} \right) = E,
\end{equation*}
This leads to separate equations for each dimension
\begin{align*}
    -\frac{\hbar^2}{2m} X''(x) & = E_x X(x), \\
    -\frac{\hbar^2}{2m} Y''(y) & = E_y Y(y), \\
    -\frac{\hbar^2}{2m} Z''(z) & = E_z Z(z),
\end{align*}
with $E_i$ as the constant of separation.
Similar to 1D case, the solution to each equation is
\begin{align*}
    X_n(x) & = \sqrt{\frac{2}{L_x}} \sin\left(\frac{n \pi x}{L_x}\right) \\
    Y_m(y) & = \sqrt{\frac{2}{L_y}} \sin\left(\frac{m \pi y}{L_y}\right) \\
    Z_l(z) & = \sqrt{\frac{2}{L_z}} \sin\left(\frac{l \pi z}{L_z}\right)
\end{align*}

Each dimension contributes independently to the expression of energy
\begin{align*}
    E_x & = \frac{\hbar^2 \pi^2 n^2}{2 m L_x^2}, \\
    E_y & = \frac{\hbar^2 \pi^2 m^2}{2 m L_y^2}, \\
    E_z & = \frac{\hbar^2 \pi^2 l^2}{2 m L_z^2}.
\end{align*}
The total energy is
\begin{equation*}
    E_{nml} = E_x + E_y + E_z = \frac{\hbar^2 \pi^2}{2 m} \left( \frac{n^2}{L_x^2} + \frac{m^2}{L_y^2} + \frac{l^2}{L_z^2} \right)
\end{equation*}

\subsubsection{Step Potential.}
The system has the potential at the shape
\begin{equation*}
    V(x) =
    \begin{cases}
        0,   & x < 0,   \\
        V_0, & 0\leq x.
    \end{cases}
\end{equation*}

As usual, the wavefunction depends on the region and the value of $E$
\begin{equation*}
    \psi(x)=
    \begin{cases}
        A e^{i k_1 x} + B e^{-i k_1 x} & x <0                \\
        C e^{i k_2 x} + D e^{-i k_2 x} & 0 \leq x,\; V_0 < E \\
        C e^{-\kappa x}                & 0 \leq x,\;E < V_0
    \end{cases}
\end{equation*}
where
\begin{equation*}
    k_1^2 =\frac{2 m E}{\hbar^2}    ,
    \qquad
    k_2 =
    \begin{cases}
        \sqrt{\dfrac{2 m (E - V_0)}{\hbar^2}}                           & V_0<E   \\
        i \kappa, \quad \kappa = \sqrt{\dfrac{2 m (V_0 - E)}{\hbar^2}}, & E < V_0
    \end{cases}
\end{equation*}

In the case of $E<V_0$, our wave function reads
\begin{equation*}
    \psi(x)=Ae^{ik_1x}+Be^{-ik_1x}+Ce^{-\kappa x}
\end{equation*}
$Ae^{ik_1x}$ corresponds to a wave incident from the left, traveling toward the potential step, while $Be^{-ik_1x}$ corresponds to a reflected wave, moving backward, away from the step.
These are oscillatory plane waves, representing the particle moving freely in the $x<0$.
$Ce^{-\kappa x}$ represents an evanescent wave, i.e., the tunneling tail inside the classically forbidden region.

Unlike the case of $E<V_0$, particle can move freely in the case of $V_0<E$.
The wavefunction in the form
\begin{equation*}
    \psi(x)=Ae^{ik_1x}+Be^{-ik_1x}+Ce^{ik_2 x}+De^{-ik_2x}
\end{equation*}
represents this.
$Ae^{ik_1x}$ represents particle incident form left, $Be^{-ik_1x}$ represents it being reflected back, and $Ce^{ik_2 x}$ represents it being transmitted.
$De^{-ik_2x}$ represents wave incident from right, although we often set $D=0$ since there is no particle incident from right.

The absolute magnitude of incident wave $A$ is arbitrary because multiplying the wavefunction by a constant does not change the reflection $R$ or transmission probabilities $T$
\begin{equation*}
    R=\frac{|B|^2}{|A|^2}=\left|\frac{k_1-k_2 }{k_1+k_2}\right|^2
    \qquad
    T = \frac{k_2}{k_1} \frac{|C|^2}{|A|^2} = \frac{k_2}{k_1} \left| \frac{2 k_1}{k_1 + k_2} \right|^2
\end{equation*}
Since both are probabilities, $R+T=1$ should apply.
We define $r=B/A$ and $t=C/A$ as the reflection and transmission amplitude respectively.

We shall now derive it.
We begin by dividing the regions into I and II.
At the region I, the Schrödinger equation reads
\begin{align*}
    -\frac{\hbar^2}{2m} \frac{d^2 \psi_{I}}{dx^2} & =  E \psi_I        \\
    \frac{d^2 \psi_{I}}{dx^2}                     & = - k_1^2 \psi_{I}
\end{align*}
with
\begin{equation*}
    k_1^2 =\frac{2 m E}{\hbar^2}
\end{equation*}
The general solution is
\begin{equation*}
    \psi_I(x) = A e^{i k_1 x} + B e^{-i k_1 x}
\end{equation*}

Now we move into region II.
The Schrödinger equation now reads
\begin{align*}
    -\frac{\hbar^2}{2m} \frac{d^2 \psi_{II}}{dx^2} + V_0 \psi_{II} & =  E \psi_{II}      \\
    \frac{d^2 \psi_{II}}{dx^2}                                     & = - k_2^2 \psi_{II}
\end{align*}
The value  of $k_2$ depends on the value of $E$
\begin{equation*}
    k_2 =
    \begin{cases}
        \sqrt{\dfrac{2 m (E - V_0)}{\hbar^2}}                           & V_0<E   \\
        i \kappa, \quad \kappa = \sqrt{\dfrac{2 m (V_0 - E)}{\hbar^2}}, & E < V_0
    \end{cases}
\end{equation*}
It also follows that the solution also depends on $E$.
The general form in the region II is
\begin{equation*}
    \psi_{II}(x)=C e^{i k_2 x} + D e^{-i k_2 x}
\end{equation*}
For the case of $V_0<E$, the exponential is complex.
As such, we have the solution
\begin{equation*}
    \psi_{II}(x) = C e^{i k_2 x} + D e^{-i k_2 x}
\end{equation*}
For the case of $E<V_0$, the exponential is real.
As such, we throw the $D$ constant to avoid divergence as $x \rightarrow \infty$
\begin{equation*}
    \psi_{II}(x) =C e^{-\kappa x}
\end{equation*}

\subsubsection{Reflection and transmission probabilities.}
The continuity for wavefunction and its first derivative state
\begin{equation*}
    \begin{cases}
        \psi_I(0)  & =\psi_{II}(0)  \\
        \psi'_I(0) & =\psi'_{II}(0) \\
    \end{cases}
\end{equation*}
In the case of the step potential, this reads
\begin{equation*}
    \begin{cases}
        A+B       & =C       \\
        ik_1(A-B) & =  ik_2C
    \end{cases}
\end{equation*}
if $D=0$.
Plugging wavefunction condition into derivative and dividing by $i$ to solve $B$ in  terms of $A$
\begin{align*}
    k_1(A-B)      & =  k_2(A+B)                 \\
    (k_1 + k_2) B & =  (k_1 - k_2) A            \\
    B             & =  \frac{k_1-k_2 }{k_1+k_2}
\end{align*}
Now solve $C$ in terms of $A$ by substituting $B$ into wavefunction condition
\begin{equation*}
    C = A + B = A + \frac{k_1 - k_2}{k_1 + k_2} A = \frac{2 k_1}{k_1 + k_2} A
\end{equation*}

\subsubsection{Finite Potential Barrier}
Generalization of step potential such that the potential form a bump relative to the surroundings
\begin{equation*}
    V(x) =
    \begin{cases}
        0,   & x < 0         \\
        V_0, & 0 \le x \le L \\
        0,   & L < x
    \end{cases}
\end{equation*}

The wavefunction in this system is
\begin{equation*}
    \psi(x) =
    \begin{cases}
        A e^{i k_1 x} + B e^{-i k_1 x},   & x < 0                        \\[1mm]
        C e^{i k_2 x} + D e^{-i k_2 x},   & 0 \le x \le L, \quad E > V_0 \\[1mm]
        C e^{\kappa x} + D e^{-\kappa x}, & 0 \le x \le L, \quad E < V_0 \\[1mm]
        F e^{i k_1 x},                    & L<x
    \end{cases}
\end{equation*}
$A$ represents incident amplitude, $B$ reflected amplitude, $C$ and $D$ amplitudes inside the barrier (right/left moving or decaying/growing), $F$ represents transmitted amplitude.

Physically, if there is only one particle coming from the left, there is no source on the right.
Therefore, we set $G=0$.

For a finite potential barrier or well, there is generally no simple closed-form expression for reflection and transmission amplitude.
They still can be written as
\begin{equation*}
    r=\frac{B }{A }\qquad t=\frac{F }{A}
\end{equation*}
For the probabilities
\begin{equation*}
    R=|r|^2=\frac{|A|^2}{|B|^2}\qquad T=\frac{k_3 }{k_1 }|t^2|=\frac{|F|^2}{|A|^2}
\end{equation*}

The derivation is mostly the same as the infinite case.
In the incident region
\begin{align*}
    -\frac{\hbar^2}{2m} \frac{d^2 \psi_{I}}{dx^2} & =  E \psi_I        \\
    \frac{d^2 \psi_{I}}{dx^2}                     & = - k_1^2 \psi_{I}
\end{align*}
with
\begin{equation*}
    k_1^2 =\frac{2 m E}{\hbar^2}
\end{equation*}
The general solution is
\begin{equation*}
    \psi_I(x) = A e^{i k_1 x} + B e^{-i k_1 x}
\end{equation*}

Now, the region II
\begin{align*}
    -\frac{\hbar^2}{2m} \frac{d^2 \psi_{II}}{dx^2} + V_0 \psi_{II} & =  E \psi_{II}      \\
    \frac{d^2 \psi_{II}}{dx^2}                                     & = - k_2^2 \psi_{II}
\end{align*}
The value  of $k_2$ depends on the value of $E$
\begin{equation*}
    k_2 =
    \begin{cases}
        \sqrt{\dfrac{2 m (E - V_0)}{\hbar^2}}                           & V_0<E   \\
        i \kappa, \quad \kappa = \sqrt{\dfrac{2 m (V_0 - E)}{\hbar^2}}, & E < V_0
    \end{cases}
\end{equation*}
The general form in the region II is
\begin{equation*}
    \psi_{II}(x)=C e^{i k_2 x} + D e^{-i k_2 x}
\end{equation*}
For the case of $V_0<E$,
\begin{equation*}
    \psi_{II}(x) = C e^{i k_2 x} + D e^{-i k_2 x}
\end{equation*}
For the case of $E<V_0$,
\begin{equation*}
    \psi_{II}(x) =C e^{-\kappa x}+ + D e^{-i\kappa x}
\end{equation*}
Here we do not throw $D$ away since $x$ does not approach infinity.

Region III is the same as region I
\begin{equation*}
    \frac{d^2 \psi_{III}}{dx^2} = - k_3^2 \psi_{III}
\end{equation*}
with
\begin{equation*}
    k_3^2 =k_1=\frac{2 m E}{\hbar^2}
\end{equation*}
The general solution is
\begin{equation*}
    \psi_I(x) = F e^{i k_1 x} + G e^{-i k_1 x}
\end{equation*}
If there are no incident waves from right, we throw $G$ away

\subsubsection{Finite Potential Well}
Generalization of potential well such that the potential form a dip relative to the surroundings
\begin{equation*}
    V(x)=
    \begin{cases}
        0,   & |x|\leq L/2 \\
        V_0, & L/2 <|x|
    \end{cases}
\end{equation*}

For $E<V_0$, the wavefunction is
\begin{equation*}
    \psi^{\text{even}}(x) =
    \begin{cases}
        F e^{\kappa (x + L/2)}, \\
        C \cos(k x),            \\
        F e^{-\kappa (x - L/2)},
    \end{cases}
    \quad
    \psi^{\text{odd}}(x) =
    \begin{cases}
        - F e^{\kappa (x + L/2)}, & x < -L/2    \\
        D \sin(k x),              & |x| \le L/2 \\
        F e^{-\kappa (x - L/2)},  & x > L/2
    \end{cases}
\end{equation*}
For $V_0<E$,
\begin{align*}
    \psi^{\text{even}}(x) & =
    \begin{cases}
        A e^{i k_1 x} , & x < -L/     \\
        C \cos(k x),    & |x| \le L/2 \\
        A e^{-i k_1 x}, & x > L/2
    \end{cases}
    \\
    \psi^{\text{odd}}(x)  & =
    \begin{cases}
        A e^{i k_1 x},   & x < -L/2    \\
        D \sin(k x),     & |x| \le L/2 \\
        -A e^{i k_1 x} , & x > L/2
    \end{cases}
\end{align*}

We have used the parameter
\begin{equation*}
    k = \sqrt{\frac{2 m E}{\hbar^2}}, \quad
    k_1 =
    \begin{cases}
        \sqrt{\dfrac{2 m (E - V_0)}{\hbar^2}}                           & V_0<E   \\
        i \kappa, \quad \kappa = \sqrt{\dfrac{2 m (V_0 - E)}{\hbar^2}}, & E < V_0
    \end{cases}
\end{equation*}

Using this parameter, we express the transcendental equation to find the allowed energy eigenvalue.
With $E<V_0$
\begin{align*}
    \sqrt{E} \tan \left( \frac{L }{2 }\sqrt{\frac{2mE}{\hbar^2}} \right) & = \sqrt{V_0-E}  \\
    \sqrt{E} \cot \left( \frac{L }{2 }\sqrt{\frac{2mE}{\hbar^2}} \right) & = -\sqrt{V_0-E}
\end{align*}
for even and odd parity respectively.
With $V_0<E$
\begin{align*}
    \sqrt{E} \tan \left( \frac{L }{2 }\sqrt{\frac{2mE}{\hbar^2}} \right) & = i\sqrt{E-V_0}  \\
    \sqrt{E} \cot \left( \frac{L }{2 }\sqrt{\frac{2mE}{\hbar^2}} \right) & = -i\sqrt{E-V_0}
\end{align*}
for even and odd parity respectively.

We shall now derive it.
In the region I, the Schrödinger equation reads
\begin{align*}
    -\frac{\hbar^2}{2 m} \frac{d^2 \psi_{I}}{dx^2} + V_0 \psi_{I} & =  E \psi_{I}   \\
    \frac{d^2 \psi_{I}}{dx^2}                                     & = k_1^2\psi_{I}
\end{align*}
with
\begin{equation*}
    k_1 =
    \begin{cases}
        \sqrt{\dfrac{2 m (E - V_0)}{\hbar^2}}                           & V_0<E   \\
        i \kappa, \quad \kappa = \sqrt{\dfrac{2 m (V_0 - E)}{\hbar^2}}, & E < V_0
    \end{cases}
\end{equation*}
The general solution is
\begin{equation*}
    \psi_{I}(x) = A e^{ik_1x} + B e^{-ik_1 x}
\end{equation*}

For the case of $V_0<E$,
\begin{equation*}
    \psi_{I}(x) = A e^{i k_1 x} + B e^{-i k_1 x}
\end{equation*}
Now enforcing odd parity $\psi(x)=-\psi(-x)$ and even  parity $\psi(x)=\psi(-x)$, we have
\begin{equation*}
    \psi_{III}^{\text{odd}}=-A e^{i k_1 x}
    \qquad
    \psi_{III}^{\text{odd}}=A e^{-i k_1 x}
\end{equation*}
where we have also considered single-sided scattering such that $B=0$.

For the case of $E<V_0$, we throw $A$ away since it diverges as $x \rightarrow -\infty$
\begin{equation*}
    \psi_{I}(x) = B e^{\kappa x}
\end{equation*}
Define a new amplitude at the boundary \(x = -L/2\) as $F = B e^{\kappa L/2}$, so that
\begin{equation*}
    \psi_{I}(x) = F e^{\kappa (x + L/2)}
\end{equation*}
when applying boundaries $\psi_{I}(-L/2) = F$.
Enforcing odd parity $\psi(x)=-\psi(-x)$ and even  parity $\psi(x)=\psi(-x)$, we have
\begin{equation*}
    \psi_{III}^{\text{odd}}(x) = -F e^{-\kappa (x - L/2)}
    \qquad
    \psi_{III}^{\text{even}}(x) = F e^{-\kappa (x - L/2)}
\end{equation*}
Now we can use this wavefunction for both odd and even parity, nothing would matter physically.
However, the $\psi_{III}^{\text{odd}}$ is negative; and by convention we want to have positive right wavefunction.
As such, we reversed the order, and choose
\begin{equation*}
    \psi_{III}(x) = F e^{-\kappa (x - L/2)}
\end{equation*}
as reference.
Now enforcing odd and even parity yield the region I wavefunction
\begin{equation*}
    \psi_{I}^{\text{odd}}(x) = -F e^{\kappa (x + L/2)},
    \qquad
    \psi_{I}^{\text{even}}(x) = F e^{\kappa (x + L/2)}
\end{equation*}

In the region II, the Schrödinger equation reads
\begin{align*}
    -\frac{\hbar^2}{2 m} \frac{d^2 \psi_{II}}{dx^2} & =  E \psi_{II}      \\
    \frac{d^2 \psi_{II}}{dx^2}                      & = - k_2^2 \psi_{II}
\end{align*}
with $k_2^2=\sqrt{2mE/\hbar^2}$.
The general solution is
\begin{equation*}
    \psi_{II}(x)=C\sin (k_2x)+D \cos (k_2x)
\end{equation*}
Enforcing the parity
\begin{equation*}
    \psi_{II}^{\text{odd}}(x) = C \sin (k_2x)
    \qquad
    \psi_{II}^{\text{even}}(x) =  D \cos (k_2x)
\end{equation*}

Now we determine the energy eigenvalue.
Consider the case of $E<V_0$ first.
For even parity, continuity at $x=L/2$ demands
\begin{equation*}
    \begin{cases}
        F=C \cos \left(  kL/2 \right) \\
        -\kappa F =-C k \sin  \left( kL/2\right)
    \end{cases}
\end{equation*}
Divide the derivative continuity with the wavefunction continuity
\begin{equation*}
    k \tan \left( \frac{k L}{2} \right) =\kappa
\end{equation*}
For even parity,
\begin{equation*}
    \begin{cases}
        D \sin (kL/2)=F \\
        kD \cos (kL/2)=\kappa F
    \end{cases}
\end{equation*}
Divide the derivative continuity with the wavefunction continuity
\begin{equation*}
    k \cot \left( \frac{k L}{2} \right) =\kappa
\end{equation*}

Now consider the case $V_0<E$.
For even parity, continuity at $x=L/2$ demands
\begin{equation*}
    \begin{cases}
        Ae^{ik_1L/2}=C \cos (kL/2) \\
        -ik_1 Ae^{ik_1L/2}=-C k \sin  \left( kL/2\right)
    \end{cases}
\end{equation*}
Divide the derivative continuity with the wavefunction continuity
\begin{equation*}
    k \tan \left( \frac{k L}{2} \right) =ik_1
\end{equation*}
For even parity,
\begin{equation*}
    \begin{cases}
        D \sin (kL/2)=-Ae^{ik_1L/2} \\
        kD \cos (kL/2)=ik_1Ae^{ik_1L/2}
    \end{cases}
\end{equation*}
Divide the derivative continuity with the wavefunction continuity
\begin{equation*}
    k \cot \left( \frac{k L}{2} \right) =ik_1
\end{equation*}

\subsection{Appendix: Simple System of Harmonic Potential}
\subsubsection{1D Harmonic Potential.}
Particle under harmonic potential is subjected to the potential in the following form
\begin{equation*}
    V(x) = \frac{1}{2} m \omega^{2} x^{2},
\end{equation*}
In the $X$ basis, the Schrödinger equation now reads
\begin{equation*}
    -\frac{\hbar^{2}}{2m} \frac{d^{2}\psi(x)}{dx^{2}} + \frac{1}{2}m\omega^{2}x^{2}\psi(x) = E\psi(x).
\end{equation*}
Solving this yields discrete energy eigenvalues
\[
    E_{n} = \hbar\omega \left(n + \frac{1}{2}\right), \quad n = 0, 1, 2, \ldots
\]
and normalized eigenfunctions
\[
    \psi_{n}(x) = \left(\frac{m\omega}{\pi\hbar}\right)^{1/4} \frac{1}{\sqrt{2^{n} n!}} H_{n}\left(\sqrt{\frac{m\omega}{\hbar}} x\right) e^{-\frac{m\omega x^{2}}{2\hbar}}
\]
where \(H_{n}\) are Hermite polynomials.
Here are the first few.
\begin{align*}
    H_{0}(x) & = 1,                     \\
    H_{1}(x) & = 2x,                    \\
    H_{2}(x) & = 4x^{2} - 2,            \\
    H_{3}(x) & = 8x^{3} - 12x,          \\
    H_{4}(x) & = 16x^{4} - 48x^{2} + 12
\end{align*}

\subsubsection{Derivation.}
From the Schrödinger equation
\begin{equation*}
    \frac{d^{2}\psi(x)}{dx^{2}} + \frac{2m }{\hbar^2 }\left(E-\omega^{2}x^{2}\right)\psi(x)
\end{equation*}
we make the substitution of $x=by$, which result in differential
\begin{align*}
    \frac{d\psi }{dx}       & =\frac{d\psi }{dy}\frac{dy }{dx}=\frac{1 }{b }\frac{d\psi }{dy}                                                                       \\
    \frac{d^2 \psi }{dx^2 } & = \frac{d }{dx }\left( \frac{1 }{b }\frac{d\psi }{dy} \right) =\frac{1 }{b}\frac{d^2\psi }{dy\;dx}=\frac{1 }{b }\frac{d^2\psi }{dy^2}
\end{align*}
Now Schrödinger equation reads
\begin{equation*}
    \frac{1 }{b^2 }\frac{d^2\psi }{dy^2 }+\frac{2m }{\hbar^2 }\left( E-\frac{1 }{2 }m\omega^2b^2y^2   \right) \psi=0
\end{equation*}
or
\begin{equation*}
    \frac{d^2\psi }{dy^2 }+\left( \frac{2mEb^2 }{\hbar^2 }-\frac{m^2\omega^2b^4y^2 }{\hbar^2}   \right) \psi=0
\end{equation*}
Now define
\begin{equation*}
    \epsilon=\frac{mEb^2 }{\hbar^2}=\frac{E }{\hbar\omega},\qquad b^2=\frac{\hbar }{m\omega}
\end{equation*}
Then
\begin{equation*}
    \frac{d^2 \psi}{dy^2}+(2\epsilon-y^2)\psi=0
\end{equation*}
Compare this to the Hermite equation
\begin{equation*}
    y_n''+(2n+1-x^2)y_n=0
\end{equation*}
which has the solution called the Hermite function
\begin{equation*}
    y_n=e^{x^2/2}\left(\frac{d}{dx}\right)^ne^{-x^2}=e^{-x^2/2}H_n(x)
\end{equation*}
On comparing $2\epsilon$ with $2n+1$, we obtain the expression for energy.
Then the solution to the Schrödinger equation is
\begin{equation*}
    \psi(y)=e^{-y^2/2}H_n(y)\qquad\text{or}\qquad\psi(x)=\exp \left[ -\frac{m\omega }{2 \hbar}x^2 \right] H_n \left[ \left( \frac{m \omega }{\hbar }^{1/2}x \right)  \right]
\end{equation*}

All that left is normalization.
To do so consider the orthogonality of Hermite polynomials
\begin{equation*}
    \int_{-\infty}^{\infty}e^{-x^2}H_n(x)H_m(x)\;dx=\sqrt{\pi}2^n n!\delta_{nm}
\end{equation*}
Therefore
\begin{align*}
    \braket{\psi|\psi} & = \int A^2\exp \left[ -\frac{m\omega }{ \hbar}x^2 \right] H_n \left\{\left[ \left( \frac{m \omega }{\hbar }^{1/2}x \right)  \right] \right\}^2\;dx                                                                                 \\
                       & = A^2 \sqrt{\frac{\hbar }{m\omega}}  \int A^2\exp \left[ -\frac{m\omega }{ \hbar}x^2 \right] H_n \left\{\left[ \left( \frac{m \omega }{\hbar }^{1/2}x \right)  \right] \right\}^2\;d \left( \sqrt{\frac{m\omega }{\hbar}x} \right) \\
    1                  & = A^2  \sqrt{\frac{\hbar }{m\omega }}(\pi)^{1/2}2^nn!                                                                                                                                                                              \\
    A                  & = \left( \frac{m\omega }{\pi \hbar 2^{2n}(n!)^2 } \right) ^{1/4}
\end{align*}
Inserting this into the wavefunction yields the normalized wavefunction.

\subsubsection{Ladder Operator.}
The ladder operator are operators that raise or lower the eigenvalue of another operator, here it is the Hamiltonian.
They are defined as
\begin{align*}
    a           & =  \left( \frac{m \omega}{2 \hbar} \right)^{1/2} X + i \left( \frac{1}{2 m \omega \hbar} \right)^{1/2} P \\
    a^{\dagger} & =  \left( \frac{m \omega}{2 \hbar} \right)^{1/2} X - i \left( \frac{1}{2 m \omega \hbar} \right)^{1/2} P
\end{align*}
which satisfy $[a,a^\dagger]=1$ and referred as lowering $a$  and raising $a^\dagger$  operators because of how they modify the quantum state’s excitation number.
They are also called destruction and creation operators since they destroy or create quanta of energy $\hbar \omega$.
We can also write the position and momentum in terms of these operators
\begin{gather*}
    X = \left( \frac{\hbar}{2m\omega} \right)^{1/2} (a + a^{\dagger}) \\
    P = i \left( \frac{m \omega \hbar}{2} \right)^{1/2} (a^{\dagger} - a)
\end{gather*}

From the physical Hamiltonian
\begin{equation*}
    H=\frac{P^2}{2m }+\frac{1 }{2 }m\omega^2 X^2=\left( a a ^\dagger+\frac{1 }{2 } \right)\hbar \omega
\end{equation*}
We define an operator
\begin{equation*}
    \hat{H }=\frac{H }{m\omega }=\left( a a ^\dagger+\frac{1 }{2 } \right)
\end{equation*}
whose eigenvalues $\epsilon$ measure energy in units of $\hbar \omega$.
We have then relation
\begin{equation*}
    [a,\hat{H } ]=a,\qquad [a ^\dagger,\hat{H }]=-a ^\dagger
\end{equation*}
From this can be obtained the action of ladder operator on the eigenstate $\ket{n}$
\begin{equation*}
    a |n\rangle = n^{1/2} |n-1\rangle\qquad a^{\dagger} |n\rangle = (n+1)^{1/2} |n+1\rangle
\end{equation*}

The following are matrix components of some operator mentioned
\begin{gather*}
    a ^\dagger\leftrightarrow\begin{bmatrix}
        0 & 0 & 0 & \cdots \\ 1^{1/2} & 0 & 0 & \\ 0 & 2^{1/2} & 0 & \\ 0 & 0 & 3^{1/2} & \\ \vdots & & &
    \end{bmatrix}                                                                                                                                         \\
    a \leftrightarrow\begin{bmatrix}
        0 & 1^{1/2} & 0 & 0 & \cdots \\ 0 & 0 & 2^{1/2} & 0 & \\ 0 & 0 & 0 & 3^{1/2} & \\ \vdots & & & &
    \end{bmatrix}                                                                                                                                          \\
    X \leftrightarrow \left( \frac{\hbar}{2m\omega} \right)^{1/2} \begin{bmatrix} 0 & 1^{1/2} & 0 & 0 & ... \\ 1^{1/2} & 0 & 2^{1/2} & 0 & \\ 0 & 2^{1/2} & 0 & 3^{1/2} & \\ 0 & 0 & 3^{1/2} & 0 & \\ \vdots & & & & \end{bmatrix}           \\
    P \leftrightarrow i \left( \frac{m \omega \hbar}{2} \right)^{1/2} \begin{bmatrix} 0 & -1^{1/2} & 0 & 0 & \cdots \\ 1^{1/2} & 0 & -2^{1/2} & 0 & \\ 0 & 2^{1/2} & 0 & -3^{1/2} & \\ 0 & 0 & 3^{1/2} & 0 & \\ \vdots & & & & \end{bmatrix} \\
    H \leftrightarrow \hbar \omega \begin{bmatrix} \frac{1}{2} & 0 & 0 & 0 & \cdots \\ 0 & \frac{3}{2} & 0 & 0 & \\ 0 & 0 & \frac{5}{2} & & \\ \vdots & & & & \end{bmatrix}
\end{gather*}

\subsubsection{3D Harmonic Potential.}
The generalized form of 3D harmonic potential is
\begin{equation*}
    V(x, y, z) = \frac{1}{2}m(\omega_{x}^{2}x^{2} + \omega_{y}^{2}y^{2} + \omega_{z}^{2}z^{2}),
\end{equation*}
Each coordinate direction oscillates independently with its own frequency, and the system loses spherical symmetry.
In the case of isotropic potential
\begin{equation*}
    V(r) = \frac{1}{2}m\omega^{2}r^{2}
\end{equation*}
This symmetry produces degenerate energy levels because the potential is identical in every direction.
Unlike isotropic potential, anisotropic potential cannot be expressed in a simple form in spherical coordinates because it lacks spherical symmetry.

In Cartesian coordinates the 3D eigenfunctions are products of 1D harmonic-oscillator eigenfunctions
\begin{equation*}
    \Psi_n(x, y, z) = \psi_{n_x}(x) \psi_{n_y}(y) \psi_{n_z}(z)
\end{equation*}
As such
\begin{equation*}
    \Psi_n(x, y, z) = \left( \frac{\alpha^2}{\pi } \right)^{3/4} \frac{H(\xi_x) H(\xi_y) H(\xi_z)}{\sqrt{2^{n}n_x!  n_y! n_z!}}  \exp \left[ -\frac{1}{2} (\xi_x^2 + \xi_y^2 + \xi_z^2) \right]
\end{equation*}
with $n=n_x+n_y+n_z$, $\alpha = \sqrt{m\omega/\hbar }$ and $ \xi_i = \alpha i$.
The corresponding energy, assuming isotropic, is
\begin{equation*}
    E_{n_{x},n_{y},n_{z}} = \hbar \omega \left(n_{x} + n_{y} + n_{z} + \frac{3}{2}  \right)
\end{equation*}
The degeneracy of the shell with total quantum number $n=n_x+n_y+n_z$ is
\begin{equation*}
    g_n=\frac{(n+1)(n+2 )}{2}
\end{equation*}

In the case of anisotropic harmonic potential, the wavefunction remains a product of 1D harmonic oscillator with differing value of $\omega$
\begin{equation*}
    \Psi_n(x, y, z) = \psi_{n_x}(x; \omega_x) \psi_{n_y}(y; \omega_y) \psi_{n_z}(z; \omega_z)
\end{equation*}
As such
\begin{equation*}
    \Psi_n(x, y, z) = \left( \frac{\alpha_x \alpha_y \alpha_z}{\pi^{3/2}} \right)^{1/2} \frac{H(\xi_x) H(\xi_y) H(\xi_z)}{\sqrt{2^{n}n_x!  n_y! n_z!}}  \exp \left[ -\frac{1}{2} (\xi_x^2 + \xi_y^2 + \xi_z^2) \right]
\end{equation*}
with $n=n_x+n_y+n_z$, $\alpha_i = \sqrt{m\omega_i/\hbar }$ and $ \xi_i = \alpha i$.
Whereas the energy is
\begin{equation*}
    E_n = \hbar \left[  \omega_x \left(n_x + \frac{1}{2}  \right) + \omega_y \left(n_y + \frac{1}{2}  \right) + \omega_z \left(n_z + \frac{1}{2}  \right)  \right]
\end{equation*}

In spherical coordinates they are products of a radial function built from generalized Laguerre polynomials and spherical harmonics
\begin{equation*}
    \Psi_{n_{r},l,m}(r, \theta, \phi) = R_{n_{r},l}(r) Y_{l}^{m}(\theta, \phi)
\end{equation*}
With the radial equation
\begin{equation*}
    R_{n_{r},l}(r) = N_{n_{r},l} r^{l} e^{-\frac{1}{2}\alpha r^{2}} L_{n_{r}}^{(l + \frac{1}{2})}(\alpha r^{2}),
\end{equation*}
where $\beta=m\omega/\hbar$$ L_n^{(a)}$ is the generalized Laguerre polynomial and the normalization constant
    \begin{equation*}
        N_{n_r,l} = \sqrt{\frac{2 \alpha^{l+\frac{3}{2}} n_r!}{\Gamma(n_r + l + \frac{3}{2})}}
    \end{equation*}
    The spherical harmonic is
    \begin{equation*}
        Y_{l}^{m}(\theta, \phi) = (-1)^{m} \sqrt{\frac{(2l + 1)}{4\pi} \frac{(l - m)!}{(l + m)!}} P_{l}^{m}(\cos\theta) e^{im\phi}
    \end{equation*}
    Both components obey the orthonormality
    \begin{align*}
        \int_{0}^{2\pi} \int_{0}^{\pi} Y_{l}^{m}(\theta, \phi) Y_{l'}^{m'*}(\theta, \phi) \sin \theta \,d\theta \,d\phi & =  \delta_{ll'} \delta_{mm'} \\
        \int_{0}^{\infty} R_{n_r,l}(r) R_{n'_r,l}(r) r^2 dr                                                             & =  \delta_{n_r n'_r}
    \end{align*}
    such that
    \begin{equation*}
        \int \Psi_{n_r,l,m}^{*}(r, \theta, \phi) \Psi_{n'_r,l',m'}(r, \theta, \phi) d^3r = \delta_{n_r n'_r} \delta_{ll'} \delta_{mm'}
    \end{equation*}

    \subsection{Appendix: Simple System of Hydrogen Atom}
    The electron of the hydrogen atom  moving in a Coulomb potential due to the proton
    \begin{equation*}
        V(r) = - \frac{e^2}{4 \pi \epsilon_0 r}
    \end{equation*}
    obey th Schrödinger equation in spherical coordinates $(r,\theta,\phi)$
    \begin{equation*}
        -\frac{\hbar^{2}}{2m}\nabla^{2}\psi(r, \theta, \phi) - \frac{e^{2}}{4\pi\varepsilon_{0}r}\psi(r, \theta, \phi) = E\psi(r, \theta, \phi)
    \end{equation*}
    since the potential is spherically symmetric.
    Using the separation of variable, the solution reads
    \begin{equation*}
        \psi_{nlm}(r, \theta, \phi) = R_{n\ell}(r) Y_{\ell}^{m}(\theta, \phi)
    \end{equation*}
$R_{n\ell}(r)$ the radial function, which depends on the principal quantum number $n$ and angular momentum quantum number $\ell$
    \begin{equation*}
        R_{n\ell}(r) = \sqrt{\left( \frac{2}{na_0} \right)^3 \frac{(n-\ell-1)!}{2n[(n+\ell)!]} e^{-r/(na_0)} \left( \frac{2r}{na_0} \right)^{\ell} L_{n-\ell-1}^{2\ell+1} \left( \frac{2r}{na_0} \right)}
    \end{equation*}
    are the spherical harmonics, which depend on $\ell$ and the magnetic quantum number $m$
    \begin{equation*}
        Y_{\ell}^{m}(\theta, \phi) = (-1)^{m} \sqrt{\frac{(2\ell + 1)(\ell - m)!}{4\pi (\ell + m)!}} P_{\ell}^{m}(\cos \theta)e^{im\phi}
    \end{equation*}

    The complete description of electron is specified by four quantum number $(n,\ell,m_l,m_s)$, where $m_s=\pm 1/2$.
    The wavefunction of hydrogen atom must be the product of spatial part $\psi_{n\ell m}$ and the spin part $\ket{1/2,m_s}$
    \begin{equation*}
        \Psi_{n\ell m_\ell m_s}=R_{n\ell}(r) Y_{\ell}^{m}(\theta, \phi) \Bigg| \frac{1}{2}, \pm \frac{1}{2} \Bigg\rangle
    \end{equation*}
    Or by using spinor
    \begin{equation*}
        \Psi_{n\ell m_\ell 1/2}=\psi_{n\ell m_l}
        \begin{bmatrix}
            1 \\0
        \end{bmatrix}
        =
        \begin{bmatrix}
            \psi_{n\ell m_l} \\0
        \end{bmatrix}
    \end{equation*}
    and
    \begin{equation*}
        \Psi_{n\ell m_\ell -1/2}=\psi_{n\ell m_l}
        \begin{bmatrix}
            0 \\1
        \end{bmatrix}
        =
        \begin{bmatrix}
            0 \\\psi_{n\ell m_l}
        \end{bmatrix}
    \end{equation*}
    As such, the ground state then may be written as
    \begin{align*}
        \Psi_{100,+\frac{1}{2}} & =  \begin{bmatrix} \psi_{100} \\ 0 \end{bmatrix} = \begin{bmatrix} \frac{1}{\sqrt{\pi a_0^3}}e^{-r/a_0} \\ 0 \end{bmatrix} \\
        \Psi_{100,-\frac{1}{2}} & =  \begin{bmatrix}0\\ \psi_{100} \end{bmatrix} = \begin{bmatrix} 0\\ \frac{1}{\sqrt{\pi a_0^3}}e^{-r/a_0}  \end{bmatrix}
    \end{align*}

    \subsubsection{Energy and Degeneracy.}
    The energy eigenvalues are quantized and depend only on the principal quantum number $n$
    \begin{equation*}
        E_{n} = -\frac{13.6 }{n^{2}} \text{ eV}
    \end{equation*}
    The dependence of $n$ only is a property of central potential.
    The additional degeneracy in $\ell$ is a property of Coulomb potential only, however.
    The degeneracy of the state $n$ is given by
    \begin{equation*}
        g_n = \sum_{\ell=0 }^{n-1 }(2l+1)=n^2
    \end{equation*}
    If we include spin, the degeneracy now
    \begin{equation*}
        g_n =2 \sum_{\ell=0 }^{n-1 }(2l+1)=2n^2
    \end{equation*}

    It is common to refer to the states with $\ell = 0, 1, 2, 3, 4, \dots$ as $s$, $p$, $d$,$f$, $g$, $h$, $\dots$ states.
    In this spectroscopic notation, $1s$ denotes the state $(n= 1, \ell=0)$; $2s$ and $2p$ the $l=0$ and $l = 1$ states at $n = 2$; $3s$, $3p$, and $3d$ the $l = 0$, 1, and 2 states at $n = 3$, and so on.
    No attempt is made to keep track of $m$.

    \subsubsection{Probabilities.}
    The probability of finding the electron between $r$ and $r+dr$ in the spherical shell is
    \begin{align*}
        P(r)\; dr & =  \int_{0}^{2\pi} \int_{0}^{\pi} |\psi_{nlm_{\ell}}(r, \theta, \phi)|^{2} r^{2} \sin \theta \;d\theta\; d\phi\; dr \\
        P(r)\; dr & =  r^{2} |R_{n\ell}(r)|^{2} \;dr \int |Y_{\ell}^{m_{\ell}}(\theta, \phi)|^{2}\; d\Omega
    \end{align*}
    Since spherical harmonics are normalized
    \begin{equation*}
        P(r) dr= r^2 |R_{n\ell}(r)|^2\;dr
    \end{equation*}
    Using this, we can find the probabilities of finding electron within radius $r=a_0$
    \begin{equation*}
        P(r \leq a_0) = \int_{0}^{a_0} P(r) dr = \int_{0}^{a_0} r^2 |R_{10}(r)|^2 dr
    \end{equation*}

    \subsubsection{Hydrogenic Identity.}
    Several additional analytic identities for hydrogenic bound states exist.
    The most frequently used are the expectation values of inverse and quadratic powers of $r$, which are
    \begin{align*}
        \langle r\rangle_{n\ell}
         & = \frac{a_{0}}{2}\bigl(3n^{2}-\ell(\ell+1)\bigr),         \\
        \langle r^{-1}\rangle_{n\ell}
         & = \frac{1}{a_{0}n^{2}},                                   \\
        \langle r^{-2}\rangle_{n\ell}
         & = \frac{1}{a_{0}^{2}n^{3}\left(\ell+\tfrac12\right)},     \\
        \langle r^{2}\rangle_{n\ell}
         & = \frac{a_{0}^{2}}{2}\left[5n^{2}+1-3\ell(\ell+1)\right],
    \end{align*}

    The Virial theorem for the Coulomb potential provides
    \[
        E_{n}
        =-\frac{1}{2}\frac{me^{4}}{\hbar^{2}}\frac{1}{n^{2}},
        \qquad
        \langle T\rangle=-E_{n},
        \qquad
        \langle V\rangle=2E_{n},
        \qquad
        \left\langle \frac{1}{r}\right\rangle=\frac{-2E_{n}}{e^{2}},
    \]

    A further identity links successive radial moments through the Coulomb radial recurrence relation
    \[
        \langle r^{k}\rangle_{n\ell}
        =\frac{(2n+k-1)(n-\ell-1)}{2Z}\langle r^{\,k-1}\rangle_{n\ell}
        -\frac{(n+\ell)(n-\ell-1)}{Z^{2}}\langle r^{\,k-2}\rangle_{n\ell},
    \]
    valid for integer $k\geq 1$ with $Z=1$ for hydrogen.
\end{document}