\documentclass[../../../main.tex]{subfiles}
\begin{document}
\subsection{Appendix: Measurement Example}
\subsubsection{State collapse example.}
Suppose there exist  state described as
\begin{equation*}
    \ket{\psi}=\frac{1 }{2}\ket{\omega,1}+\frac{1 }{2}\ket{\psi}+\sum_{i=3} c_i \ket{\omega_i}
\end{equation*}
It is also known that the orthonormal basis $\ket{\omega,1  }$ and $\ket{\omega,2}$ is degenerate and so do their eigenvalues $\omega_1=\omega_2=\omega$.
If measurement reveal the result $\omega$, the collapsed state is written as
\begin{equation*}
    \ket{\psi}\xrightarrow{\omega\text{ obtained}}\frac{\mathbb{P}_\omega \ket{\omega }}{\braket{\mathbb{P}_\omega\psi|\mathbb{P }_\omega \psi}}^{1/2}
\end{equation*}
where the projection operator $\mathbb{P }_\omega$ for the eigenspace is written as
\begin{equation*}
    \mathbb{P }\omega=\ket{\omega,1}\bra{\omega,1}+\ket{\omega,2}\bra{\omega,2}
\end{equation*}
As such, its action on the state vector is
\begin{equation*}
    \mathbb{P }\omega \ket{\psi}=\ket{\omega,1}\braket{\omega,1|\psi}+\ket{\omega,2}\braket{\omega,2|\psi}
    =\frac{1 }{2 }\ket{\omega,1}+\frac{1 }{2 }\ket{\omega,2}
\end{equation*}
Therefore
\begin{equation*}
    \ket{\psi}=\frac{1}{\left( 1/4+1/4 \right) ^{1/2}}\left(\frac{1 }{2 }\ket{\omega,1}+\frac{1 }{2 }\ket{\omega,2}\right)
    =\frac{1 }{\sqrt{2} }\ket{\omega,1}+\frac{1 }{\sqrt{2} }\ket{\omega,2}
\end{equation*}

\subsection{Appendix: Exercise 4.2.1 from Shankar's Book}
Consider the following operators on a Hilbert space $\mathbb{V}^3(C)$
\begin{equation*}
    L_x=\frac{1 }{\sqrt{2 }}\begin{bmatrix}
        0 & 1 & 0 \\
        1 & 0 & 1 \\
        0 & 1 & 0
    \end{bmatrix}\;
    L_y=\frac{1 }{\sqrt{2 }}\begin{bmatrix}
        0 & -i & 0  \\
        i & 0  & -i \\
        0 & i  & 0
    \end{bmatrix}\;
    L_z=\frac{1 }{\sqrt{2 }}\begin{bmatrix}
        1 & 0 & 0  \\
        0 & 0 & 0  \\
        0 & 0 & -1
    \end{bmatrix}
\end{equation*}
\subsubsection{First question.}
What are the possible values one can obtain if $L_z$ is measured?

Since the $L_z$ is already diagonalized, the eigenvalues are its diagonal entries: $1,0,-1$.

\subsubsection{Second question.}
Take the state in which $L_z=1$.
In this state what are $\braket{L_x}$, $\braket{L_x^2}$ and $\braket{\Delta L_x^2}$?

The state at which $L_z=1$ is written in braket notation as
\begin{align*}
    L_z \ket{\psi} & =  1 \ket{\psi} \\
    \frac{1 }{\sqrt{2 }}\begin{bmatrix}
                            1 & 0 & 0  \\
                            0 & 0 & 0  \\
                            0 & 0 & -1 \\
                        \end{bmatrix}
    \ket{\psi}     & =
    \begin{bmatrix}
        1 &   &   \\
          & 1 &   \\
          &   & 1 \\
    \end{bmatrix}
    \ket{\psi}
\end{align*}
It can be easily seen that the state in question is
\begin{align*}
    \bra{\psi}=\begin{bmatrix}
                   1 & 0 & 0
               \end{bmatrix}
\end{align*}

Then let us determine the expectation value of $L_x$
\begin{align*}
    \braket{L_x}=\braket{\psi|L_x|\psi} & =
    \begin{bmatrix}
        1 & 0 & 0
    \end{bmatrix}
    \frac{1 }{\sqrt{2 }}\begin{bmatrix}
                              & 1 &   \\
                            1 &   & 1 \\
                              & 1 &   \\
                        \end{bmatrix}
    \begin{bmatrix}
        1 \\
        0 \\
        0 \\
    \end{bmatrix}                          \\
                                        & =
    \frac{1 }{\sqrt{2}}
    \begin{bmatrix}
        1 & 0 & 0
    \end{bmatrix}
    \begin{bmatrix}
        0 \\
        1 \\
        0 \\
    \end{bmatrix}
    =
    0
\end{align*}

Then $\braket{L_x^2}$
\begin{align*}
    \braket{L_z^2} & =
    \begin{bmatrix}
        1 & 0 & 0
    \end{bmatrix}
    \frac{1 }{2}
    \begin{bmatrix}
          & 1 &   \\
        1 &   & 1 \\
          & 1 &   \\
    \end{bmatrix}
    \begin{bmatrix}
          & 1 &   \\
        1 &   & 1 \\
          & 1 &   \\
    \end{bmatrix}
    \begin{bmatrix}
        1 \\
        0 \\
        0 \\
    \end{bmatrix}     \\
                   & =
    \frac{1 }{2}
    \begin{bmatrix}
        1 & 0 & 0
    \end{bmatrix}
    \begin{bmatrix}
        1 &   & 1 \\
          & 2 &   \\
        1 &   & 1 \\
    \end{bmatrix}
    \begin{bmatrix}
        1 \\
        0 \\
        0 \\
    \end{bmatrix}     \\
                   & =
    \frac{1 }{2}
    \begin{bmatrix}
        1 & 0 & 0
    \end{bmatrix}
    \begin{bmatrix}
        1 \\
        0 \\
        1 \\
    \end{bmatrix}
    =
    \frac{1 }{2 }
\end{align*}

And the uncertainty $\Delta L_x$
\begin{equation*}
    \Delta L_x =  \left( \braket{L_x^2}-\braket{L_x}^2 \right) ^{1/2 }=\left( \frac{1 }{2 } -0\right) ^{1/2 }=\frac{1 }{\sqrt{2 }}
\end{equation*}

\subsubsection{Third question.}
Find the normalized eigenstates and the eigenvalues of $L_x$ in $L_z$ basis.

First solve the eigenvalue problem
\begin{align*}
    \left( L_x-\omega I  \right) \ket{\omega} & = 0
    \begin{bmatrix}
        -\omega & 1       &         \\
        1       & -\omega & 1       \\
                & 1       & -\omega \\
    \end{bmatrix}
    \ket{\omega}=0
\end{align*}
We have the equation
\begin{equation*}
    \det
    \begin{pmatrix}
        -\omega    & 1/\sqrt{2 } &            \\
        1/\sqrt{2} & -\omega     & 1/\sqrt{2} \\
                   & 1/\sqrt{2}  & -\omega    \\
    \end{pmatrix}
    =0
\end{equation*}
and the characteristic equation
\begin{align*}
    -\omega \left( \omega^2- \frac{1 }{2} \right) +\frac{\omega }{2} & = 0 \\
    \omega \left( -\omega^2+1  \right)                               & = 0
\end{align*}
The eigenvalues are thus $\omega=0,\pm 1$.

Moving on to find the eigenstates.
First the eigenstate corresponding to the eigenvalue $\omega=1$
\begin{align*}
    \left( L_x -\omega I  \right) \ket{L_x=1} & = 0 \\
    \begin{bmatrix}
        -1         & 1/\sqrt{2 } &            \\
        1/\sqrt{2} & -1          & 1/\sqrt{2} \\
                   & 1/\sqrt{2}  & -1         \\
    \end{bmatrix}
    \ket{L_x=1}                               & = 0 \\
    \begin{bmatrix}
        -x+\frac{y }{\sqrt{2 }}                      \\
        \frac{x }{\sqrt{2 }}-y +\frac{z }{\sqrt{2 }} \\
        \frac{y }{\sqrt{2 }}-z
    \end{bmatrix}
                                              & = 0
    \begin{cases}
        \sqrt{2 }x=y \\
        1/2-y+y/2=0  \\
        y=\sqrt{2 }z
    \end{cases}
\end{align*}
If we choose $x=1$, the eigenstate $\ket{L_x'=1}$ and its normalized state $\ket{L_x=1}$ are
\begin{equation*}
    \ket{L_x'=1}=
    \begin{bmatrix}
        1        \\
        \sqrt{2} \\
        1        \\
    \end{bmatrix}
    \qquad
    \ket{L_x=1}=
    \frac{1}{\sqrt{1+2+1}}
    \begin{bmatrix}
        1        \\
        \sqrt{2} \\
        1        \\
    \end{bmatrix}
    =
    \begin{bmatrix}
        1/2        \\
        \sqrt{2}/2 \\
        1/2        \\
    \end{bmatrix}
\end{equation*}

Then the eigenstate corresponding to $\omega=0$
\begin{align*}
    \left( L_x -\omega I  \right) \ket{L_x=0} & = 0 \\
    \begin{bmatrix}
        0          & 1/\sqrt{2 } &            \\
        1/\sqrt{2} & 0           & 1/\sqrt{2} \\
                   & 1/\sqrt{2}  & 0          \\
    \end{bmatrix}
    \ket{L_x=0}                               & = 0 \\
    \begin{bmatrix}
        y/\sqrt{2}              \\
        x/\sqrt{2 }+z/\sqrt{2 } \\
        y/\sqrt{2}
    \end{bmatrix}
                                              & = 0
    \begin{cases}
        y/\sqrt{2}=0 \\
        x+z=0
    \end{cases}
\end{align*}
If we choose $x=1$, the normalized eigenstate is
\begin{equation*}
    \ket{L_x=1}=
    \frac{1}{\sqrt{1+1}}
    \begin{bmatrix}
        1  \\
        0  \\
        -1 \\
    \end{bmatrix}
    =
    \begin{bmatrix}
        1/2  \\
        0    \\
        -1/2 \\
    \end{bmatrix}
\end{equation*}

Then the eigenstate corresponding to $\omega=-1$
\begin{align*}
    \left( L_x -\omega I  \right) \ket{L_x=-1} & = 0 \\
    \begin{bmatrix}
        1          & 1/\sqrt{2 } &            \\
        1/\sqrt{2} & 1           & 1/\sqrt{2} \\
                   & 1/\sqrt{2}  & 1          \\
    \end{bmatrix}
    \ket{L_x=-1}                               & = 0 \\
    \begin{bmatrix}
        x+y/\sqrt{2}               \\
        x/\sqrt{2 }+y+z/\sqrt{2}=0 \\
        y/\sqrt{2 }+z              \\
    \end{bmatrix}
                                               & = 0
    \begin{cases}
        - \sqrt{2}x=y \\
        -y/2+y-y/2=0  \\
        y=-\sqrt{2}z  \\
    \end{cases}
\end{align*}
If we choose $x=1$, the normalized eigenstate is
\begin{equation*}
    \ket{L_x=1}=
    \frac{1}{\sqrt{1+1+2}}
    \begin{bmatrix}
        1         \\
        -\sqrt{2} \\
        1         \\
    \end{bmatrix}
    =
    \begin{bmatrix}
        1/2         \\
        -1/\sqrt{2} \\
        1/2         \\
    \end{bmatrix}
\end{equation*}

\subsubsection{Fourth question.}
If the particle is in the state with $L_z = -1$, and $L_x$ is measured, what are the possible outcomes and their probabilities?

State at which $L_z=-1$ can be obtained from
\begin{align*}
    L_z \ket{\psi} & =-  1 \ket{\psi} \\
    \frac{1 }{\sqrt{2 }}\begin{bmatrix}
                            1 & 0 & 0  \\
                            0 & 0 & 0  \\
                            0 & 0 & -1 \\
                        \end{bmatrix}
    \ket{\psi}     & =
    \begin{bmatrix}
        -1 &    &    \\
           & -1 &    \\
           &    & -1 \\
    \end{bmatrix}
    \ket{\psi}
\end{align*}
The state in question is
\begin{equation*}
    \ket{\psi}=
    \begin{bmatrix}
        0 \\0\\-1\\
    \end{bmatrix}
\end{equation*}

Since the eigenvalues of $L_x$ are $\omega=0,\pm 1$, their probabilities are
\begin{align*}
    P(L_x=1)  & = \left(
    \begin{bmatrix}
            1/2 & \sqrt{2}/2 & 2 & 1/2
        \end{bmatrix}
    \begin{bmatrix}
            0 \\0\\-1
        \end{bmatrix}
    \right) ^2=\frac{1 }{4}  \\
    P(L_x=0)  & = \left(
    \begin{bmatrix}
            1/\sqrt{2} & 0 & 2 & -1/\sqrt{2}
        \end{bmatrix}
    \begin{bmatrix}
            0 \\0\\-1
        \end{bmatrix}
    \right) ^2=\frac{1 }{/2} \\
    P(L_x=-1) & = \left(
    \begin{bmatrix}
            -1/2 & 0 & 2 & -1/2
        \end{bmatrix}
    \begin{bmatrix}
            0 \\0\\-1
        \end{bmatrix}
    \right) ^2=\frac{1 }{4}  \\
\end{align*}

\subsubsection{Fifth question.}
Consider the state, in bra to save space
\begin{equation*}
    \bra{\psi}=\begin{bmatrix}
        \frac{1 }{2} & \frac{1 }{2} & \frac{1 }{\sqrt{2}}
    \end{bmatrix}
\end{equation*}
in the $L_z$ basis.
If $L_z^2$ is measured in this state and a result $+1$ is obtained, what is the state after the measurement?
How probable was this result?
If $L_z$ is measured, what are the outcomes and respective probabilities?

The square of $L_z$ operator is
\begin{equation*}
    L_z=\frac{1 }{2}
    \begin{bmatrix}
        1 & 0 & 0  \\
        0 & 0 & 0  \\
        0 & 0 & -1
    \end{bmatrix}
    \begin{bmatrix}
        1 & 0 & 0  \\
        0 & 0 & 0  \\
        0 & 0 & -1
    \end{bmatrix}
    =
    \begin{bmatrix}
        1 & 0 & 0 \\
        0 & 0 & 0 \\
        0 & 0 & 1
    \end{bmatrix}
\end{equation*}
with eigenvalue of 0 and degenerate value of 1.

The eigenstate corresponding the degenerate eigenvalue is
\begin{align*}
    (L_x^2-\omega I)\ket{L_x^2=1} & = 0 \\
    \begin{bmatrix}
        0 & 0  & 0 \\
        0 & -1 & 0 \\
        0 & 0  & 0
    \end{bmatrix}
    \ket{L_x^2=1}                 & = 0 \\
    \begin{bmatrix}
        0 & -y & 0
    \end{bmatrix}
    =0
\end{align*}
With arbitrary choices of $x$ and $z$, we choose such that the two eigenstate are orthonormal to each other
\begin{equation*}
    \ket{L_x^2=1,1}=
    \begin{bmatrix}
        1 \\0\\0
    \end{bmatrix}\qquad
    \ket{L_x^2=1,2}=
    \begin{bmatrix}
        0 \\0\\1
    \end{bmatrix}
\end{equation*}

Now the eigenstate corresponding to the eigenvalue of 0
\begin{align*}
    (L_x^2-\omega I) \ket{L_x^2} & = 0 \\
    \begin{bmatrix}
        1 &   &   \\
          & 0 &   \\
          &   & 1
    \end{bmatrix}
    \ket{L_x^2}                  & = 0 \\
    \begin{bmatrix}
        x \\0\\z
    \end{bmatrix}
                                 & = 0
\end{align*}
Since $y$ is arbitrary, we choose so that it normalize the eigenstate
\begin{equation*}
    \ket{L_x^2=0}=
    \begin{bmatrix}
        0 \\1\\0
    \end{bmatrix}
\end{equation*}

From each eigenstate, we can construct the state vector in terms of the eigenstates
\begin{align*}
    \ket{\psi } & =  \sum_{i} \ket{L_x^2=i}\braket{L_x^2=1|\psi}                                            \\
                & = \frac{1 }{2 }\ket{L_x^2=0}+\frac{1 }{2 }\ket{L_x^2,1}+\frac{1 }{\sqrt{2 }}\ket{L_x^2,2}
\end{align*}
It can be easily seen that the probability of measuring the eigenstates $P(L_x^2=1)=3/4$

We know that the state changes after the measurement of the degenerate eigenvalue.
To find changed state, we construct the projection operator on this eigenspace
\begin{equation*}
    \mathbb{P }_1=\sum_i \ket{L_x^2,i}\bra{L_x^2,i} =
    \begin{bmatrix}
        1 &   &   \\
          & 0 &   \\
          &   & 0 \\
    \end{bmatrix}
    +
    \begin{bmatrix}
        0 &   &   \\
          & 0 &   \\
        1 &   & 0 \\
    \end{bmatrix}
\end{equation*}
So its action on the state is
\begin{align*}
    P_1 \ket{\psi} & =   \begin{bmatrix}
                             1 &   &   \\
                               & 0 &   \\
                               &   & 0
                         \end{bmatrix}
    \begin{bmatrix}
        \frac{1 }{2} \\ \frac{1 }{2} \\ \frac{1 }{\sqrt{2}}
    \end{bmatrix}
    +
    \begin{bmatrix}
        0 &   &   \\
          & 0 &   \\
        1 &   & 0
    \end{bmatrix}
    \begin{bmatrix}
        \frac{1 }{2} \\ \frac{1 }{2} \\ \frac{1 }{\sqrt{2}}
    \end{bmatrix}                   \\
                   & =
    \begin{bmatrix}
        1/2 \\0\\0
    \end{bmatrix}
    +
    \begin{bmatrix}
        0 \\0\\1/\sqrt{2 }
    \end{bmatrix}                                                                   \\
                   & = \frac{1 }{2 }\ket{L_x^2=1,1}+\frac{1 }{\sqrt{2 }}\ket{L_x^2=1,2}
\end{align*}

\subsubsection{Sixth question.}
A particle is in a state for which the probabilities are $P(L_z=1)=1/4$, $P(L_z=O)=1/2$, and $P(L_z =-1)= 1 /4$.
Convince yourself that the most general, normalized state with this property is
\begin{equation*}
    \ket{\psi}=\frac{e^{i\delta_1 }}{2 }\ket{L_z=1}+\frac{e^{i \delta_2 }}{\sqrt{2 }}\ket{L_z=0}+\frac{e^{i \delta_3 }}{2}\ket{L_z=-1}
\end{equation*}
It was stated earlier on that if $\ket{\psi}$ is a normalized state then the state $e^{i\delta}\ket{y}$ is a physically equivalent normalized state.
Does this mean that the factors $e^{i\delta}$ multiplying the $L_z$ eigenstates are irrelevant?

No.
The vectors $\ket{\psi}$ and $e^{i\theta}\ket{\psi}$ are physically equivalent only in the sense that they generate the same probability distribution for any observable.
This does not mean that when the vector $\psi$ appears as a part of a linear combination it can be multiplied by an arbitrary phase factor

\subsection{Appendix: Shankar's Example to Determine the Expectation Value and Uncertainty of Position Operator}
Suppose that the wave function is $X$ basis $\braket{x|\psi}=\psi(x)$ is a Gaussian, that is
\begin{equation*}
    \psi(x)=A \exp \left[ \-\frac{(X-a)^2 }{2\Delta^2} \right]
\end{equation*}
The normalization of $\psi(x )$ is
\begin{equation*}
    \psi(x)=\frac{1 }{(\pi\Delta^2)^{1/4}}\exp \left[ -\frac{(x-a )^2}{2\Delta^2} \right]
\end{equation*}

Form this, the particle is most likely to be found around $x =a$, and chances of finding it away from this point drop rapidly beyond a distance $\Delta$.
Mathematically
\begin{equation*}
    \braket{X}=a,\quad\Delta X=\frac{\Delta}{\sqrt{2}}
\end{equation*}

\subsubsection{Normalization.}
From the  unity of eigenstate,
\begin{align*}
    \braket{\psi|\psi} & = \int_{-\infty}^{\infty} \braket{\psi|x}\braket{x|\psi}\;dx                    \\
                       & = \int_{-\infty}^{\infty} A^2\exp \left[ -\frac{(x-a)^2}{\Delta^2} \right] \;dx \\
                       & = A^2 \sqrt{\frac{\pi }{1/\Delta^2}}=A^2 \sqrt{\pi \Delta^2}
\end{align*}
As such, $A=(\pi\Delta^2)^{-1/4}$.

\subsubsection{Expectation value.}
With the now normalized state, we can determine the expectation  value of position operator $X$.
The definition of expectation value give
\begin{align*}
    \braket{\psi|X|\psi} & = \int_{-\infty}^{\infty} \braket{\psi|X|x}\braket{x|\psi}\;dx                                                   \\
                         & = \int_{-\infty}^{\infty} \left(\int_{-\infty}^{\infty}  \braket{x|X|x'}\braket{x'|\psi}\;dx' \right)\psi(x)\;dx \\
                         & = \int_{-\infty}^{\infty} \left(\int_{-\infty}^{\infty}  x\delta(x-x')\psi(x')\;dx' \right)\psi(x)\;dx           \\
                         & = \int_{-\infty}^{\infty} \left(\int_{-\infty}^{\infty}  \delta(x'-x)x\psi(x')\;dx' \right)\psi(x)\;dx           \\
                         & = \int_{-\infty}^{\infty} \psi ^*(x)x\psi(x)                                                                     \\
                         & = \frac{1 }{(\pi \Delta^2)^{1/2}}\int_{-\infty}^{\infty} x \exp\left[ -\frac{(x-a)^2}{\Delta^2} \right] dx       \\
                         & = \frac{1 }{(\pi\Delta^2)^{1/2}}\int_{-\infty}^{\infty} (x+a)\exp \left( -\frac{u^2 }{\Delta^2} \right) du       \\
    \braket{X}           & =\frac{1 }{(\pi\Delta^2)^{1/2}} \left( 0+a (\pi\Delta)^{1/2} \right) =a
\end{align*}

\subsubsection{Uncertainty.}
Using the expression for uncertainty
\begin{equation*}
    \Delta\Omega=\left( \braket{\Omega^2}-\braket{\Omega}^2 \right) ^{1/2}
\end{equation*}
we need to determine the expectation value of the square of position operator.
We simply need to evaluate
\begin{align*}
    \braket{X^2} & = \frac{1 }{(\pi\Delta^2)^{1/2}} \int_{-\infty}^{\infty} \psi ^*(x)x^2\psi(x)\;dx                                   \\
                 & = \frac{1 }{(\pi\Delta^2)^{1/2}}\int_{-\infty}^{\infty} (u^2+2ua+a^2)\exp \left( -\frac{u ^2}{\Delta^2} \right)\;dx \\
    \braket{X^2} & = \frac{\Delta^2 }{2}+0+a^2
\end{align*}
So
\begin{equation*}
    \Delta X=\left[ \frac{\Delta^2}{2}+a^{2}-a^{2} \right]^{1/2}=\frac{\Delta}{\sqrt{2}}
\end{equation*}

\subsection{Appendix: Shankar's Example to Determine the Expectation Value and Uncertainty of Momentum Operator}
The solution of the Dirac delta $\braket{x|p}=\psi_p(x)$ is
\begin{equation*}
    \psi_p(x)=\frac{1 }{\sqrt{2\pi \hbar}}e^{ipx/\hbar}
\end{equation*}

The expectation value and uncertainty the of momentum operator is
\begin{equation*}
    \braket{P}=0,\quad\Delta P=\frac{\hbar}{\Delta\sqrt{2}}
\end{equation*}

\subsection{Appendix: Free Particle}
The Schrödinger equation in this case
\begin{equation*}
    i \hbar \ket{\dot{\psi}}=H \ket{\psi}=\frac{p^2 }{2m }\ket{\psi}
\end{equation*}
The normal mode are the solution in the form of $\ket{\psi }= \ket{E }e^{-iE t}/\hbar$.
Feeding the normal mode into the Schrödinger equation
\begin{equation*}
    H \ket{E}=\frac{P^2 }{2m }\ket{E}=E \ket{E}
\end{equation*}
Since $P$ is also an eigenstate of $P^2 $, we feed the trial solution $\ket{p}$
\begin{align*}
    \frac{P^2 }{2m }\ket{p }                    & =  E \ket{p } \\
    \left( \frac{p^2 }{2m }-E  \right) \ket{p } & = 0
\end{align*}
Since $\ket{p}$ is not a null vector, therefore
\begin{equation*}
    p=\pm \sqrt{2mE}
\end{equation*}
The energy eigenvalue $E$ is doubly degenerate because two distinct momentum eigenvalues $p=\pm \sqrt{2mE}$.

Suppose we expand the Hamiltonian in $X$ basis
\begin{equation*}
    -\frac{\hbar^2 }{2m }\frac{d^2 }{dx } \psi(x)=E\psi
\end{equation*}
or
\begin{equation*}
    \frac{d^2 }{dx^2}\psi+k^2\psi=0
\end{equation*}
with
\begin{equation*}
    k^2=\frac{2mE }{\hbar^2}
\end{equation*}
The solution is
\begin{equation*}
    \psi(x)=Ae^{ikx}+Be^{-ikx}
\end{equation*}
From $E=p^2/2m$, we also obtain the relation $p=\hbar k$.

\subsection{Infinite Potential Well}
Consider the one--dimensional infinite square well of width \(L\).
The potential shape is
\begin{equation*}
    V(x)=
    \begin{cases}
        0,      & |x|<L/2      \\
        \infty, & |x| \leq L/2
    \end{cases}
\end{equation*}
We begin by partitioning space into three regions I, II, and III.
According to the Schrödinger equation, the wave function outside the well is simply zero since the potential is infinite.
\begin{equation*}
    -\frac{\hbar^2 }{2m }\frac{d^2 }{dx^2 }\psi_I+\infty \psi_I=E\psi,\implies \psi_I(x)=\psi_{III}=0
\end{equation*}

Next we move to region $II$.
The Schrödinger equation now reads
\begin{align*}
    -\frac{\hbar^2}{2m} \frac{d^2 \psi(x)}{dx^2} & =  E \psi(x)   \\
    \frac{d^2 \psi(x)}{dx^2}                     & = -k^2 \psi(x)
\end{align*}
with $k^2 =2 m E/\hbar^2.$
Recall the boundaries condition of
\begin{equation*}
    \psi \left( -\frac{L }{2 } \right) =\psi \left( \frac{L }{2 } \right) =0
\end{equation*}
Because the potential is symmetric, the Hamiltonian commutes with the parity operator $[H,\Pi] = 0$.
For even parity, we have
\begin{equation*}
    \psi(-x) = \psi(x) \implies B = 0, \quad \psi(x) = A \cos(kx)
\end{equation*}
Applying the second boundaries condition
\begin{equation*}
    \cos\left(\frac{k L}{2}\right) = 0\implies k^{\text{even}} = \frac{(2n-1)\pi}{L},
\end{equation*}
This yield the even parity wave function
\begin{equation*}
    \psi_n^{\text{even}}(x) =A \cos\left(\frac{(2n-1)\pi x}{L}\right)
\end{equation*}
Next for odd parity
\begin{equation*}
    \psi(-x) = -\psi(x) \implies A = 0, \quad \psi(x) = B \sin(kx)
\end{equation*}
Applying the first boundaries condition
\begin{equation*}
    \sin\left(\frac{k L}{2}\right) = 0 \implies k^{\text{odd}} = \frac{2 n \pi}{L}
\end{equation*}
This yield the odd parity wave function
\begin{equation*}
    \psi_n^{\text{odd}}(x) = B \sin\left(\frac{2 n \pi x}{L}\right)
\end{equation*}

Next we begin the normalization step.
To normalize $\psi_n^{\text{even}}$, we perform
\begin{align*}
    B^2\int_{-L/2}^{L/2} \sin^2\left(\frac{2 n \pi x}{L}\right) dx & =  \int_0^{L/2} \left[1 - \cos\left(\frac{4 n \pi x}{L}\right)\right] dx                 \\
                                                                   & = \frac{L }{2 }+ \frac{\sin\left(\frac{4 n \pi x}{L}\right)}{4 n \pi / L} \Bigg|_0^{L/2} \\
                                                                   & = \frac{L }{2 }+\frac{\sin(2 n \pi) - \sin 0}{4 n \pi / L}                               \\
    B^2\int_{-L/2}^{L/2} \sin^2\left(\frac{2 n \pi x}{L}\right) dx & =\frac{L }{2} =1                                                                         \\
    B                                                              & = \sqrt{\frac{L }{2}}
\end{align*}
Then we normalize $\psi_n^{\text{odd}}$
\begin{align*}
    A^2\int_{-L/2}^{L/2} \cos^2\left(\frac{(2n-1) \pi x}{L}\right) dx & =  \int_0^{L/2} \left[1 + \cos\left(\frac{2 (2n-1) \pi x}{L}\right)\right] dx                   \\
                                                                      & = \frac{L }{2 }+\frac{\sin\left(\frac{2 (2n-1) \pi x}{L}\right)}{2(2n-1)\pi / L} \Bigg|_0^{L/2} \\
                                                                      & = \frac{L }{2 }+ \frac{\sin((2n-1)\pi)-\sin0}{2(2n-1)\pi / L}                                   \\
    A^2\int_{-L/2}^{L/2} \cos^2\left(\frac{(2n-1) \pi x}{L}\right) dx & = \frac{L }{2 }=1                                                                               \\
    A                                                                 & =  \sqrt{\frac{L }{2}}
\end{align*}
We also have the energy eigenstate
\begin{equation*}
    E_n^{\text{even}} = \frac{\hbar^2}{2 m} \left[\frac{(2n-1)\pi}{L}\right]^2, \qquad
    E_n^{\text{odd}} = \frac{\hbar^2}{2 m} \left(\frac{2 n \pi}{L}\right)^2
\end{equation*}

\subsubsection{Antisymmetric potential.}
The potential is in the shape
\begin{equation*}
    V(x)=
    \begin{cases}
        \infty, & x \leq0  \\
        0,      & 0<x<L    \\
        \infty, & L \leq L
    \end{cases}
\end{equation*}
As usual, the wave function at infinite potential is zero $\psi(x \leq 0)=\psi(x\leq L)=0$.
In $0 < x < L$, where $ V = 0$,
\begin{align*}
    \frac{d^2\psi}{dx^2} & = -\frac{2mE}{\hbar^2}\psi \\
    \frac{d^2\psi}{dx^2} & = -k^2\psi                 \\
\end{align*}
with $k^2 =2 m E/\hbar^2$.
The solution reads
\begin{equation*}
    \psi(x)=A\sin \biggl(k x\biggr)+ B \cos \biggl(kx\biggr)
\end{equation*}
Imposing boundary condition $\psi(x)=0$, we see that $B=0$
\begin{equation*}
    \psi(x)=A\sin \biggl(\frac{\sqrt{2mE}}{\hbar} x\biggr)
\end{equation*}
Imposing the second boundary condition $\psi(L)=0$, and considering that sinus function goes to zero at $n\pi$,
\begin{equation*}
    k=n\pi
\end{equation*}
and the energy eigenstate
\begin{equation*}
    \psi(x)=\begin{cases}
        0                                          & x\leq 0   \\
        \sqrt{\dfrac{2}{L}}\sin( \dfrac{n\pi}{L}x) & 0 < x < L \\
        0                                          & L \leq x
    \end{cases}
\end{equation*}
where normalization requires $A = \sqrt{2/L}$.
From this, we obtain, the energy eigenvalue of our particle
\begin{equation*}
    E_n=\frac{\pi^2\hbar^2}{2mL^2}n^2
\end{equation*}
This is different from the symmetric potential case.
Even if the “shape” or magnitude of the potential is similar, changing its symmetry changes the mathematical constraints on eigenfunctions, which directly affects the allowed energies.
In quantum mechanics, small changes in the Hamiltonian can produce different spectra, because energies are eigenvalues of the operator, not just classical measures of potential height.


\end{document}