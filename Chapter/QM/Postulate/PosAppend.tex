\documentclass[../../../main.tex]{subfiles}
\begin{document}
\subsection{Appendix: Measurement Example}
\subsubsection{State collapse example.}
Suppose there exist  state described as 
\begin{equation*}
    \ket{\psi}=\frac{1 }{2}\ket{\omega,1}+\frac{1 }{2}\ket{\psi}+\sum_{i=3} c_i \ket{\omega_i}
\end{equation*}
It is also known that the orthonormal basis $\ket{\omega,1  }$ and $\ket{\omega,2}$ is degenerate and so do their eigenvalues $\omega_1=\omega_2=\omega$.
If measurement reveal the result $\omega$, the collapsed state is written as 
\begin{equation*}
    \ket{\psi}\xrightarrow{\omega\text{ obtained}}\frac{\mathbb{P}_\omega \ket{\omega }}{\braket{\mathbb{P}_\omega\psi|\mathbb{P }_\omega \psi}}^{1/2}
\end{equation*}
where the projection operator $\mathbb{P }_\omega$ for the eigenspace is written as 
\begin{equation*}
    \mathbb{P }\omega=\ket{\omega,1}\bra{\omega,1}+\ket{\omega,2}\bra{\omega,2}
\end{equation*} 
As such, its action on the state vector is 
\begin{equation*}
    \mathbb{P }\omega \ket{\psi}=\ket{\omega,1}\braket{\omega,1|\psi}+\ket{\omega,2}\braket{\omega,2|\psi}
    =\frac{1 }{2 }\ket{\omega,1}+\frac{1 }{2 }\ket{\omega,2}
\end{equation*}
Therefore
\begin{equation*}
    \ket{\psi}=\frac{1}{\left( 1/4+1/4 \right) ^{1/2}}\left(\frac{1 }{2 }\ket{\omega,1}+\frac{1 }{2 }\ket{\omega,2}\right)
    =\frac{1 }{\sqrt{2} }\ket{\omega,1}+\frac{1 }{\sqrt{2} }\ket{\omega,2}
\end{equation*}

\subsection{Appendix: Exercise 4.2.1 from Shankar's Book}
Consider the following operators on a Hilbert space $\mathbb{V}^3(C)$
\begin{equation*}
    L_x=\frac{1 }{\sqrt{2 }}\begin{bmatrix}
        0&1&0\\
        1&0&1\\
        0&1&0
    \end{bmatrix}\;
    L_y=\frac{1 }{\sqrt{2 }}\begin{bmatrix}
        0&-i&0\\
        i&0&-i\\
        0&i&0
    \end{bmatrix}\;
    L_z=\frac{1 }{\sqrt{2 }}\begin{bmatrix}
        1&0&0\\
        0&0&0\\
        0&0&-1
    \end{bmatrix}\;
\end{equation*}
\subsubsection{First question.}
What are the possible values one can obtain if $L_z$ is measured?
\subsubsection{Second question.}
Take the state in which $L_z=1$. 
In this state what are $\braket{L_x}$, $\braket{L_x^2}$ and $\braket{\Delta L_x^2}$?

\subsubsection{Third question.}
Find the normalized eigenstates and the eigenvalues of $L_x$ in $L_z$ basis.

\subsubsection{Fourth question.}
If the particle is in the state with $L_z = -1$, and $L_x$ is measured, what are the possible outcomes and their probabilities?

\subsubsection{Fifth question.}
Consider the state, in bra to save space 
\begin{equation*}
    \bra{\psi}=\begin{bmatrix}
        \frac{1 }{2}&\frac{1 }{2}&\frac{1 }{\sqrt{2}}
    \end{bmatrix}
\end{equation*}
in the $L_z$ basis.
If $L_z^2$ is measured in this state and a result $+1$ is obtained, what is the state after the measurement? 
How probable was this result? 
If $L_z$ is measured, what are the outcomes and respective probabilities?

\subsubsection{Sixth question.}
A particle is in a state for which the probabilities are $P(L_z=1)=1/4$, $P(L_z=O)=1/2$, and $P(L_z =-1)= 1 /4$. 
Convince yourself that the most general, normalized state with this property is
\begin{equation*}
    \ket{\psi}=\frac{e^{i\delta_1 }}{2 }\ket{L_z=1}+\frac{e^{i \delta_2 }}{\sqrt{2 }}\ket{L_z=0}+\frac{e^{i \delta_3 }}{2}\ket{L_z=-1}
\end{equation*}
It was stated earlier on that if $\ket{\psi}$ is a normalized state then the state $e^{i\delta}\ket{y}$ is a physically equivalent normalized state. 
Does this mean that the factors $e^{i\delta}$ multiplying the $L_z$ eigenstates are irrelevant?
\end{document}